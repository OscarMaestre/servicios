%% Generated by Sphinx.
\def\sphinxdocclass{report}
\documentclass[letterpaper,10pt,spanish]{sphinxmanual}
\ifdefined\pdfpxdimen
   \let\sphinxpxdimen\pdfpxdimen\else\newdimen\sphinxpxdimen
\fi \sphinxpxdimen=.75bp\relax

\PassOptionsToPackage{warn}{textcomp}
\usepackage[utf8]{inputenc}
\ifdefined\DeclareUnicodeCharacter
% support both utf8 and utf8x syntaxes
  \ifdefined\DeclareUnicodeCharacterAsOptional
    \def\sphinxDUC#1{\DeclareUnicodeCharacter{"#1}}
  \else
    \let\sphinxDUC\DeclareUnicodeCharacter
  \fi
  \sphinxDUC{00A0}{\nobreakspace}
  \sphinxDUC{2500}{\sphinxunichar{2500}}
  \sphinxDUC{2502}{\sphinxunichar{2502}}
  \sphinxDUC{2514}{\sphinxunichar{2514}}
  \sphinxDUC{251C}{\sphinxunichar{251C}}
  \sphinxDUC{2572}{\textbackslash}
\fi
\usepackage{cmap}
\usepackage[T1]{fontenc}
\usepackage{amsmath,amssymb,amstext}
\usepackage{babel}



\usepackage{times}
\expandafter\ifx\csname T@LGR\endcsname\relax
\else
% LGR was declared as font encoding
  \substitutefont{LGR}{\rmdefault}{cmr}
  \substitutefont{LGR}{\sfdefault}{cmss}
  \substitutefont{LGR}{\ttdefault}{cmtt}
\fi
\expandafter\ifx\csname T@X2\endcsname\relax
  \expandafter\ifx\csname T@T2A\endcsname\relax
  \else
  % T2A was declared as font encoding
    \substitutefont{T2A}{\rmdefault}{cmr}
    \substitutefont{T2A}{\sfdefault}{cmss}
    \substitutefont{T2A}{\ttdefault}{cmtt}
  \fi
\else
% X2 was declared as font encoding
  \substitutefont{X2}{\rmdefault}{cmr}
  \substitutefont{X2}{\sfdefault}{cmss}
  \substitutefont{X2}{\ttdefault}{cmtt}
\fi


\usepackage[Sonny]{fncychap}
\ChNameVar{\Large\normalfont\sffamily}
\ChTitleVar{\Large\normalfont\sffamily}
\usepackage{sphinx}

\fvset{fontsize=\small}
\usepackage{geometry}


% Include hyperref last.
\usepackage{hyperref}
% Fix anchor placement for figures with captions.
\usepackage{hypcap}% it must be loaded after hyperref.
% Set up styles of URL: it should be placed after hyperref.
\urlstyle{same}


\usepackage{sphinxmessages}
\setcounter{tocdepth}{1}



\title{Apuntes de Servicios y Procesos}
\date{18 de junio de 2021}
\release{1.3}
\author{Oscar Gomez}
\newcommand{\sphinxlogo}{\vbox{}}
\renewcommand{\releasename}{Versión}
\makeindex
\begin{document}

\ifdefined\shorthandoff
  \ifnum\catcode`\=\string=\active\shorthandoff{=}\fi
  \ifnum\catcode`\"=\active\shorthandoff{"}\fi
\fi

\pagestyle{empty}
\sphinxmaketitle
\pagestyle{plain}
\sphinxtableofcontents
\pagestyle{normal}
\phantomsection\label{\detokenize{index::doc}}



\chapter{Programación multiproceso}
\label{\detokenize{textos/tema1:programacion-multiproceso}}\label{\detokenize{textos/tema1::doc}}

\section{Ejecutables. Procesos. Servicios.}
\label{\detokenize{textos/tema1:ejecutables-procesos-servicios}}

\subsection{Ejecutables}
\label{\detokenize{textos/tema1:ejecutables}}
Un ejecutable es un archivo con la estructura necesaria para que el sistema operativo pueda poner en marcha el programa que hay dentro. En Windows, los ejecutables suelen ser archivos con la extension .EXE.

Se pueden utilizar «desensambladores» para averiguar la secuencia de instrucciones que hay en un EXE. Incluso existen desensambladores en línea como \sphinxurl{http://onlinedisassembler.com}

Sin embargo, Java genera ficheros .JAR o .CLASS. Estos ficheros \sphinxstyleemphasis{no son ejecutables} sino que son archivos que el intérprete de JAVA (el archivo \sphinxcode{\sphinxupquote{java.exe}}) leerá y ejecutará.

El intérprete toma el programa y lo traduce a instrucciones del microprocesador en el que estemos, que puede ser x86 o un x64 o lo que sea. Ese proceso se hace «al instante» o JIT (Just\sphinxhyphen{}In\sphinxhyphen{}Time).

Un archivo .CLASS puede desensamblarse utilizando el comando \sphinxcode{\sphinxupquote{javap \sphinxhyphen{}c \textless{}archivo.class\textgreater{}}} . Cuando se hace así, se obtiene un listado de «instrucciones» que no se corresponden con las instrucciones del microprocesador, sino con «instrucciones virtuales de Java». El intérprte Java (el archivo \sphinxcode{\sphinxupquote{java.exe}}) traducirá en el momento del arranque dichas instrucciones virtuales Java a instrucciones reales del microprocesador.

Este último aspecto es el esgrimido por Java para defender que su ejecución puede ser más rápida que la de un EXE, ya que Java puede averiguar en qué microprocesador se está ejecutando y así generar el código más óptimo posible.

Un EXE puede que no contenga las instrucciones de los microprocesadores más modernos. Como todos son compatibles no es un gran problema, sin embargo, puede que no aprovechemos al 100\% la capacidad de nuestro micro.


\subsection{Procesos}
\label{\detokenize{textos/tema1:procesos}}
Es un archivo que está en ejecución y bajo el control del sistema operativo. Un proceso puede atravesar diversas etapas en su «ciclo de vida». Los estados en los que puede estar son:
\begin{itemize}
\item {} 
En ejecución: está dentro del microprocesador.

\item {} 
Pausado/detenido/en espera: el proceso tiene que seguir en ejecución pero en ese momento el S.O tomó la decisión de dejar paso a otro.

\item {} 
Interrumpido: el proceso tiene que seguir en ejecución pero \sphinxstyleemphasis{el usuario} ha decidido interrumpir la ejecución.

\item {} 
Existen otros estados pero ya son muy dependientes del sistema operativo concreto.

\end{itemize}


\subsection{Servicios}
\label{\detokenize{textos/tema1:servicios}}
Un servicio es un proceso que no muestra ninguna ventana ni gráfico en pantalla porque no está pensado para que el usuario lo maneje directamente.

Habitualmente, un servicio es un programa que atiende a otro programa.


\section{Hilos.}
\label{\detokenize{textos/tema1:hilos}}
Un hilo es un concepto más avanzado que un proceso: al hablar de procesos cada uno tiene su propio espacio en memoria. Si abrimos 20 procesos cada uno de ellos consume 20x de memoria RAM. Un hilo es un proceso mucho más ligero, en el que el código y los datos se comparten de una forma distinta.

Un proceso no tiene acceso a los datos de otro procesos. Sin embargo un hilo sí accede a los datos de otro hilo. Esto complicará algunas cuestiones a la hora de programar.


\section{Programación concurrente.}
\label{\detokenize{textos/tema1:programacion-concurrente}}
La programación concurrente es la parte de la programación que se ocupa de crear programas que pueden tener varios procesos/hilos que colaboran para ejecutar un trabajo y aprovechar al máximo el rendimiento de sistemas multinúcleo. En el caso de la programación concurrente un solo ordenador puede ejecutar varias tareas a la vez (lo que supone que tiene 2 o más núcleos).

Por otro lado se denomina programación paralela a la capacidad de un núcleo de ejecutar dos o más tareas a la vez, normalmente repartiendo el tiempo de proceso entre las tareas.


\section{Programación paralela y distribuida.}
\label{\detokenize{textos/tema1:programacion-paralela-y-distribuida}}
Dentro de la programación concurrente tenemos la paralela y la distribuida:
\begin{itemize}
\item {} 
En general se denomina «programación paralela» a la creación de software que se ejecuta siempre en un solo ordenador (con varios núcleos o no)

\item {} 
Se denomina «programación distribuida» a la creación de software que se ejecuta en ordenadores distintos y que se comunican a través de una red.

\end{itemize}


\section{Creación de procesos.}
\label{\detokenize{textos/tema1:creacion-de-procesos}}
En Java es posible crear procesos utilizando algunas clases que el entorno ofrece para esta tarea. En este tema, veremos en profundidad la clase ProcessBuilder.

El ejemplo siguiente muestra como lanzar un proceso de Acrobat Reader:

\begin{sphinxVerbatim}[commandchars=\\\{\}]
\PYG{k+kd}{public} \PYG{k+kd}{class} \PYG{n+nc}{LanzadorProcesos} \PYG{p}{\PYGZob{}}
        \PYG{k+kd}{public} \PYG{k+kt}{void} \PYG{n+nf}{ejecutar}\PYG{p}{(}\PYG{n}{String} \PYG{n}{ruta}\PYG{p}{)}\PYG{p}{\PYGZob{}}

                \PYG{n}{ProcessBuilder} \PYG{n}{pb}\PYG{p}{;}
                \PYG{k}{try} \PYG{p}{\PYGZob{}}
                        \PYG{n}{pb} \PYG{o}{=} \PYG{k}{new} \PYG{n}{ProcessBuilder}\PYG{p}{(}\PYG{n}{ruta}\PYG{p}{)}\PYG{p}{;}
                        \PYG{n}{pb}\PYG{p}{.}\PYG{n+na}{start}\PYG{p}{(}\PYG{p}{)}\PYG{p}{;}
                \PYG{p}{\PYGZcb{}} \PYG{k}{catch} \PYG{p}{(}\PYG{n}{Exception} \PYG{n}{e}\PYG{p}{)} \PYG{p}{\PYGZob{}}
                        \PYG{c+c1}{// TODO Auto\PYGZhy{}generated catch block}
                        \PYG{n}{e}\PYG{p}{.}\PYG{n+na}{printStackTrace}\PYG{p}{(}\PYG{p}{)}\PYG{p}{;}
                \PYG{p}{\PYGZcb{}}

        \PYG{p}{\PYGZcb{}}
        \PYG{c+cm}{/**}
\PYG{c+cm}{         * @param args}
\PYG{c+cm}{         */}
        \PYG{k+kd}{public} \PYG{k+kd}{static} \PYG{k+kt}{void} \PYG{n+nf}{main}\PYG{p}{(}\PYG{n}{String}\PYG{o}{[}\PYG{o}{]} \PYG{n}{args}\PYG{p}{)} \PYG{p}{\PYGZob{}}
                \PYG{n}{String} \PYG{n}{ruta}\PYG{o}{=}
                        \PYG{l+s}{\PYGZdq{}}\PYG{l+s}{C:}\PYG{l+s}{\PYGZbs{}\PYGZbs{}}\PYG{l+s}{Program Files (x86)}\PYG{l+s}{\PYGZbs{}\PYGZbs{}}\PYG{l+s}{Adobe}\PYG{l+s}{\PYGZbs{}\PYGZbs{}}\PYG{l+s}{Reader 11.0}\PYG{l+s}{\PYGZbs{}\PYGZbs{}}\PYG{l+s}{Reader}\PYG{l+s}{\PYGZbs{}\PYGZbs{}}\PYG{l+s}{AcroRd32.exe}\PYG{l+s}{\PYGZdq{}}\PYG{p}{;}
                \PYG{n}{LanzadorProcesos} \PYG{n}{lp}\PYG{o}{=}\PYG{k}{new} \PYG{n}{LanzadorProcesos}\PYG{p}{(}\PYG{p}{)}\PYG{p}{;}
                \PYG{n}{lp}\PYG{p}{.}\PYG{n+na}{ejecutar}\PYG{p}{(}\PYG{n}{ruta}\PYG{p}{)}\PYG{p}{;}
                \PYG{n}{System}\PYG{p}{.}\PYG{n+na}{out}\PYG{p}{.}\PYG{n+na}{println}\PYG{p}{(}\PYG{l+s}{\PYGZdq{}}\PYG{l+s}{Finalizado}\PYG{l+s}{\PYGZdq{}}\PYG{p}{)}\PYG{p}{;}
        \PYG{p}{\PYGZcb{}}

\PYG{p}{\PYGZcb{}}
\end{sphinxVerbatim}

Supongamos que necesitamos crear un programa que aproveche al máximo el número de CPUs para realizar alguna tarea intensiva. Supongamos que dicha tarea consiste en sumar números.

Enunciado: crear una clase Java que sea capaz de sumar todos los números comprendidos entre dos valores incluyendo ambos valores.

Para resolverlo crearemos una clase \sphinxcode{\sphinxupquote{Sumador}} que tenga un método que acepte dos números \sphinxcode{\sphinxupquote{n1}} y \sphinxcode{\sphinxupquote{n2}} y que devuelva la suma de todo el intervalor.

Además, incluiremos un método \sphinxcode{\sphinxupquote{main}} que ejecute la operación de suma tomando los números de la línea de comandos (es decir, se pasan como argumentos al main).

El código de dicha clase podría ser algo así:

\begin{sphinxVerbatim}[commandchars=\\\{\}]
\PYG{k+kn}{package} \PYG{n+nn}{com.ies}\PYG{p}{;}

\PYG{k+kd}{public} \PYG{k+kd}{class} \PYG{n+nc}{Sumador} \PYG{p}{\PYGZob{}}
        \PYG{k+kd}{public} \PYG{k+kt}{int} \PYG{n+nf}{sumar}\PYG{p}{(}\PYG{k+kt}{int} \PYG{n}{n1}\PYG{p}{,} \PYG{k+kt}{int} \PYG{n}{n2}\PYG{p}{)}\PYG{p}{\PYGZob{}}
                \PYG{k+kt}{int} \PYG{n}{resultado}\PYG{o}{=}\PYG{l+m+mi}{0}\PYG{p}{;}
                \PYG{k}{for} \PYG{p}{(}\PYG{k+kt}{int} \PYG{n}{i}\PYG{o}{=}\PYG{n}{n1}\PYG{p}{;}\PYG{n}{i}\PYG{o}{\PYGZlt{}}\PYG{o}{=}\PYG{n}{n2}\PYG{p}{;}\PYG{n}{i}\PYG{o}{+}\PYG{o}{+}\PYG{p}{)}\PYG{p}{\PYGZob{}}
                        \PYG{n}{resultado}\PYG{o}{=}\PYG{n}{resultado}\PYG{o}{+}\PYG{n}{i}\PYG{p}{;}
                \PYG{p}{\PYGZcb{}}
                \PYG{k}{return} \PYG{n}{resultado}\PYG{p}{;}
        \PYG{p}{\PYGZcb{}}
        \PYG{k+kd}{public} \PYG{k+kd}{static} \PYG{k+kt}{void} \PYG{n+nf}{main}\PYG{p}{(}\PYG{n}{String}\PYG{o}{[}\PYG{o}{]} \PYG{n}{args}\PYG{p}{)}\PYG{p}{\PYGZob{}}
                \PYG{n}{Sumador} \PYG{n}{s}\PYG{o}{=}\PYG{k}{new} \PYG{n}{Sumador}\PYG{p}{(}\PYG{p}{)}\PYG{p}{;}
                \PYG{k+kt}{int} \PYG{n}{n1}\PYG{o}{=}\PYG{n}{Integer}\PYG{p}{.}\PYG{n+na}{parseInt}\PYG{p}{(}\PYG{n}{args}\PYG{o}{[}\PYG{l+m+mi}{0}\PYG{o}{]}\PYG{p}{)}\PYG{p}{;}
                \PYG{k+kt}{int} \PYG{n}{n2}\PYG{o}{=}\PYG{n}{Integer}\PYG{p}{.}\PYG{n+na}{parseInt}\PYG{p}{(}\PYG{n}{args}\PYG{o}{[}\PYG{l+m+mi}{1}\PYG{o}{]}\PYG{p}{)}\PYG{p}{;}
                \PYG{k+kt}{int} \PYG{n}{resultado}\PYG{o}{=}\PYG{n}{s}\PYG{p}{.}\PYG{n+na}{sumar}\PYG{p}{(}\PYG{n}{n1}\PYG{p}{,} \PYG{n}{n2}\PYG{p}{)}\PYG{p}{;}
                \PYG{n}{System}\PYG{p}{.}\PYG{n+na}{out}\PYG{p}{.}\PYG{n+na}{println}\PYG{p}{(}\PYG{n}{resultado}\PYG{p}{)}\PYG{p}{;}
        \PYG{p}{\PYGZcb{}}
\PYG{p}{\PYGZcb{}}
\end{sphinxVerbatim}

Para ejecutar este programa desde dentro de Eclipse es necesario indicar que deseamos enviar \sphinxstyleemphasis{argumentos} al programa. Por ejemplo, si deseamos sumar los números del 2 al 10, deberemos ir a la venta «Run configuration» y en la pestaña «Arguments» indicar los argumentos (que en este caso son los dos números a indicar).

\begin{figure}[htbp]
\centering
\capstart

\noindent\sphinxincludegraphics{{configuraciones}.png}
\caption{Modificando los argumentos del programa}\label{\detokenize{textos/tema1:id1}}\end{figure}

Una vez hecha la prueba de la clase sumador, le quitamos el main, y crearemos una clase que sea capaz de lanzar varios procesos. La clase \sphinxcode{\sphinxupquote{Sumador}} se quedará así:

\begin{sphinxVerbatim}[commandchars=\\\{\}]
\PYG{k+kd}{public} \PYG{k+kd}{class} \PYG{n+nc}{Sumador} \PYG{p}{\PYGZob{}}
        \PYG{k+kd}{public} \PYG{k+kt}{int} \PYG{n+nf}{sumar}\PYG{p}{(}\PYG{k+kt}{int} \PYG{n}{n1}\PYG{p}{,} \PYG{k+kt}{int} \PYG{n}{n2}\PYG{p}{)}\PYG{p}{\PYGZob{}}
                \PYG{k+kt}{int} \PYG{n}{resultado}\PYG{o}{=}\PYG{l+m+mi}{0}\PYG{p}{;}
                \PYG{k}{for} \PYG{p}{(}\PYG{k+kt}{int} \PYG{n}{i}\PYG{o}{=}\PYG{n}{n1}\PYG{p}{;}\PYG{n}{i}\PYG{o}{\PYGZlt{}}\PYG{o}{=}\PYG{n}{n2}\PYG{p}{;}\PYG{n}{i}\PYG{o}{+}\PYG{o}{+}\PYG{p}{)}\PYG{p}{\PYGZob{}}
                        \PYG{n}{resultado}\PYG{o}{=}\PYG{n}{resultado}\PYG{o}{+}\PYG{n}{i}\PYG{p}{;}
                \PYG{p}{\PYGZcb{}}
                \PYG{k}{return} \PYG{n}{resultado}\PYG{p}{;}
        \PYG{p}{\PYGZcb{}}
\PYG{p}{\PYGZcb{}}
\end{sphinxVerbatim}

Y ahora tendremos una clase que lanza procesos de esta forma:

\begin{sphinxVerbatim}[commandchars=\\\{\}]
\PYG{k+kn}{package} \PYG{n+nn}{com.ies}\PYG{p}{;}

\PYG{k+kd}{public} \PYG{k+kd}{class} \PYG{n+nc}{Lanzador} \PYG{p}{\PYGZob{}}
        \PYG{k+kd}{public} \PYG{k+kt}{void} \PYG{n+nf}{lanzarSumador}\PYG{p}{(}\PYG{n}{Integer} \PYG{n}{n1}\PYG{p}{,}
                        \PYG{n}{Integer} \PYG{n}{n2}\PYG{p}{)}\PYG{p}{\PYGZob{}}
                \PYG{n}{String} \PYG{n}{clase}\PYG{o}{=}\PYG{l+s}{\PYGZdq{}}\PYG{l+s}{com.ies.Sumador}\PYG{l+s}{\PYGZdq{}}\PYG{p}{;}
                \PYG{n}{ProcessBuilder} \PYG{n}{pb}\PYG{p}{;}
                \PYG{k}{try} \PYG{p}{\PYGZob{}}
                        \PYG{n}{pb} \PYG{o}{=} \PYG{k}{new} \PYG{n}{ProcessBuilder}\PYG{p}{(}
                                        \PYG{l+s}{\PYGZdq{}}\PYG{l+s}{java}\PYG{l+s}{\PYGZdq{}}\PYG{p}{,}\PYG{n}{clase}\PYG{p}{,}
                                        \PYG{n}{n1}\PYG{p}{.}\PYG{n+na}{toString}\PYG{p}{(}\PYG{p}{)}\PYG{p}{,}
                                        \PYG{n}{n2}\PYG{p}{.}\PYG{n+na}{toString}\PYG{p}{(}\PYG{p}{)}\PYG{p}{)}\PYG{p}{;}
                        \PYG{n}{pb}\PYG{p}{.}\PYG{n+na}{start}\PYG{p}{(}\PYG{p}{)}\PYG{p}{;}
                \PYG{p}{\PYGZcb{}} \PYG{k}{catch} \PYG{p}{(}\PYG{n}{Exception} \PYG{n}{e}\PYG{p}{)} \PYG{p}{\PYGZob{}}
                        \PYG{c+c1}{// TODO Auto\PYGZhy{}generated catch block}
                        \PYG{n}{e}\PYG{p}{.}\PYG{n+na}{printStackTrace}\PYG{p}{(}\PYG{p}{)}\PYG{p}{;}
                \PYG{p}{\PYGZcb{}}
        \PYG{p}{\PYGZcb{}}
        \PYG{k+kd}{public} \PYG{k+kd}{static} \PYG{k+kt}{void} \PYG{n+nf}{main}\PYG{p}{(}\PYG{n}{String}\PYG{o}{[}\PYG{o}{]} \PYG{n}{args}\PYG{p}{)}\PYG{p}{\PYGZob{}}
                \PYG{n}{Lanzador} \PYG{n}{l}\PYG{o}{=}\PYG{k}{new} \PYG{n}{Lanzador}\PYG{p}{(}\PYG{p}{)}\PYG{p}{;}
                \PYG{n}{l}\PYG{p}{.}\PYG{n+na}{lanzarSumador}\PYG{p}{(}\PYG{l+m+mi}{1}\PYG{p}{,} \PYG{l+m+mi}{51}\PYG{p}{)}\PYG{p}{;}
                \PYG{n}{l}\PYG{p}{.}\PYG{n+na}{lanzarSumador}\PYG{p}{(}\PYG{l+m+mi}{51}\PYG{p}{,} \PYG{l+m+mi}{100}\PYG{p}{)}\PYG{p}{;}
                \PYG{n}{System}\PYG{p}{.}\PYG{n+na}{out}\PYG{p}{.}\PYG{n+na}{println}\PYG{p}{(}\PYG{l+s}{\PYGZdq{}}\PYG{l+s}{Ok}\PYG{l+s}{\PYGZdq{}}\PYG{p}{)}\PYG{p}{;}
        \PYG{p}{\PYGZcb{}}
\PYG{p}{\PYGZcb{}}
\end{sphinxVerbatim}


\section{Comunicación entre procesos.}
\label{\detokenize{textos/tema1:comunicacion-entre-procesos}}
Las operaciones multiproceso pueden implicar que sea necesario comunicar información entre muchos procesos, lo que obliga a la necesidad de utilizar mecanismos específicos de comunicación que ofrecerá Java o a diseñar alguno separado que evite los problemas que puedan aparecer.

En el ejemplo, el segundo proceso suele sobreescribir el resultado del primero, así que modificaremos el código del lanzador para que cada proceso use su propio fichero de resultados.

\begin{sphinxVerbatim}[commandchars=\\\{\}]
\PYG{k+kd}{public} \PYG{k+kd}{class} \PYG{n+nc}{Lanzador} \PYG{p}{\PYGZob{}}
        \PYG{k+kd}{public} \PYG{k+kt}{void} \PYG{n+nf}{lanzarSumador}\PYG{p}{(}\PYG{n}{Integer} \PYG{n}{n1}\PYG{p}{,}
                        \PYG{n}{Integer} \PYG{n}{n2}\PYG{p}{,} \PYG{n}{String} \PYG{n}{fichResultado}\PYG{p}{)}\PYG{p}{\PYGZob{}}
                \PYG{n}{String} \PYG{n}{clase}\PYG{o}{=}\PYG{l+s}{\PYGZdq{}}\PYG{l+s}{com.ies.Sumador}\PYG{l+s}{\PYGZdq{}}\PYG{p}{;}
                \PYG{n}{ProcessBuilder} \PYG{n}{pb}\PYG{p}{;}
                \PYG{k}{try} \PYG{p}{\PYGZob{}}
                        \PYG{n}{pb} \PYG{o}{=} \PYG{k}{new} \PYG{n}{ProcessBuilder}\PYG{p}{(}
                                        \PYG{l+s}{\PYGZdq{}}\PYG{l+s}{java}\PYG{l+s}{\PYGZdq{}}\PYG{p}{,}\PYG{n}{clase}\PYG{p}{,}
                                        \PYG{n}{n1}\PYG{p}{.}\PYG{n+na}{toString}\PYG{p}{(}\PYG{p}{)}\PYG{p}{,}
                                        \PYG{n}{n2}\PYG{p}{.}\PYG{n+na}{toString}\PYG{p}{(}\PYG{p}{)}\PYG{p}{)}\PYG{p}{;}

                        \PYG{n}{pb}\PYG{p}{.}\PYG{n+na}{redirectError}\PYG{p}{(}\PYG{k}{new} \PYG{n}{File}\PYG{p}{(}\PYG{l+s}{\PYGZdq{}}\PYG{l+s}{errores.txt}\PYG{l+s}{\PYGZdq{}}\PYG{p}{)}\PYG{p}{)}\PYG{p}{;}
                        \PYG{n}{pb}\PYG{p}{.}\PYG{n+na}{redirectOutput}\PYG{p}{(}\PYG{k}{new} \PYG{n}{File}\PYG{p}{(}\PYG{n}{fichResultado}\PYG{p}{)}\PYG{p}{)}\PYG{p}{;}
                        \PYG{n}{pb}\PYG{p}{.}\PYG{n+na}{start}\PYG{p}{(}\PYG{p}{)}\PYG{p}{;}
                \PYG{p}{\PYGZcb{}} \PYG{k}{catch} \PYG{p}{(}\PYG{n}{Exception} \PYG{n}{e}\PYG{p}{)} \PYG{p}{\PYGZob{}}
                        \PYG{c+c1}{// TODO Auto\PYGZhy{}generated catch block}
                        \PYG{n}{e}\PYG{p}{.}\PYG{n+na}{printStackTrace}\PYG{p}{(}\PYG{p}{)}\PYG{p}{;}
                \PYG{p}{\PYGZcb{}}
        \PYG{p}{\PYGZcb{}}
        \PYG{k+kd}{public} \PYG{k+kd}{static} \PYG{k+kt}{void} \PYG{n+nf}{main}\PYG{p}{(}\PYG{n}{String}\PYG{o}{[}\PYG{o}{]} \PYG{n}{args}\PYG{p}{)}\PYG{p}{\PYGZob{}}
                \PYG{n}{Lanzador} \PYG{n}{l}\PYG{o}{=}\PYG{k}{new} \PYG{n}{Lanzador}\PYG{p}{(}\PYG{p}{)}\PYG{p}{;}
                \PYG{n}{l}\PYG{p}{.}\PYG{n+na}{lanzarSumador}\PYG{p}{(}\PYG{l+m+mi}{1}\PYG{p}{,} \PYG{l+m+mi}{5}\PYG{p}{,} \PYG{l+s}{\PYGZdq{}}\PYG{l+s}{result1.txt}\PYG{l+s}{\PYGZdq{}}\PYG{p}{)}\PYG{p}{;}
                \PYG{n}{l}\PYG{p}{.}\PYG{n+na}{lanzarSumador}\PYG{p}{(}\PYG{l+m+mi}{6}\PYG{p}{,}\PYG{l+m+mi}{10}\PYG{p}{,} \PYG{l+s}{\PYGZdq{}}\PYG{l+s}{result2.txt}\PYG{l+s}{\PYGZdq{}}\PYG{p}{)}\PYG{p}{;}
                \PYG{n}{System}\PYG{p}{.}\PYG{n+na}{out}\PYG{p}{.}\PYG{n+na}{println}\PYG{p}{(}\PYG{l+s}{\PYGZdq{}}\PYG{l+s}{Ok}\PYG{l+s}{\PYGZdq{}}\PYG{p}{)}\PYG{p}{;}
        \PYG{p}{\PYGZcb{}}
\PYG{p}{\PYGZcb{}}
\end{sphinxVerbatim}

Cuando se lanza un programa desde Eclipse no ocurre lo mismo que cuando se lanza desde Windows. Eclipse trabaja con unos directorios predefinidos y puede ser necesario indicar a nuestro programa cual es la ruta donde hay que buscar algo.

Usando el método \sphinxcode{\sphinxupquote{.directory(new File("c:\textbackslash{}\textbackslash{}dir\textbackslash{}\textbackslash{}))}} se puede indicar a Java donde está el archivo que se desea ejecutar.


\section{Ejercicio}
\label{\detokenize{textos/tema1:ejercicio}}
Crear un programa que permita parametrizar el lanzamiento de sumadores, que vuelque el contenido de las sumas en ficheros y que permita al programa principal recuperar las sumas de los ficheros parciales.

En el listado siguiente se muestra la clase Sumador

\begin{sphinxVerbatim}[commandchars=\\\{\}]
\PYG{k+kn}{package} \PYG{n+nn}{es.ies.multiproceso}\PYG{p}{;}

\PYG{k+kd}{public} \PYG{k+kd}{class} \PYG{n+nc}{Sumador} \PYG{p}{\PYGZob{}}
        \PYG{c+cm}{/** Suma todos los valores incluidos}
\PYG{c+cm}{         * entre dos valores}
\PYG{c+cm}{         * @param n1 Limite 1}
\PYG{c+cm}{         * @param n2 Limite 2}
\PYG{c+cm}{         * @return La suma de dichos valores}
\PYG{c+cm}{         */}
        \PYG{k+kd}{public} \PYG{k+kd}{static} \PYG{k+kt}{int} \PYG{n+nf}{sumar}\PYG{p}{(}\PYG{k+kt}{int} \PYG{n}{n1}\PYG{p}{,} \PYG{k+kt}{int} \PYG{n}{n2}\PYG{p}{)}\PYG{p}{\PYGZob{}}
                \PYG{k+kt}{int} \PYG{n}{suma}\PYG{o}{=}\PYG{l+m+mi}{0}\PYG{p}{;}
                \PYG{k}{if} \PYG{p}{(}\PYG{n}{n1}\PYG{o}{\PYGZgt{}}\PYG{n}{n2}\PYG{p}{)}\PYG{p}{\PYGZob{}}
                        \PYG{k+kt}{int} \PYG{n}{aux}\PYG{o}{=}\PYG{n}{n1}\PYG{p}{;}
                        \PYG{n}{n1}\PYG{o}{=}\PYG{n}{n2}\PYG{p}{;}
                        \PYG{n}{n2}\PYG{o}{=}\PYG{n}{aux}\PYG{p}{;}
                \PYG{p}{\PYGZcb{}}
                \PYG{k}{for} \PYG{p}{(}\PYG{k+kt}{int} \PYG{n}{i}\PYG{o}{=}\PYG{n}{n1}\PYG{p}{;} \PYG{n}{i}\PYG{o}{\PYGZlt{}}\PYG{o}{=}\PYG{n}{n2}\PYG{p}{;} \PYG{n}{i}\PYG{o}{+}\PYG{o}{+}\PYG{p}{)}\PYG{p}{\PYGZob{}}
                        \PYG{n}{suma}\PYG{o}{=}\PYG{n}{suma}\PYG{o}{+}\PYG{n}{i}\PYG{p}{;}
                \PYG{p}{\PYGZcb{}}
                \PYG{k}{return} \PYG{n}{suma}\PYG{p}{;}
        \PYG{p}{\PYGZcb{}}

        \PYG{k+kd}{public} \PYG{k+kd}{static} \PYG{k+kt}{void} \PYG{n+nf}{main}\PYG{p}{(}\PYG{n}{String}\PYG{o}{[}\PYG{o}{]} \PYG{n}{args}\PYG{p}{)}\PYG{p}{\PYGZob{}}
                \PYG{k+kt}{int} \PYG{n}{n1}\PYG{o}{=}\PYG{n}{Integer}\PYG{p}{.}\PYG{n+na}{parseInt}\PYG{p}{(}\PYG{n}{args}\PYG{o}{[}\PYG{l+m+mi}{0}\PYG{o}{]}\PYG{p}{)}\PYG{p}{;}
                \PYG{k+kt}{int} \PYG{n}{n2}\PYG{o}{=}\PYG{n}{Integer}\PYG{p}{.}\PYG{n+na}{parseInt}\PYG{p}{(}\PYG{n}{args}\PYG{o}{[}\PYG{l+m+mi}{1}\PYG{o}{]}\PYG{p}{)}\PYG{p}{;}
                \PYG{k+kt}{int} \PYG{n}{suma}\PYG{o}{=}\PYG{n}{sumar}\PYG{p}{(}\PYG{n}{n1}\PYG{p}{,} \PYG{n}{n2}\PYG{p}{)}\PYG{p}{;}
                \PYG{n}{System}\PYG{p}{.}\PYG{n+na}{out}\PYG{p}{.}\PYG{n+na}{println}\PYG{p}{(}\PYG{n}{suma}\PYG{p}{)}\PYG{p}{;}
                \PYG{n}{System}\PYG{p}{.}\PYG{n+na}{out}\PYG{p}{.}\PYG{n+na}{flush}\PYG{p}{(}\PYG{p}{)}\PYG{p}{;}
        \PYG{p}{\PYGZcb{}}
\PYG{p}{\PYGZcb{}}
\end{sphinxVerbatim}

En el listado siguiente se muestra la clase Main

\begin{sphinxVerbatim}[commandchars=\\\{\}]
\PYG{k+kd}{public} \PYG{k+kd}{class} \PYG{n+nc}{Main} \PYG{p}{\PYGZob{}}

        \PYG{k+kd}{static} \PYG{k+kd}{final} \PYG{k+kt}{int} \PYG{n}{NUM\PYGZus{}PROCESOS}\PYG{o}{=}\PYG{l+m+mi}{4}\PYG{p}{;}
        \PYG{k+kd}{static} \PYG{k+kd}{final} \PYG{n}{String} \PYG{n}{PREFIJO\PYGZus{}FICHEROS}\PYG{o}{=}\PYG{l+s}{\PYGZdq{}}\PYG{l+s}{fich}\PYG{l+s}{\PYGZdq{}}\PYG{p}{;}


        \PYG{k+kd}{public} \PYG{k+kd}{static} \PYG{k+kt}{void} \PYG{n+nf}{lanzarSumador}\PYG{p}{(}
                        \PYG{k+kt}{int} \PYG{n}{n1}\PYG{p}{,} \PYG{k+kt}{int} \PYG{n}{n2}\PYG{p}{,}\PYG{n}{String} \PYG{n}{fichResultados}\PYG{p}{)} \PYG{k+kd}{throws} \PYG{n}{IOException}\PYG{p}{\PYGZob{}}
                \PYG{n}{String} \PYG{n}{comando}\PYG{p}{;}
                \PYG{n}{comando}\PYG{o}{=}\PYG{l+s}{\PYGZdq{}}\PYG{l+s}{es.ies.multiproceso.Sumador}\PYG{l+s}{\PYGZdq{}}\PYG{p}{;}

                \PYG{n}{File} \PYG{n}{directorioSumador}\PYG{p}{;}
                \PYG{n}{directorioSumador}\PYG{o}{=}\PYG{k}{new} \PYG{n}{File}\PYG{p}{(}\PYG{l+s}{\PYGZdq{}}\PYG{l+s}{C:}\PYG{l+s}{\PYGZbs{}\PYGZbs{}}\PYG{l+s}{Users}\PYG{l+s}{\PYGZbs{}\PYGZbs{}}\PYG{l+s}{\PYGZdq{}}\PYG{o}{+}
                \PYG{l+s}{\PYGZdq{}}\PYG{l+s}{ogomez}\PYG{l+s}{\PYGZbs{}\PYGZbs{}}\PYG{l+s}{workspace}\PYG{l+s}{\PYGZbs{}\PYGZbs{}}\PYG{l+s}{\PYGZdq{}}\PYG{o}{+}
                \PYG{l+s}{\PYGZdq{}}\PYG{l+s}{MultiProceso1}\PYG{l+s}{\PYGZbs{}\PYGZbs{}}\PYG{l+s}{bin}\PYG{l+s}{\PYGZbs{}\PYGZbs{}}\PYG{l+s}{\PYGZdq{}}\PYG{p}{)}\PYG{p}{;}
                \PYG{n}{File} \PYG{n}{fichResultado}\PYG{o}{=}\PYG{k}{new} \PYG{n}{File}\PYG{p}{(}\PYG{n}{fichResultados}\PYG{p}{)}\PYG{p}{;}
                \PYG{n}{ProcessBuilder} \PYG{n}{pb}\PYG{p}{;}
                \PYG{n}{pb}\PYG{o}{=}\PYG{k}{new} \PYG{n}{ProcessBuilder}\PYG{p}{(}\PYG{l+s}{\PYGZdq{}}\PYG{l+s}{java}\PYG{l+s}{\PYGZdq{}}\PYG{p}{,}
                                \PYG{n}{comando}\PYG{p}{,}
                                \PYG{n}{String}\PYG{p}{.}\PYG{n+na}{valueOf}\PYG{p}{(}\PYG{n}{n1}\PYG{p}{)}\PYG{p}{,}
                                \PYG{n}{String}\PYG{p}{.}\PYG{n+na}{valueOf}\PYG{p}{(}\PYG{n}{n2}\PYG{p}{)} \PYG{p}{)}\PYG{p}{;}
                \PYG{n}{pb}\PYG{p}{.}\PYG{n+na}{directory}\PYG{p}{(}\PYG{n}{directorioSumador}\PYG{p}{)}\PYG{p}{;}
                \PYG{n}{pb}\PYG{p}{.}\PYG{n+na}{redirectOutput}\PYG{p}{(}\PYG{n}{fichResultado}\PYG{p}{)}\PYG{p}{;}
                \PYG{n}{pb}\PYG{p}{.}\PYG{n+na}{start}\PYG{p}{(}\PYG{p}{)}\PYG{p}{;}
        \PYG{p}{\PYGZcb{}}

        \PYG{k+kd}{public} \PYG{k+kd}{static} \PYG{k+kt}{int} \PYG{n+nf}{getResultadoFichero}\PYG{p}{(}
                        \PYG{n}{String} \PYG{n}{nombreFichero}\PYG{p}{)}\PYG{p}{\PYGZob{}}

                \PYG{k+kt}{int} \PYG{n}{suma}\PYG{o}{=}\PYG{l+m+mi}{0}\PYG{p}{;}
                \PYG{k}{try} \PYG{p}{\PYGZob{}}
                        \PYG{n}{FileInputStream} \PYG{n}{fichero}\PYG{o}{=}
                                        \PYG{k}{new} \PYG{n}{FileInputStream}\PYG{p}{(}
                                                        \PYG{n}{nombreFichero}\PYG{p}{)}\PYG{p}{;}
                        \PYG{n}{InputStreamReader} \PYG{n}{fir}\PYG{o}{=}
                                        \PYG{k}{new} \PYG{n}{InputStreamReader}\PYG{p}{(}
                                                        \PYG{n}{fichero}\PYG{p}{)}\PYG{p}{;}
                        \PYG{n}{BufferedReader} \PYG{n}{br}\PYG{o}{=}\PYG{k}{new} \PYG{n}{BufferedReader}\PYG{p}{(}\PYG{n}{fir}\PYG{p}{)}\PYG{p}{;}
                        \PYG{n}{String} \PYG{n}{linea}\PYG{o}{=}\PYG{n}{br}\PYG{p}{.}\PYG{n+na}{readLine}\PYG{p}{(}\PYG{p}{)}\PYG{p}{;}
                        \PYG{n}{suma}\PYG{o}{=} \PYG{k}{new} \PYG{n}{Integer}\PYG{p}{(}\PYG{n}{linea}\PYG{p}{)}\PYG{p}{;}
                        \PYG{k}{return} \PYG{n}{suma}\PYG{p}{;}
                \PYG{p}{\PYGZcb{}} \PYG{k}{catch} \PYG{p}{(}\PYG{n}{FileNotFoundException} \PYG{n}{e}\PYG{p}{)} \PYG{p}{\PYGZob{}}
                        \PYG{n}{System}\PYG{p}{.}\PYG{n+na}{out}\PYG{p}{.}\PYG{n+na}{println}\PYG{p}{(}
                                \PYG{l+s}{\PYGZdq{}}\PYG{l+s}{No se pudo abrir }\PYG{l+s}{\PYGZdq{}}\PYG{o}{+}\PYG{n}{nombreFichero}\PYG{p}{)}\PYG{p}{;}

                \PYG{p}{\PYGZcb{}} \PYG{k}{catch} \PYG{p}{(}\PYG{n}{IOException} \PYG{n}{e}\PYG{p}{)} \PYG{p}{\PYGZob{}}
                        \PYG{n}{System}\PYG{p}{.}\PYG{n+na}{out}\PYG{p}{.}\PYG{n+na}{println}\PYG{p}{(}
                                        \PYG{l+s}{\PYGZdq{}}\PYG{l+s}{No hay nada en }\PYG{l+s}{\PYGZdq{}}\PYG{o}{+}\PYG{n}{nombreFichero}\PYG{p}{)}\PYG{p}{;}
                \PYG{p}{\PYGZcb{}}
                \PYG{k}{return} \PYG{n}{suma}\PYG{p}{;}
        \PYG{p}{\PYGZcb{}}


        \PYG{k+kd}{public} \PYG{k+kd}{static} \PYG{k+kt}{int} \PYG{n+nf}{getSumaTotal}\PYG{p}{(}\PYG{k+kt}{int} \PYG{n}{numFicheros}\PYG{p}{)}\PYG{p}{\PYGZob{}}
                \PYG{k+kt}{int} \PYG{n}{sumaTotal}\PYG{o}{=}\PYG{l+m+mi}{0}\PYG{p}{;}
                \PYG{k}{for} \PYG{p}{(}\PYG{k+kt}{int} \PYG{n}{i}\PYG{o}{=}\PYG{l+m+mi}{1}\PYG{p}{;} \PYG{n}{i}\PYG{o}{\PYGZlt{}}\PYG{o}{=}\PYG{n}{NUM\PYGZus{}PROCESOS}\PYG{p}{;}\PYG{n}{i}\PYG{o}{+}\PYG{o}{+}\PYG{p}{)}\PYG{p}{\PYGZob{}}
                        \PYG{n}{sumaTotal}\PYG{o}{+}\PYG{o}{=}\PYG{n}{getResultadoFichero}\PYG{p}{(}
                                \PYG{n}{PREFIJO\PYGZus{}FICHEROS}\PYG{o}{+}\PYG{n}{String}\PYG{p}{.}\PYG{n+na}{valueOf}\PYG{p}{(}\PYG{n}{i}\PYG{p}{)} \PYG{p}{)}\PYG{p}{;}
                \PYG{p}{\PYGZcb{}}
                \PYG{k}{return} \PYG{n}{sumaTotal}\PYG{p}{;}
        \PYG{p}{\PYGZcb{}}

        \PYG{c+cm}{/* Recibe dos parámetros y hará}
\PYG{c+cm}{         * la suma de los valores comprendidos}
\PYG{c+cm}{         * entre ambos parametros}
\PYG{c+cm}{         */}
        \PYG{k+kd}{public} \PYG{k+kd}{static} \PYG{k+kt}{void} \PYG{n+nf}{main}\PYG{p}{(}\PYG{n}{String}\PYG{o}{[}\PYG{o}{]} \PYG{n}{args}\PYG{p}{)} \PYG{k+kd}{throws} \PYG{n}{IOException}\PYG{p}{,} \PYG{n}{InterruptedException}\PYG{p}{\PYGZob{}}
                \PYG{k+kt}{int} \PYG{n}{n1}\PYG{o}{=}\PYG{n}{Integer}\PYG{p}{.}\PYG{n+na}{parseInt}\PYG{p}{(}\PYG{n}{args}\PYG{o}{[}\PYG{l+m+mi}{0}\PYG{o}{]}\PYG{p}{)}\PYG{p}{;}
                \PYG{k+kt}{int} \PYG{n}{n2}\PYG{o}{=}\PYG{n}{Integer}\PYG{p}{.}\PYG{n+na}{parseInt}\PYG{p}{(}\PYG{n}{args}\PYG{o}{[}\PYG{l+m+mi}{1}\PYG{o}{]}\PYG{p}{)}\PYG{p}{;}
                \PYG{k+kt}{int} \PYG{n}{salto}\PYG{o}{=}\PYG{p}{(} \PYG{n}{n2} \PYG{o}{/} \PYG{n}{NUM\PYGZus{}PROCESOS} \PYG{p}{)} \PYG{o}{\PYGZhy{}} \PYG{l+m+mi}{1}\PYG{p}{;}
                \PYG{k}{for} \PYG{p}{(}\PYG{k+kt}{int} \PYG{n}{i}\PYG{o}{=}\PYG{l+m+mi}{1}\PYG{p}{;}\PYG{n}{i}\PYG{o}{\PYGZlt{}}\PYG{o}{=}\PYG{n}{NUM\PYGZus{}PROCESOS}\PYG{p}{;} \PYG{n}{i}\PYG{o}{+}\PYG{o}{+}\PYG{p}{)}\PYG{p}{\PYGZob{}}
                        \PYG{n}{System}\PYG{p}{.}\PYG{n+na}{out}\PYG{p}{.}\PYG{n+na}{println}\PYG{p}{(}\PYG{l+s}{\PYGZdq{}}\PYG{l+s}{n1:}\PYG{l+s}{\PYGZdq{}}\PYG{o}{+}\PYG{n}{n1}\PYG{p}{)}\PYG{p}{;}
                        \PYG{k+kt}{int} \PYG{n}{resultadoSumaConSalto}\PYG{o}{=}\PYG{n}{n1}\PYG{o}{+}\PYG{n}{salto}\PYG{p}{;}
                        \PYG{n}{System}\PYG{p}{.}\PYG{n+na}{out}\PYG{p}{.}\PYG{n+na}{println}\PYG{p}{(}\PYG{l+s}{\PYGZdq{}}\PYG{l+s}{n2:}\PYG{l+s}{\PYGZdq{}}\PYG{o}{+}\PYG{n}{resultadoSumaConSalto}\PYG{p}{)}\PYG{p}{;}
                        \PYG{n}{lanzarSumador}\PYG{p}{(}\PYG{n}{n1}\PYG{p}{,} \PYG{n}{n1}\PYG{o}{+}\PYG{n}{salto} \PYG{p}{,}
                                        \PYG{n}{PREFIJO\PYGZus{}FICHEROS}\PYG{o}{+}\PYG{n}{String}\PYG{p}{.}\PYG{n+na}{valueOf}\PYG{p}{(}\PYG{n}{i}\PYG{p}{)}\PYG{p}{)}\PYG{p}{;}
                        \PYG{n}{n1}\PYG{o}{=}\PYG{n}{n1} \PYG{o}{+} \PYG{n}{salto} \PYG{o}{+} \PYG{l+m+mi}{1}\PYG{p}{;}
                        \PYG{n}{System}\PYG{p}{.}\PYG{n+na}{out}\PYG{p}{.}\PYG{n+na}{println}\PYG{p}{(}\PYG{l+s}{\PYGZdq{}}\PYG{l+s}{Suma lanzada...}\PYG{l+s}{\PYGZdq{}}\PYG{p}{)}\PYG{p}{;}
                \PYG{p}{\PYGZcb{}}
                \PYG{n}{Thread}\PYG{p}{.}\PYG{n+na}{sleep}\PYG{p}{(}\PYG{l+m+mi}{5000}\PYG{p}{)}\PYG{p}{;}
                \PYG{k+kt}{int} \PYG{n}{sumaTotal}\PYG{o}{=}\PYG{n}{getSumaTotal}\PYG{p}{(}\PYG{n}{NUM\PYGZus{}PROCESOS}\PYG{p}{)}\PYG{p}{;}
                \PYG{n}{System}\PYG{p}{.}\PYG{n+na}{out}\PYG{p}{.}\PYG{n+na}{println}\PYG{p}{(}\PYG{l+s}{\PYGZdq{}}\PYG{l+s}{La suma total es:}\PYG{l+s}{\PYGZdq{}}\PYG{o}{+}
                                                \PYG{n}{sumaTotal}\PYG{p}{)}\PYG{p}{;}
        \PYG{p}{\PYGZcb{}}
\PYG{p}{\PYGZcb{}}
\end{sphinxVerbatim}


\section{Ejercicio resuelto}
\label{\detokenize{textos/tema1:ejercicio-resuelto}}
Crear un programa que sea capaz de contar cuantas vocales hay en un fichero. El programa padre debe lanzar cinco procesos hijo, donde cada uno de ellos se ocupará de contar una vocal concreta (que puede ser minúscula o mayúscula). Cada subproceso que cuenta vocales deberá dejar el resultado en un fichero. El programa padre se ocupará de recuperar los resultados de los ficheros, sumar todos los subtotales y mostrar el resultado final en pantalla.


\subsection{Operaciones con ficheros}
\label{\detokenize{textos/tema1:operaciones-con-ficheros}}
Tanto en este ejercicio como en muchos otros vamos a necesitar realizar ciertas operaciones con ficheros. Para aumentar nuestra productividad utilizaremos una clase llamada \sphinxcode{\sphinxupquote{UtilidadesFicheros}} que nos permita realizar ciertas tareas como las siguientes:
\begin{itemize}
\item {} 
Obtener un objeto \sphinxcode{\sphinxupquote{BufferedReader}} a partir de un nombre de fichero de tipo \sphinxcode{\sphinxupquote{String}}. Llamaremos a este método \sphinxcode{\sphinxupquote{getBufferedReader}} y lo utilizaremos para poder manejar un fichero mediante una clase de muy alto nivel que nos facilita la lectura de ficheros. En el listado adjunto se puede consultar dicho método.

\end{itemize}

\begin{sphinxVerbatim}[commandchars=\\\{\}]
\PYG{k+kd}{public} \PYG{k+kd}{static} \PYG{n}{BufferedReader} \PYG{n+nf}{getBufferedReader}\PYG{p}{(}
    \PYG{n}{String} \PYG{n}{nombreFichero}\PYG{p}{)} \PYG{k+kd}{throws}
\PYG{n}{FileNotFoundException} \PYG{p}{\PYGZob{}}
    \PYG{n}{FileReader} \PYG{n}{lector}\PYG{p}{;}
    \PYG{n}{lector} \PYG{o}{=} \PYG{k}{new} \PYG{n}{FileReader}\PYG{p}{(}\PYG{n}{nombreFichero}\PYG{p}{)}\PYG{p}{;}
    \PYG{n}{BufferedReader} \PYG{n}{bufferedReader}\PYG{p}{;}
    \PYG{n}{bufferedReader} \PYG{o}{=} \PYG{k}{new} \PYG{n}{BufferedReader}\PYG{p}{(}\PYG{n}{lector}\PYG{p}{)}\PYG{p}{;}
    \PYG{k}{return} \PYG{n}{bufferedReader}\PYG{p}{;}
\PYG{p}{\PYGZcb{}}
\end{sphinxVerbatim}
\begin{itemize}
\item {} 
De la misma forma crearemos un método \sphinxcode{\sphinxupquote{getPrintWriter}} que nos devuelva un objeto de la clase \sphinxcode{\sphinxupquote{PrintWriter}} para poder realizar fácilmente operaciones de escritura en ficheros. En el listado adjunto se muestra el código.

\end{itemize}

\begin{sphinxVerbatim}[commandchars=\\\{\}]
\PYG{k+kd}{public} \PYG{k+kd}{static} \PYG{n}{PrintWriter} \PYG{n+nf}{getPrintWriter}\PYG{p}{(}
    \PYG{n}{String} \PYG{n}{nombreFichero}\PYG{p}{)} \PYG{k+kd}{throws} \PYG{n}{IOException} \PYG{p}{\PYGZob{}}
    \PYG{n}{PrintWriter} \PYG{n}{printWriter}\PYG{p}{;}
    \PYG{n}{FileWriter} \PYG{n}{fileWriter}\PYG{p}{;}
    \PYG{n}{fileWriter} \PYG{o}{=} \PYG{k}{new} \PYG{n}{FileWriter}\PYG{p}{(}\PYG{n}{nombreFichero}\PYG{p}{)}\PYG{p}{;}
    \PYG{n}{printWriter} \PYG{o}{=} \PYG{k}{new} \PYG{n}{PrintWriter}\PYG{p}{(}\PYG{n}{fileWriter}\PYG{p}{)}\PYG{p}{;}
    \PYG{k}{return} \PYG{n}{printWriter}\PYG{p}{;}
\PYG{c+c1}{//Fin de getPrintWriter}
\PYG{p}{\PYGZcb{}}
\end{sphinxVerbatim}
\begin{itemize}
\item {} 
Aunque nos estamos anticipando, puede ser útil tener un método que dado un nombre de fichero nos devuelva un \sphinxcode{\sphinxupquote{ArrayList\textless{}String\textgreater{}}} con todas las líneas que hay en el fichero. \sphinxstylestrong{En este ejercicio no lo usaremos}. Este método podría implementarse así:

\end{itemize}

\begin{sphinxVerbatim}[commandchars=\\\{\}]
\PYG{k+kd}{public} \PYG{k+kd}{static} \PYG{n}{ArrayList}\PYG{o}{\PYGZlt{}}\PYG{n}{String}\PYG{o}{\PYGZgt{}} \PYG{n+nf}{getLineasFichero}\PYG{p}{(}
    \PYG{n}{String} \PYG{n}{nombreFichero}\PYG{p}{)} \PYG{k+kd}{throws} \PYG{n}{IOException} \PYG{p}{\PYGZob{}}
    \PYG{n}{ArrayList}\PYG{o}{\PYGZlt{}}\PYG{n}{String}\PYG{o}{\PYGZgt{}} \PYG{n}{lineas} \PYG{o}{=} \PYG{k}{new} \PYG{n}{ArrayList}\PYG{o}{\PYGZlt{}}\PYG{n}{String}\PYG{o}{\PYGZgt{}}\PYG{p}{(}\PYG{p}{)}\PYG{p}{;}
    \PYG{n}{BufferedReader} \PYG{n}{bfr} \PYG{o}{=} \PYG{n}{getBufferedReader}\PYG{p}{(}\PYG{n}{nombreFichero}\PYG{p}{)}\PYG{p}{;}
    \PYG{c+c1}{//Leemos líneas del fichero...}
    \PYG{n}{String} \PYG{n}{linea} \PYG{o}{=} \PYG{n}{bfr}\PYG{p}{.}\PYG{n+na}{readLine}\PYG{p}{(}\PYG{p}{)}\PYG{p}{;}
    \PYG{k}{while} \PYG{p}{(}\PYG{n}{linea} \PYG{o}{!}\PYG{o}{=} \PYG{k+kc}{null}\PYG{p}{)} \PYG{p}{\PYGZob{}}
        \PYG{c+c1}{//Y las añadimos al array}
        \PYG{n}{lineas}\PYG{p}{.}\PYG{n+na}{add}\PYG{p}{(}\PYG{n}{linea}\PYG{p}{)}\PYG{p}{;}
        \PYG{n}{linea} \PYG{o}{=} \PYG{n}{bfr}\PYG{p}{.}\PYG{n+na}{readLine}\PYG{p}{(}\PYG{p}{)}\PYG{p}{;}
    \PYG{p}{\PYGZcb{}}
    \PYG{c+c1}{//Fin del bucle que lee líneas}
    \PYG{k}{return} \PYG{n}{lineas}\PYG{p}{;}
\PYG{p}{\PYGZcb{}}
\end{sphinxVerbatim}


\subsection{Procesado de ficheros}
\label{\detokenize{textos/tema1:procesado-de-ficheros}}
Necesitaremos una clase \sphinxcode{\sphinxupquote{ProcesadorFichero}} que nos permita procesar los ficheros de la forma pedida. Antes de crear un programa multiproceso empezaremos por crear una clase simple que nos resuelva este problema.

Dicha clase tendrá un método \sphinxcode{\sphinxupquote{hacerRecuento}} que resuelva el problema de contar el número de apariciones en un fichero dejando el total de apariciones en un fichero de salida distintos. En el código adjunto podemos ver como podría implementarse dicho método.

\begin{sphinxVerbatim}[commandchars=\\\{\}]
\PYG{k+kd}{public} \PYG{k+kd}{static} \PYG{n}{ArrayList}\PYG{o}{\PYGZlt{}}\PYG{n}{String}\PYG{o}{\PYGZgt{}} \PYG{n+nf}{getLineasFichero}\PYG{p}{(}
    \PYG{n}{String} \PYG{n}{nombreFichero}\PYG{p}{)} \PYG{k+kd}{throws} \PYG{n}{IOException} \PYG{p}{\PYGZob{}}
    \PYG{n}{ArrayList}\PYG{o}{\PYGZlt{}}\PYG{n}{String}\PYG{o}{\PYGZgt{}} \PYG{n}{lineas} \PYG{o}{=} \PYG{k}{new} \PYG{n}{ArrayList}\PYG{o}{\PYGZlt{}}\PYG{n}{String}\PYG{o}{\PYGZgt{}}\PYG{p}{(}\PYG{p}{)}\PYG{p}{;}
    \PYG{n}{BufferedReader} \PYG{n}{bfr} \PYG{o}{=} \PYG{n}{getBufferedReader}\PYG{p}{(}\PYG{n}{nombreFichero}\PYG{p}{)}\PYG{p}{;}
    \PYG{c+c1}{//Leemos líneas del fichero...}
    \PYG{n}{String} \PYG{n}{linea} \PYG{o}{=} \PYG{n}{bfr}\PYG{p}{.}\PYG{n+na}{readLine}\PYG{p}{(}\PYG{p}{)}\PYG{p}{;}
    \PYG{k}{while} \PYG{p}{(}\PYG{n}{linea} \PYG{o}{!}\PYG{o}{=} \PYG{k+kc}{null}\PYG{p}{)} \PYG{p}{\PYGZob{}}
        \PYG{c+c1}{//Y las añadimos al array}
        \PYG{n}{lineas}\PYG{p}{.}\PYG{n+na}{add}\PYG{p}{(}\PYG{n}{linea}\PYG{p}{)}\PYG{p}{;}
        \PYG{n}{linea} \PYG{o}{=} \PYG{n}{bfr}\PYG{p}{.}\PYG{n+na}{readLine}\PYG{p}{(}\PYG{p}{)}\PYG{p}{;}
    \PYG{p}{\PYGZcb{}}
    \PYG{c+c1}{//Fin del bucle que lee líneas}
    \PYG{k}{return} \PYG{n}{lineas}\PYG{p}{;}
\PYG{p}{\PYGZcb{}}
\end{sphinxVerbatim}

La clase \sphinxcode{\sphinxupquote{ProcesadorFichero}} será lanzada desde otra clase que llamaremos \sphinxcode{\sphinxupquote{Lanzador}}. Nuestra \sphinxcode{\sphinxupquote{ProcesadorFichero}} recogerá en su método \sphinxcode{\sphinxupquote{main}} los siguientes parámetros:
\begin{enumerate}
\sphinxsetlistlabels{\arabic}{enumi}{enumii}{}{.}%
\item {} 
El nombre del fichero a procesar. Lo llamaremos \sphinxcode{\sphinxupquote{nombreFicheroEntrada}} y estará en la posición 0 de los argumentos.

\item {} 
La letra de la que hay que hacer el recuento de apariciones. La llamaremos \sphinxcode{\sphinxupquote{letra}} y estará en la posición 1.

\item {} 
El nombre del fichero donde se dejarán los resultados. Lo llamaremos \sphinxcode{\sphinxupquote{nombreFicheroResultado}} y estará en la posición 1 de los argumentos.

\end{enumerate}

El código del main se muestra a continuación.

\begin{sphinxVerbatim}[commandchars=\\\{\}]
\PYG{c+cm}{/**}
\PYG{c+cm}{	 * Dado un fichero pasado como argumento, contará cuantas}
\PYG{c+cm}{	 * apariciones hay de una cierta vocal (pasada como argumento)}
\PYG{c+cm}{	 * y dejará la cantidad en otro fichero (también pasado como}
\PYG{c+cm}{	 * argumento)}
\PYG{c+cm}{	 * @throws IOException}
\PYG{c+cm}{	 * @throws FileNotFoundException  */}
\PYG{k+kd}{public} \PYG{k+kd}{static} \PYG{k+kt}{void} \PYG{n+nf}{main}\PYG{p}{(}\PYG{n}{String}\PYG{o}{[}\PYG{o}{]} \PYG{n}{args}\PYG{p}{)} \PYG{k+kd}{throws}
\PYG{n}{FileNotFoundException}\PYG{p}{,} \PYG{n}{IOException} \PYG{p}{\PYGZob{}}
    \PYG{n}{String} \PYG{n}{nombreFicheroEntrada} \PYG{o}{=} \PYG{n}{args}\PYG{o}{[}\PYG{l+m+mi}{0}\PYG{o}{]}\PYG{p}{;}
    \PYG{n}{String} \PYG{n}{letra} \PYG{o}{=} \PYG{n}{args}\PYG{o}{[}\PYG{l+m+mi}{1}\PYG{o}{]}\PYG{p}{;}
    \PYG{n}{String} \PYG{n}{nombreFicheroResultado} \PYG{o}{=} \PYG{n}{args}\PYG{o}{[}\PYG{l+m+mi}{2}\PYG{o}{]}\PYG{p}{;}
    \PYG{n}{hacerRecuento}\PYG{p}{(}\PYG{n}{nombreFicheroEntrada}\PYG{p}{,} \PYG{n}{letra}\PYG{p}{,} \PYG{n}{nombreFicheroResultado}\PYG{p}{)}\PYG{p}{;}
\PYG{c+c1}{//Fin del main}
\PYG{p}{\PYGZcb{}}
\end{sphinxVerbatim}


\section{Lanzamiento de los procesos}
\label{\detokenize{textos/tema1:lanzamiento-de-los-procesos}}
La clase \sphinxcode{\sphinxupquote{Lanzador}} tendrá un método \sphinxcode{\sphinxupquote{main}} que se encargará de varias cosas:
\begin{enumerate}
\sphinxsetlistlabels{\arabic}{enumi}{enumii}{}{.}%
\item {} 
Recoger el primer parámetro (\sphinxcode{\sphinxupquote{args{[}0{]}}}), que contendrá el fichero a procesar.

\item {} 
Recoger el segundo parámetro, que contendrá el directorio de \sphinxcode{\sphinxupquote{CLASSPATH}} donde habrá que buscar la clase \sphinxcode{\sphinxupquote{UtilidadesFicheros}}.

\item {} 
Recoger el tercer parámetro, que contendrá el directorio donde habrá que buscar la clase \sphinxcode{\sphinxupquote{ProcesadorFicheros}}.

\item {} 
Una vez recogidos los parámetros, se lanzarán los procesos utilizando la clase \sphinxcode{\sphinxupquote{ProcessBuilder}}.

\item {} 
Los procesos se ejecutarán y después recogeremos los resultados de los ficheros.

\end{enumerate}

En el código siguiente puede verse el método \sphinxcode{\sphinxupquote{main}} de esta clase \sphinxcode{\sphinxupquote{Lanzador}}. La recogida de resultados se deja como ejercicio:

\begin{sphinxVerbatim}[commandchars=\\\{\}]
\PYG{k+kd}{public} \PYG{k+kd}{static} \PYG{k+kt}{void} \PYG{n+nf}{main}\PYG{p}{(}\PYG{n}{String}\PYG{o}{[}\PYG{o}{]} \PYG{n}{args}\PYG{p}{)} \PYG{k+kd}{throws}
\PYG{n}{IOException}\PYG{p}{,} \PYG{n}{InterruptedException} \PYG{p}{\PYGZob{}}
    \PYG{n}{String} \PYG{n}{ficheroEntrada}\PYG{p}{;}
    \PYG{n}{ficheroEntrada} \PYG{o}{=} \PYG{n}{args}\PYG{o}{[}\PYG{l+m+mi}{0}\PYG{o}{]}\PYG{p}{;}
    \PYG{n}{String} \PYG{n}{classpathUtilidades}\PYG{p}{;}
    \PYG{n}{classpathUtilidades} \PYG{o}{=} \PYG{n}{args}\PYG{o}{[}\PYG{l+m+mi}{1}\PYG{o}{]}\PYG{p}{;}
    \PYG{n}{String} \PYG{n}{classpathProcesadorFichero}\PYG{p}{;}
    \PYG{n}{classpathProcesadorFichero} \PYG{o}{=} \PYG{n}{args}\PYG{o}{[}\PYG{l+m+mi}{2}\PYG{o}{]}\PYG{p}{;}
    \PYG{n}{String}\PYG{o}{[}\PYG{o}{]} \PYG{n}{vocales} \PYG{o}{=} \PYG{p}{\PYGZob{}} \PYG{l+s}{\PYGZdq{}}\PYG{l+s}{A}\PYG{l+s}{\PYGZdq{}}\PYG{p}{,} \PYG{l+s}{\PYGZdq{}}\PYG{l+s}{E}\PYG{l+s}{\PYGZdq{}}\PYG{p}{,} \PYG{l+s}{\PYGZdq{}}\PYG{l+s}{I}\PYG{l+s}{\PYGZdq{}}\PYG{p}{,} \PYG{l+s}{\PYGZdq{}}\PYG{l+s}{O}\PYG{l+s}{\PYGZdq{}}\PYG{p}{,} \PYG{l+s}{\PYGZdq{}}\PYG{l+s}{U}\PYG{l+s}{\PYGZdq{}} \PYG{p}{\PYGZcb{}}\PYG{p}{;}
    \PYG{n}{String} \PYG{n}{classPath}\PYG{p}{;}
    \PYG{n}{classPath} \PYG{o}{=} \PYG{n}{classpathProcesadorFichero} \PYG{o}{+} \PYG{l+s}{\PYGZdq{}}\PYG{l+s}{:}\PYG{l+s}{\PYGZdq{}} \PYG{o}{+} \PYG{n}{classpathUtilidades}\PYG{p}{;}
    \PYG{n}{System}\PYG{p}{.}\PYG{n+na}{out}\PYG{p}{.}\PYG{n+na}{println}\PYG{p}{(}\PYG{l+s}{\PYGZdq{}}\PYG{l+s}{Usando classpath:}\PYG{l+s}{\PYGZdq{}} \PYG{o}{+} \PYG{n}{classPath}\PYG{p}{)}\PYG{p}{;}
    \PYG{c+cm}{/* Se lanzan los procesos*/}
    \PYG{k}{for} \PYG{p}{(}\PYG{k+kt}{int} \PYG{n}{i} \PYG{o}{=} \PYG{l+m+mi}{0}\PYG{p}{;} \PYG{n}{i} \PYG{o}{\PYGZlt{}} \PYG{n}{vocales}\PYG{p}{.}\PYG{n+na}{length}\PYG{p}{;} \PYG{n}{i}\PYG{o}{+}\PYG{o}{+}\PYG{p}{)} \PYG{p}{\PYGZob{}}
        \PYG{n}{String} \PYG{n}{fichErrores} \PYG{o}{=} \PYG{l+s}{\PYGZdq{}}\PYG{l+s}{Errores\PYGZus{}}\PYG{l+s}{\PYGZdq{}} \PYG{o}{+} \PYG{n}{vocales}\PYG{o}{[}\PYG{n}{i}\PYG{o}{]} \PYG{o}{+}
                             \PYG{l+s}{\PYGZdq{}}\PYG{l+s}{.txt}\PYG{l+s}{\PYGZdq{}}\PYG{p}{;}
        \PYG{n}{ProcessBuilder} \PYG{n}{pb}\PYG{p}{;}
        \PYG{n}{pb} \PYG{o}{=} \PYG{k}{new} \PYG{n}{ProcessBuilder}\PYG{p}{(}\PYG{l+s}{\PYGZdq{}}\PYG{l+s}{java}\PYG{l+s}{\PYGZdq{}}\PYG{p}{,} \PYG{l+s}{\PYGZdq{}}\PYG{l+s}{\PYGZhy{}cp}\PYG{l+s}{\PYGZdq{}}\PYG{p}{,} \PYG{n}{classPath}\PYG{p}{,}
                                \PYG{l+s}{\PYGZdq{}}\PYG{l+s}{com.ies.ProcesadorFichero}\PYG{l+s}{\PYGZdq{}}\PYG{p}{,} \PYG{n}{ficheroEntrada}\PYG{p}{,}
                                \PYG{n}{vocales}\PYG{o}{[}\PYG{n}{i}\PYG{o}{]}\PYG{p}{,} \PYG{n}{vocales}\PYG{o}{[}\PYG{n}{i}\PYG{o}{]} \PYG{o}{+} \PYG{l+s}{\PYGZdq{}}\PYG{l+s}{.txt}\PYG{l+s}{\PYGZdq{}}\PYG{p}{)}\PYG{p}{;}
        \PYG{c+c1}{//Si hay algún error, almacenarlo en un fichero}
        \PYG{n}{pb}\PYG{p}{.}\PYG{n+na}{redirectError}\PYG{p}{(}\PYG{k}{new} \PYG{n}{File}\PYG{p}{(}\PYG{n}{fichErrores}\PYG{p}{)}\PYG{p}{)}\PYG{p}{;}
        \PYG{n}{pb}\PYG{p}{.}\PYG{n+na}{start}\PYG{p}{(}\PYG{p}{)}\PYG{p}{;}
        \PYG{c+c1}{//Fin del for}
    \PYG{p}{\PYGZcb{}}
    \PYG{c+cm}{/* Esperamos un poco*/}
    \PYG{n}{Thread}\PYG{p}{.}\PYG{n+na}{sleep}\PYG{p}{(}\PYG{l+m+mi}{3000}\PYG{p}{)}\PYG{p}{;}
    \PYG{c+cm}{/* La recogida de resultados se deja como}
\PYG{c+cm}{    		 * ejercicio al lector. ;) */}
\PYG{p}{\PYGZcb{}}
\end{sphinxVerbatim}


\section{Ejercicio propuesto (I)}
\label{\detokenize{textos/tema1:ejercicio-propuesto-i}}
Se desea crear un programa que procese ficheros aprovechando el paralelismo de la máquina. Se tienen cinco ficheros con los siguientes nombres:
\begin{itemize}
\item {} 
\sphinxcode{\sphinxupquote{informatica.txt}}

\item {} 
\sphinxcode{\sphinxupquote{gerencia.txt}}

\item {} 
\sphinxcode{\sphinxupquote{contabilidad.txt}}

\item {} 
\sphinxcode{\sphinxupquote{comercio.txt}}

\item {} 
\sphinxcode{\sphinxupquote{rrhh.txt}}

\end{itemize}

En cada fichero hay una lista de cantidades enteras que representa las contabilidades de distintos departamentos. Hay una cantidad en cada línea. Se desea que el programa creado sume la cantidad total que suman todas las cantidades de los cinco ficheros haciendo uso del paralelismo.


\subsection{Solución al ejercicio propuesto (I)}
\label{\detokenize{textos/tema1:solucion-al-ejercicio-propuesto-i}}
Este ejercicio es bastante parecido al anterior, con la salvedad de que ahora necesitaremos sumar cantidades en lugar de buscar elementos. Por lo demás,  también generaremos un fichero de resultados por cada fichero de datos y los llamaremos igual pero añadiendo la extensión «.res». Es decir, la suma de las cantidades de \sphinxcode{\sphinxupquote{informatica.txt}} se dejará en \sphinxcode{\sphinxupquote{informatica.txt.res}}.

Una vez generados todos los fichero \sphinxcode{\sphinxupquote{Resultado}} tendremos que sumar todas las cantidades de estos ficheros. Su estructura es muy simple, todos contienen una única línea con una cantidad. Dado que esta operación de sumar cantidades almacenadas en distintos ficheros es una operación muy habitual, añadiremos un método a \sphinxcode{\sphinxupquote{UtilidadesFicheros}} que se encargue de hacer esta operación. Obsérvese que usaremos tipos de datos \sphinxcode{\sphinxupquote{long}} y no \sphinxcode{\sphinxupquote{int}} dado que no sabemos como de grandes serán las cantidades a procesar.

A continuación se muestra el código de este método.

\begin{sphinxVerbatim}[commandchars=\\\{\}]
\PYG{c+c1}{//Fin de getLineasFichero}
\PYG{k+kd}{public} \PYG{k+kd}{static} \PYG{k+kt}{long} \PYG{n+nf}{getSuma}\PYG{p}{(}\PYG{n}{String}\PYG{o}{[}\PYG{o}{]}
                           \PYG{n}{listaNombresFichero}\PYG{p}{)} \PYG{p}{\PYGZob{}}
    \PYG{k+kt}{long} \PYG{n}{suma} \PYG{o}{=} \PYG{l+m+mi}{0}\PYG{p}{;}
    \PYG{n}{ArrayList}\PYG{o}{\PYGZlt{}}\PYG{n}{String}\PYG{o}{\PYGZgt{}} \PYG{n}{lineas}\PYG{p}{;}
    \PYG{n}{String} \PYG{n}{lineaCantidad}\PYG{p}{;}
    \PYG{k+kt}{long} \PYG{n}{cantidad}\PYG{p}{;}
    \PYG{k}{for} \PYG{p}{(}\PYG{n}{String} \PYG{n}{nombreFichero} \PYG{p}{:} \PYG{n}{listaNombresFichero}\PYG{p}{)} \PYG{p}{\PYGZob{}}
        \PYG{k}{try} \PYG{p}{\PYGZob{}}
            \PYG{c+c1}{//Recuperamos todas las lineas}
            \PYG{n}{lineas} \PYG{o}{=} \PYG{n}{getLineasFichero}\PYG{p}{(}\PYG{n}{nombreFichero}\PYG{p}{)}\PYG{p}{;}
            \PYG{c+c1}{//Pero solo nos interesa la primera}
            \PYG{n}{lineaCantidad} \PYG{o}{=} \PYG{n}{lineas}\PYG{p}{.}\PYG{n+na}{get}\PYG{p}{(}\PYG{l+m+mi}{0}\PYG{p}{)}\PYG{p}{;}
            \PYG{c+c1}{//Convertimos la linea a número}
            \PYG{n}{cantidad} \PYG{o}{=} \PYG{n}{Long}\PYG{p}{.}\PYG{n+na}{parseLong}\PYG{p}{(}\PYG{n}{lineaCantidad}\PYG{p}{)}\PYG{p}{;}
            \PYG{c+c1}{//Y se incrementa la suma total}
            \PYG{n}{suma} \PYG{o}{=} \PYG{n}{suma} \PYG{o}{+} \PYG{n}{cantidad}\PYG{p}{;}
        \PYG{p}{\PYGZcb{}} \PYG{k}{catch} \PYG{p}{(}\PYG{n}{IOException} \PYG{n}{e}\PYG{p}{)} \PYG{p}{\PYGZob{}}
            \PYG{n}{System}\PYG{p}{.}\PYG{n+na}{err}\PYG{p}{.}\PYG{n+na}{println}\PYG{p}{(}\PYG{l+s}{\PYGZdq{}}\PYG{l+s}{Fallo al procesar el fichero }\PYG{l+s}{\PYGZdq{}}
                               \PYG{o}{+} \PYG{n}{nombreFichero}\PYG{p}{)}\PYG{p}{;}
            \PYG{c+c1}{//Fin del catch}
        \PYG{p}{\PYGZcb{}}
        \PYG{c+c1}{//Fin del for que recorre los nombres de fichero}
    \PYG{p}{\PYGZcb{}}
    \PYG{k}{return} \PYG{n}{suma}\PYG{p}{;}
\PYG{p}{\PYGZcb{}}
\end{sphinxVerbatim}

Una vez creado este método que usaremos al final, pasemos a crear una clase \sphinxcode{\sphinxupquote{ProcesadorContabilidad}} que leerá un fichero, sumará las cantidades y las dejará en el fichero de resultados. Esta clase solo tendrá un método \sphinxcode{\sphinxupquote{main}} que realizará esta tarea.

Como nuestro programa está formado por varias clases que están dispersas por distintos directorios, nuestro \sphinxcode{\sphinxupquote{main}} tomará el \sphinxcode{\sphinxupquote{CLASSPATH}} de los argumentos.

\begin{sphinxVerbatim}[commandchars=\\\{\}]
\PYG{k+kd}{public} \PYG{k+kd}{static} \PYG{k+kt}{void} \PYG{n+nf}{main}\PYG{p}{(}\PYG{n}{String}\PYG{o}{[}\PYG{o}{]} \PYG{n}{args}\PYG{p}{)} \PYG{k+kd}{throws}
\PYG{n}{IOException} \PYG{p}{\PYGZob{}}
    \PYG{n}{String} \PYG{n}{nombreFichero} \PYG{o}{=} \PYG{n}{args}\PYG{o}{[}\PYG{l+m+mi}{0}\PYG{o}{]}\PYG{p}{;}
    \PYG{n}{String} \PYG{n}{nombreFicheroResultado} \PYG{o}{=} \PYG{n}{args}\PYG{o}{[}\PYG{l+m+mi}{1}\PYG{o}{]}\PYG{p}{;}
    \PYG{n}{ArrayList}\PYG{o}{\PYGZlt{}}\PYG{n}{String}\PYG{o}{\PYGZgt{}} \PYG{n}{cantidades}\PYG{p}{;}
    \PYG{k+kt}{long} \PYG{n}{total} \PYG{o}{=} \PYG{l+m+mi}{0}\PYG{p}{;}
    \PYG{k}{try} \PYG{p}{\PYGZob{}}
        \PYG{c+c1}{//Extraemos las cantidades}
        \PYG{n}{cantidades} \PYG{o}{=} \PYG{n}{UtilidadesFicheros}\PYG{p}{.}\PYG{n+na}{getLineasFichero}\PYG{p}{(}
            \PYG{n}{nombreFichero}\PYG{p}{)}\PYG{p}{;}
        \PYG{c+c1}{//Y las sumamos una por una}
        \PYG{k}{for} \PYG{p}{(}\PYG{n}{String} \PYG{n}{lineaCantidad} \PYG{p}{:} \PYG{n}{cantidades}\PYG{p}{)} \PYG{p}{\PYGZob{}}
            \PYG{k+kt}{long} \PYG{n}{cantidad} \PYG{o}{=} \PYG{n}{Long}\PYG{p}{.}\PYG{n+na}{parseLong}\PYG{p}{(}\PYG{n}{lineaCantidad}\PYG{p}{)}\PYG{p}{;}
            \PYG{n}{total} \PYG{o}{=} \PYG{n}{total} \PYG{o}{+} \PYG{n}{cantidad}\PYG{p}{;}
            \PYG{c+c1}{//Fin del for que recorre las cantidades}
        \PYG{p}{\PYGZcb{}}
        \PYG{c+c1}{//Almacenamos el total en un fichero}
        \PYG{n}{PrintWriter} \PYG{n}{pw}\PYG{p}{;}
        \PYG{n}{pw} \PYG{o}{=} \PYG{n}{UtilidadesFicheros}\PYG{p}{.}\PYG{n+na}{getPrintWriter}\PYG{p}{(}
                 \PYG{n}{nombreFicheroResultado}\PYG{p}{)}\PYG{p}{;}
        \PYG{n}{pw}\PYG{p}{.}\PYG{n+na}{println}\PYG{p}{(}\PYG{n}{total}\PYG{p}{)}\PYG{p}{;}
        \PYG{n}{pw}\PYG{p}{.}\PYG{n+na}{close}\PYG{p}{(}\PYG{p}{)}\PYG{p}{;}
    \PYG{p}{\PYGZcb{}}\PYG{c+c1}{//Fin del try}
    \PYG{k}{catch} \PYG{p}{(}\PYG{n}{IOException} \PYG{n}{e}\PYG{p}{)} \PYG{p}{\PYGZob{}}
        \PYG{n}{System}\PYG{p}{.}\PYG{n+na}{err}\PYG{p}{.}\PYG{n+na}{println}\PYG{p}{(}\PYG{l+s}{\PYGZdq{}}\PYG{l+s}{No se pudo procesar el fichero }\PYG{l+s}{\PYGZdq{}}
                           \PYG{o}{+} \PYG{n}{nombreFichero}\PYG{p}{)}\PYG{p}{;}
        \PYG{n}{e}\PYG{p}{.}\PYG{n+na}{printStackTrace}\PYG{p}{(}\PYG{p}{)}\PYG{p}{;}
    \PYG{p}{\PYGZcb{}}
\PYG{c+c1}{//Fin del main}
\PYG{p}{\PYGZcb{}}
\end{sphinxVerbatim}

Por último, solo queda crear la clase \sphinxcode{\sphinxupquote{Lanzador}} que recogerá los nombres de fichero que hay que sumar y lanzará un proceso de la clase \sphinxcode{\sphinxupquote{ProcesadorContabilidad}} para cada uno de ellos. Por último, sumará todos los resultados y los dejará en un fichero que llamaremos (por ejemplo) \sphinxcode{\sphinxupquote{Resultado\_global.txt}}

\begin{sphinxVerbatim}[commandchars=\\\{\}]
\PYG{k+kd}{public} \PYG{k+kd}{static} \PYG{k+kt}{void} \PYG{n+nf}{main}\PYG{p}{(}\PYG{n}{String}\PYG{o}{[}\PYG{o}{]} \PYG{n}{args}\PYG{p}{)} \PYG{k+kd}{throws}
\PYG{n}{IOException} \PYG{p}{\PYGZob{}}
    \PYG{n}{String} \PYG{n}{classpath} \PYG{o}{=} \PYG{n}{args}\PYG{o}{[}\PYG{l+m+mi}{0}\PYG{o}{]}\PYG{p}{;}
    \PYG{n}{String}\PYG{o}{[}\PYG{o}{]} \PYG{n}{ficheros} \PYG{o}{=} \PYG{p}{\PYGZob{}} \PYG{l+s}{\PYGZdq{}}\PYG{l+s}{informatica.txt}\PYG{l+s}{\PYGZdq{}}\PYG{p}{,} \PYG{l+s}{\PYGZdq{}}\PYG{l+s}{gerencia.txt}\PYG{l+s}{\PYGZdq{}}\PYG{p}{,} \PYG{l+s}{\PYGZdq{}}\PYG{l+s}{contabilidad.txt}\PYG{l+s}{\PYGZdq{}}\PYG{p}{,} \PYG{l+s}{\PYGZdq{}}\PYG{l+s}{comercio.txt}\PYG{l+s}{\PYGZdq{}}\PYG{p}{,} \PYG{l+s}{\PYGZdq{}}\PYG{l+s}{rrhh.txt}\PYG{l+s}{\PYGZdq{}} \PYG{p}{\PYGZcb{}}\PYG{p}{;}
    \PYG{c+c1}{//Los nombres de los ficheros de resultados}
    \PYG{c+c1}{//se generarán y luego se almacenarán aquí}
    \PYG{n}{String}\PYG{o}{[}\PYG{o}{]} \PYG{n}{ficherosResultado}\PYG{p}{;}
    \PYG{n}{ficherosResultado} \PYG{o}{=} \PYG{k}{new} \PYG{n}{String}\PYG{o}{[}\PYG{n}{ficheros}\PYG{p}{.}\PYG{n+na}{length}\PYG{o}{]}\PYG{p}{;}
    \PYG{c+cm}{/* Lanzamos los procesos*/}
    \PYG{n}{ProcessBuilder}\PYG{o}{[}\PYG{o}{]} \PYG{n}{constructores}\PYG{p}{;}
    \PYG{n}{constructores} \PYG{o}{=} \PYG{k}{new} \PYG{n}{ProcessBuilder}\PYG{o}{[}\PYG{n}{ficheros}\PYG{p}{.}\PYG{n+na}{length}\PYG{o}{]}\PYG{p}{;}
    \PYG{k}{for} \PYG{p}{(}\PYG{k+kt}{int} \PYG{n}{i} \PYG{o}{=} \PYG{l+m+mi}{0}\PYG{p}{;} \PYG{n}{i} \PYG{o}{\PYGZlt{}} \PYG{n}{ficheros}\PYG{p}{.}\PYG{n+na}{length}\PYG{p}{;} \PYG{n}{i}\PYG{o}{+}\PYG{o}{+}\PYG{p}{)} \PYG{p}{\PYGZob{}}
        \PYG{n}{String} \PYG{n}{fichResultado}\PYG{p}{,} \PYG{n}{fichErrores}\PYG{p}{;}
        \PYG{n}{fichResultado} \PYG{o}{=} \PYG{n}{ficheros}\PYG{o}{[}\PYG{n}{i}\PYG{o}{]} \PYG{o}{+} \PYG{n}{SUFIJO\PYGZus{}RESULTADO}\PYG{p}{;}
        \PYG{n}{fichErrores} \PYG{o}{=} \PYG{n}{ficheros}\PYG{o}{[}\PYG{n}{i}\PYG{o}{]} \PYG{o}{+} \PYG{n}{SUFIJO\PYGZus{}ERRORES}\PYG{p}{;}
        \PYG{n}{ficherosResultado}\PYG{o}{[}\PYG{n}{i}\PYG{o}{]} \PYG{o}{=} \PYG{n}{fichResultado}\PYG{p}{;}
        \PYG{n}{constructores}\PYG{o}{[}\PYG{n}{i}\PYG{o}{]} \PYG{o}{=} \PYG{k}{new} \PYG{n}{ProcessBuilder}\PYG{p}{(}\PYG{p}{)}\PYG{p}{;}
        \PYG{n}{constructores}\PYG{o}{[}\PYG{n}{i}\PYG{o}{]}\PYG{p}{.}\PYG{n+na}{command}\PYG{p}{(}\PYG{l+s}{\PYGZdq{}}\PYG{l+s}{java}\PYG{l+s}{\PYGZdq{}}\PYG{p}{,} \PYG{l+s}{\PYGZdq{}}\PYG{l+s}{\PYGZhy{}cp}\PYG{l+s}{\PYGZdq{}}\PYG{p}{,} \PYG{n}{classpath}\PYG{p}{,}
                                 \PYG{l+s}{\PYGZdq{}}\PYG{l+s}{com.ies.ProcesadorContabilidad}\PYG{l+s}{\PYGZdq{}}\PYG{p}{,} \PYG{n}{ficheros}\PYG{o}{[}\PYG{n}{i}\PYG{o}{]}\PYG{p}{,}
                                 \PYG{n}{fichResultado}\PYG{p}{)}\PYG{p}{;}
        \PYG{c+c1}{//El fichero de errores se generará, aunque}
        \PYG{c+c1}{//puede que vacío}
        \PYG{n}{constructores}\PYG{o}{[}\PYG{n}{i}\PYG{o}{]}\PYG{p}{.}\PYG{n+na}{redirectError}\PYG{p}{(}\PYG{k}{new} \PYG{n}{File}\PYG{p}{(}
                                           \PYG{n}{fichErrores}\PYG{p}{)}\PYG{p}{)}\PYG{p}{;}
        \PYG{n}{constructores}\PYG{o}{[}\PYG{n}{i}\PYG{o}{]}\PYG{p}{.}\PYG{n+na}{start}\PYG{p}{(}\PYG{p}{)}\PYG{p}{;}
        \PYG{c+c1}{//Fin del for que recorre los ficheros}
    \PYG{p}{\PYGZcb{}}
    \PYG{c+c1}{//Calculamos las sumas de cantidades}
    \PYG{k+kt}{long} \PYG{n}{total} \PYG{o}{=} \PYG{n}{UtilidadesFicheros}\PYG{p}{.}\PYG{n+na}{getSuma}\PYG{p}{(}\PYG{n}{ficherosResultado}\PYG{p}{)}\PYG{p}{;}
    \PYG{c+c1}{//Y las almacenamos}
    \PYG{n}{PrintWriter} \PYG{n}{pw} \PYG{o}{=} \PYG{n}{UtilidadesFicheros}\PYG{p}{.}\PYG{n+na}{getPrintWriter}\PYG{p}{(}\PYG{n}{RESULTADOS\PYGZus{}GLOBALES}\PYG{p}{)}\PYG{p}{;}
    \PYG{n}{pw}\PYG{p}{.}\PYG{n+na}{println}\PYG{p}{(}\PYG{n}{total}\PYG{p}{)}\PYG{p}{;}
    \PYG{n}{pw}\PYG{p}{.}\PYG{n+na}{close}\PYG{p}{(}\PYG{p}{)}\PYG{p}{;}
\PYG{c+c1}{//Fin del main}
\PYG{p}{\PYGZcb{}}
\end{sphinxVerbatim}


\section{Gestión de procesos.}
\label{\detokenize{textos/tema1:gestion-de-procesos}}
La gestión de procesos se realiza de dos formas \sphinxstylestrong{muy distintas} en función de los dos grandes sistemas operativos: Windows y Linux.
\begin{itemize}
\item {} 
En Windows toda la gestión de procesos se realiza desde el «Administrador de tareas» al cual se accede con Ctrl+Alt+Supr. Existen otros programas más sofisticados que proporcionan algo más de información sobre los procesos, como Process Explorer (antes conocido con el nombre de ProcessViewer).

\end{itemize}


\section{Comandos para la gestión de procesos en sistemas libres y propietarios.}
\label{\detokenize{textos/tema1:comandos-para-la-gestion-de-procesos-en-sistemas-libres-y-propietarios}}
En sistemas Windows, no existen apenas comandos para gestionar procesos. Puede obligarse al sistema operativo a arrancar la aplicación asociada a un archivo con el comando \sphinxcode{\sphinxupquote{START}}. Es decir, si se ejecuta lo siguiente:

\begin{sphinxVerbatim}[commandchars=\\\{\}]
\PYG{n}{START} \PYG{n}{documento}\PYG{o}{.}\PYG{n}{pdf}
\end{sphinxVerbatim}

se abrirá el visor de archivos PDF el cual cargará automáticamente el fichero \sphinxcode{\sphinxupquote{documento.pdf}}

En GNU/Linux se puede utilizar un terminal de consola para la gestión de procesos, lo que implica que no solo se pueden arrancar procesos si no tambien detenerlos, reanudarlos, terminarlos y modificar su prioridad de ejecución.
\begin{itemize}
\item {} 
Para arrancar un proceso, simplemente tenemos que escribir el nombre del comando correspondiente. Desde GNU/Linux se pueden controlar los servicios que se ejecutan con un comando llamado \sphinxcode{\sphinxupquote{service}}. Por ejemplo, se puede usar \sphinxcode{\sphinxupquote{sudo service apache2 stop}} para parar el servidor web y \sphinxcode{\sphinxupquote{sudo service apache2 start}} para volver a ponerlo en marcha. También se puede reiniciar un servicio (tal vez para que relea un fichero de configuración que hemos cambiado) con \sphinxcode{\sphinxupquote{sudo service apache2 restart}}.

\item {} 
Se puede detener y/o terminar un proceso con el comando \sphinxcode{\sphinxupquote{kill}}. Se puede usar este comando para \sphinxstylestrong{terminar un proceso} sin guardar nada usando \sphinxcode{\sphinxupquote{kill \sphinxhyphen{}SIGKILL \textless{}numproceso\textgreater{}}} o \sphinxcode{\sphinxupquote{kill \sphinxhyphen{}9 \textless{}numproceso\textgreater{}}}. Se puede pausar un proceso con \sphinxcode{\sphinxupquote{kill \sphinxhyphen{}SIGSTOP \textless{}numproceso\textgreater{}}} y rearrancarlo con \sphinxcode{\sphinxupquote{kill \sphinxhyphen{}SIGCONT}}

\item {} 
Se puede enviar un proceso a segundo plano con comandos como \sphinxcode{\sphinxupquote{bg}} o al arrancar el proceso escribir el nombre del comando terminado en \sphinxcode{\sphinxupquote{\&}}.

\item {} 
Se puede devolver un proceso a primer plano con el comando \sphinxcode{\sphinxupquote{fg}}.

\end{itemize}


\subsection{Prioridades}
\label{\detokenize{textos/tema1:prioridades}}
En sistemas como GNU/Linux se puede modificar la prioridad con que se ejecuta un proceso. Esto implica dos posibilidades
\begin{itemize}
\item {} 
Si pensamos que un programa que necesitamos ejecutar es muy importante podemos darle más prioridad para que reciba «más turnos» del planificador.

\item {} 
Y por el contrario, si pensamos que un programa no es muy necesario podemos quitarle prioridad y reservar «más turnos de planificador» para otros posibles procesos.

\end{itemize}

El comando \sphinxcode{\sphinxupquote{nice}} permite indicar prioridades entre \sphinxhyphen{}20 y 19. El \sphinxhyphen{}20 implica que un proceso reciba la \sphinxstylestrong{máxima prioridad}, y el 19 supone asignar la \sphinxstylestrong{mínima prioridad}


\section{Sincronización entre procesos.}
\label{\detokenize{textos/tema1:sincronizacion-entre-procesos}}
Cuando se lanza más de un proceso de una misma sección de código no se sabe qué proceso ejecutará qué instrucción en un cierto momento, lo que es muy peligroso:

\begin{sphinxVerbatim}[commandchars=\\\{\}]
\PYG{k+kt}{int} \PYG{n}{i}\PYG{p}{,}\PYG{n}{j}\PYG{p}{;}
\PYG{n}{i}\PYG{o}{=}\PYG{l+m+mi}{0}\PYG{p}{;}
\PYG{k}{if} \PYG{p}{(}\PYG{n}{i}\PYG{o}{\PYGZgt{}}\PYG{o}{=}\PYG{l+m+mi}{2}\PYG{p}{)}\PYG{p}{\PYGZob{}}
        \PYG{n}{i}\PYG{o}{=}\PYG{n}{i}\PYG{o}{+}\PYG{l+m+mi}{1}\PYG{p}{;}
        \PYG{n}{j}\PYG{o}{=}\PYG{n}{j}\PYG{o}{+}\PYG{l+m+mi}{1}
\PYG{p}{\PYGZcb{}}
\PYG{n}{System}\PYG{p}{.}\PYG{n+na}{out}\PYG{p}{.}\PYG{n+na}{println}\PYG{p}{(}\PYG{l+s}{\PYGZdq{}}\PYG{l+s}{Ok}\PYG{l+s}{\PYGZdq{}}\PYG{p}{)}\PYG{p}{;}
\PYG{n}{i}\PYG{o}{=}\PYG{n}{i}\PYG{o}{*}\PYG{l+m+mi}{2}\PYG{p}{;}
\PYG{n}{j}\PYG{o}{=}\PYG{n}{j}\PYG{o}{\PYGZhy{}}\PYG{l+m+mi}{1}\PYG{p}{;}
\end{sphinxVerbatim}

Si dos o más procesos avanzan por esta sección de código es perfectamente posible que unas veces nuestro programa multiproceso se ejecute bien y otras no.

En todo programa multiproceso pueden encontrarse estas zonas de código «peligrosas» que deben protegerse especialmente utilizando ciertos mecanismos. El nombre global para todos los lenguajes es denominar a estos trozos «secciones críticas».


\subsection{Mecanismos para controlar secciones críticas}
\label{\detokenize{textos/tema1:mecanismos-para-controlar-secciones-criticas}}
Los mecanismos más típicos son los ofrecidos por UNIX/Windows:
\begin{itemize}
\item {} 
Semáforos.

\item {} 
Colas de mensajes.

\item {} 
Tuberías (pipes)

\item {} 
Bloques de memoria compartida.

\end{itemize}

En realidad algunos de estos mecanismos se utilizan más para intercomunicar procesos, aunque para los programadores Java la forma de resolver el problema de la «sección crítica» es más simple.

En Java, si el programador piensa que un trozo de código es peligroso puede ponerle la palabra clave \sphinxcode{\sphinxupquote{synchronized}} y la máquina virtual Java protege el código automáticamente.

\begin{sphinxVerbatim}[commandchars=\\\{\}]
    \PYG{c+cm}{/* La máquina virtual Java evitará que más de un proceso/hilo acceda a este método*/}
    \PYG{k+kd}{synchronized}
            \PYG{k+kd}{public} \PYG{k+kt}{void} \PYG{n+nf}{actualizarPension}\PYG{p}{(}\PYG{k+kt}{int} \PYG{n}{nuevoValor}\PYG{p}{)}\PYG{p}{\PYGZob{}}
            \PYG{c+cm}{/*..trozo de código largo omitido*/}
            \PYG{k}{this}\PYG{p}{.}\PYG{n+na}{pension}\PYG{o}{=}\PYG{n}{nuevoValor}
    \PYG{p}{\PYGZcb{}}


    \PYG{c+cm}{/* Otro ejemplo, ahora no hemos protegido un método entero,}
\PYG{c+cm}{sino solo un pequeño trozo de código.*/}
    \PYG{k}{for} \PYG{p}{(}\PYG{k+kt}{int} \PYG{n}{i}\PYG{o}{=}\PYG{l+m+mi}{0}\PYG{p}{;} \PYG{n}{i}\PYG{o}{=}\PYG{n}{i}\PYG{o}{+}\PYG{l+m+mi}{1}\PYG{p}{;} \PYG{n}{i}\PYG{o}{+}\PYG{o}{+}\PYG{p}{)}\PYG{p}{\PYGZob{}}
            \PYG{c+cm}{/* Código omitido*/}
            \PYG{k+kd}{synchronized} \PYG{p}{\PYGZob{}}
                    \PYG{n}{i}\PYG{o}{=}\PYG{n}{i}\PYG{o}{*}\PYG{l+m+mi}{2}\PYG{p}{;}
                    \PYG{n}{j}\PYG{o}{=}\PYG{n}{j}\PYG{o}{+}\PYG{l+m+mi}{1}\PYG{p}{;}
            \PYG{p}{\PYGZcb{}}
\end{sphinxVerbatim}


\section{Documentación}
\label{\detokenize{textos/tema1:documentacion}}
Para hacer la documentación tradicionalmente hemos usado JavaDOC. Sin embargo, las versiones más modernas de Java incluyen las \sphinxstylestrong{anotaciones}.

Una anotación es un texto que pueden utilizar otras herramientas (no solo el Javadoc) para comprender mejor qué hace ese código o como documentarlo.

Cualquiera puede crear sus propias anotaciones simplemente definiéndolas como un interfaz Java. Sin embargo tendremos que programar nuestras propias clases para extraer la información que proporcionan dichas anotaciones.


\section{Depuración.}
\label{\detokenize{textos/tema1:depuracion}}
¿Como se depura un programa multiproceso/multihilo? Por desgracia puede ser muy difícil:
\begin{enumerate}
\sphinxsetlistlabels{\arabic}{enumi}{enumii}{}{.}%
\item {} 
No todos los depuradores son capaces.

\item {} 
A veces cuando un depurador interviene en un proceso puede ocurrir que el resto de procesos consigan ejecutarse en el orden correcto y dar lugar a que el programa parezca que funciona bien.

\item {} 
Un error muy típico es la \sphinxcode{\sphinxupquote{NullPointerException}}, que en muchos casos se deben a la utilización de referencias Java no inicializadas o incluso a la devolución de valores NULL que luego no se comprueban en alguna parte del código.

\item {} 
Se puede usar el método \sphinxcode{\sphinxupquote{redirectError}} pasándole un objeto de tipo \sphinxcode{\sphinxupquote{File}} para que los mensajes de error vayan a un fichero.

\item {} 
Se debe recordar que la «visión» que tiene Eclipse del sistema puede ser \sphinxstylestrong{muy diferente} de la visión que tiene el proceso lanzado. Un problema muy común es que el proceso lanzado no encuentre clases, lo que obligará a indicar el \sphinxcode{\sphinxupquote{CLASSPATH}}.

\item {} 
Un buen método para determinar errores consiste en utilizar el entorno de consola para lanzar comandos para ver «como es el sistema» que ve un proceso fuera de Eclipse (o de cualquier otro entorno).

\end{enumerate}

En general todos los fallos en un programa multiproceso vienen derivado de no usar \sphinxcode{\sphinxupquote{synchronized}} de la forma correcta.


\section{Examen}
\label{\detokenize{textos/tema1:examen}}
Fecha por determinar


\chapter{Programación multihilo}
\label{\detokenize{textos/tema2:programacion-multihilo}}\label{\detokenize{textos/tema2::doc}}

\section{Recursos compartidos por los hilos.}
\label{\detokenize{textos/tema2:recursos-compartidos-por-los-hilos}}
Cuando creamos varios objetos de una clase, puede ocurrir que varios hilos de ejecución accedan a un objeto. Es importante recordar que \sphinxstylestrong{todos los campos del objeto son compartidos entre todos los hilos}.

Supongamos una clase como esta:

\begin{sphinxVerbatim}[commandchars=\\\{\}]
\PYG{k+kd}{public} \PYG{k+kd}{class} \PYG{n+nf}{Empleado}\PYG{p}{(}\PYG{p}{)}\PYG{p}{\PYGZob{}}
        \PYG{k+kt}{int} \PYG{n}{numHorasTrabajadas}\PYG{o}{=}\PYG{l+m+mi}{0}\PYG{p}{;}
        \PYG{k+kd}{public} \PYG{k+kt}{void} \PYG{k+kt}{int} \PYG{n+nf}{incrementarHoras}\PYG{p}{(}\PYG{p}{)}\PYG{p}{\PYGZob{}}
                \PYG{n}{numHorasTrabajadas}\PYG{o}{+}\PYG{o}{+}\PYG{p}{;}
        \PYG{p}{\PYGZcb{}}
\PYG{p}{\PYGZcb{}}
\end{sphinxVerbatim}

Si varios hilos ejecutan sin querer el método \sphinxcode{\sphinxupquote{incrementar}} en teoría debería ocurrir que el número se incrementase tantas veces como procesos. Sin embargo, \sphinxstylestrong{es muy probable que eso no ocurra}


\section{Estados de un hilo. Cambios de estado.}
\label{\detokenize{textos/tema2:estados-de-un-hilo-cambios-de-estado}}
Aunque no lo vemos, un hilo cambia de estado: puede pasar de la nada a la ejecución. De la ejecución al estado «en espera». De ahí puede volver a estar en ejecución. De cualquier estado se puede pasar al estado «finalizado».

El programador no necesita controlar esto, lo hace el sistema operativo. Sin embargo un programa multihilo mal hecho puede dar lugar problemas como los siguientes:
\begin{itemize}
\item {} 
Interbloqueo. Se produce cuando las peticiones y las esperas se entrelazan de forma que ningún proceso puede avanzar.

\item {} 
Inanición. Ningún proceso consigue hacer ninguna tarea útil y por lo tanto hay que esperar a que el administrador del sistema detecte el interbloqueo y mate procesos (o hasta que alguien reinicie el equipo).

\end{itemize}


\section{Elementos relacionados con la programación de hilos. Librerías y clases.}
\label{\detokenize{textos/tema2:elementos-relacionados-con-la-programacion-de-hilos-librerias-y-clases}}
Para crear programas multihilo en Java se pueden hacer dos cosas:
\begin{enumerate}
\sphinxsetlistlabels{\arabic}{enumi}{enumii}{}{.}%
\item {} 
Heredar de la clase Thread.

\item {} 
Implementar la interfaz Runnable.

\end{enumerate}

Los documentos de Java aconsejan el segundo. Lo único que hay que hacer es algo como esto.

\begin{sphinxVerbatim}[commandchars=\\\{\}]
\PYG{k+kd}{class} \PYG{n+nc}{EjecutorTareaCompleja} \PYG{k+kd}{implements} \PYG{n}{Runnable}\PYG{p}{\PYGZob{}}
        \PYG{k+kd}{private} \PYG{n}{String} \PYG{n}{nombre}\PYG{p}{;}
        \PYG{k+kt}{int} \PYG{n}{numEjecucion}\PYG{p}{;}
        \PYG{k+kd}{public} \PYG{n+nf}{EjecutorTareaCompleja}\PYG{p}{(}\PYG{n}{String} \PYG{n}{nombre}\PYG{p}{)}\PYG{p}{\PYGZob{}}
                \PYG{k}{this}\PYG{p}{.}\PYG{n+na}{nombre}\PYG{o}{=}\PYG{n}{nombre}\PYG{p}{;}
        \PYG{p}{\PYGZcb{}}
        \PYG{n+nd}{@Override}
        \PYG{k+kd}{public} \PYG{k+kt}{void} \PYG{n+nf}{run}\PYG{p}{(}\PYG{p}{)} \PYG{p}{\PYGZob{}}
                \PYG{n}{String} \PYG{n}{cad}\PYG{p}{;}
                \PYG{k}{while} \PYG{p}{(}\PYG{n}{numEjecucion}\PYG{o}{\PYGZlt{}}\PYG{l+m+mi}{100}\PYG{p}{)}\PYG{p}{\PYGZob{}}
                        \PYG{k}{for} \PYG{p}{(}\PYG{k+kt}{double} \PYG{n}{i}\PYG{o}{=}\PYG{l+m+mi}{0}\PYG{p}{;} \PYG{n}{i}\PYG{o}{\PYGZlt{}}\PYG{l+m+mf}{4999.99}\PYG{p}{;} \PYG{n}{i}\PYG{o}{=}\PYG{n}{i}\PYG{o}{+}\PYG{l+m+mf}{0.04}\PYG{p}{)}
                        \PYG{p}{\PYGZob{}}
                                \PYG{n}{Math}\PYG{p}{.}\PYG{n+na}{sqrt}\PYG{p}{(}\PYG{n}{i}\PYG{p}{)}\PYG{p}{;}
                        \PYG{p}{\PYGZcb{}}
                        \PYG{n}{cad}\PYG{o}{=}\PYG{l+s}{\PYGZdq{}}\PYG{l+s}{Soy el hilo }\PYG{l+s}{\PYGZdq{}}\PYG{o}{+}\PYG{k}{this}\PYG{p}{.}\PYG{n+na}{nombre}\PYG{p}{;}
                        \PYG{n}{cad}\PYG{o}{+}\PYG{o}{=}\PYG{l+s}{\PYGZdq{}}\PYG{l+s}{ y mi valor de i es }\PYG{l+s}{\PYGZdq{}}\PYG{o}{+}\PYG{n}{numEjecucion}\PYG{p}{;}
                        \PYG{n}{System}\PYG{p}{.}\PYG{n+na}{out}\PYG{p}{.}\PYG{n+na}{println}\PYG{p}{(}\PYG{n}{cad}\PYG{p}{)}\PYG{p}{;}
                        \PYG{n}{numEjecucion}\PYG{o}{+}\PYG{o}{+}\PYG{p}{;}
                \PYG{p}{\PYGZcb{}}
        \PYG{p}{\PYGZcb{}}

\PYG{p}{\PYGZcb{}}
\PYG{k+kd}{public} \PYG{k+kd}{class} \PYG{n+nc}{LanzaHilos} \PYG{p}{\PYGZob{}}

        \PYG{c+cm}{/**}
\PYG{c+cm}{         * @param args}
\PYG{c+cm}{         */}
        \PYG{k+kd}{public} \PYG{k+kd}{static} \PYG{k+kt}{void} \PYG{n+nf}{main}\PYG{p}{(}\PYG{n}{String}\PYG{o}{[}\PYG{o}{]} \PYG{n}{args}\PYG{p}{)} \PYG{p}{\PYGZob{}}
                \PYG{k+kt}{int} \PYG{n}{NUM\PYGZus{}HILOS}\PYG{o}{=}\PYG{l+m+mi}{500}\PYG{p}{;}
                \PYG{n}{EjecutorTareaCompleja} \PYG{n}{op}\PYG{p}{;}
                \PYG{k}{for} \PYG{p}{(}\PYG{k+kt}{int} \PYG{n}{i}\PYG{o}{=}\PYG{l+m+mi}{0}\PYG{p}{;} \PYG{n}{i}\PYG{o}{\PYGZlt{}}\PYG{n}{NUM\PYGZus{}HILOS}\PYG{p}{;} \PYG{n}{i}\PYG{o}{+}\PYG{o}{+}\PYG{p}{)}
                \PYG{p}{\PYGZob{}}
                        \PYG{n}{op}\PYG{o}{=}\PYG{k}{new} \PYG{n}{EjecutorTareaCompleja} \PYG{p}{(}\PYG{l+s}{\PYGZdq{}}\PYG{l+s}{Operacion }\PYG{l+s}{\PYGZdq{}}\PYG{o}{+}\PYG{n}{i}\PYG{p}{)}\PYG{p}{;}
                        \PYG{n}{Thread} \PYG{n}{hilo}\PYG{o}{=}\PYG{k}{new} \PYG{n}{Thread}\PYG{p}{(}\PYG{n}{op}\PYG{p}{)}\PYG{p}{;}
                        \PYG{n}{hilo}\PYG{p}{.}\PYG{n+na}{start}\PYG{p}{(}\PYG{p}{)}\PYG{p}{;}
                \PYG{p}{\PYGZcb{}}
        \PYG{p}{\PYGZcb{}}
\PYG{p}{\PYGZcb{}}
\end{sphinxVerbatim}

\begin{sphinxadmonition}{warning}{Advertencia:}
Este código tiene un problema \sphinxstylestrong{muy grave} y es que no se controla el acceso a variables compartidas, es decir \sphinxstylestrong{HAY UNA SECCIÓN CRÍTICA QUE NO ESTÁ PROTEGIDA} por lo que el resultado de la ejecución no muestra ningún sentido aunque el programa esté bien.
\end{sphinxadmonition}


\section{Gestión de hilos.}
\label{\detokenize{textos/tema2:gestion-de-hilos}}
Con los hilos se pueden efectuar diversas operaciones que sean de utilidad al programador (y al administrador de sistemas a veces).

Por ejemplo, un hilo puede tener un nombre. Si queremos asignar un nombre a un hilo podemos usar el método \sphinxcode{\sphinxupquote{setName("Nombre que sea")}}. También podemos obtener un objeto que represente el hilo de ejecución con \sphinxcode{\sphinxupquote{currentThread}} que nos devolverá un objeto de la clase \sphinxcode{\sphinxupquote{Thread}}.

Otra operación de utilidad al gestionar hilos es indicar la prioridad que queremos darle a un hilo. En realidad esta prioridad es indicativa, el sistema operativo no está obligado a respetarla aunque por lo general lo hacen. Se puede indicar la prioridad con \sphinxcode{\sphinxupquote{setPriority(10)}}. La máxima prioridad posible es \sphinxcode{\sphinxupquote{MAX\_PRIORITY}}, y la mínima es \sphinxcode{\sphinxupquote{MIN\_PRIORITY}}.

Cuando lanzamos una operación también podemos usar el método \sphinxcode{\sphinxupquote{Thread.sleep(numero)}} y poner nuestro hilo «a dormir».

Cuando se trabaja con prioridades en hilos \sphinxstylestrong{no hay garantías de que un hilo termine cuando esperemos}.

Podemos terminar un hilo de ejecución llamando al método \sphinxcode{\sphinxupquote{join}}. Este método devuelve el control al hilo principal que lanzó el hilo secundario con la posibilidad de elegir un tiempo de espera en milisegundos.

El siguiente programa ilustra el uso de estos métodos:

\begin{sphinxVerbatim}[commandchars=\\\{\}]
\PYG{k+kd}{class} \PYG{n+nc}{Calculador} \PYG{k+kd}{implements} \PYG{n}{Runnable}\PYG{p}{\PYGZob{}}
        \PYG{n+nd}{@Override}
        \PYG{k+kd}{public} \PYG{k+kt}{void} \PYG{n+nf}{run}\PYG{p}{(}\PYG{p}{)} \PYG{p}{\PYGZob{}}
                \PYG{k+kt}{int} \PYG{n}{num}\PYG{o}{=}\PYG{l+m+mi}{0}\PYG{p}{;}
                \PYG{k}{while}\PYG{p}{(}\PYG{n}{num}\PYG{o}{\PYGZlt{}}\PYG{l+m+mi}{5}\PYG{p}{)}\PYG{p}{\PYGZob{}}
                        \PYG{n}{System}\PYG{p}{.}\PYG{n+na}{out}\PYG{p}{.}\PYG{n+na}{println}\PYG{p}{(}\PYG{l+s}{\PYGZdq{}}\PYG{l+s}{Calculando...}\PYG{l+s}{\PYGZdq{}}\PYG{p}{)}\PYG{p}{;}
                        \PYG{k}{try} \PYG{p}{\PYGZob{}}
                                \PYG{k+kt}{long} \PYG{n}{tiempo}\PYG{o}{=}\PYG{p}{(}\PYG{k+kt}{long}\PYG{p}{)} \PYG{p}{(}\PYG{l+m+mi}{1000}\PYG{o}{*}\PYG{n}{Math}\PYG{p}{.}\PYG{n+na}{random}\PYG{p}{(}\PYG{p}{)}\PYG{o}{*}\PYG{l+m+mi}{10}\PYG{p}{)}\PYG{p}{;}
                                \PYG{k}{if} \PYG{p}{(}\PYG{n}{tiempo}\PYG{o}{\PYGZgt{}}\PYG{l+m+mi}{8000}\PYG{p}{)}\PYG{p}{\PYGZob{}}
                                        \PYG{n}{Thread} \PYG{n}{hilo}\PYG{o}{=}\PYG{n}{Thread}\PYG{p}{.}\PYG{n+na}{currentThread}\PYG{p}{(}\PYG{p}{)}\PYG{p}{;}
                                        \PYG{n}{System}\PYG{p}{.}\PYG{n+na}{out}\PYG{p}{.}\PYG{n+na}{println}\PYG{p}{(}
                                                        \PYG{l+s}{\PYGZdq{}}\PYG{l+s}{Terminando hilo:}\PYG{l+s}{\PYGZdq{}}\PYG{o}{+}
                                                                        \PYG{n}{hilo}\PYG{p}{.}\PYG{n+na}{getName}\PYG{p}{(}\PYG{p}{)}
                                        \PYG{p}{)}\PYG{p}{;}
                                        \PYG{n}{hilo}\PYG{p}{.}\PYG{n+na}{join}\PYG{p}{(}\PYG{p}{)}\PYG{p}{;}
                                \PYG{p}{\PYGZcb{}}
                                \PYG{n}{Thread}\PYG{p}{.}\PYG{n+na}{sleep}\PYG{p}{(}\PYG{n}{tiempo}\PYG{p}{)}\PYG{p}{;}
                        \PYG{p}{\PYGZcb{}} \PYG{k}{catch} \PYG{p}{(}\PYG{n}{InterruptedException} \PYG{n}{e}\PYG{p}{)} \PYG{p}{\PYGZob{}}
                                \PYG{c+c1}{// TODO Auto\PYGZhy{}generated catch block}
                                \PYG{n}{e}\PYG{p}{.}\PYG{n+na}{printStackTrace}\PYG{p}{(}\PYG{p}{)}\PYG{p}{;}
                        \PYG{p}{\PYGZcb{}}
                        \PYG{n}{System}\PYG{p}{.}\PYG{n+na}{out}\PYG{p}{.}\PYG{n+na}{println}\PYG{p}{(}\PYG{l+s}{\PYGZdq{}}\PYG{l+s}{Calculado y reiniciando.}\PYG{l+s}{\PYGZdq{}}\PYG{p}{)}\PYG{p}{;}
                        \PYG{n}{num}\PYG{o}{+}\PYG{o}{+}\PYG{p}{;}
                \PYG{p}{\PYGZcb{}}
                \PYG{n}{Thread} \PYG{n}{hilo}\PYG{o}{=}\PYG{n}{Thread}\PYG{p}{.}\PYG{n+na}{currentThread}\PYG{p}{(}\PYG{p}{)}\PYG{p}{;}
                \PYG{n}{String} \PYG{n}{miNombre}\PYG{o}{=}\PYG{n}{hilo}\PYG{p}{.}\PYG{n+na}{getName}\PYG{p}{(}\PYG{p}{)}\PYG{p}{;}
                \PYG{n}{System}\PYG{p}{.}\PYG{n+na}{out}\PYG{p}{.}\PYG{n+na}{println}\PYG{p}{(}\PYG{l+s}{\PYGZdq{}}\PYG{l+s}{Hilo terminado:}\PYG{l+s}{\PYGZdq{}}\PYG{o}{+}\PYG{n}{miNombre}\PYG{p}{)}\PYG{p}{;}
        \PYG{p}{\PYGZcb{}}
\PYG{p}{\PYGZcb{}}

\PYG{k+kd}{public} \PYG{k+kd}{class} \PYG{n+nc}{LanzadorHilos} \PYG{p}{\PYGZob{}}
        \PYG{k+kd}{public} \PYG{k+kd}{static} \PYG{k+kt}{void} \PYG{n+nf}{main}\PYG{p}{(}\PYG{n}{String}\PYG{o}{[}\PYG{o}{]} \PYG{n}{args}\PYG{p}{)} \PYG{p}{\PYGZob{}}
                \PYG{n}{Calculador} \PYG{n}{vHilos}\PYG{o}{[}\PYG{o}{]}\PYG{o}{=}\PYG{k}{new} \PYG{n}{Calculador}\PYG{o}{[}\PYG{l+m+mi}{5}\PYG{o}{]}\PYG{p}{;}
                \PYG{k}{for} \PYG{p}{(}\PYG{k+kt}{int} \PYG{n}{i}\PYG{o}{=}\PYG{l+m+mi}{0}\PYG{p}{;} \PYG{n}{i}\PYG{o}{\PYGZlt{}}\PYG{l+m+mi}{5}\PYG{p}{;}\PYG{n}{i}\PYG{o}{+}\PYG{o}{+}\PYG{p}{)}\PYG{p}{\PYGZob{}}
                        \PYG{n}{vHilos}\PYG{o}{[}\PYG{n}{i}\PYG{o}{]}\PYG{o}{=}\PYG{k}{new} \PYG{n}{Calculador}\PYG{p}{(}\PYG{p}{)}\PYG{p}{;}
                        \PYG{n}{Thread} \PYG{n}{hilo}\PYG{o}{=}\PYG{k}{new} \PYG{n}{Thread}\PYG{p}{(}\PYG{n}{vHilos}\PYG{o}{[}\PYG{n}{i}\PYG{o}{]}\PYG{p}{)}\PYG{p}{;}
                        \PYG{n}{hilo}\PYG{p}{.}\PYG{n+na}{setName}\PYG{p}{(}\PYG{l+s}{\PYGZdq{}}\PYG{l+s}{Hilo }\PYG{l+s}{\PYGZdq{}}\PYG{o}{+}\PYG{n}{i}\PYG{p}{)}\PYG{p}{;}
                        \PYG{k}{if} \PYG{p}{(}\PYG{n}{i}\PYG{o}{=}\PYG{o}{=}\PYG{l+m+mi}{0}\PYG{p}{)}\PYG{p}{\PYGZob{}}
                                \PYG{n}{hilo}\PYG{p}{.}\PYG{n+na}{setPriority}\PYG{p}{(}\PYG{n}{Thread}\PYG{p}{.}\PYG{n+na}{MAX\PYGZus{}PRIORITY}\PYG{p}{)}\PYG{p}{;}
                        \PYG{p}{\PYGZcb{}}
                        \PYG{n}{hilo}\PYG{p}{.}\PYG{n+na}{start}\PYG{p}{(}\PYG{p}{)}\PYG{p}{;}
                \PYG{p}{\PYGZcb{}}
        \PYG{p}{\PYGZcb{}}
\PYG{p}{\PYGZcb{}}
\end{sphinxVerbatim}

Ejercicio: crear un programa que lance 10 hilos de ejecución donde a cada hilo se le pasará la base y la altura de un triángulo, y cada hilo ejecutará el cálculo del área de dicho triángulo informando de qué base y qué altura recibió y cual es el área resultado.

Una posibilidad (quizá incorrecta) sería esta:

Para implementar la clase \sphinxcode{\sphinxupquote{CalculadorAreas}}

\begin{sphinxVerbatim}[commandchars=\\\{\}]
\PYG{k+kd}{class} \PYG{n+nc}{CalculadorAreas} \PYG{k+kd}{implements} \PYG{n}{Runnable} \PYG{p}{\PYGZob{}}

    \PYG{k+kt}{float} \PYG{n}{base}\PYG{p}{,} \PYG{n}{altura}\PYG{p}{,} \PYG{n}{area}\PYG{p}{;}

    \PYG{k+kd}{public} \PYG{k+kt}{int} \PYG{n}{contador} \PYG{o}{=} \PYG{l+m+mi}{0}\PYG{p}{;}

    \PYG{k+kd}{public} \PYG{n+nf}{CalculadorAreas}\PYG{p}{(}\PYG{k+kt}{float} \PYG{n}{b}\PYG{p}{,} \PYG{k+kt}{float} \PYG{n}{a}\PYG{p}{)} \PYG{p}{\PYGZob{}}
        \PYG{k}{this}\PYG{p}{.}\PYG{n+na}{base} \PYG{o}{=} \PYG{n}{b}\PYG{p}{;}
        \PYG{k}{this}\PYG{p}{.}\PYG{n+na}{altura} \PYG{o}{=} \PYG{n}{a}\PYG{p}{;}
    \PYG{p}{\PYGZcb{}}

    \PYG{k+kd}{public} \PYG{k+kd}{synchronized} \PYG{k+kt}{void} \PYG{n+nf}{incrementarContador}\PYG{p}{(}\PYG{p}{)} \PYG{p}{\PYGZob{}}
        \PYG{n}{contador} \PYG{o}{=} \PYG{n}{contador} \PYG{o}{+} \PYG{l+m+mi}{1}\PYG{p}{;}
    \PYG{p}{\PYGZcb{}}

    \PYG{n+nd}{@Override}
    \PYG{k+kd}{public} \PYG{k+kt}{void} \PYG{n+nf}{run}\PYG{p}{(}\PYG{p}{)} \PYG{p}{\PYGZob{}}
        \PYG{n}{Random} \PYG{n}{generador}\PYG{p}{;}
        \PYG{n}{generador} \PYG{o}{=} \PYG{k}{new} \PYG{n}{Random}\PYG{p}{(}\PYG{p}{)}\PYG{p}{;}
        \PYG{n}{area} \PYG{o}{=} \PYG{n}{base} \PYG{o}{*} \PYG{n}{altura} \PYG{o}{/} \PYG{l+m+mi}{2}\PYG{p}{;}
        \PYG{k}{try} \PYG{p}{\PYGZob{}}
            \PYG{n}{Thread}\PYG{p}{.}\PYG{n+na}{sleep}\PYG{p}{(}\PYG{n}{generador}\PYG{p}{.}\PYG{n+na}{nextInt}\PYG{p}{(}\PYG{l+m+mi}{650}\PYG{p}{)}\PYG{p}{)}\PYG{p}{;}
        \PYG{p}{\PYGZcb{}} \PYG{k}{catch} \PYG{p}{(}\PYG{n}{InterruptedException} \PYG{n}{e}\PYG{p}{)} \PYG{p}{\PYGZob{}}
            \PYG{c+c1}{// TODO Auto\PYGZhy{}generated catch block}
            \PYG{n}{e}\PYG{p}{.}\PYG{n+na}{printStackTrace}\PYG{p}{(}\PYG{p}{)}\PYG{p}{;}
        \PYG{p}{\PYGZcb{}}
        \PYG{k}{this}\PYG{p}{.}\PYG{n+na}{incrementarContador}\PYG{p}{(}\PYG{p}{)}\PYG{p}{;}
    \PYG{p}{\PYGZcb{}}
\PYG{p}{\PYGZcb{}}
\end{sphinxVerbatim}

Para implementar la clase \sphinxcode{\sphinxupquote{AreasEnParalelo}} que lanza los hilos para hacer muchos cálculos de áreas en paralelo usaremos esto:

\begin{sphinxVerbatim}[commandchars=\\\{\}]
\PYG{k+kd}{public} \PYG{k+kd}{class} \PYG{n+nc}{AreasEnParalelo} \PYG{p}{\PYGZob{}}

    \PYG{k+kd}{public} \PYG{k+kd}{static} \PYG{k+kt}{void} \PYG{n+nf}{main}\PYG{p}{(}\PYG{n}{String}\PYG{o}{[}\PYG{o}{]} \PYG{n}{args}\PYG{p}{)} \PYG{k+kd}{throws}
    \PYG{n}{InterruptedException} \PYG{p}{\PYGZob{}}
        \PYG{n}{CalculadorAreas} \PYG{n}{ca} \PYG{o}{=} \PYG{k}{new} \PYG{n}{CalculadorAreas}\PYG{p}{(}\PYG{l+m+mi}{1}\PYG{p}{,} \PYG{l+m+mi}{2}\PYG{p}{)}\PYG{p}{;}
        \PYG{k+kd}{final} \PYG{k+kt}{int} \PYG{n}{MAX\PYGZus{}HILOS} \PYG{o}{=} \PYG{l+m+mi}{10000}\PYG{p}{;}
        \PYG{n}{Thread}\PYG{o}{[}\PYG{o}{]} \PYG{n}{hilos} \PYG{o}{=} \PYG{k}{new} \PYG{n}{Thread}\PYG{o}{[}\PYG{n}{MAX\PYGZus{}HILOS}\PYG{o}{]}\PYG{p}{;}
        \PYG{k}{for} \PYG{p}{(}\PYG{k+kt}{int} \PYG{n}{i} \PYG{o}{=} \PYG{l+m+mi}{0}\PYG{p}{;} \PYG{n}{i} \PYG{o}{\PYGZlt{}} \PYG{n}{MAX\PYGZus{}HILOS}\PYG{p}{;} \PYG{n}{i}\PYG{o}{+}\PYG{o}{+}\PYG{p}{)} \PYG{p}{\PYGZob{}}
            \PYG{n}{hilos}\PYG{o}{[}\PYG{n}{i}\PYG{o}{]} \PYG{o}{=} \PYG{k}{new} \PYG{n}{Thread}\PYG{p}{(}\PYG{n}{ca}\PYG{p}{)}\PYG{p}{;}
            \PYG{n}{hilos}\PYG{o}{[}\PYG{n}{i}\PYG{o}{]}\PYG{p}{.}\PYG{n+na}{start}\PYG{p}{(}\PYG{p}{)}\PYG{p}{;}
        \PYG{p}{\PYGZcb{}}
        \PYG{k}{for} \PYG{p}{(}\PYG{k+kt}{int} \PYG{n}{i} \PYG{o}{=} \PYG{l+m+mi}{0}\PYG{p}{;} \PYG{n}{i} \PYG{o}{\PYGZlt{}} \PYG{n}{MAX\PYGZus{}HILOS}\PYG{p}{;} \PYG{n}{i}\PYG{o}{+}\PYG{o}{+}\PYG{p}{)} \PYG{p}{\PYGZob{}}
            \PYG{n}{hilos}\PYG{o}{[}\PYG{n}{i}\PYG{o}{]}\PYG{p}{.}\PYG{n+na}{join}\PYG{p}{(}\PYG{p}{)}\PYG{p}{;}
        \PYG{p}{\PYGZcb{}}
        \PYG{n}{System}\PYG{p}{.}\PYG{n+na}{out}\PYG{p}{.}\PYG{n+na}{println}\PYG{p}{(}\PYG{l+s}{\PYGZdq{}}\PYG{l+s}{Total de calculos:}\PYG{l+s}{\PYGZdq{}} \PYG{o}{+} \PYG{n}{ca}\PYG{p}{.}\PYG{n+na}{contador}\PYG{p}{)}\PYG{p}{;}
    \PYG{p}{\PYGZcb{}}
\PYG{p}{\PYGZcb{}}
\end{sphinxVerbatim}

\begin{sphinxVerbatim}[commandchars=\\\{\}]
\PYG{k+kn}{package} \PYG{n+nn}{com.ies}\PYG{p}{;}

\PYG{k+kn}{import} \PYG{n+nn}{java.util.Random}\PYG{p}{;}

\PYG{k+kd}{class} \PYG{n+nc}{CalculadorAreas} \PYG{k+kd}{implements} \PYG{n}{Runnable}\PYG{p}{\PYGZob{}}
        \PYG{k+kt}{int} \PYG{n}{base}\PYG{p}{,} \PYG{n}{altura}\PYG{p}{;}
        \PYG{k+kd}{public} \PYG{n+nf}{CalculadorAreas}\PYG{p}{(}\PYG{k+kt}{int} \PYG{n}{base}\PYG{p}{,} \PYG{k+kt}{int} \PYG{n}{altura}\PYG{p}{)}\PYG{p}{\PYGZob{}}
                \PYG{k}{this}\PYG{p}{.}\PYG{n+na}{base}\PYG{o}{=}\PYG{n}{base}\PYG{p}{;}
                \PYG{k}{this}\PYG{p}{.}\PYG{n+na}{altura}\PYG{o}{=}\PYG{n}{altura}\PYG{p}{;}
        \PYG{p}{\PYGZcb{}}
        \PYG{n+nd}{@Override}
        \PYG{k+kd}{public} \PYG{k+kt}{void} \PYG{n+nf}{run}\PYG{p}{(}\PYG{p}{)} \PYG{p}{\PYGZob{}}
                \PYG{k+kt}{float} \PYG{n}{area}\PYG{o}{=}\PYG{k}{this}\PYG{p}{.}\PYG{n+na}{base}\PYG{o}{*}\PYG{k}{this}\PYG{p}{.}\PYG{n+na}{altura}\PYG{o}{/}\PYG{l+m+mi}{2}\PYG{p}{;}
                \PYG{n}{System}\PYG{p}{.}\PYG{n+na}{out}\PYG{p}{.}\PYG{n+na}{print}\PYG{p}{(}\PYG{l+s}{\PYGZdq{}}\PYG{l+s}{Base:}\PYG{l+s}{\PYGZdq{}}\PYG{o}{+}\PYG{k}{this}\PYG{p}{.}\PYG{n+na}{base}\PYG{p}{)}\PYG{p}{;}
                \PYG{n}{System}\PYG{p}{.}\PYG{n+na}{out}\PYG{p}{.}\PYG{n+na}{print}\PYG{p}{(}\PYG{l+s}{\PYGZdq{}}\PYG{l+s}{Altura:}\PYG{l+s}{\PYGZdq{}}\PYG{o}{+}\PYG{k}{this}\PYG{p}{.}\PYG{n+na}{altura}\PYG{p}{)}\PYG{p}{;}
                \PYG{n}{System}\PYG{p}{.}\PYG{n+na}{out}\PYG{p}{.}\PYG{n+na}{println}\PYG{p}{(}\PYG{l+s}{\PYGZdq{}}\PYG{l+s}{Area:}\PYG{l+s}{\PYGZdq{}}\PYG{o}{+}\PYG{n}{area}\PYG{p}{)}\PYG{p}{;}
        \PYG{p}{\PYGZcb{}}

\PYG{p}{\PYGZcb{}}
\PYG{k+kd}{public} \PYG{k+kd}{class} \PYG{n+nc}{AreasEnParalelo} \PYG{p}{\PYGZob{}}

        \PYG{k+kd}{public} \PYG{k+kd}{static} \PYG{k+kt}{void} \PYG{n+nf}{main}\PYG{p}{(}\PYG{n}{String}\PYG{o}{[}\PYG{o}{]} \PYG{n}{args}\PYG{p}{)} \PYG{p}{\PYGZob{}}
                \PYG{n}{Random} \PYG{n}{generador}\PYG{o}{=}\PYG{k}{new} \PYG{n}{Random}\PYG{p}{(}\PYG{p}{)}\PYG{p}{;}
                \PYG{k+kt}{int} \PYG{n}{numHilos}\PYG{o}{=}\PYG{l+m+mi}{10000}\PYG{p}{;}
                \PYG{k+kt}{int} \PYG{n}{baseMaxima}\PYG{o}{=}\PYG{l+m+mi}{3}\PYG{p}{;}
                \PYG{k+kt}{int} \PYG{n}{alturaMaxima}\PYG{o}{=}\PYG{l+m+mi}{5}\PYG{p}{;}
                \PYG{k}{for} \PYG{p}{(}\PYG{k+kt}{int} \PYG{n}{i}\PYG{o}{=}\PYG{l+m+mi}{0}\PYG{p}{;} \PYG{n}{i}\PYG{o}{\PYGZlt{}}\PYG{n}{numHilos}\PYG{p}{;} \PYG{n}{i}\PYG{o}{+}\PYG{o}{+}\PYG{p}{)}\PYG{p}{\PYGZob{}}
                        \PYG{c+c1}{//Sumamos 1 para evitar casos como base=0}
                        \PYG{k+kt}{int} \PYG{n}{base}\PYG{o}{=}\PYG{l+m+mi}{1}\PYG{o}{+}\PYG{n}{generador}\PYG{p}{.}\PYG{n+na}{nextInt}\PYG{p}{(}\PYG{n}{baseMaxima}\PYG{p}{)}\PYG{p}{;}
                        \PYG{k+kt}{int} \PYG{n}{altura}\PYG{o}{=}\PYG{l+m+mi}{1}\PYG{o}{+}\PYG{n}{generador}\PYG{p}{.}\PYG{n+na}{nextInt}\PYG{p}{(}\PYG{n}{alturaMaxima}\PYG{p}{)}\PYG{p}{;}
                        \PYG{n}{CalculadorAreas} \PYG{n}{ca}\PYG{o}{=}
                                        \PYG{k}{new} \PYG{n}{CalculadorAreas}\PYG{p}{(}\PYG{n}{base}\PYG{p}{,} \PYG{n}{altura}\PYG{p}{)}\PYG{p}{;}
                        \PYG{n}{Thread} \PYG{n}{hiloAsociado}\PYG{o}{=}\PYG{k}{new} \PYG{n}{Thread}\PYG{p}{(}\PYG{n}{ca}\PYG{p}{)}\PYG{p}{;}
                        \PYG{n}{hiloAsociado}\PYG{p}{.}\PYG{n+na}{start}\PYG{p}{(}\PYG{p}{)}\PYG{p}{;}
                \PYG{p}{\PYGZcb{}}
        \PYG{p}{\PYGZcb{}}
\PYG{p}{\PYGZcb{}}
\end{sphinxVerbatim}

Las secciones siguientes sirven como resumen de como crear una aplicación multihilo


\section{Creación, inicio y finalización.}
\label{\detokenize{textos/tema2:creacion-inicio-y-finalizacion}}\begin{itemize}
\item {} 
Podemos heredar de Thread o implementar \sphinxcode{\sphinxupquote{Runnable}}. Usaremos el segundo recordando implementar el método \sphinxcode{\sphinxupquote{public void run()}}.

\item {} 
Para crear un hilo asociado a un objeto usaremos algo como:

\end{itemize}

\begin{sphinxVerbatim}[commandchars=\\\{\}]
\PYG{n}{Thread} \PYG{n}{hilo}\PYG{o}{=}\PYG{k}{new} \PYG{n}{Thread}\PYG{p}{(}\PYG{n}{objetoDeClase}\PYG{p}{)}
\end{sphinxVerbatim}

Lo más habitual es guardar en un vector todos los hilos que hagan algo, y no en un objeto suelto.
\begin{itemize}
\item {} 
Cada objeto que tenga un hilo asociado debe iniciarse así:

\end{itemize}

\begin{sphinxVerbatim}[commandchars=\\\{\}]
\PYG{n}{hilo}\PYG{p}{.}\PYG{n+na}{start}\PYG{p}{(}\PYG{p}{)}\PYG{p}{;}
\end{sphinxVerbatim}
\begin{itemize}
\item {} 
Todo programa multihilo tiene un «hilo principal», el cual deberá esperar a que terminen los hilos asociados ejecutando el método \sphinxcode{\sphinxupquote{join()}}  .

\end{itemize}


\section{Sincronización de hilos.}
\label{\detokenize{textos/tema2:sincronizacion-de-hilos}}
Cuando un método acceda a una variable miembro que esté compartida deberemos proteger dicha sección crítica, usando \sphinxcode{\sphinxupquote{synchronized}}. Se puede poner todo el método \sphinxcode{\sphinxupquote{synchronized}} o marcar un trozo de código más pequeño.


\section{Información entre hilos.}
\label{\detokenize{textos/tema2:informacion-entre-hilos}}
Todos los hilos comparten todo, así que obtener información es tan sencillo como consultar un miembro.
En realidad podemos comunicar los hilos con otro mecanismo llamado \sphinxcode{\sphinxupquote{sockets de red}}, pero se ve en el tema siguiente.


\section{Prioridades de los hilos.}
\label{\detokenize{textos/tema2:prioridades-de-los-hilos}}
Podemos asignar distintas prioridades a los hilos usando los campos estáticos \sphinxcode{\sphinxupquote{MAX\_PRIORITY}} y \sphinxcode{\sphinxupquote{MIN\_PRIORITY}}. Usando valores entre estas dos constantes podemos hacer que un hilo reciba más procesador que otro (se hace en contadas ocasiones).

Para ello se usa el método \sphinxcode{\sphinxupquote{setPriority(valor)}}


\section{Gestión de prioridades.}
\label{\detokenize{textos/tema2:gestion-de-prioridades}}
En realidad un sistema operativo no está obligado a respetar las prioridades, sino que se lo tomará como «recomendaciones». En general hasta ahora todos respetan hasta cierto punto las prioridades que pone el programador pero no debe tomarse como algo absoluto.


\section{Programación de aplicaciones multihilo.}
\label{\detokenize{textos/tema2:programacion-de-aplicaciones-multihilo}}
La estructura típica de un programa multihilo es esta:

\begin{sphinxVerbatim}[commandchars=\\\{\}]
\PYG{k+kd}{class} \PYG{n+nc}{TareaCompleja} \PYG{k+kd}{implements} \PYG{n}{Runnable}\PYG{p}{\PYGZob{}}
        \PYG{n+nd}{@Override}
        \PYG{k+kd}{public} \PYG{k+kt}{void} \PYG{n+nf}{run}\PYG{p}{(}\PYG{p}{)} \PYG{p}{\PYGZob{}}
                \PYG{k}{for} \PYG{p}{(}\PYG{k+kt}{int} \PYG{n}{i}\PYG{o}{=}\PYG{l+m+mi}{0}\PYG{p}{;} \PYG{n}{i}\PYG{o}{\PYGZlt{}}\PYG{l+m+mi}{100}\PYG{p}{;}\PYG{n}{i}\PYG{o}{+}\PYG{o}{+}\PYG{p}{)}\PYG{p}{\PYGZob{}}
                        \PYG{k+kt}{int} \PYG{n}{a}\PYG{o}{=}\PYG{n}{i}\PYG{o}{*}\PYG{l+m+mi}{3}\PYG{p}{;}
                \PYG{p}{\PYGZcb{}}
                \PYG{n}{Thread} \PYG{n}{hiloActual}\PYG{o}{=}\PYG{n}{Thread}\PYG{p}{.}\PYG{n+na}{currentThread}\PYG{p}{(}\PYG{p}{)}\PYG{p}{;}
                \PYG{n}{String} \PYG{n}{miNombre}\PYG{o}{=}\PYG{n}{hiloActual}\PYG{p}{.}\PYG{n+na}{getName}\PYG{p}{(}\PYG{p}{)}\PYG{p}{;}
                \PYG{n}{System}\PYG{p}{.}\PYG{n+na}{out}\PYG{p}{.}\PYG{n+na}{println}\PYG{p}{(}
                                \PYG{l+s}{\PYGZdq{}}\PYG{l+s}{Finalizado el hilo}\PYG{l+s}{\PYGZdq{}}\PYG{o}{+}\PYG{n}{miNombre}\PYG{p}{)}\PYG{p}{;}
        \PYG{p}{\PYGZcb{}}
\PYG{p}{\PYGZcb{}}
\PYG{k+kd}{public} \PYG{k+kd}{class} \PYG{n+nc}{LanzadorHilos} \PYG{p}{\PYGZob{}}
        \PYG{k+kd}{public} \PYG{k+kd}{static} \PYG{k+kt}{void} \PYG{n+nf}{main}\PYG{p}{(}\PYG{n}{String}\PYG{o}{[}\PYG{o}{]} \PYG{n}{args}\PYG{p}{)} \PYG{p}{\PYGZob{}}
                \PYG{k+kt}{int} \PYG{n}{NUM\PYGZus{}HILOS}\PYG{o}{=}\PYG{l+m+mi}{100}\PYG{p}{;}
                \PYG{n}{Thread}\PYG{o}{[}\PYG{o}{]} \PYG{n}{hilosAsociados}\PYG{p}{;}

                \PYG{n}{hilosAsociados}\PYG{o}{=}\PYG{k}{new} \PYG{n}{Thread}\PYG{o}{[}\PYG{n}{NUM\PYGZus{}HILOS}\PYG{o}{]}\PYG{p}{;}
                \PYG{k}{for} \PYG{p}{(}\PYG{k+kt}{int} \PYG{n}{i}\PYG{o}{=}\PYG{l+m+mi}{0}\PYG{p}{;}\PYG{n}{i}\PYG{o}{\PYGZlt{}}\PYG{n}{NUM\PYGZus{}HILOS}\PYG{p}{;}\PYG{n}{i}\PYG{o}{+}\PYG{o}{+}\PYG{p}{)}\PYG{p}{\PYGZob{}}
                        \PYG{n}{TareaCompleja} \PYG{n}{t}\PYG{o}{=}\PYG{k}{new} \PYG{n}{TareaCompleja}\PYG{p}{(}\PYG{p}{)}\PYG{p}{;}
                        \PYG{n}{Thread} \PYG{n}{hilo}\PYG{o}{=}\PYG{k}{new} \PYG{n}{Thread}\PYG{p}{(}\PYG{n}{t}\PYG{p}{)}\PYG{p}{;}
                        \PYG{n}{hilo}\PYG{p}{.}\PYG{n+na}{setName}\PYG{p}{(}\PYG{l+s}{\PYGZdq{}}\PYG{l+s}{Hilo: }\PYG{l+s}{\PYGZdq{}}\PYG{o}{+}\PYG{n}{i}\PYG{p}{)}\PYG{p}{;}
                        \PYG{n}{hilo}\PYG{p}{.}\PYG{n+na}{start}\PYG{p}{(}\PYG{p}{)}\PYG{p}{;}
                        \PYG{n}{hilosAsociados}\PYG{o}{[}\PYG{n}{i}\PYG{o}{]}\PYG{o}{=}\PYG{n}{hilo}\PYG{p}{;}
                \PYG{p}{\PYGZcb{}}

                \PYG{c+cm}{/* Despues de crear todo, nos aseguramos}
\PYG{c+cm}{                 * de esperar que todos los hilos acaben. */}

                \PYG{k}{for} \PYG{p}{(}\PYG{k+kt}{int} \PYG{n}{i}\PYG{o}{=}\PYG{l+m+mi}{0}\PYG{p}{;} \PYG{n}{i}\PYG{o}{\PYGZlt{}}\PYG{n}{NUM\PYGZus{}HILOS}\PYG{p}{;} \PYG{n}{i}\PYG{o}{+}\PYG{o}{+}\PYG{p}{)}\PYG{p}{\PYGZob{}}
                        \PYG{n}{Thread} \PYG{n}{hilo}\PYG{o}{=}\PYG{n}{hilosAsociados}\PYG{o}{[}\PYG{n}{i}\PYG{o}{]}\PYG{p}{;}
                        \PYG{k}{try} \PYG{p}{\PYGZob{}}
                                \PYG{c+c1}{//Espera a que el hilo acabe}
                                \PYG{n}{hilo}\PYG{p}{.}\PYG{n+na}{join}\PYG{p}{(}\PYG{p}{)}\PYG{p}{;}
                        \PYG{p}{\PYGZcb{}} \PYG{k}{catch} \PYG{p}{(}\PYG{n}{InterruptedException} \PYG{n}{e}\PYG{p}{)} \PYG{p}{\PYGZob{}}
                                \PYG{n}{System}\PYG{p}{.}\PYG{n+na}{out}\PYG{p}{.}\PYG{n+na}{print}\PYG{p}{(}\PYG{l+s}{\PYGZdq{}}\PYG{l+s}{Algun hilo acabó }\PYG{l+s}{\PYGZdq{}}\PYG{p}{)}\PYG{p}{;}
                                \PYG{n}{System}\PYG{p}{.}\PYG{n+na}{out}\PYG{p}{.}\PYG{n+na}{println}\PYG{p}{(}\PYG{l+s}{\PYGZdq{}}\PYG{l+s}{ antes de tiempo!}\PYG{l+s}{\PYGZdq{}}\PYG{p}{)}\PYG{p}{;}
                        \PYG{p}{\PYGZcb{}}
                \PYG{p}{\PYGZcb{}}
                \PYG{n}{System}\PYG{p}{.}\PYG{n+na}{out}\PYG{p}{.}\PYG{n+na}{println}\PYG{p}{(}\PYG{l+s}{\PYGZdq{}}\PYG{l+s}{El principal ha terminado}\PYG{l+s}{\PYGZdq{}}\PYG{p}{)}\PYG{p}{;}
        \PYG{p}{\PYGZcb{}}
\PYG{p}{\PYGZcb{}}
\end{sphinxVerbatim}

Supongamos que la tarea es más compleja y que el bucle se ejecuta un número al azar de veces. Esto significaría que nuestro bucle es algo como esto:

\begin{sphinxVerbatim}[commandchars=\\\{\}]
\PYG{n}{Random} \PYG{n}{generador}\PYG{o}{=} \PYG{k}{new} \PYG{n}{Random}\PYG{p}{(}\PYG{p}{)}\PYG{p}{;}
\PYG{k+kt}{int} \PYG{n}{numAzar}\PYG{o}{=}\PYG{p}{(}\PYG{l+m+mi}{1}\PYG{o}{+}\PYG{n}{generador}\PYG{p}{.}\PYG{n+na}{nextInt}\PYG{p}{(}\PYG{l+m+mi}{5}\PYG{p}{)}\PYG{p}{)}\PYG{o}{*}\PYG{l+m+mi}{100}\PYG{p}{;}
\PYG{k}{for} \PYG{p}{(}\PYG{k+kt}{int} \PYG{n}{i}\PYG{o}{=}\PYG{l+m+mi}{0}\PYG{p}{;} \PYG{n}{i}\PYG{o}{\PYGZlt{}}\PYG{n}{numAzar}\PYG{p}{;}\PYG{n}{i}\PYG{o}{+}\PYG{o}{+}\PYG{p}{)}\PYG{p}{\PYGZob{}}
        \PYG{k+kt}{int} \PYG{n}{a}\PYG{o}{=}\PYG{n}{i}\PYG{o}{*}\PYG{l+m+mi}{3}\PYG{p}{;}
\PYG{p}{\PYGZcb{}}
\end{sphinxVerbatim}

¿Como podríamos modificar el programa para que podamos saber cuantas multiplicaciones se han hecho en total entre todos los hilos?

Aquí entra el problema de la sincronización. Supongamos una clase contador muy simple como esta:

\begin{sphinxVerbatim}[commandchars=\\\{\}]
\PYG{k+kd}{class} \PYG{n+nc}{Contador}\PYG{p}{\PYGZob{}}
        \PYG{k+kt}{int} \PYG{n}{cuenta}\PYG{p}{;}
        \PYG{k+kd}{public} \PYG{n+nf}{Contador}\PYG{p}{(}\PYG{p}{)}\PYG{p}{\PYGZob{}}
                \PYG{n}{cuenta}\PYG{o}{=}\PYG{l+m+mi}{0}\PYG{p}{;}
        \PYG{p}{\PYGZcb{}}
        \PYG{k+kd}{public} \PYG{k+kt}{void} \PYG{n+nf}{incrementar}\PYG{p}{(}\PYG{p}{)}\PYG{p}{\PYGZob{}}
                \PYG{n}{cuenta}\PYG{o}{=}\PYG{n}{cuenta}\PYG{o}{+}\PYG{l+m+mi}{1}\PYG{p}{;}
        \PYG{p}{\PYGZcb{}}
        \PYG{k+kd}{public} \PYG{k+kt}{int} \PYG{n+nf}{getCuenta}\PYG{p}{(}\PYG{p}{)}\PYG{p}{\PYGZob{}}
                \PYG{k}{return} \PYG{n}{cuenta}\PYG{p}{;}
        \PYG{p}{\PYGZcb{}}
\PYG{p}{\PYGZcb{}}
\end{sphinxVerbatim}

De esta forma podríamos construir un objeto contador y pasárselo a todos los hilos para que en ese único objeto se almacene el recuento final. El problema es que en la programación multihilo \sphinxstylestrong{SI EL OBJETO CONTADOR SE COMPARTE ENTRE VARIOS HILOS LA CUENTA FINAL RESULTANTE ES MUY POSIBLE QUE ESTÉ MAL}

Esta clase debería tener protegidas sus secciones críticas

\begin{sphinxVerbatim}[commandchars=\\\{\}]
\PYG{k+kd}{class} \PYG{n+nc}{Contador}\PYG{p}{\PYGZob{}}
        \PYG{k+kt}{int} \PYG{n}{cuenta}\PYG{p}{;}
        \PYG{k+kd}{public} \PYG{n+nf}{Contador}\PYG{p}{(}\PYG{p}{)}\PYG{p}{\PYGZob{}}
                \PYG{n}{cuenta}\PYG{o}{=}\PYG{l+m+mi}{0}\PYG{p}{;}
        \PYG{p}{\PYGZcb{}}
        \PYG{k+kd}{public} \PYG{k+kd}{synchronized} \PYG{k+kt}{void} \PYG{n+nf}{incrementar}\PYG{p}{(}\PYG{p}{)}\PYG{p}{\PYGZob{}}
                \PYG{n}{cuenta}\PYG{o}{=}\PYG{n}{cuenta}\PYG{o}{+}\PYG{l+m+mi}{1}\PYG{p}{;}
        \PYG{p}{\PYGZcb{}}
        \PYG{k+kd}{public} \PYG{k+kd}{synchronized} \PYG{k+kt}{int} \PYG{n+nf}{getCuenta}\PYG{p}{(}\PYG{p}{)}\PYG{p}{\PYGZob{}}
                \PYG{k}{return} \PYG{n}{cuenta}\PYG{p}{;}
        \PYG{p}{\PYGZcb{}}
\PYG{p}{\PYGZcb{}}
\end{sphinxVerbatim}

Se puede aprovechar todavía más rendimiento si en un método marcamos como sección crítica (o sincronizada) exclusivamente el código peligroso:

\begin{sphinxVerbatim}[commandchars=\\\{\}]
\PYG{k+kd}{public}  \PYG{k+kt}{void} \PYG{n+nf}{incrementar}\PYG{p}{(}\PYG{p}{)}\PYG{p}{\PYGZob{}}
        \PYG{n}{System}\PYG{p}{.}\PYG{n+na}{out}\PYG{p}{.}\PYG{n+na}{println}\PYG{p}{(}\PYG{l+s}{\PYGZdq{}}\PYG{l+s}{Otras cosas}\PYG{l+s}{\PYGZdq{}}\PYG{p}{)}\PYG{p}{;}
        \PYG{k+kd}{synchronized}\PYG{p}{(}\PYG{k}{this}\PYG{p}{)}\PYG{p}{\PYGZob{}}
                \PYG{n}{cuenta}\PYG{o}{=}\PYG{n}{cuenta}\PYG{o}{+}\PYG{l+m+mi}{1}\PYG{p}{;}
        \PYG{p}{\PYGZcb{}}
        \PYG{n}{System}\PYG{p}{.}\PYG{n+na}{out}\PYG{p}{.}\PYG{n+na}{println}\PYG{p}{(}\PYG{l+s}{\PYGZdq{}}\PYG{l+s}{Mas cosas...}\PYG{l+s}{\PYGZdq{}}\PYG{p}{)}\PYG{p}{;}
        \PYG{k+kd}{synchronized}\PYG{p}{(}\PYG{k}{this}\PYG{p}{)}\PYG{p}{\PYGZob{}}
                \PYG{k}{if} \PYG{p}{(}\PYG{n}{cuenta}\PYG{o}{\PYGZgt{}}\PYG{l+m+mi}{300}\PYG{p}{)}\PYG{p}{\PYGZob{}}
                        \PYG{n}{System}\PYG{p}{.}\PYG{n+na}{out}\PYG{p}{.}\PYG{n+na}{println}\PYG{p}{(}\PYG{l+s}{\PYGZdq{}}\PYG{l+s}{Este hilo trabaja mucho}\PYG{l+s}{\PYGZdq{}}\PYG{p}{)}\PYG{p}{;}
                \PYG{p}{\PYGZcb{}}
        \PYG{p}{\PYGZcb{}}
\PYG{p}{\PYGZcb{}}
\end{sphinxVerbatim}


\section{Problema: filósofos}
\label{\detokenize{textos/tema2:problema-filosofos}}
En una mesa hay procesos que simulan el comportamiento de unos filósofos que intentan comer de un plato. Cada filósofo tiene un cubierto a su izquierda y uno a su derecha y para poder comer tiene que conseguir los dos. Si lo consigue, mostrará un mensaje en pantalla que indique «Filosofo 2 comiendo».

Despues de comer, soltará los cubiertos y esperará al azar un tiempo entre 1000 y 5000 milisegundos, indicando por pantalla «El filósofo 2 está pensando».

En general todos los objetos de la clase Filósofo está en un bucle infinito dedicándose a comer y a pensar.

Simular este problema en un programa Java que muestre el progreso de todos sin caer en problemas de sincronización ni de inanición.

\begin{figure}[htbp]
\centering
\capstart

\noindent\sphinxincludegraphics{{Filosofos}.png}
\caption{Esquema de los filósofos}\label{\detokenize{textos/tema2:id5}}\end{figure}


\subsection{Boceto de solución}
\label{\detokenize{textos/tema2:boceto-de-solucion}}
\begin{sphinxVerbatim}[commandchars=\\\{\}]
\PYG{k+kn}{import} \PYG{n+nn}{java.util.Random}\PYG{p}{;}


\PYG{k+kd}{public} \PYG{k+kd}{class} \PYG{n+nc}{Filosofo}  \PYG{k+kd}{implements} \PYG{n}{Runnable}\PYG{p}{\PYGZob{}}
        \PYG{k+kd}{public} \PYG{k+kt}{void} \PYG{n+nf}{run}\PYG{p}{(}\PYG{p}{)}\PYG{p}{\PYGZob{}}
                \PYG{n}{String} \PYG{n}{miNombre}\PYG{o}{=}\PYG{n}{Thread}\PYG{p}{.}\PYG{n+na}{currentThread}\PYG{p}{(}\PYG{p}{)}\PYG{p}{.}\PYG{n+na}{getName}\PYG{p}{(}\PYG{p}{)}\PYG{p}{;}
                \PYG{n}{Random} \PYG{n}{generador}\PYG{o}{=}\PYG{k}{new} \PYG{n}{Random}\PYG{p}{(}\PYG{p}{)}\PYG{p}{;}
                \PYG{k}{while} \PYG{p}{(}\PYG{k+kc}{true}\PYG{p}{)}\PYG{p}{\PYGZob{}}
                        \PYG{c+cm}{/* Comer*/}
                        \PYG{c+cm}{/* Intentar coger palillos*/}
                        \PYG{c+cm}{/* Si los coge:*/}
                        \PYG{n}{System}\PYG{p}{.}\PYG{n+na}{out}\PYG{p}{.}\PYG{n+na}{println}\PYG{p}{(}\PYG{n}{miNombre}\PYG{o}{+}\PYG{l+s}{\PYGZdq{}}\PYG{l+s}{ comiendo...}\PYG{l+s}{\PYGZdq{}}\PYG{p}{)}\PYG{p}{;}
                        \PYG{k+kt}{int} \PYG{n}{milisegs}\PYG{o}{=}\PYG{p}{(}\PYG{l+m+mi}{1}\PYG{o}{+}\PYG{n}{generador}\PYG{p}{.}\PYG{n+na}{nextInt}\PYG{p}{(}\PYG{l+m+mi}{5}\PYG{p}{)}\PYG{p}{)}\PYG{o}{*}\PYG{l+m+mi}{1000}\PYG{p}{;}
                        \PYG{n}{esperarTiempoAzar}\PYG{p}{(}\PYG{n}{miNombre}\PYG{p}{,} \PYG{n}{milisegs}\PYG{p}{)}\PYG{p}{;}
                        \PYG{c+cm}{/* Pensando...*/}
                        \PYG{c+c1}{//Recordemos soltar los palillos}
                        \PYG{n}{System}\PYG{p}{.}\PYG{n+na}{out}\PYG{p}{.}\PYG{n+na}{println}\PYG{p}{(}\PYG{n}{miNombre}\PYG{o}{+}\PYG{l+s}{\PYGZdq{}}\PYG{l+s}{  pensando...}\PYG{l+s}{\PYGZdq{}}\PYG{p}{)}\PYG{p}{;}                           \PYG{n}{milisegs}\PYG{o}{=}\PYG{p}{(}\PYG{l+m+mi}{1}\PYG{o}{+}\PYG{n}{generador}\PYG{p}{.}\PYG{n+na}{nextInt}\PYG{p}{(}\PYG{l+m+mi}{5}\PYG{p}{)}\PYG{p}{)}\PYG{o}{*}\PYG{l+m+mi}{1000}\PYG{p}{;}
                        \PYG{n}{esperarTiempoAzar}\PYG{p}{(}\PYG{n}{miNombre}\PYG{p}{,} \PYG{n}{milisegs}\PYG{p}{)}\PYG{p}{;}
                \PYG{p}{\PYGZcb{}}
        \PYG{p}{\PYGZcb{}}

\PYG{k+kd}{private} \PYG{k+kt}{void} \PYG{n+nf}{esperarTiempoAzar}\PYG{p}{(}\PYG{n}{String} \PYG{n}{miNombre}\PYG{p}{,} \PYG{k+kt}{int} \PYG{n}{milisegs}\PYG{p}{)} \PYG{p}{\PYGZob{}}
                \PYG{k}{try} \PYG{p}{\PYGZob{}}
                        \PYG{n}{Thread}\PYG{p}{.}\PYG{n+na}{sleep}\PYG{p}{(}\PYG{n}{milisegs}\PYG{p}{)}\PYG{p}{;}
                \PYG{p}{\PYGZcb{}} \PYG{k}{catch} \PYG{p}{(}\PYG{n}{InterruptedException} \PYG{n}{e}\PYG{p}{)} \PYG{p}{\PYGZob{}}
                        \PYG{n}{System}\PYG{p}{.}\PYG{n+na}{out}\PYG{p}{.}\PYG{n+na}{println}\PYG{p}{(}
                                \PYG{n}{miNombre}\PYG{o}{+}\PYG{l+s}{\PYGZdq{}}\PYG{l+s}{ interrumpido!!. Saliendo...}\PYG{l+s}{\PYGZdq{}}\PYG{p}{)}\PYG{p}{;}
                        \PYG{k}{return} \PYG{p}{;}
                \PYG{p}{\PYGZcb{}}
        \PYG{p}{\PYGZcb{}}
\PYG{p}{\PYGZcb{}}
\end{sphinxVerbatim}


\section{Solución completa al problema de los filósofos}
\label{\detokenize{textos/tema2:solucion-completa-al-problema-de-los-filosofos}}

\subsection{Gestor de recursos compartidos (palillos)}
\label{\detokenize{textos/tema2:gestor-de-recursos-compartidos-palillos}}
\begin{sphinxVerbatim}[commandchars=\\\{\}]
\PYG{k+kn}{package} \PYG{n+nn}{com.iesmaestre.filosofos}\PYG{p}{;}

\PYG{k+kd}{public} \PYG{k+kd}{class} \PYG{n+nc}{GestorPalillos} \PYG{p}{\PYGZob{}}
        \PYG{k+kt}{boolean} \PYG{n}{palilloLibre}\PYG{o}{[}\PYG{o}{]}\PYG{p}{;}
        \PYG{k+kd}{public} \PYG{n+nf}{GestorPalillos}\PYG{p}{(}\PYG{k+kt}{int} \PYG{n}{numPalillos}\PYG{p}{)}\PYG{p}{\PYGZob{}}
                \PYG{n}{palilloLibre}\PYG{o}{=}\PYG{k}{new}  \PYG{k+kt}{boolean}\PYG{o}{[}\PYG{n}{numPalillos}\PYG{o}{]}\PYG{p}{;}
                \PYG{k}{for} \PYG{p}{(}\PYG{k+kt}{int} \PYG{n}{i}\PYG{o}{=}\PYG{l+m+mi}{0}\PYG{p}{;} \PYG{n}{i}\PYG{o}{\PYGZlt{}}\PYG{n}{numPalillos}\PYG{p}{;} \PYG{n}{i}\PYG{o}{+}\PYG{o}{+}\PYG{p}{)}\PYG{p}{\PYGZob{}}
                        \PYG{n}{palilloLibre}\PYG{o}{[}\PYG{n}{i}\PYG{o}{]}\PYG{o}{=}\PYG{k+kc}{true}\PYG{p}{;}
                \PYG{p}{\PYGZcb{}} \PYG{c+c1}{//Fin del for}
        \PYG{p}{\PYGZcb{}} \PYG{c+c1}{//Fin del constructor}
        \PYG{k+kd}{public} \PYG{k+kd}{synchronized} \PYG{k+kt}{boolean}
                \PYG{n+nf}{intentarCogerPalillos}\PYG{p}{(}\PYG{k+kt}{int} \PYG{n}{pos1}\PYG{p}{,} \PYG{k+kt}{int} \PYG{n}{pos2}\PYG{p}{)}
        \PYG{p}{\PYGZob{}}
                \PYG{k+kt}{boolean} \PYG{n}{seConsigue}\PYG{o}{=}\PYG{k+kc}{false}\PYG{p}{;}
                \PYG{k}{if} \PYG{p}{(}
                                \PYG{p}{(}\PYG{n}{palilloLibre}\PYG{o}{[}\PYG{n}{pos1}\PYG{o}{]}\PYG{p}{)}
                                \PYG{o}{\PYGZam{}}\PYG{o}{\PYGZam{}}
                                \PYG{p}{(}\PYG{n}{palilloLibre}\PYG{o}{[}\PYG{n}{pos2}\PYG{o}{]}\PYG{p}{)} \PYG{p}{)}
                \PYG{p}{\PYGZob{}}
                        \PYG{n}{palilloLibre}\PYG{o}{[}\PYG{n}{pos1}\PYG{o}{]}\PYG{o}{=}\PYG{k+kc}{false}\PYG{p}{;}
                        \PYG{n}{palilloLibre}\PYG{o}{[}\PYG{n}{pos2}\PYG{o}{]}\PYG{o}{=}\PYG{k+kc}{false}\PYG{p}{;}
                        \PYG{n}{seConsigue}\PYG{o}{=}\PYG{k+kc}{true}\PYG{p}{;}
                \PYG{p}{\PYGZcb{}} \PYG{c+c1}{//Fin del if}
                \PYG{k}{return} \PYG{n}{seConsigue}\PYG{p}{;}
        \PYG{p}{\PYGZcb{}}

        \PYG{k+kd}{public} \PYG{k+kt}{void} \PYG{n+nf}{liberarPalillos}\PYG{p}{(}\PYG{k+kt}{int} \PYG{n}{pos1}\PYG{p}{,} \PYG{k+kt}{int} \PYG{n}{pos2}\PYG{p}{)}\PYG{p}{\PYGZob{}}
                \PYG{n}{palilloLibre}\PYG{o}{[}\PYG{n}{pos1}\PYG{o}{]}\PYG{o}{=}\PYG{k+kc}{true}\PYG{p}{;}
                \PYG{n}{palilloLibre}\PYG{o}{[}\PYG{n}{pos2}\PYG{o}{]}\PYG{o}{=}\PYG{k+kc}{true}\PYG{p}{;}
        \PYG{p}{\PYGZcb{}}
\PYG{p}{\PYGZcb{}}
\end{sphinxVerbatim}


\subsection{Simulación de un filósofo}
\label{\detokenize{textos/tema2:simulacion-de-un-filosofo}}
\begin{sphinxVerbatim}[commandchars=\\\{\}]
\PYG{k+kn}{package} \PYG{n+nn}{com.iesmaestre.filosofos}\PYG{p}{;}

\PYG{k+kn}{import} \PYG{n+nn}{java.util.Random}\PYG{p}{;}

\PYG{k+kd}{public} \PYG{k+kd}{class} \PYG{n+nc}{Filosofo} \PYG{k+kd}{implements} \PYG{n}{Runnable}\PYG{p}{\PYGZob{}}
        \PYG{n}{GestorPalillos} \PYG{n}{gestorPalillos}\PYG{p}{;}
        \PYG{k+kt}{int} \PYG{n}{posPalilloIzq}\PYG{p}{,}\PYG{n}{posPalilloDer}\PYG{p}{;}
        \PYG{k+kd}{public} \PYG{n+nf}{Filosofo}\PYG{p}{(}\PYG{n}{GestorPalillos} \PYG{n}{g}\PYG{p}{,} \PYG{k+kt}{int} \PYG{n}{pIzq}\PYG{p}{,} \PYG{k+kt}{int} \PYG{n}{pDer}\PYG{p}{)}\PYG{p}{\PYGZob{}}
                \PYG{k}{this}\PYG{p}{.}\PYG{n+na}{gestorPalillos}\PYG{o}{=}\PYG{n}{g}\PYG{p}{;}
                \PYG{k}{this}\PYG{p}{.}\PYG{n+na}{posPalilloDer}\PYG{o}{=}\PYG{n}{pDer}\PYG{p}{;}
                \PYG{k}{this}\PYG{p}{.}\PYG{n+na}{posPalilloIzq}\PYG{o}{=}\PYG{n}{pIzq}\PYG{p}{;}
        \PYG{p}{\PYGZcb{}}
        \PYG{k+kd}{public} \PYG{k+kt}{void} \PYG{n+nf}{run}\PYG{p}{(}\PYG{p}{)} \PYG{p}{\PYGZob{}}
                \PYG{k}{while} \PYG{p}{(}\PYG{k+kc}{true}\PYG{p}{)}\PYG{p}{\PYGZob{}}
                        \PYG{k+kt}{boolean} \PYG{n}{palillosCogidos}\PYG{p}{;}
                        \PYG{n}{palillosCogidos}\PYG{o}{=}
                          \PYG{k}{this}\PYG{p}{.}\PYG{n+na}{gestorPalillos}\PYG{p}{.}\PYG{n+na}{intentarCogerPalillos}\PYG{p}{(}
                                          \PYG{n}{posPalilloIzq}\PYG{p}{,} \PYG{n}{posPalilloDer}\PYG{p}{)}\PYG{p}{;}
                        \PYG{k}{if} \PYG{p}{(}\PYG{n}{palillosCogidos}\PYG{p}{)}\PYG{p}{\PYGZob{}}
                                \PYG{n}{comer}\PYG{p}{(}\PYG{p}{)}\PYG{p}{;}
                                \PYG{k}{this}\PYG{p}{.}\PYG{n+na}{gestorPalillos}\PYG{p}{.}\PYG{n+na}{liberarPalillos}\PYG{p}{(}
                                                \PYG{n}{posPalilloIzq}\PYG{p}{,}
                                                \PYG{n}{posPalilloDer}\PYG{p}{)}\PYG{p}{;}
                                \PYG{n}{dormir}\PYG{p}{(}\PYG{p}{)}\PYG{p}{;}
                        \PYG{p}{\PYGZcb{}} \PYG{c+c1}{//Fin del if}
                \PYG{p}{\PYGZcb{}} \PYG{c+c1}{//Fin del while true}
        \PYG{p}{\PYGZcb{}} \PYG{c+c1}{//Fin del run()}

        \PYG{k+kd}{private} \PYG{k+kt}{void} \PYG{n+nf}{comer}\PYG{p}{(}\PYG{p}{)} \PYG{p}{\PYGZob{}}
                \PYG{n}{System}\PYG{p}{.}\PYG{n+na}{out}\PYG{p}{.}\PYG{n+na}{println}\PYG{p}{(}\PYG{l+s}{\PYGZdq{}}\PYG{l+s}{Filosofo }\PYG{l+s}{\PYGZdq{}}\PYG{o}{+}
                                \PYG{n}{Thread}\PYG{p}{.}\PYG{n+na}{currentThread}\PYG{p}{(}\PYG{p}{)}\PYG{p}{.}\PYG{n+na}{getName}\PYG{p}{(}\PYG{p}{)}\PYG{o}{+}
                                \PYG{l+s}{\PYGZdq{}}\PYG{l+s}{ comiendo}\PYG{l+s}{\PYGZdq{}}\PYG{p}{)}\PYG{p}{;}
                \PYG{n}{esperarTiempoAzar}\PYG{p}{(}\PYG{p}{)}\PYG{p}{;}
        \PYG{p}{\PYGZcb{}}
        \PYG{k+kd}{private} \PYG{k+kt}{void} \PYG{n+nf}{esperarTiempoAzar}\PYG{p}{(}\PYG{p}{)} \PYG{p}{\PYGZob{}}
                \PYG{n}{Random} \PYG{n}{generador}\PYG{o}{=}\PYG{k}{new} \PYG{n}{Random}\PYG{p}{(}\PYG{p}{)}\PYG{p}{;}
                \PYG{k+kt}{int} \PYG{n}{msAzar}\PYG{o}{=}\PYG{n}{generador}\PYG{p}{.}\PYG{n+na}{nextInt}\PYG{p}{(}\PYG{l+m+mi}{3}\PYG{p}{)}\PYG{p}{;}
                \PYG{k}{try} \PYG{p}{\PYGZob{}}
                        \PYG{n}{Thread}\PYG{p}{.}\PYG{n+na}{sleep}\PYG{p}{(}\PYG{n}{msAzar}\PYG{p}{)}\PYG{p}{;}
                \PYG{p}{\PYGZcb{}} \PYG{k}{catch} \PYG{p}{(}\PYG{n}{InterruptedException} \PYG{n}{ex}\PYG{p}{)} \PYG{p}{\PYGZob{}}
                        \PYG{n}{System}\PYG{p}{.}\PYG{n+na}{out}\PYG{p}{.}\PYG{n+na}{println}\PYG{p}{(}\PYG{l+s}{\PYGZdq{}}\PYG{l+s}{Fallo la espera}\PYG{l+s}{\PYGZdq{}}\PYG{p}{)}\PYG{p}{;}
                \PYG{p}{\PYGZcb{}}
        \PYG{p}{\PYGZcb{}}
        \PYG{k+kd}{private} \PYG{k+kt}{void} \PYG{n+nf}{dormir}\PYG{p}{(}\PYG{p}{)}\PYG{p}{\PYGZob{}}
                \PYG{n}{System}\PYG{p}{.}\PYG{n+na}{out}\PYG{p}{.}\PYG{n+na}{println}\PYG{p}{(}\PYG{l+s}{\PYGZdq{}}\PYG{l+s}{Filosofo }\PYG{l+s}{\PYGZdq{}}\PYG{o}{+}
                                \PYG{n}{Thread}\PYG{p}{.}\PYG{n+na}{currentThread}\PYG{p}{(}\PYG{p}{)}\PYG{p}{.}\PYG{n+na}{getName}\PYG{p}{(}\PYG{p}{)}\PYG{o}{+}
                                \PYG{l+s}{\PYGZdq{}}\PYG{l+s}{ durmiendo (zzzzzz)}\PYG{l+s}{\PYGZdq{}}\PYG{p}{)}\PYG{p}{;}
                \PYG{n}{esperarTiempoAzar}\PYG{p}{(}\PYG{p}{)}\PYG{p}{;}
        \PYG{p}{\PYGZcb{}}
\PYG{p}{\PYGZcb{}}
\end{sphinxVerbatim}


\subsection{Lanzador de hilos}
\label{\detokenize{textos/tema2:lanzador-de-hilos}}
\begin{sphinxVerbatim}[commandchars=\\\{\}]
\PYG{k+kn}{package} \PYG{n+nn}{com.iesmaestre.filosofos}\PYG{p}{;}
\PYG{k+kd}{public} \PYG{k+kd}{class} \PYG{n+nc}{Lanzador} \PYG{p}{\PYGZob{}}
        \PYG{k+kd}{public} \PYG{k+kd}{static} \PYG{k+kt}{void} \PYG{n+nf}{main}\PYG{p}{(}\PYG{n}{String}\PYG{o}{[}\PYG{o}{]} \PYG{n}{args}\PYG{p}{)} \PYG{p}{\PYGZob{}}
                \PYG{k+kt}{int} \PYG{n}{NUM\PYGZus{}PROCESOS}\PYG{o}{=}\PYG{l+m+mi}{5}\PYG{p}{;}
                \PYG{n}{Filosofo} \PYG{n}{filosofos}\PYG{o}{[}\PYG{o}{]}\PYG{o}{=}\PYG{k}{new} \PYG{n}{Filosofo}\PYG{o}{[}\PYG{n}{NUM\PYGZus{}PROCESOS}\PYG{o}{]}\PYG{p}{;}
                \PYG{n}{GestorPalillos} \PYG{n}{gestorPalillos}\PYG{p}{;}
                \PYG{n}{gestorPalillos}\PYG{o}{=}\PYG{k}{new} \PYG{n}{GestorPalillos}\PYG{p}{(}\PYG{n}{NUM\PYGZus{}PROCESOS}\PYG{p}{)}\PYG{p}{;}
                \PYG{n}{Thread} \PYG{n}{hilos}\PYG{o}{[}\PYG{o}{]}\PYG{o}{=}\PYG{k}{new} \PYG{n}{Thread}\PYG{o}{[}\PYG{n}{NUM\PYGZus{}PROCESOS}\PYG{o}{]}\PYG{p}{;}
                \PYG{k}{for} \PYG{p}{(}\PYG{k+kt}{int} \PYG{n}{i}\PYG{o}{=}\PYG{l+m+mi}{1}\PYG{p}{;} \PYG{n}{i}\PYG{o}{\PYGZlt{}}\PYG{n}{NUM\PYGZus{}PROCESOS}\PYG{p}{;} \PYG{n}{i}\PYG{o}{+}\PYG{o}{+}\PYG{p}{)}\PYG{p}{\PYGZob{}}
                        \PYG{n}{filosofos}\PYG{o}{[}\PYG{n}{i}\PYG{o}{]}\PYG{o}{=}\PYG{k}{new} \PYG{n}{Filosofo}\PYG{p}{(}
                                \PYG{n}{gestorPalillos}\PYG{p}{,} \PYG{n}{i}\PYG{p}{,} \PYG{n}{i}\PYG{o}{\PYGZhy{}}\PYG{l+m+mi}{1}\PYG{p}{)}\PYG{p}{;}
                        \PYG{n}{hilos}\PYG{o}{[}\PYG{n}{i}\PYG{o}{]}\PYG{o}{=}\PYG{k}{new} \PYG{n}{Thread}\PYG{p}{(}\PYG{n}{filosofos}\PYG{o}{[}\PYG{n}{i}\PYG{o}{]}\PYG{p}{)}\PYG{p}{;}
                        \PYG{n}{hilos}\PYG{o}{[}\PYG{n}{i}\PYG{o}{]}\PYG{p}{.}\PYG{n+na}{start}\PYG{p}{(}\PYG{p}{)}\PYG{p}{;}
                \PYG{p}{\PYGZcb{}}
                \PYG{n}{filosofos}\PYG{o}{[}\PYG{l+m+mi}{0}\PYG{o}{]}\PYG{o}{=}\PYG{k}{new} \PYG{n}{Filosofo}\PYG{p}{(}
                                \PYG{n}{gestorPalillos}\PYG{p}{,} \PYG{l+m+mi}{0}\PYG{p}{,} \PYG{l+m+mi}{4}\PYG{p}{)}\PYG{p}{;}
                \PYG{n}{hilos}\PYG{o}{[}\PYG{l+m+mi}{0}\PYG{o}{]}\PYG{o}{=}\PYG{k}{new} \PYG{n}{Thread}\PYG{p}{(}\PYG{n}{filosofos}\PYG{o}{[}\PYG{l+m+mi}{0}\PYG{o}{]}\PYG{p}{)}\PYG{p}{;}
                \PYG{n}{hilos}\PYG{o}{[}\PYG{l+m+mi}{0}\PYG{o}{]}\PYG{p}{.}\PYG{n+na}{start}\PYG{p}{(}\PYG{p}{)}\PYG{p}{;}
        \PYG{p}{\PYGZcb{}} \PYG{c+c1}{//Fin del main}
\PYG{p}{\PYGZcb{}} \PYG{c+c1}{//Fin de la clase}
\end{sphinxVerbatim}


\section{Problema: simulador de casino}
\label{\detokenize{textos/tema2:problema-simulador-de-casino}}
Se desea simular los posibles beneficios de diversas estrategias de juego en un casino. La ruleta francesa es un juego en el que hay una ruleta con 37 números (del 0 al 36). Cada 3000 milisegundos el croupier saca un número al azar y los diversos hilos apuestan para ver si ganan. Todos los hilos empiezan con 1.000 euros y la banca (que controla la ruleta) con 50.000. Cuando los jugadores pierden dinero, la banca incrementa su saldo.
\begin{itemize}
\item {} 
Se puede jugar a un número concreto. Habrá 4 hilos que eligen números al azar del 1 al 36 (no el 0) y restarán 10 euros de su saldo para apostar a ese ese número. Si sale su número su saldo se incrementa en 360 euros (36 veces lo apostado).

\item {} 
Se puede jugar a par/impar. Habrá 4 hilos que eligen al azar si apuestan a que saldrá un número par o un número impar. Siempre restan 10 euros para apostar y si ganan incrementan su saldo en 20 euros.

\item {} 
Se puede jugar a la «martingala». Habrá 4 hilos que eligen números al azar. Elegirán un número y empezarán restando 10 euros de su saldo para apostar a ese número. Si ganan incrementan su saldo en 360 euros. Si pierden jugarán el doble de su apuesta anterior (es decir, 20, luego 40, luego 80, y así sucesivamente)

\item {} 
La banca acepta todas las apuestas pero nunca paga más dinero del que tiene.

\item {} 
Si sale el 0, todo el mundo pierde y la banca se queda con todo el dinero.

\end{itemize}


\section{Problema: barberos}
\label{\detokenize{textos/tema2:problema-barberos}}
En una peluquería hay barberos y sillas para los clientes (siempre hay más sillas que clientes). Sin embargo, en esta peluquería no siempre hay trabajo por lo que los barberos duermen cuando no hay clientes a los que afeitar. Un cliente puede llegar a la barbería y encontrar alguna silla libre, en cuyo caso, el cliente se sienta y esperará que algún barbero le afeite. Puede ocurrir que el cliente llegue y no haya sillas libres, en cuyo caso se marcha. Simular el comportamiento de la barbería mediante un programa Java teniendo en cuenta que:
\begin{itemize}
\item {} 
Se generan clientes continuamente, algunos encuentran silla y otros no. Los que no consigan silla desaparecen (terminan su ejecución)

\item {} 
Puede haber más sillas que barberos y al revés (poner constantes para poder cambiar fácilmente entre ejecuciones).

\item {} 
Se recuerda que no debe haber inanición, es decir ningún cliente debería quedarse en una silla esperando un tiempo infinito.

\end{itemize}

\begin{figure}[htbp]
\centering
\capstart

\noindent\sphinxincludegraphics{{barberodormilon}.png}
\caption{Los  barberos dormilones}\label{\detokenize{textos/tema2:id6}}\end{figure}


\section{Una (mala) solución al problema de los barberos}
\label{\detokenize{textos/tema2:una-mala-solucion-al-problema-de-los-barberos}}
Prueba la siguiente solución:


\subsection{Clase Barbero}
\label{\detokenize{textos/tema2:clase-barbero}}
\begin{sphinxVerbatim}[commandchars=\\\{\}]
\PYG{k+kd}{public} \PYG{k+kd}{class} \PYG{n+nc}{Barbero} \PYG{k+kd}{implements} \PYG{n}{Runnable} \PYG{p}{\PYGZob{}}
        \PYG{k+kd}{private} \PYG{n}{String}              \PYG{n}{nombre}\PYG{p}{;}
        \PYG{k+kd}{private} \PYG{n}{GestorConcurrencia}  \PYG{n}{gc}\PYG{p}{;}
        \PYG{k+kd}{private} \PYG{n}{Random}              \PYG{n}{generador}\PYG{p}{;}
        \PYG{k+kd}{private} \PYG{k+kt}{int}                 \PYG{n}{MAX\PYGZus{}ESPERA\PYGZus{}SEGS}\PYG{o}{=}\PYG{l+m+mi}{5}\PYG{p}{;}
        \PYG{k+kd}{public} \PYG{n+nf}{Barbero}\PYG{p}{(}\PYG{n}{GestorConcurrencia} \PYG{n}{gc}\PYG{p}{,}\PYG{n}{String} \PYG{n}{nombre}\PYG{p}{)}\PYG{p}{\PYGZob{}}
                \PYG{k}{this}\PYG{p}{.}\PYG{n+na}{nombre}     \PYG{o}{=}\PYG{n}{nombre}\PYG{p}{;}
                \PYG{k}{this}\PYG{p}{.}\PYG{n+na}{gc}         \PYG{o}{=}\PYG{n}{gc}\PYG{p}{;}
                \PYG{k}{this}\PYG{p}{.}\PYG{n+na}{generador}  \PYG{o}{=}\PYG{k}{new} \PYG{n}{Random}\PYG{p}{(}\PYG{p}{)}\PYG{p}{;}
        \PYG{p}{\PYGZcb{}}

        \PYG{k+kd}{public} \PYG{k+kt}{void} \PYG{n+nf}{esperarTiempoAzar}\PYG{p}{(}\PYG{k+kt}{int} \PYG{n}{max}\PYG{p}{)}\PYG{p}{\PYGZob{}}
                \PYG{c+cm}{/* Se calculan unos milisegundos al azar*/}
                \PYG{k+kt}{int} \PYG{n}{msgs}\PYG{o}{=}\PYG{p}{(}\PYG{l+m+mi}{1}\PYG{o}{+}\PYG{n}{generador}\PYG{p}{.}\PYG{n+na}{nextInt}\PYG{p}{(}\PYG{n}{max}\PYG{p}{)}\PYG{p}{)}\PYG{o}{*}\PYG{l+m+mi}{1000}\PYG{p}{;}
                \PYG{k}{try} \PYG{p}{\PYGZob{}}
                        \PYG{n}{Thread}\PYG{p}{.}\PYG{n+na}{currentThread}\PYG{p}{(}\PYG{p}{)}\PYG{p}{.}\PYG{n+na}{sleep}\PYG{p}{(}\PYG{n}{msgs}\PYG{p}{)}\PYG{p}{;}
                \PYG{p}{\PYGZcb{}} \PYG{k}{catch} \PYG{p}{(}\PYG{n}{InterruptedException} \PYG{n}{e}\PYG{p}{)} \PYG{p}{\PYGZob{}}
                        \PYG{c+c1}{// TODO Auto\PYGZhy{}generated catch block}
                        \PYG{n}{e}\PYG{p}{.}\PYG{n+na}{printStackTrace}\PYG{p}{(}\PYG{p}{)}\PYG{p}{;}
                \PYG{p}{\PYGZcb{}}
        \PYG{p}{\PYGZcb{}}
        \PYG{k+kd}{public} \PYG{k+kt}{void} \PYG{n+nf}{run}\PYG{p}{(}\PYG{p}{)}\PYG{p}{\PYGZob{}}
                \PYG{k}{while} \PYG{p}{(}\PYG{k+kc}{true}\PYG{p}{)}\PYG{p}{\PYGZob{}}
                        \PYG{k+kt}{int} \PYG{n}{num\PYGZus{}silla}\PYG{o}{=}\PYG{n}{gc}\PYG{p}{.}\PYG{n+na}{atenderAlgunCliente}\PYG{p}{(}\PYG{p}{)}\PYG{p}{;}
                        \PYG{k}{while} \PYG{p}{(}\PYG{n}{num\PYGZus{}silla}\PYG{o}{=}\PYG{o}{=}\PYG{o}{\PYGZhy{}}\PYG{l+m+mi}{1}\PYG{p}{)}\PYG{p}{\PYGZob{}}
                                \PYG{c+cm}{/* Mientras no haya nadie a quien}
\PYG{c+cm}{                                 * atender, dormimos}
\PYG{c+cm}{                                 */}
                                \PYG{n}{esperarTiempoAzar}\PYG{p}{(}\PYG{n}{MAX\PYGZus{}ESPERA\PYGZus{}SEGS}\PYG{p}{)}\PYG{p}{;}
                                \PYG{n}{num\PYGZus{}silla}\PYG{o}{=}\PYG{n}{gc}\PYG{p}{.}\PYG{n+na}{atenderAlgunCliente}\PYG{p}{(}\PYG{p}{)}\PYG{p}{;}
                        \PYG{p}{\PYGZcb{}}
                        \PYG{c+cm}{/* Si llegamos aqui es que había algún cliente}
\PYG{c+cm}{                         * Simulamos un tiempo de afeitado}
\PYG{c+cm}{                         */}
                        \PYG{n}{esperarTiempoAzar}\PYG{p}{(}\PYG{n}{MAX\PYGZus{}ESPERA\PYGZus{}SEGS}\PYG{p}{)}\PYG{p}{;}
                        \PYG{c+cm}{/* Tras ese tiempo de afeitado se}
\PYG{c+cm}{                         * libera la silla}
\PYG{c+cm}{                         */}
                        \PYG{n}{gc}\PYG{p}{.}\PYG{n+na}{liberarSilla}\PYG{p}{(}\PYG{n}{num\PYGZus{}silla}\PYG{p}{)}\PYG{p}{;}
                        \PYG{c+cm}{/* Y vuelta a empezar*/}
                \PYG{p}{\PYGZcb{}}
        \PYG{p}{\PYGZcb{}}
\PYG{p}{\PYGZcb{}}
\end{sphinxVerbatim}


\subsection{Clase Cliente}
\label{\detokenize{textos/tema2:clase-cliente}}
\begin{sphinxVerbatim}[commandchars=\\\{\}]
\PYG{k+kd}{public} \PYG{k+kd}{class} \PYG{n+nc}{Cliente} \PYG{k+kd}{implements} \PYG{n}{Runnable}\PYG{p}{\PYGZob{}}
        \PYG{n}{GestorConcurrencia}      \PYG{n}{gc}\PYG{p}{;}
        \PYG{k+kd}{public} \PYG{n+nf}{Cliente}\PYG{p}{(}\PYG{n}{GestorConcurrencia} \PYG{n}{gc}\PYG{p}{)}\PYG{p}{\PYGZob{}}
                \PYG{k}{this}\PYG{p}{.}\PYG{n+na}{gc}         \PYG{o}{=}\PYG{n}{gc}\PYG{p}{;}
        \PYG{p}{\PYGZcb{}}
        \PYG{k+kd}{public} \PYG{k+kt}{void} \PYG{n+nf}{run}\PYG{p}{(}\PYG{p}{)}\PYG{p}{\PYGZob{}}
                \PYG{c+cm}{/* Los clientes no esperan que haya}
\PYG{c+cm}{                 * sillas libres, no hay bucle infinito.}
\PYG{c+cm}{                 * Si no hay sillas libres se van...}
\PYG{c+cm}{                 */}
                \PYG{n}{gc}\PYG{p}{.}\PYG{n+na}{getSillaLibre}\PYG{p}{(}\PYG{p}{)}\PYG{p}{;}
        \PYG{p}{\PYGZcb{}}
\PYG{p}{\PYGZcb{}}
\end{sphinxVerbatim}


\subsection{Clase GestorConcurrencia}
\label{\detokenize{textos/tema2:clase-gestorconcurrencia}}
\begin{sphinxVerbatim}[commandchars=\\\{\}]
\PYG{k+kd}{public} \PYG{k+kd}{class} \PYG{n+nc}{GestorConcurrencia} \PYG{p}{\PYGZob{}}
        \PYG{c+cm}{/* Vector que indica cuantas sillas hay y}
\PYG{c+cm}{         * si están libres o no}
\PYG{c+cm}{         */}
        \PYG{k+kt}{boolean}\PYG{o}{[}\PYG{o}{]} \PYG{n}{sillasLibres}\PYG{p}{;}
        \PYG{c+cm}{/* Indica si el cliente sentado en esa}
\PYG{c+cm}{         * silla está atendido por un barbero o no}
\PYG{c+cm}{         */}
        \PYG{k+kt}{boolean}\PYG{o}{[}\PYG{o}{]} \PYG{n}{clienteEstaAtendido}\PYG{p}{;}

        \PYG{k+kd}{public} \PYG{n+nf}{GestorConcurrencia}\PYG{p}{(}\PYG{k+kt}{int} \PYG{n}{numSillas}\PYG{p}{)}\PYG{p}{\PYGZob{}}
                \PYG{c+cm}{/*Construimos los vectores...*/}
                \PYG{n}{sillasLibres}            \PYG{o}{=}\PYG{k}{new} \PYG{k+kt}{boolean}\PYG{o}{[}\PYG{n}{numSillas}\PYG{o}{]}\PYG{p}{;}
                \PYG{n}{clienteEstaAtendido}     \PYG{o}{=}\PYG{k}{new} \PYG{k+kt}{boolean}\PYG{o}{[}\PYG{n}{numSillas}\PYG{o}{]}\PYG{p}{;}
                \PYG{c+cm}{/* ... los inicializamos*/}
                \PYG{k}{for} \PYG{p}{(}\PYG{k+kt}{int} \PYG{n}{i}\PYG{o}{=}\PYG{l+m+mi}{0}\PYG{p}{;} \PYG{n}{i}\PYG{o}{\PYGZlt{}}\PYG{n}{numSillas}\PYG{p}{;}\PYG{n}{i}\PYG{o}{+}\PYG{o}{+}\PYG{p}{)}\PYG{p}{\PYGZob{}}
                        \PYG{n}{sillasLibres}\PYG{o}{[}\PYG{n}{i}\PYG{o}{]}                 \PYG{o}{=}\PYG{k+kc}{true}\PYG{p}{;}
                        \PYG{n}{clienteEstaAtendido}\PYG{o}{[}\PYG{n}{i}\PYG{o}{]}  \PYG{o}{=}\PYG{k+kc}{false}\PYG{p}{;}
                \PYG{p}{\PYGZcb{}}
        \PYG{p}{\PYGZcb{}}

        \PYG{c+cm}{/**}
\PYG{c+cm}{         * Permite obtener una silla libre, usado por la}
\PYG{c+cm}{         * clase Cliente para saber si puede sentarse}
\PYG{c+cm}{         * en algún sitio o irse}
\PYG{c+cm}{         * @return Devuelve el número de la primera silla}
\PYG{c+cm}{         * que está libre o \PYGZhy{}1 si no hay ninguna}
\PYG{c+cm}{         */}
        \PYG{k+kd}{public} \PYG{k+kd}{synchronized} \PYG{k+kt}{int} \PYG{n+nf}{getSillaLibre}\PYG{p}{(}\PYG{p}{)}\PYG{p}{\PYGZob{}}
                \PYG{k}{for} \PYG{p}{(}\PYG{k+kt}{int} \PYG{n}{i}\PYG{o}{=}\PYG{l+m+mi}{0}\PYG{p}{;} \PYG{n}{i}\PYG{o}{\PYGZlt{}}\PYG{n}{sillasLibres}\PYG{p}{.}\PYG{n+na}{length}\PYG{p}{;} \PYG{n}{i}\PYG{o}{+}\PYG{o}{+}\PYG{p}{)}\PYG{p}{\PYGZob{}}
                        \PYG{c+cm}{/* Si está libre la silla ...*/}
                        \PYG{k}{if} \PYG{p}{(}\PYG{n}{sillasLibres}\PYG{o}{[}\PYG{n}{i}\PYG{o}{]}\PYG{p}{)} \PYG{p}{\PYGZob{}}
                                \PYG{c+cm}{/* ...se marca como ocupada*/}
                                \PYG{n}{sillasLibres}\PYG{o}{[}\PYG{n}{i}\PYG{o}{]}\PYG{o}{=}\PYG{k+kc}{false}\PYG{p}{;}
                                \PYG{n}{System}\PYG{p}{.}\PYG{n+na}{out}\PYG{p}{.}\PYG{n+na}{println}\PYG{p}{(}
                                        \PYG{l+s}{\PYGZdq{}}\PYG{l+s}{Cliente sentado en silla }\PYG{l+s}{\PYGZdq{}}\PYG{o}{+}\PYG{n}{i}
                                                \PYG{p}{)}\PYG{p}{;}
                                \PYG{c+cm}{/*.. y devolvemos i...*/}
                                \PYG{k}{return} \PYG{n}{i}\PYG{p}{;}
                        \PYG{p}{\PYGZcb{}}
                \PYG{p}{\PYGZcb{}}
                \PYG{c+cm}{/* Si llegamos aquí es que no había nada libre*/}
                \PYG{k}{return} \PYG{o}{\PYGZhy{}}\PYG{l+m+mi}{1}\PYG{p}{;}
        \PYG{p}{\PYGZcb{}}

        \PYG{c+cm}{/**}
\PYG{c+cm}{         * Nos dice qué silla tiene algun cliente}
\PYG{c+cm}{         * que no está atendido}
\PYG{c+cm}{         * @return un número de silla o \PYGZhy{}1 si no}
\PYG{c+cm}{         * hay clientes sin atender}
\PYG{c+cm}{         */}
        \PYG{k+kd}{public} \PYG{k+kd}{synchronized} \PYG{k+kt}{int} \PYG{n+nf}{atenderAlgunCliente}\PYG{p}{(}\PYG{p}{)}\PYG{p}{\PYGZob{}}
                \PYG{k}{for} \PYG{p}{(}\PYG{k+kt}{int} \PYG{n}{i}\PYG{o}{=}\PYG{l+m+mi}{0}\PYG{p}{;} \PYG{n}{i}\PYG{o}{\PYGZlt{}}\PYG{n}{sillasLibres}\PYG{p}{.}\PYG{n+na}{length}\PYG{p}{;} \PYG{n}{i}\PYG{o}{+}\PYG{o}{+}\PYG{p}{)}\PYG{p}{\PYGZob{}}
                        \PYG{c+cm}{/* Si una silla está ocupada (no libre, false)}
\PYG{c+cm}{                         * y está marcado como \PYGZdq{}sin atender\PYGZdq{} (false)}
\PYG{c+cm}{                         * entonces la marcamos como atendida}
\PYG{c+cm}{                         */}
                        \PYG{k}{if} \PYG{p}{(}\PYG{n}{clienteEstaAtendido}\PYG{o}{[}\PYG{n}{i}\PYG{o}{]}\PYG{o}{=}\PYG{o}{=}\PYG{k+kc}{false}\PYG{p}{)}\PYG{p}{\PYGZob{}}
                                \PYG{n}{clienteEstaAtendido}\PYG{o}{[}\PYG{n}{i}\PYG{o}{]}\PYG{o}{=}\PYG{k+kc}{true}\PYG{p}{;}
                                \PYG{n}{System}\PYG{p}{.}\PYG{n+na}{out}\PYG{p}{.}\PYG{n+na}{println}\PYG{p}{(}
                                                \PYG{l+s}{\PYGZdq{}}\PYG{l+s}{Afeitando cliente en silla }\PYG{l+s}{\PYGZdq{}}\PYG{o}{+}\PYG{n}{i}\PYG{p}{)}\PYG{p}{;}
                                \PYG{k}{return} \PYG{n}{i}\PYG{p}{;}
                        \PYG{p}{\PYGZcb{}}
                \PYG{p}{\PYGZcb{}}
                \PYG{k}{return} \PYG{o}{\PYGZhy{}}\PYG{l+m+mi}{1}\PYG{p}{;}
        \PYG{p}{\PYGZcb{}}

        \PYG{c+cm}{/* El cliente de esa silla, se marcha, por lo}
\PYG{c+cm}{         * que se marca esa silla como \PYGZdq{}libre\PYGZdq{}}
\PYG{c+cm}{         * y como \PYGZdq{}sin atender\PYGZdq{}}
\PYG{c+cm}{         */}
        \PYG{k+kd}{public} \PYG{k+kd}{synchronized} \PYG{k+kt}{void} \PYG{n+nf}{liberarSilla}\PYG{p}{(}\PYG{k+kt}{int} \PYG{n}{i}\PYG{p}{)}\PYG{p}{\PYGZob{}}
                \PYG{n}{sillasLibres}\PYG{o}{[}\PYG{n}{i}\PYG{o}{]}                 \PYG{o}{=}\PYG{k+kc}{true}\PYG{p}{;}
                \PYG{n}{clienteEstaAtendido}\PYG{o}{[}\PYG{n}{i}\PYG{o}{]}  \PYG{o}{=}\PYG{k+kc}{false}\PYG{p}{;}
                \PYG{n}{System}\PYG{p}{.}\PYG{n+na}{out}\PYG{p}{.}\PYG{n+na}{println}\PYG{p}{(}
                                \PYG{l+s}{\PYGZdq{}}\PYG{l+s}{Se marcha el cliente de la silla }\PYG{l+s}{\PYGZdq{}}\PYG{o}{+}\PYG{n}{i}\PYG{p}{)}\PYG{p}{;}
        \PYG{p}{\PYGZcb{}}
\PYG{p}{\PYGZcb{}}
\end{sphinxVerbatim}


\subsection{Clase Lanzador}
\label{\detokenize{textos/tema2:clase-lanzador}}
\begin{sphinxVerbatim}[commandchars=\\\{\}]
\PYG{k+kd}{public} \PYG{k+kd}{class} \PYG{n+nc}{Lanzador} \PYG{p}{\PYGZob{}}

        \PYG{k+kd}{public} \PYG{k+kd}{static} \PYG{k+kt}{void} \PYG{n+nf}{main}\PYG{p}{(}\PYG{n}{String}\PYG{o}{[}\PYG{o}{]} \PYG{n}{args}\PYG{p}{)} \PYG{p}{\PYGZob{}}

                \PYG{k+kt}{int} \PYG{n}{MAX\PYGZus{}BARBEROS}        \PYG{o}{=}\PYG{l+m+mi}{2}\PYG{p}{;}
                \PYG{k+kt}{int} \PYG{n}{MAX\PYGZus{}SILLAS}          \PYG{o}{=}\PYG{n}{MAX\PYGZus{}BARBEROS}\PYG{o}{+}\PYG{l+m+mi}{1}\PYG{p}{;}
                \PYG{k+kt}{int} \PYG{n}{MAX\PYGZus{}CLIENTES}        \PYG{o}{=}\PYG{n}{MAX\PYGZus{}BARBEROS}\PYG{o}{*}\PYG{l+m+mi}{10}\PYG{p}{;}
                \PYG{k+kt}{int} \PYG{n}{MAX\PYGZus{}ESPERA\PYGZus{}SEGS}     \PYG{o}{=} \PYG{l+m+mi}{3}\PYG{p}{;}
                \PYG{n}{GestorConcurrencia} \PYG{n}{gc}\PYG{p}{;}
                \PYG{n}{gc}\PYG{o}{=}\PYG{k}{new} \PYG{n}{GestorConcurrencia}\PYG{p}{(}\PYG{n}{MAX\PYGZus{}SILLAS}\PYG{p}{)}\PYG{p}{;}

                \PYG{n}{Thread}\PYG{o}{[}\PYG{o}{]} \PYG{n}{vhBarberos}     \PYG{o}{=}\PYG{k}{new} \PYG{n}{Thread}\PYG{o}{[}\PYG{n}{MAX\PYGZus{}BARBEROS}\PYG{o}{]}\PYG{p}{;}
                \PYG{k}{for} \PYG{p}{(}\PYG{k+kt}{int} \PYG{n}{i}\PYG{o}{=}\PYG{l+m+mi}{0}\PYG{p}{;} \PYG{n}{i}\PYG{o}{\PYGZlt{}}\PYG{n}{MAX\PYGZus{}BARBEROS}\PYG{p}{;}\PYG{n}{i}\PYG{o}{+}\PYG{o}{+}\PYG{p}{)}\PYG{p}{\PYGZob{}}
                        \PYG{n}{Barbero} \PYG{n}{b}\PYG{o}{=}\PYG{k}{new} \PYG{n}{Barbero}\PYG{p}{(}\PYG{n}{gc}\PYG{p}{,} \PYG{l+s}{\PYGZdq{}}\PYG{l+s}{Barbero }\PYG{l+s}{\PYGZdq{}}\PYG{o}{+}\PYG{n}{i}\PYG{p}{)}\PYG{p}{;}
                        \PYG{n}{Thread} \PYG{n}{hilo}\PYG{o}{=}\PYG{k}{new} \PYG{n}{Thread}\PYG{p}{(}\PYG{n}{b}\PYG{p}{)}\PYG{p}{;}
                        \PYG{n}{vhBarberos}\PYG{o}{[}\PYG{n}{i}\PYG{o}{]}\PYG{o}{=}\PYG{n}{hilo}\PYG{p}{;}
                        \PYG{n}{hilo}\PYG{p}{.}\PYG{n+na}{start}\PYG{p}{(}\PYG{p}{)}\PYG{p}{;}
                \PYG{p}{\PYGZcb{}}

                \PYG{c+cm}{/* Generamos unos cuantos clientes}
\PYG{c+cm}{                 * a intervalos aleatorios}
\PYG{c+cm}{                 */}
                \PYG{n}{Random} \PYG{n}{generador}\PYG{o}{=}\PYG{k}{new} \PYG{n}{Random}\PYG{p}{(}\PYG{p}{)}\PYG{p}{;}
                \PYG{k}{for} \PYG{p}{(}\PYG{k+kt}{int} \PYG{n}{i}\PYG{o}{=}\PYG{l+m+mi}{0}\PYG{p}{;} \PYG{n}{i}\PYG{o}{\PYGZlt{}}\PYG{n}{MAX\PYGZus{}CLIENTES}\PYG{p}{;} \PYG{n}{i}\PYG{o}{+}\PYG{o}{+}\PYG{p}{)}\PYG{p}{\PYGZob{}}
                        \PYG{n}{Cliente} \PYG{n}{c}                       \PYG{o}{=}\PYG{k}{new} \PYG{n}{Cliente}\PYG{p}{(}\PYG{n}{gc}\PYG{p}{)}\PYG{p}{;}
                        \PYG{n}{Thread} \PYG{n}{hiloCliente}      \PYG{o}{=}\PYG{k}{new} \PYG{n}{Thread}\PYG{p}{(}\PYG{n}{c}\PYG{p}{)}\PYG{p}{;}
                        \PYG{n}{hiloCliente}\PYG{p}{.}\PYG{n+na}{start}\PYG{p}{(}\PYG{p}{)}\PYG{p}{;}

                        \PYG{k+kt}{int} \PYG{n}{msegs}\PYG{o}{=}\PYG{n}{generador}\PYG{p}{.}\PYG{n+na}{nextInt}\PYG{p}{(}\PYG{l+m+mi}{3}\PYG{p}{)}\PYG{o}{*}\PYG{l+m+mi}{1000}\PYG{p}{;}
                        \PYG{k}{try} \PYG{p}{\PYGZob{}}
                                \PYG{n}{Thread}\PYG{p}{.}\PYG{n+na}{sleep}\PYG{p}{(}\PYG{n}{msegs}\PYG{p}{)}\PYG{p}{;}
                        \PYG{p}{\PYGZcb{}} \PYG{k}{catch} \PYG{p}{(}\PYG{n}{InterruptedException} \PYG{n}{e}\PYG{p}{)} \PYG{p}{\PYGZob{}}
                                \PYG{c+c1}{// TODO Auto\PYGZhy{}generated catch block}
                                \PYG{n}{e}\PYG{p}{.}\PYG{n+na}{printStackTrace}\PYG{p}{(}\PYG{p}{)}\PYG{p}{;}
                        \PYG{p}{\PYGZcb{}}
                \PYG{p}{\PYGZcb{}} \PYG{c+cm}{/* Fin del for*/}
        \PYG{p}{\PYGZcb{}}
\PYG{p}{\PYGZcb{}}
\end{sphinxVerbatim}


\subsection{Críticas a la solución anterior}
\label{\detokenize{textos/tema2:criticas-a-la-solucion-anterior}}
¿Cual es el problema?

El problema está en que la forma que tiene el gestor de concurrencia de decirle a un barbero qué silla tiene un cliente sin afeitar es incorrecta: como siempre se empieza a buscar por el principio del vector, los clientes sentados al final \sphinxstylestrong{nunca son atendidos}. Hay que corregir esa asignación para \sphinxstyleemphasis{evitar que los procesos sufran de inanición}.


\subsection{Clase cliente}
\label{\detokenize{textos/tema2:id1}}
\begin{sphinxVerbatim}[commandchars=\\\{\}]
\PYG{k+kd}{public} \PYG{k+kd}{class} \PYG{n+nc}{Cliente}  \PYG{p}{\PYGZob{}}
        \PYG{n}{GestorSillas} \PYG{n}{gestorSillas}\PYG{p}{;}
        \PYG{k+kd}{public} \PYG{n+nf}{Cliente}\PYG{p}{(}\PYG{n}{GestorSillas} \PYG{n}{g}\PYG{p}{)}\PYG{p}{\PYGZob{}}
                \PYG{k}{this}\PYG{p}{.}\PYG{n+na}{gestorSillas} \PYG{o}{=} \PYG{n}{g}\PYG{p}{;}
        \PYG{p}{\PYGZcb{}}
        \PYG{k+kd}{public} \PYG{k+kt}{void} \PYG{n+nf}{entrarEnBarberia}\PYG{p}{(}\PYG{p}{)}\PYG{p}{\PYGZob{}}
                \PYG{k+kt}{int} \PYG{n}{posSillaLibre} \PYG{o}{=} \PYG{k}{this}\PYG{p}{.}\PYG{n+na}{gestorSillas}\PYG{p}{.}\PYG{n+na}{getPosSillaLibre}\PYG{p}{(}\PYG{p}{)}\PYG{p}{;}
                \PYG{k}{if} \PYG{p}{(}\PYG{n}{posSillaLibre}\PYG{o}{=}\PYG{o}{=}\PYG{o}{\PYGZhy{}}\PYG{l+m+mi}{1}\PYG{p}{)}\PYG{p}{\PYGZob{}}
                        \PYG{n}{System}\PYG{p}{.}\PYG{n+na}{out}\PYG{p}{.}\PYG{n+na}{println}\PYG{p}{(}\PYG{l+s}{\PYGZdq{}}\PYG{l+s}{No habia sillas libres, me marcho}\PYG{l+s}{\PYGZdq{}}\PYG{p}{)}\PYG{p}{;}
                        \PYG{k}{return} \PYG{p}{;}
                \PYG{p}{\PYGZcb{}}
                \PYG{n}{System}\PYG{p}{.}\PYG{n+na}{out}\PYG{p}{.}\PYG{n+na}{println}\PYG{p}{(}\PYG{l+s}{\PYGZdq{}}\PYG{l+s}{Me siento en la silla:}\PYG{l+s}{\PYGZdq{}}\PYG{o}{+}\PYG{n}{posSillaLibre}\PYG{p}{)}\PYG{p}{;}
        \PYG{p}{\PYGZcb{}}
\PYG{p}{\PYGZcb{}}
\end{sphinxVerbatim}


\subsection{Clase Barbero}
\label{\detokenize{textos/tema2:id2}}
\begin{sphinxVerbatim}[commandchars=\\\{\}]
\PYG{k+kd}{public} \PYG{k+kd}{class} \PYG{n+nc}{Barbero} \PYG{k+kd}{implements} \PYG{n}{Runnable}\PYG{p}{\PYGZob{}}
        \PYG{n}{GestorSillas} \PYG{n}{gestorSillas}\PYG{p}{;}
        \PYG{k+kt}{boolean} \PYG{n}{barberiaAbierta}\PYG{p}{;}
        \PYG{k+kd}{public} \PYG{n+nf}{Barbero} \PYG{p}{(}\PYG{n}{GestorSillas} \PYG{n}{g}\PYG{p}{)}\PYG{p}{\PYGZob{}}
                \PYG{n}{gestorSillas}\PYG{o}{=}\PYG{n}{g}\PYG{p}{;}
                \PYG{n}{barberiaAbierta}\PYG{o}{=}\PYG{k+kc}{true}\PYG{p}{;}
        \PYG{p}{\PYGZcb{}}

        \PYG{k+kd}{public} \PYG{k+kt}{void} \PYG{n+nf}{cerrarBarberia}\PYG{p}{(}\PYG{p}{)}\PYG{p}{\PYGZob{}}
                \PYG{k}{this}\PYG{p}{.}\PYG{n+na}{barberiaAbierta}\PYG{o}{=}\PYG{k+kc}{false}\PYG{p}{;}
        \PYG{p}{\PYGZcb{}}
        \PYG{n+nd}{@Override}
        \PYG{k+kd}{public} \PYG{k+kt}{void} \PYG{n+nf}{run}\PYG{p}{(}\PYG{p}{)} \PYG{p}{\PYGZob{}}
           \PYG{k}{while}\PYG{p}{(}\PYG{n}{barberiaAbierta}\PYG{p}{)}\PYG{p}{\PYGZob{}}
                   \PYG{k+kt}{int} \PYG{n}{posSillaClienteSinAtender}\PYG{p}{;}
                   \PYG{n}{posSillaClienteSinAtender}\PYG{o}{=}
                                   \PYG{k}{this}\PYG{p}{.}\PYG{n+na}{gestorSillas}\PYG{p}{.}\PYG{n+na}{getSiguienteCliente}\PYG{p}{(}\PYG{p}{)}\PYG{p}{;}
                   \PYG{k}{if} \PYG{p}{(}\PYG{n}{posSillaClienteSinAtender}\PYG{o}{=}\PYG{o}{=}\PYG{o}{\PYGZhy{}}\PYG{l+m+mi}{1}\PYG{p}{)}\PYG{p}{\PYGZob{}}
                           \PYG{n}{esperarTiempoAzar}\PYG{p}{(}\PYG{l+m+mi}{3}\PYG{p}{)}\PYG{p}{;}
                   \PYG{p}{\PYGZcb{}} \PYG{k}{else} \PYG{p}{\PYGZob{}}
                           \PYG{n}{System}\PYG{p}{.}\PYG{n+na}{out}\PYG{p}{.}\PYG{n+na}{println}\PYG{p}{(}\PYG{l+s}{\PYGZdq{}}\PYG{l+s}{Barbero atendiendo silla:}\PYG{l+s}{\PYGZdq{}} \PYG{o}{+}
                                           \PYG{n}{posSillaClienteSinAtender}\PYG{p}{)}\PYG{p}{;}
                           \PYG{n}{esperarTiempoAzar}\PYG{p}{(}\PYG{l+m+mi}{3}\PYG{p}{)}\PYG{p}{;}
                           \PYG{k}{this}\PYG{p}{.}\PYG{n+na}{gestorSillas}\PYG{p}{.}\PYG{n+na}{liberarSilla}\PYG{p}{(}\PYG{n}{posSillaClienteSinAtender}\PYG{p}{)}\PYG{p}{;}
                   \PYG{p}{\PYGZcb{}}
           \PYG{p}{\PYGZcb{}}
        \PYG{p}{\PYGZcb{}}

        \PYG{k+kd}{public} \PYG{k+kd}{static} \PYG{k+kt}{void} \PYG{n+nf}{esperarTiempoAzar}\PYG{p}{(}\PYG{k+kt}{int} \PYG{n}{max}\PYG{p}{)}\PYG{p}{\PYGZob{}}
                \PYG{n}{Random} \PYG{n}{generador}\PYG{o}{=}\PYG{k}{new} \PYG{n}{Random}\PYG{p}{(}\PYG{p}{)}\PYG{p}{;}
                \PYG{c+cm}{/* Se calculan unos milisegundos al azar*/}
                \PYG{k+kt}{int} \PYG{n}{msgs}\PYG{o}{=}\PYG{p}{(}\PYG{l+m+mi}{1}\PYG{o}{+}\PYG{n}{generador}\PYG{p}{.}\PYG{n+na}{nextInt}\PYG{p}{(}\PYG{n}{max}\PYG{p}{)}\PYG{p}{)}\PYG{o}{*}\PYG{l+m+mi}{1000}\PYG{p}{;}
                \PYG{k}{try} \PYG{p}{\PYGZob{}}
                                \PYG{n}{Thread}\PYG{p}{.}\PYG{n+na}{currentThread}\PYG{p}{(}\PYG{p}{)}\PYG{p}{.}\PYG{n+na}{sleep}\PYG{p}{(}\PYG{n}{msgs}\PYG{p}{)}\PYG{p}{;}
                \PYG{p}{\PYGZcb{}} \PYG{k}{catch} \PYG{p}{(}\PYG{n}{InterruptedException} \PYG{n}{e}\PYG{p}{)} \PYG{p}{\PYGZob{}}
                                \PYG{c+c1}{// TODO Auto\PYGZhy{}generated catch block}
                                \PYG{n}{e}\PYG{p}{.}\PYG{n+na}{printStackTrace}\PYG{p}{(}\PYG{p}{)}\PYG{p}{;}
                \PYG{p}{\PYGZcb{}}
        \PYG{p}{\PYGZcb{}}

\PYG{p}{\PYGZcb{}}
\end{sphinxVerbatim}


\subsection{Clase GestorSillas}
\label{\detokenize{textos/tema2:clase-gestorsillas}}
\begin{sphinxVerbatim}[commandchars=\\\{\}]
\PYG{k+kd}{public} \PYG{k+kd}{class} \PYG{n+nc}{GestorSillas} \PYG{p}{\PYGZob{}}
        \PYG{k+kd}{private} \PYG{k+kt}{int} \PYG{n}{MAX\PYGZus{}SILLAS}\PYG{p}{;}
        \PYG{k+kd}{private} \PYG{k+kt}{boolean}\PYG{o}{[}\PYG{o}{]} \PYG{n}{estaSillaLibre}\PYG{p}{;}
        \PYG{k+kd}{private} \PYG{k+kt}{boolean}\PYG{o}{[}\PYG{o}{]} \PYG{n}{clienteEstaAtendido}\PYG{p}{;}
        \PYG{k+kd}{private} \PYG{k+kt}{int} \PYG{n}{siguienteClienteParaAtender}\PYG{o}{=}\PYG{l+m+mi}{0}\PYG{p}{;}
        \PYG{n}{GestorSillas}\PYG{p}{(}\PYG{k+kt}{int} \PYG{n}{num}\PYG{p}{)}\PYG{p}{\PYGZob{}}
                \PYG{n}{MAX\PYGZus{}SILLAS}\PYG{o}{=}\PYG{n}{num}\PYG{p}{;}
                \PYG{n}{estaSillaLibre}\PYG{o}{=}\PYG{k}{new} \PYG{k+kt}{boolean}\PYG{o}{[}\PYG{n}{MAX\PYGZus{}SILLAS}\PYG{o}{]}\PYG{p}{;}
                \PYG{n}{clienteEstaAtendido}\PYG{o}{=}\PYG{k}{new} \PYG{k+kt}{boolean}\PYG{o}{[}\PYG{n}{MAX\PYGZus{}SILLAS}\PYG{o}{]}\PYG{p}{;}
                \PYG{k}{for} \PYG{p}{(}\PYG{k+kt}{int} \PYG{n}{i}\PYG{o}{=}\PYG{l+m+mi}{0}\PYG{p}{;} \PYG{n}{i}\PYG{o}{\PYGZlt{}}\PYG{n}{MAX\PYGZus{}SILLAS}\PYG{p}{;} \PYG{n}{i}\PYG{o}{+}\PYG{o}{+}\PYG{p}{)}\PYG{p}{\PYGZob{}}
                        \PYG{n}{estaSillaLibre}\PYG{o}{[}\PYG{n}{i}\PYG{o}{]}       \PYG{o}{=} \PYG{k+kc}{true}\PYG{p}{;}
                        \PYG{n}{clienteEstaAtendido}\PYG{o}{[}\PYG{n}{i}\PYG{o}{]}  \PYG{o}{=} \PYG{k+kc}{false}\PYG{p}{;}
                \PYG{p}{\PYGZcb{}}
        \PYG{p}{\PYGZcb{}}
        \PYG{c+cm}{/**}
\PYG{c+cm}{         * Nos dice el numero de silla que está libre}
\PYG{c+cm}{         * @return Devuelve una posición o \PYGZhy{}1 si está}
\PYG{c+cm}{         * todo ocupado}
\PYG{c+cm}{         */}
        \PYG{k+kd}{public} \PYG{k+kd}{synchronized} \PYG{k+kt}{int} \PYG{n+nf}{getPosSillaLibre}\PYG{p}{(}\PYG{p}{)}\PYG{p}{\PYGZob{}}
                \PYG{k+kt}{int} \PYG{n}{posSilla}\PYG{o}{=}\PYG{o}{\PYGZhy{}}\PYG{l+m+mi}{1}\PYG{p}{;}
                \PYG{k}{for} \PYG{p}{(}\PYG{k+kt}{int} \PYG{n}{pos}\PYG{o}{=}\PYG{l+m+mi}{0}\PYG{p}{;} \PYG{n}{pos}\PYG{o}{\PYGZlt{}}\PYG{n}{MAX\PYGZus{}SILLAS}\PYG{p}{;}\PYG{n}{pos}\PYG{o}{+}\PYG{o}{+}\PYG{p}{)}\PYG{p}{\PYGZob{}}
                        \PYG{k}{if} \PYG{p}{(}\PYG{n}{estaSillaLibre}\PYG{o}{[}\PYG{n}{pos}\PYG{o}{]}\PYG{o}{=}\PYG{o}{=}\PYG{k+kc}{true}\PYG{p}{)}\PYG{p}{\PYGZob{}}
                                \PYG{n}{estaSillaLibre}\PYG{o}{[}\PYG{n}{pos}\PYG{o}{]}\PYG{o}{=}\PYG{k+kc}{false}\PYG{p}{;}
                                \PYG{k}{return} \PYG{n}{pos}\PYG{p}{;}
                        \PYG{p}{\PYGZcb{}}
                \PYG{p}{\PYGZcb{}}
                \PYG{k}{return} \PYG{n}{posSilla}\PYG{p}{;}
        \PYG{p}{\PYGZcb{}}

        \PYG{k+kd}{public} \PYG{k+kt}{void} \PYG{n+nf}{liberarSilla}\PYG{p}{(}\PYG{k+kt}{int} \PYG{n}{pos}\PYG{p}{)}\PYG{p}{\PYGZob{}}
                \PYG{n}{estaSillaLibre}\PYG{o}{[}\PYG{n}{pos}\PYG{o}{]}\PYG{o}{=}\PYG{k+kc}{true}\PYG{p}{;}
                \PYG{n}{clienteEstaAtendido}\PYG{o}{[}\PYG{n}{pos}\PYG{o}{]}\PYG{o}{=}\PYG{k+kc}{false}\PYG{p}{;}
        \PYG{p}{\PYGZcb{}}
        \PYG{k+kd}{public} \PYG{k+kd}{synchronized} \PYG{k+kt}{int} \PYG{n+nf}{getSiguienteCliente}\PYG{p}{(}\PYG{p}{)}\PYG{p}{\PYGZob{}}
                \PYG{k+kt}{int} \PYG{n}{pos}\PYG{o}{=}\PYG{o}{\PYGZhy{}}\PYG{l+m+mi}{1}\PYG{p}{;}
                \PYG{k+kt}{boolean} \PYG{n}{salir}\PYG{p}{;}
                \PYG{k+kt}{int} \PYG{n}{i}\PYG{p}{;}
                \PYG{n}{salir}\PYG{o}{=}\PYG{k+kc}{false}\PYG{p}{;}
                \PYG{n}{i}\PYG{o}{=}\PYG{k}{this}\PYG{p}{.}\PYG{n+na}{siguienteClienteParaAtender}\PYG{p}{;}
                \PYG{k}{while}\PYG{p}{(}\PYG{o}{!}\PYG{n}{salir}\PYG{p}{)}\PYG{p}{\PYGZob{}}
                        \PYG{k}{if} \PYG{p}{(}
                                        \PYG{p}{(}\PYG{k}{this}\PYG{p}{.}\PYG{n+na}{estaSillaLibre}\PYG{o}{[}\PYG{n}{i}\PYG{o}{]}\PYG{o}{=}\PYG{o}{=}\PYG{k+kc}{false}\PYG{p}{)} \PYG{o}{\PYGZam{}}\PYG{o}{\PYGZam{}}
                                        \PYG{p}{(}\PYG{k}{this}\PYG{p}{.}\PYG{n+na}{clienteEstaAtendido}\PYG{o}{[}\PYG{n}{i}\PYG{o}{]}\PYG{o}{=}\PYG{o}{=}\PYG{k+kc}{false}\PYG{p}{)}
                        \PYG{p}{)}
                        \PYG{p}{\PYGZob{}}
                                \PYG{k}{this}\PYG{p}{.}\PYG{n+na}{clienteEstaAtendido}\PYG{o}{[}\PYG{n}{i}\PYG{o}{]}\PYG{o}{=}\PYG{k+kc}{true}\PYG{p}{;}
                                \PYG{k}{this}\PYG{p}{.}\PYG{n+na}{siguienteClienteParaAtender}\PYG{o}{=} \PYG{p}{(}\PYG{n}{i}\PYG{o}{+}\PYG{l+m+mi}{1}\PYG{p}{)} \PYG{o}{\PYGZpc{}} \PYG{n}{MAX\PYGZus{}SILLAS}\PYG{p}{;}
                                \PYG{k}{return} \PYG{n}{i}\PYG{p}{;}

                        \PYG{p}{\PYGZcb{}}
                        \PYG{n}{i}\PYG{o}{+}\PYG{o}{+}\PYG{p}{;}
                        \PYG{k}{if} \PYG{p}{(}\PYG{n}{i}\PYG{o}{=}\PYG{o}{=}\PYG{k}{this}\PYG{p}{.}\PYG{n+na}{MAX\PYGZus{}SILLAS}\PYG{p}{)}\PYG{p}{\PYGZob{}}
                                \PYG{n}{i}\PYG{o}{=}\PYG{l+m+mi}{0}\PYG{p}{;}
                        \PYG{p}{\PYGZcb{}}
                        \PYG{k}{if} \PYG{p}{(}\PYG{n}{i}\PYG{o}{=}\PYG{o}{=}\PYG{k}{this}\PYG{p}{.}\PYG{n+na}{siguienteClienteParaAtender}\PYG{p}{)} \PYG{n}{salir}\PYG{o}{=}\PYG{k+kc}{true}\PYG{p}{;}

                \PYG{p}{\PYGZcb{}}

                \PYG{k}{return} \PYG{n}{pos}\PYG{p}{;}
        \PYG{p}{\PYGZcb{}}
\PYG{p}{\PYGZcb{}}
\end{sphinxVerbatim}


\subsection{Clase Lanzador}
\label{\detokenize{textos/tema2:id3}}
\begin{sphinxVerbatim}[commandchars=\\\{\}]
\PYG{k+kd}{public} \PYG{k+kd}{class} \PYG{n+nc}{Lanzador} \PYG{p}{\PYGZob{}}

        \PYG{k+kd}{public} \PYG{k+kd}{static} \PYG{k+kt}{void} \PYG{n+nf}{main}\PYG{p}{(}\PYG{n}{String}\PYG{o}{[}\PYG{o}{]} \PYG{n}{args}\PYG{p}{)} \PYG{k+kd}{throws} \PYG{n}{InterruptedException} \PYG{p}{\PYGZob{}}
                \PYG{k+kt}{int} \PYG{n}{MAX\PYGZus{}BARBEROS} \PYG{o}{=} \PYG{l+m+mi}{2}\PYG{p}{;}
                \PYG{k+kt}{int} \PYG{n}{MAX\PYGZus{}SILLAS}   \PYG{o}{=} \PYG{l+m+mi}{3}\PYG{p}{;}
                \PYG{k+kt}{int} \PYG{n}{MAX\PYGZus{}CLIENTES} \PYG{o}{=} \PYG{l+m+mi}{1000}\PYG{p}{;}
                \PYG{n}{Barbero}\PYG{o}{[}\PYG{o}{]} \PYG{n}{barberos}\PYG{p}{;}
                \PYG{n}{Thread}\PYG{o}{[}\PYG{o}{]}  \PYG{n}{hilos}\PYG{p}{;}

                \PYG{n}{barberos}\PYG{o}{=}\PYG{k}{new} \PYG{n}{Barbero}\PYG{o}{[}\PYG{n}{MAX\PYGZus{}BARBEROS}\PYG{o}{]}\PYG{p}{;}
                \PYG{n}{hilos}   \PYG{o}{=}\PYG{k}{new} \PYG{n}{Thread} \PYG{o}{[}\PYG{n}{MAX\PYGZus{}BARBEROS}\PYG{o}{]}\PYG{p}{;}

                \PYG{n}{GestorSillas} \PYG{n}{gestorSillas}\PYG{o}{=}\PYG{k}{new} \PYG{n}{GestorSillas}\PYG{p}{(}\PYG{n}{MAX\PYGZus{}SILLAS}\PYG{p}{)}\PYG{p}{;}

                \PYG{k}{for} \PYG{p}{(}\PYG{k+kt}{int} \PYG{n}{i}\PYG{o}{=}\PYG{l+m+mi}{0}\PYG{p}{;} \PYG{n}{i}\PYG{o}{\PYGZlt{}}\PYG{n}{MAX\PYGZus{}BARBEROS}\PYG{p}{;} \PYG{n}{i}\PYG{o}{+}\PYG{o}{+}\PYG{p}{)}\PYG{p}{\PYGZob{}}
                        \PYG{n}{barberos}\PYG{o}{[}\PYG{n}{i}\PYG{o}{]}\PYG{o}{=}\PYG{k}{new} \PYG{n}{Barbero}\PYG{p}{(}\PYG{n}{gestorSillas}\PYG{p}{)}\PYG{p}{;}
                        \PYG{n}{hilos}\PYG{o}{[}\PYG{n}{i}\PYG{o}{]}   \PYG{o}{=}\PYG{k}{new} \PYG{n}{Thread}\PYG{p}{(}\PYG{n}{barberos}\PYG{o}{[}\PYG{n}{i}\PYG{o}{]}\PYG{p}{)}\PYG{p}{;}

                        \PYG{n}{hilos}\PYG{o}{[}\PYG{n}{i}\PYG{o}{]}\PYG{p}{.}\PYG{n+na}{start}\PYG{p}{(}\PYG{p}{)}\PYG{p}{;}
                \PYG{p}{\PYGZcb{}} \PYG{c+c1}{//Fin del for}


                \PYG{k}{for} \PYG{p}{(}\PYG{k+kt}{int} \PYG{n}{i}\PYG{o}{=}\PYG{l+m+mi}{0}\PYG{p}{;} \PYG{n}{i}\PYG{o}{\PYGZlt{}} \PYG{n}{MAX\PYGZus{}CLIENTES}\PYG{p}{;} \PYG{n}{i}\PYG{o}{+}\PYG{o}{+}\PYG{p}{)}\PYG{p}{\PYGZob{}}
                        \PYG{n}{Cliente} \PYG{n}{c}\PYG{o}{=}\PYG{k}{new} \PYG{n}{Cliente}\PYG{p}{(}\PYG{n}{gestorSillas}\PYG{p}{)}\PYG{p}{;}
                        \PYG{n}{c}\PYG{p}{.}\PYG{n+na}{entrarEnBarberia}\PYG{p}{(}\PYG{p}{)}\PYG{p}{;}
                \PYG{p}{\PYGZcb{}}
                \PYG{n}{Barbero}\PYG{p}{.}\PYG{n+na}{esperarTiempoAzar}\PYG{p}{(}\PYG{l+m+mi}{30}\PYG{p}{)}\PYG{p}{;}
                \PYG{c+cm}{/* La jornada ha terminado, \PYGZdq{}cerramos\PYGZdq{} los barberos*/}
                \PYG{k}{for} \PYG{p}{(}\PYG{k+kt}{int} \PYG{n}{i}\PYG{o}{=}\PYG{l+m+mi}{0}\PYG{p}{;} \PYG{n}{i}\PYG{o}{\PYGZlt{}}\PYG{n}{MAX\PYGZus{}BARBEROS}\PYG{p}{;} \PYG{n}{i}\PYG{o}{+}\PYG{o}{+}\PYG{p}{)}\PYG{p}{\PYGZob{}}
                        \PYG{n}{barberos}\PYG{o}{[}\PYG{n}{i}\PYG{o}{]}\PYG{p}{.}\PYG{n+na}{cerrarBarberia}\PYG{p}{(}\PYG{p}{)}\PYG{p}{;}
                        \PYG{n}{hilos}\PYG{o}{[}\PYG{n}{i}\PYG{o}{]}\PYG{p}{.}\PYG{n+na}{join}\PYG{p}{(}\PYG{p}{)}\PYG{p}{;}
                \PYG{p}{\PYGZcb{}}
                \PYG{n}{System}\PYG{p}{.}\PYG{n+na}{out}\PYG{p}{.}\PYG{n+na}{println}\PYG{p}{(}\PYG{l+s}{\PYGZdq{}}\PYG{l+s}{Barberia cerrada.}\PYG{l+s}{\PYGZdq{}}\PYG{p}{)}\PYG{p}{;}
        \PYG{p}{\PYGZcb{}} \PYG{c+c1}{//Fin del main}
\PYG{p}{\PYGZcb{}} \PYG{c+c1}{//Fin de la clase}
\end{sphinxVerbatim}


\section{Problema: productores y consumidores.}
\label{\detokenize{textos/tema2:problema-productores-y-consumidores}}
En un cierto programa se tienen procesos que producen números y procesos que leen esos números. Todos los números se introducen en una cola (o vector) limitada.

Todo el mundo lee y escribe de/en esa cola. Cuando un productor quiere poner un número tendrá que comprobar si la cola está llena. Si está llena, espera un tiempo al azar. Si no está llena pone su número en la última posición libre.

Cuando un lector quiere leer, examina si la cola está vacía. Si lo está espera un tiempo al azar, y sino coge el número que haya al principio de la cola y ese número \sphinxstyleemphasis{ya no está disponible para el siguiente}.

Crear un programa que simule el comportamiento de estos procesos evitando problemas de entrelazado e inanición.


\section{Solución}
\label{\detokenize{textos/tema2:solucion}}
En primer lugar necesitamos una cola que sea «a prueba de hilos» y que permita encolar y desencolar elementos de una manera segura.

Sabemos que antes de encolar hay que comprobar si la cola está llena. También sabemos que antes de desencolar hay que comprobar si la cola está vacía.

Sin embargo \sphinxstylestrong{¡¡CUIDADO!!} es posible programar mal aunque las operaciones de nuestra cola estén protegidas con el método «synchronized». Supongamos que un productor tiene código como este:

\begin{sphinxVerbatim}[commandchars=\\\{\}]
\PYG{k}{while} \PYG{p}{(}\PYG{k+kc}{true}\PYG{p}{)}\PYG{p}{\PYGZob{}}
        \PYG{k}{if} \PYG{p}{(}\PYG{o}{!}\PYG{n}{cola}\PYG{p}{.}\PYG{n+na}{estaLlena}\PYG{p}{(}\PYG{p}{)}\PYG{p}{)}\PYG{p}{\PYGZob{}}
                \PYG{n}{cola}\PYG{p}{.}\PYG{n+na}{encolar}\PYG{p}{(}\PYG{l+m+mi}{4}\PYG{p}{)}\PYG{p}{;}
        \PYG{p}{\PYGZcb{}}
\PYG{p}{\PYGZcb{}}
\end{sphinxVerbatim}

Puede que pensemos que este código está bien pero \sphinxstylestrong{ESTÁ MAL}. Está mal porque puede que en el tiempo transcurrido entre que un productor ejecuta \sphinxcode{\sphinxupquote{estaLlena}} y \sphinxcode{\sphinxupquote{encolar}} OTRO HILO SE HAYA COLADO ENTRE MEDIAS Y HAYA ALTERADO LA EJECUCIÓN. Eso significa que \sphinxstylestrong{aunque llamemos a dos métodos synchronized uno detrás de otro es posible que la ejecución sea incorrecta}. Así, necesitaremos hacer operaciones \sphinxcode{\sphinxupquote{encolar}} y \sphinxcode{\sphinxupquote{desencolar}} que funcionen de manera atómica y nos avisen de si consiguen hacer la operación o no.


\subsection{Una cola limitada en tamaño y thread\sphinxhyphen{}safe}
\label{\detokenize{textos/tema2:una-cola-limitada-en-tamano-y-thread-safe}}
\begin{sphinxVerbatim}[commandchars=\\\{\}]
\PYG{k+kd}{public} \PYG{k+kd}{class} \PYG{n+nc}{Cola} \PYG{p}{\PYGZob{}}
        \PYG{k+kd}{private} \PYG{k+kt}{int} \PYG{n}{MAX\PYGZus{}ELEMENTOS}\PYG{p}{;}
        \PYG{n}{LinkedList}\PYG{o}{\PYGZlt{}}\PYG{n}{Integer}\PYG{o}{\PYGZgt{}} \PYG{n}{cola}\PYG{p}{;}
        \PYG{k+kd}{public} \PYG{n+nf}{Cola} \PYG{p}{(}\PYG{k+kt}{int} \PYG{n}{max}\PYG{p}{)}\PYG{p}{\PYGZob{}}
                \PYG{n}{cola}\PYG{o}{=}\PYG{k}{new} \PYG{n}{LinkedList}\PYG{o}{\PYGZlt{}}\PYG{n}{Integer}\PYG{o}{\PYGZgt{}}\PYG{p}{(}\PYG{p}{)}\PYG{p}{;}
                \PYG{k}{this}\PYG{p}{.}\PYG{n+na}{MAX\PYGZus{}ELEMENTOS}\PYG{o}{=}\PYG{n}{max}\PYG{p}{;}
        \PYG{p}{\PYGZcb{}}
        \PYG{c+cm}{/* En realidad, si estamos seguro de que nadie}
\PYG{c+cm}{        llamará a este método podriamos ponerla como no}
\PYG{c+cm}{        synchronized*/}
        \PYG{k+kd}{public} \PYG{k+kd}{synchronized} \PYG{k+kt}{boolean} \PYG{n+nf}{estaVacia}\PYG{p}{(}\PYG{p}{)}\PYG{p}{\PYGZob{}}
                \PYG{k+kt}{int} \PYG{n}{numElementos}\PYG{o}{=}\PYG{n}{cola}\PYG{p}{.}\PYG{n+na}{size}\PYG{p}{(}\PYG{p}{)}\PYG{p}{;}
                \PYG{k}{if} \PYG{p}{(}\PYG{n}{numElementos}\PYG{o}{=}\PYG{o}{=}\PYG{l+m+mi}{0}\PYG{p}{)}\PYG{p}{\PYGZob{}}
                        \PYG{k}{return} \PYG{k+kc}{true}\PYG{p}{;}
                \PYG{p}{\PYGZcb{}}
                \PYG{k}{return} \PYG{k+kc}{false}\PYG{p}{;}
        \PYG{p}{\PYGZcb{}}
        \PYG{c+cm}{/* Igual que antes, si estamos seguro de que nadie}
\PYG{c+cm}{        llamará a este método podriamos ponerla como no}
\PYG{c+cm}{        synchronized*/}
        \PYG{k+kd}{public} \PYG{k+kd}{synchronized} \PYG{k+kt}{boolean} \PYG{n+nf}{estaLlena}\PYG{p}{(}\PYG{p}{)}\PYG{p}{\PYGZob{}}
                \PYG{k+kt}{int} \PYG{n}{numElementos}\PYG{o}{=}\PYG{n}{cola}\PYG{p}{.}\PYG{n+na}{size}\PYG{p}{(}\PYG{p}{)}\PYG{p}{;}
                \PYG{k}{if} \PYG{p}{(}\PYG{n}{numElementos}\PYG{o}{=}\PYG{o}{=}\PYG{k}{this}\PYG{p}{.}\PYG{n+na}{MAX\PYGZus{}ELEMENTOS}\PYG{p}{)}\PYG{p}{\PYGZob{}}
                        \PYG{k}{return} \PYG{k+kc}{true}\PYG{p}{;}
                \PYG{p}{\PYGZcb{}}
                \PYG{k}{return} \PYG{k+kc}{false}\PYG{p}{;}
        \PYG{p}{\PYGZcb{}}
        \PYG{c+cm}{/* Devuelve true si se pudo hacer y false si no se pudo*/}
        \PYG{k+kd}{public} \PYG{k+kd}{synchronized} \PYG{k+kt}{boolean} \PYG{n+nf}{encolar}\PYG{p}{(}\PYG{k+kt}{int} \PYG{n}{numero}\PYG{p}{)}\PYG{p}{\PYGZob{}}
                \PYG{k}{if} \PYG{p}{(}\PYG{n}{estaLlena}\PYG{p}{(}\PYG{p}{)}\PYG{p}{)}\PYG{p}{\PYGZob{}}
                        \PYG{k}{return} \PYG{k+kc}{false}\PYG{p}{;}
                \PYG{p}{\PYGZcb{}}
                \PYG{n}{cola}\PYG{p}{.}\PYG{n+na}{addLast}\PYG{p}{(}\PYG{n}{numero}\PYG{p}{)}\PYG{p}{;}
                \PYG{k}{return} \PYG{k+kc}{true}\PYG{p}{;}
        \PYG{p}{\PYGZcb{}}
        \PYG{k+kd}{public} \PYG{k+kd}{synchronized} \PYG{k+kt}{int} \PYG{n+nf}{desencolar}\PYG{p}{(}\PYG{p}{)}\PYG{p}{\PYGZob{}}
                \PYG{c+cm}{/* Necesitamos un número especial que actúe}
\PYG{c+cm}{                como comprobador de errores*/}
                \PYG{k}{if} \PYG{p}{(}\PYG{n}{estaVacia}\PYG{p}{(}\PYG{p}{)}\PYG{p}{)}\PYG{p}{\PYGZob{}}
                        \PYG{k}{return} \PYG{o}{\PYGZhy{}}\PYG{l+m+mi}{1}\PYG{p}{;}
                \PYG{p}{\PYGZcb{}}
                \PYG{k+kt}{int} \PYG{n}{numero}\PYG{o}{=}\PYG{n}{cola}\PYG{p}{.}\PYG{n+na}{removeFirst}\PYG{p}{(}\PYG{p}{)}\PYG{p}{;}
                \PYG{k}{return} \PYG{n}{numero}\PYG{p}{;}
        \PYG{p}{\PYGZcb{}}

\PYG{p}{\PYGZcb{}}
\end{sphinxVerbatim}


\subsection{La clase Productor}
\label{\detokenize{textos/tema2:la-clase-productor}}
\begin{sphinxVerbatim}[commandchars=\\\{\}]
\PYG{k+kd}{public} \PYG{k+kd}{class} \PYG{n+nc}{Productor} \PYG{k+kd}{implements} \PYG{n}{Runnable}\PYG{p}{\PYGZob{}}
        \PYG{n}{Cola} \PYG{n}{colaCompartida}\PYG{p}{;}
        \PYG{k+kd}{public} \PYG{n+nf}{Productor}\PYG{p}{(}\PYG{n}{Cola} \PYG{n}{cola}\PYG{p}{)}\PYG{p}{\PYGZob{}}
                \PYG{k}{this}\PYG{p}{.}\PYG{n+na}{colaCompartida}\PYG{o}{=}\PYG{n}{cola}\PYG{p}{;}
        \PYG{p}{\PYGZcb{}}
        \PYG{k+kd}{public} \PYG{k+kt}{void} \PYG{n+nf}{run}\PYG{p}{(}\PYG{p}{)} \PYG{p}{\PYGZob{}}
           \PYG{k}{while} \PYG{p}{(}\PYG{k+kc}{true}\PYG{p}{)}\PYG{p}{\PYGZob{}}
                   \PYG{k+kt}{int} \PYG{n}{num}\PYG{o}{=}\PYG{n}{Utilidades}\PYG{p}{.}\PYG{n+na}{numAzar}\PYG{p}{(}\PYG{l+m+mi}{10}\PYG{p}{)}\PYG{p}{;}
                   \PYG{k}{while} \PYG{p}{(}\PYG{n}{colaCompartida}\PYG{p}{.}\PYG{n+na}{encolar}\PYG{p}{(}\PYG{n}{num}\PYG{p}{)}\PYG{o}{=}\PYG{o}{=}\PYG{k+kc}{false}\PYG{p}{)}\PYG{p}{\PYGZob{}}
                           \PYG{n}{Utilidades}\PYG{p}{.}\PYG{n+na}{esperarTiempoAzar}\PYG{p}{(}\PYG{l+m+mi}{3000}\PYG{p}{)}\PYG{p}{;}
                   \PYG{p}{\PYGZcb{}} \PYG{c+cm}{/*Fin del while*/}

                   \PYG{n}{System}\PYG{p}{.}\PYG{n+na}{out}\PYG{p}{.}\PYG{n+na}{println}\PYG{p}{(}\PYG{l+s}{\PYGZdq{}}\PYG{l+s}{Productor encoló el numero:}\PYG{l+s}{\PYGZdq{}}\PYG{o}{+}\PYG{n}{num}\PYG{p}{)}\PYG{p}{;}
           \PYG{p}{\PYGZcb{}} \PYG{c+cm}{/*Fin del while externo*/}
        \PYG{p}{\PYGZcb{}} \PYG{c+cm}{/*Fin de run()*/}
\PYG{p}{\PYGZcb{}} \PYG{c+cm}{/*Fin de la clase*/}
\end{sphinxVerbatim}


\subsection{La clase Consumidor}
\label{\detokenize{textos/tema2:la-clase-consumidor}}
\begin{sphinxVerbatim}[commandchars=\\\{\}]
\PYG{k+kd}{public} \PYG{k+kd}{class} \PYG{n+nc}{Consumidor} \PYG{k+kd}{implements} \PYG{n}{Runnable}\PYG{p}{\PYGZob{}}
        \PYG{n}{Cola} \PYG{n}{colaCompartida}\PYG{p}{;}
        \PYG{k+kd}{public} \PYG{n+nf}{Consumidor}\PYG{p}{(}\PYG{n}{Cola} \PYG{n}{cola}\PYG{p}{)}\PYG{p}{\PYGZob{}}
                \PYG{k}{this}\PYG{p}{.}\PYG{n+na}{colaCompartida}\PYG{o}{=}\PYG{n}{cola}\PYG{p}{;}
        \PYG{p}{\PYGZcb{}}
        \PYG{n+nd}{@Override}
        \PYG{k+kd}{public} \PYG{k+kt}{void} \PYG{n+nf}{run}\PYG{p}{(}\PYG{p}{)} \PYG{p}{\PYGZob{}}
                \PYG{k+kt}{int} \PYG{n}{num}\PYG{p}{;}
                \PYG{k}{while} \PYG{p}{(}\PYG{k+kc}{true}\PYG{p}{)}\PYG{p}{\PYGZob{}}
                        \PYG{n}{num}\PYG{o}{=}\PYG{n}{colaCompartida}\PYG{p}{.}\PYG{n+na}{desencolar}\PYG{p}{(}\PYG{p}{)}\PYG{p}{;}
                        \PYG{k}{if} \PYG{p}{(}\PYG{n}{num}\PYG{o}{!}\PYG{o}{=}\PYG{o}{\PYGZhy{}}\PYG{l+m+mi}{1}\PYG{p}{)}\PYG{p}{\PYGZob{}}
                                \PYG{n}{System}\PYG{p}{.}\PYG{n+na}{out}\PYG{p}{.}\PYG{n+na}{println}\PYG{p}{(}\PYG{l+s}{\PYGZdq{}}\PYG{l+s}{Consumidor recuperó el numero:}\PYG{l+s}{\PYGZdq{}}\PYG{o}{+}\PYG{n}{num}\PYG{p}{)}\PYG{p}{;}
                        \PYG{p}{\PYGZcb{}} \PYG{c+cm}{/* Fin del if*/}
                \PYG{p}{\PYGZcb{}} \PYG{c+cm}{/*Fin del bucle infinito*/}
        \PYG{p}{\PYGZcb{}} \PYG{c+cm}{/* Fin del run()*/}
\PYG{p}{\PYGZcb{}} \PYG{c+cm}{/*Fin de la clase Consumidor*/}
\end{sphinxVerbatim}


\subsection{Un lanzador}
\label{\detokenize{textos/tema2:un-lanzador}}
\begin{sphinxVerbatim}[commandchars=\\\{\}]
\PYG{k+kd}{public} \PYG{k+kd}{class} \PYG{n+nc}{Lanzador} \PYG{p}{\PYGZob{}}

        \PYG{k+kd}{public} \PYG{k+kd}{static} \PYG{k+kt}{void} \PYG{n+nf}{main}\PYG{p}{(}\PYG{n}{String}\PYG{o}{[}\PYG{o}{]} \PYG{n}{args}\PYG{p}{)} \PYG{k+kd}{throws} \PYG{n}{InterruptedException} \PYG{p}{\PYGZob{}}
                \PYG{k+kt}{int} \PYG{n}{MAX\PYGZus{}PRODUCTORES}     \PYG{o}{=} \PYG{l+m+mi}{5}\PYG{p}{;}
                \PYG{k+kt}{int} \PYG{n}{MAX\PYGZus{}CONSUMIDORES}    \PYG{o}{=} \PYG{l+m+mi}{7}\PYG{p}{;}
                \PYG{k+kt}{int} \PYG{n}{MAX\PYGZus{}ELEMENTOS}       \PYG{o}{=} \PYG{l+m+mi}{10}\PYG{p}{;}

                \PYG{n}{Thread}\PYG{o}{[}\PYG{o}{]} \PYG{n}{hilosProductor}\PYG{p}{;}
                \PYG{n}{Thread}\PYG{o}{[}\PYG{o}{]} \PYG{n}{hilosConsumidor}\PYG{p}{;}

                \PYG{n}{hilosProductor}   \PYG{o}{=} \PYG{k}{new} \PYG{n}{Thread}\PYG{o}{[}\PYG{n}{MAX\PYGZus{}PRODUCTORES}\PYG{o}{]}\PYG{p}{;}
                \PYG{n}{hilosConsumidor}  \PYG{o}{=} \PYG{k}{new} \PYG{n}{Thread}\PYG{o}{[}\PYG{n}{MAX\PYGZus{}CONSUMIDORES}\PYG{o}{]}\PYG{p}{;}

                \PYG{n}{Cola} \PYG{n}{colaCompartida}\PYG{o}{=}\PYG{k}{new} \PYG{n}{Cola}\PYG{p}{(}\PYG{n}{MAX\PYGZus{}ELEMENTOS}\PYG{p}{)}\PYG{p}{;}

                \PYG{c+cm}{/*Construimos los productores*/}
                \PYG{k}{for} \PYG{p}{(}\PYG{k+kt}{int} \PYG{n}{i}\PYG{o}{=}\PYG{l+m+mi}{0}\PYG{p}{;} \PYG{n}{i}\PYG{o}{\PYGZlt{}}\PYG{n}{MAX\PYGZus{}PRODUCTORES}\PYG{p}{;} \PYG{n}{i}\PYG{o}{+}\PYG{o}{+}\PYG{p}{)}\PYG{p}{\PYGZob{}}
                        \PYG{n}{Productor} \PYG{n}{productor}\PYG{o}{=}\PYG{k}{new} \PYG{n}{Productor}\PYG{p}{(}\PYG{n}{colaCompartida}\PYG{p}{)}\PYG{p}{;}
                        \PYG{n}{hilosProductor}\PYG{o}{[}\PYG{n}{i}\PYG{o}{]}\PYG{o}{=}\PYG{k}{new} \PYG{n}{Thread}\PYG{p}{(}\PYG{n}{productor}\PYG{p}{)}\PYG{p}{;}
                        \PYG{n}{hilosProductor}\PYG{o}{[}\PYG{n}{i}\PYG{o}{]}\PYG{p}{.}\PYG{n+na}{start}\PYG{p}{(}\PYG{p}{)}\PYG{p}{;}
                \PYG{p}{\PYGZcb{}}
                \PYG{c+cm}{/*Construimos los consumidores*/}
                \PYG{k}{for} \PYG{p}{(}\PYG{k+kt}{int} \PYG{n}{i}\PYG{o}{=}\PYG{l+m+mi}{0}\PYG{p}{;} \PYG{n}{i}\PYG{o}{\PYGZlt{}}\PYG{n}{MAX\PYGZus{}CONSUMIDORES}\PYG{p}{;} \PYG{n}{i}\PYG{o}{+}\PYG{o}{+}\PYG{p}{)}\PYG{p}{\PYGZob{}}
                        \PYG{n}{Consumidor} \PYG{n}{consumidor}\PYG{o}{=}\PYG{k}{new} \PYG{n}{Consumidor}\PYG{p}{(}\PYG{n}{colaCompartida}\PYG{p}{)}\PYG{p}{;}
                        \PYG{n}{hilosConsumidor}\PYG{o}{[}\PYG{n}{i}\PYG{o}{]}\PYG{o}{=}\PYG{k}{new} \PYG{n}{Thread}\PYG{p}{(}\PYG{n}{consumidor}\PYG{p}{)}\PYG{p}{;}
                        \PYG{n}{hilosConsumidor}\PYG{o}{[}\PYG{n}{i}\PYG{o}{]}\PYG{p}{.}\PYG{n+na}{start}\PYG{p}{(}\PYG{p}{)}\PYG{p}{;}
                \PYG{p}{\PYGZcb{}}

                \PYG{c+cm}{/* Esperamos a que acaben todos los hilos, primero}
\PYG{c+cm}{                productores y luego consumidores}
\PYG{c+cm}{                */}
                \PYG{k}{for} \PYG{p}{(}\PYG{k+kt}{int} \PYG{n}{i}\PYG{o}{=}\PYG{l+m+mi}{0}\PYG{p}{;} \PYG{n}{i}\PYG{o}{\PYGZlt{}}\PYG{n}{MAX\PYGZus{}PRODUCTORES}\PYG{p}{;} \PYG{n}{i}\PYG{o}{+}\PYG{o}{+}\PYG{p}{)}\PYG{p}{\PYGZob{}}
                        \PYG{n}{hilosProductor}\PYG{o}{[}\PYG{n}{i}\PYG{o}{]}\PYG{p}{.}\PYG{n+na}{join}\PYG{p}{(}\PYG{p}{)}\PYG{p}{;}
                \PYG{p}{\PYGZcb{}}
                \PYG{k}{for} \PYG{p}{(}\PYG{k+kt}{int} \PYG{n}{i}\PYG{o}{=}\PYG{l+m+mi}{0}\PYG{p}{;} \PYG{n}{i}\PYG{o}{\PYGZlt{}}\PYG{n}{MAX\PYGZus{}CONSUMIDORES}\PYG{p}{;} \PYG{n}{i}\PYG{o}{+}\PYG{o}{+}\PYG{p}{)}\PYG{p}{\PYGZob{}}
                        \PYG{n}{hilosConsumidor}\PYG{o}{[}\PYG{n}{i}\PYG{o}{]}\PYG{p}{.}\PYG{n+na}{join}\PYG{p}{(}\PYG{p}{)}\PYG{p}{;}
                \PYG{p}{\PYGZcb{}}
        \PYG{p}{\PYGZcb{}}

\PYG{p}{\PYGZcb{}}
\end{sphinxVerbatim}


\section{Ejercicio}
\label{\detokenize{textos/tema2:ejercicio}}
En unos grandes almacenes hay 300 clientes agolpados en la puerta para intentar conseguir un producto del cual solo hay 100 unidades.

Por la puerta solo cabe una persona, pero la paciencia de los clientes es limitada por lo que solo se harán un máximo de 10 intentos para entrar por la puerta. Si despues de 10 intentos la puerta no se ha encontrado libre ni una sola vez, el cliente desiste y se marcha.

Cuando se consigue entrar por la puerta el cliente puede encontrarse con dos situaciones:
\begin{enumerate}
\sphinxsetlistlabels{\arabic}{enumi}{enumii}{}{.}%
\item {} 
Quedan productos: el cliente cogerá uno y se marchará.

\item {} 
No quedan productos: el cliente simplemente se marchará.

\end{enumerate}

Realizar la simulación en Java de dicha situación.

Una posible solución sería la siguiente:


\subsection{Clase cliente}
\label{\detokenize{textos/tema2:id4}}
\begin{sphinxVerbatim}[commandchars=\\\{\}]
\PYG{k+kn}{package} \PYG{n+nn}{grandesalmacenes}\PYG{p}{;}

\PYG{k+kn}{import} \PYG{n+nn}{java.util.Random}\PYG{p}{;}

\PYG{k+kd}{public} \PYG{k+kd}{class} \PYG{n+nc}{Cliente} \PYG{k+kd}{implements} \PYG{n}{Runnable}\PYG{p}{\PYGZob{}}

        \PYG{n}{Puerta}      \PYG{n}{puerta}\PYG{p}{;}
        \PYG{n}{Almacen}     \PYG{n}{almacen}\PYG{p}{;}
        \PYG{n}{String}      \PYG{n}{nombre}\PYG{p}{;}
        \PYG{n}{Random}      \PYG{n}{generador}\PYG{p}{;}
        \PYG{k+kd}{final} \PYG{k+kt}{int} \PYG{n}{MAX\PYGZus{}INTENTOS}  \PYG{o}{=}   \PYG{l+m+mi}{10}\PYG{p}{;}
        \PYG{k+kd}{public} \PYG{n+nf}{Cliente}\PYG{p}{(}\PYG{n}{Puerta} \PYG{n}{p}\PYG{p}{,} \PYG{n}{Almacen} \PYG{n}{a}\PYG{p}{,} \PYG{n}{String} \PYG{n}{nombre}\PYG{p}{)}\PYG{p}{\PYGZob{}}
                \PYG{k}{this}\PYG{p}{.}\PYG{n+na}{puerta}     \PYG{o}{=}   \PYG{n}{p}\PYG{p}{;}
                \PYG{k}{this}\PYG{p}{.}\PYG{n+na}{almacen}    \PYG{o}{=}   \PYG{n}{a}\PYG{p}{;}
                \PYG{k}{this}\PYG{p}{.}\PYG{n+na}{nombre}     \PYG{o}{=}   \PYG{n}{nombre}\PYG{p}{;}
                \PYG{n}{generador}       \PYG{o}{=}   \PYG{k}{new} \PYG{n}{Random}\PYG{p}{(}\PYG{p}{)}\PYG{p}{;}
        \PYG{p}{\PYGZcb{}}

        \PYG{k+kd}{public} \PYG{k+kt}{void} \PYG{n+nf}{esperar}\PYG{p}{(}\PYG{p}{)}\PYG{p}{\PYGZob{}}
                \PYG{k}{try} \PYG{p}{\PYGZob{}}
                        \PYG{k+kt}{int} \PYG{n}{ms\PYGZus{}azar} \PYG{o}{=} \PYG{n}{generador}\PYG{p}{.}\PYG{n+na}{nextInt}\PYG{p}{(}\PYG{l+m+mi}{100}\PYG{p}{)}\PYG{p}{;}
                        \PYG{n}{Thread}\PYG{p}{.}\PYG{n+na}{sleep}\PYG{p}{(}\PYG{n}{ms\PYGZus{}azar}\PYG{p}{)}\PYG{p}{;}
                \PYG{p}{\PYGZcb{}} \PYG{k}{catch} \PYG{p}{(}\PYG{n}{InterruptedException} \PYG{n}{ex}\PYG{p}{)} \PYG{p}{\PYGZob{}}
                        \PYG{n}{System}\PYG{p}{.}\PYG{n+na}{out}\PYG{p}{.}\PYG{n+na}{println}\PYG{p}{(}\PYG{l+s}{\PYGZdq{}}\PYG{l+s}{Falló la espera}\PYG{l+s}{\PYGZdq{}}\PYG{p}{)}\PYG{p}{;}
                \PYG{p}{\PYGZcb{}}
        \PYG{p}{\PYGZcb{}}
        \PYG{n+nd}{@Override}
        \PYG{k+kd}{public} \PYG{k+kt}{void} \PYG{n+nf}{run}\PYG{p}{(}\PYG{p}{)} \PYG{p}{\PYGZob{}}
                \PYG{k}{for} \PYG{p}{(}\PYG{k+kt}{int} \PYG{n}{i}\PYG{o}{=}\PYG{l+m+mi}{0}\PYG{p}{;} \PYG{n}{i}\PYG{o}{\PYGZlt{}}\PYG{n}{MAX\PYGZus{}INTENTOS}\PYG{p}{;} \PYG{n}{i}\PYG{o}{+}\PYG{o}{+}\PYG{p}{)}\PYG{p}{\PYGZob{}}
                        \PYG{k}{if} \PYG{p}{(}\PYG{o}{!}\PYG{n}{puerta}\PYG{p}{.}\PYG{n+na}{estaOcupada}\PYG{p}{(}\PYG{p}{)}\PYG{p}{)}\PYG{p}{\PYGZob{}}
                                \PYG{k}{if} \PYG{p}{(}\PYG{n}{puerta}\PYG{p}{.}\PYG{n+na}{intentarEntrar}\PYG{p}{(}\PYG{p}{)}\PYG{p}{)}\PYG{p}{\PYGZob{}}
                                        \PYG{n}{esperar}\PYG{p}{(}\PYG{p}{)}\PYG{p}{;}
                                        \PYG{n}{puerta}\PYG{p}{.}\PYG{n+na}{liberarPuerta}\PYG{p}{(}\PYG{p}{)}\PYG{p}{;}
                                        \PYG{k}{if} \PYG{p}{(}\PYG{n}{almacen}\PYG{p}{.}\PYG{n+na}{cogerProducto}\PYG{p}{(}\PYG{p}{)}\PYG{p}{)}\PYG{p}{\PYGZob{}}
                                                \PYG{n}{System}\PYG{p}{.}\PYG{n+na}{out}\PYG{p}{.}\PYG{n+na}{println}\PYG{p}{(}
                                                                \PYG{k}{this}\PYG{p}{.}\PYG{n+na}{nombre} \PYG{o}{+} \PYG{l+s}{\PYGZdq{}}\PYG{l+s}{: cogí un producto!}\PYG{l+s}{\PYGZdq{}}\PYG{p}{)}\PYG{p}{;}
                                                \PYG{k}{return} \PYG{p}{;}
                                        \PYG{p}{\PYGZcb{}}
                                        \PYG{k}{else} \PYG{p}{\PYGZob{}}
                                                \PYG{n}{System}\PYG{p}{.}\PYG{n+na}{out}\PYG{p}{.}\PYG{n+na}{println}\PYG{p}{(}
                                                           \PYG{k}{this}\PYG{p}{.}\PYG{n+na}{nombre}\PYG{o}{+}
                                                           \PYG{l+s}{\PYGZdq{}}\PYG{l+s}{: ops, crucé pero no cogí nada}\PYG{l+s}{\PYGZdq{}}\PYG{p}{)}\PYG{p}{;}
                                                \PYG{k}{return} \PYG{p}{;}
                                        \PYG{p}{\PYGZcb{}} \PYG{c+c1}{//Fin del else}
                                \PYG{p}{\PYGZcb{}} \PYG{c+c1}{//Fin del if}
                        \PYG{p}{\PYGZcb{}} \PYG{k}{else}\PYG{p}{\PYGZob{}}
                           \PYG{n}{esperar}\PYG{p}{(}\PYG{p}{)}\PYG{p}{;}
                        \PYG{p}{\PYGZcb{}}

                \PYG{p}{\PYGZcb{}} \PYG{c+c1}{//Fin del for}
                \PYG{c+c1}{//Se superó el máximo de reintentos y abandonamos}
                \PYG{n}{System}\PYG{p}{.}\PYG{n+na}{out}\PYG{p}{.}\PYG{n+na}{println}\PYG{p}{(}\PYG{k}{this}\PYG{p}{.}\PYG{n+na}{nombre}\PYG{o}{+}
                                \PYG{l+s}{\PYGZdq{}}\PYG{l+s}{: lo intenté muchas veces y no pude}\PYG{l+s}{\PYGZdq{}}\PYG{p}{)}\PYG{p}{;}
        \PYG{p}{\PYGZcb{}}

\PYG{p}{\PYGZcb{}}
\end{sphinxVerbatim}


\subsection{Clase Almacén}
\label{\detokenize{textos/tema2:clase-almacen}}
\begin{sphinxVerbatim}[commandchars=\\\{\}]
\PYG{k+kn}{package} \PYG{n+nn}{grandesalmacenes}\PYG{p}{;}

\PYG{k+kd}{public} \PYG{k+kd}{class} \PYG{n+nc}{Almacen} \PYG{p}{\PYGZob{}}
        \PYG{k+kd}{private} \PYG{k+kt}{int} \PYG{n}{numProductos}\PYG{p}{;}
        \PYG{k+kd}{public} \PYG{n+nf}{Almacen}\PYG{p}{(}\PYG{k+kt}{int} \PYG{n}{nProductos}\PYG{p}{)}\PYG{p}{\PYGZob{}}
                \PYG{k}{this}\PYG{p}{.}\PYG{n+na}{numProductos}\PYG{o}{=}\PYG{n}{nProductos}\PYG{p}{;}
        \PYG{p}{\PYGZcb{}}
        \PYG{k+kd}{public} \PYG{k+kt}{boolean} \PYG{n+nf}{cogerProducto}\PYG{p}{(}\PYG{p}{)}\PYG{p}{\PYGZob{}}
                \PYG{k}{if} \PYG{p}{(}\PYG{k}{this}\PYG{p}{.}\PYG{n+na}{numProductos}\PYG{o}{\PYGZgt{}}\PYG{l+m+mi}{0}\PYG{p}{)}\PYG{p}{\PYGZob{}}
                        \PYG{k}{this}\PYG{p}{.}\PYG{n+na}{numProductos}\PYG{o}{\PYGZhy{}}\PYG{o}{\PYGZhy{}}\PYG{p}{;}
                        \PYG{k}{return} \PYG{k+kc}{true}\PYG{p}{;}
                \PYG{p}{\PYGZcb{}}
                \PYG{k}{return} \PYG{k+kc}{false}\PYG{p}{;}
        \PYG{p}{\PYGZcb{}}
\PYG{p}{\PYGZcb{}}
\end{sphinxVerbatim}


\subsection{Clase Puerta}
\label{\detokenize{textos/tema2:clase-puerta}}
\begin{sphinxVerbatim}[commandchars=\\\{\}]
\PYG{k+kn}{package} \PYG{n+nn}{grandesalmacenes}\PYG{p}{;}

\PYG{k+kn}{import} \PYG{n+nn}{java.util.Random}\PYG{p}{;}
\PYG{k+kn}{import} \PYG{n+nn}{java.util.logging.Level}\PYG{p}{;}
\PYG{k+kn}{import} \PYG{n+nn}{java.util.logging.Logger}\PYG{p}{;}


\PYG{k+kd}{public} \PYG{k+kd}{class} \PYG{n+nc}{Puerta} \PYG{p}{\PYGZob{}}
        \PYG{k+kt}{boolean} \PYG{n}{ocupada}\PYG{p}{;}

        \PYG{n}{Puerta}\PYG{p}{(}\PYG{p}{)}\PYG{p}{\PYGZob{}}
                \PYG{k}{this}\PYG{p}{.}\PYG{n+na}{ocupada}\PYG{o}{=}\PYG{k+kc}{false}\PYG{p}{;}

        \PYG{p}{\PYGZcb{}}
        \PYG{k+kd}{public} \PYG{k+kt}{boolean} \PYG{n+nf}{estaOcupada}\PYG{p}{(}\PYG{p}{)}\PYG{p}{\PYGZob{}}
                \PYG{k}{return} \PYG{k}{this}\PYG{p}{.}\PYG{n+na}{ocupada}\PYG{p}{;}
        \PYG{p}{\PYGZcb{}}
        \PYG{k+kd}{public} \PYG{k+kd}{synchronized} \PYG{k+kt}{void} \PYG{n+nf}{liberarPuerta}\PYG{p}{(}\PYG{p}{)}\PYG{p}{\PYGZob{}}
                \PYG{k}{this}\PYG{p}{.}\PYG{n+na}{ocupada}\PYG{o}{=}\PYG{k+kc}{false}\PYG{p}{;}
        \PYG{p}{\PYGZcb{}}
        \PYG{k+kd}{public} \PYG{k+kd}{synchronized} \PYG{k+kt}{boolean} \PYG{n+nf}{intentarEntrar}\PYG{p}{(}\PYG{p}{)}\PYG{p}{\PYGZob{}}
                \PYG{k}{if} \PYG{p}{(}\PYG{k}{this}\PYG{p}{.}\PYG{n+na}{ocupada}\PYG{p}{)} \PYG{k}{return} \PYG{k+kc}{false}\PYG{p}{;}
                \PYG{c+cm}{/* Si llegamos aquí, la puerta estaba libre}
\PYG{c+cm}{                pero la pondremos a ocupada un tiempo}
\PYG{c+cm}{                y luego la volveremos a poner a \PYGZdq{}libre\PYGZdq{}*/}
                \PYG{k}{this}\PYG{p}{.}\PYG{n+na}{ocupada}\PYG{o}{=}\PYG{k+kc}{true}\PYG{p}{;}
                \PYG{k}{return} \PYG{k+kc}{true}\PYG{p}{;}
        \PYG{p}{\PYGZcb{}}
\PYG{p}{\PYGZcb{}}
\end{sphinxVerbatim}


\subsection{Clase lanzadora (GrandesAlmacenes)}
\label{\detokenize{textos/tema2:clase-lanzadora-grandesalmacenes}}
\begin{sphinxVerbatim}[commandchars=\\\{\}]
\PYG{k+kn}{package} \PYG{n+nn}{grandesalmacenes}\PYG{p}{;}
\PYG{k+kd}{public} \PYG{k+kd}{class} \PYG{n+nc}{GrandesAlmacenes} \PYG{p}{\PYGZob{}}
        \PYG{k+kd}{public} \PYG{k+kd}{static} \PYG{k+kt}{void} \PYG{n+nf}{main}\PYG{p}{(}\PYG{n}{String}\PYG{o}{[}\PYG{o}{]} \PYG{n}{args}\PYG{p}{)} \PYG{k+kd}{throws} \PYG{n}{InterruptedException} \PYG{p}{\PYGZob{}}
                \PYG{k+kd}{final} \PYG{k+kt}{int} \PYG{n}{NUM\PYGZus{}CLIENTES}  \PYG{o}{=} \PYG{l+m+mi}{300}\PYG{p}{;}
                \PYG{k+kd}{final} \PYG{k+kt}{int} \PYG{n}{NUM\PYGZus{}PRODUCTOS} \PYG{o}{=} \PYG{l+m+mi}{100}\PYG{p}{;}

                \PYG{n}{Cliente}\PYG{o}{[}\PYG{o}{]}   \PYG{n}{cliente} \PYG{o}{=}   \PYG{k}{new} \PYG{n}{Cliente}\PYG{o}{[}\PYG{n}{NUM\PYGZus{}CLIENTES}\PYG{o}{]}\PYG{p}{;}
                \PYG{n}{Almacen}     \PYG{n}{almacen} \PYG{o}{=}   \PYG{k}{new} \PYG{n}{Almacen}\PYG{p}{(}\PYG{n}{NUM\PYGZus{}PRODUCTOS}\PYG{p}{)}\PYG{p}{;}
                \PYG{n}{Puerta}      \PYG{n}{puerta}  \PYG{o}{=}   \PYG{k}{new} \PYG{n}{Puerta}\PYG{p}{(}\PYG{p}{)}\PYG{p}{;}

                \PYG{n}{Thread}\PYG{o}{[}\PYG{o}{]}    \PYG{n}{hilosAsociados}\PYG{o}{=}\PYG{k}{new} \PYG{n}{Thread}\PYG{o}{[}\PYG{n}{NUM\PYGZus{}CLIENTES}\PYG{o}{]}\PYG{p}{;}

                \PYG{k}{for} \PYG{p}{(}\PYG{k+kt}{int} \PYG{n}{i}\PYG{o}{=}\PYG{l+m+mi}{0}\PYG{p}{;} \PYG{n}{i}\PYG{o}{\PYGZlt{}}\PYG{n}{NUM\PYGZus{}CLIENTES}\PYG{p}{;} \PYG{n}{i}\PYG{o}{+}\PYG{o}{+}\PYG{p}{)}\PYG{p}{\PYGZob{}}
                        \PYG{n}{String} \PYG{n}{nombreHilo}   \PYG{o}{=} \PYG{l+s}{\PYGZdq{}}\PYG{l+s}{Cliente }\PYG{l+s}{\PYGZdq{}}\PYG{o}{+}\PYG{n}{i}\PYG{p}{;}
                        \PYG{n}{cliente}\PYG{o}{[}\PYG{n}{i}\PYG{o}{]}          \PYG{o}{=} \PYG{k}{new} \PYG{n}{Cliente}\PYG{p}{(}\PYG{n}{puerta}\PYG{p}{,} \PYG{n}{almacen}\PYG{p}{,}
                                                                                                \PYG{n}{nombreHilo}\PYG{p}{)}\PYG{p}{;}
                        \PYG{n}{hilosAsociados}\PYG{o}{[}\PYG{n}{i}\PYG{o}{]}   \PYG{o}{=} \PYG{k}{new} \PYG{n}{Thread}\PYG{p}{(}\PYG{n}{cliente}\PYG{o}{[}\PYG{n}{i}\PYG{o}{]}\PYG{p}{)}\PYG{p}{;}
                        \PYG{c+c1}{//Intentamos arrancar el hilo}
                        \PYG{n}{hilosAsociados}\PYG{o}{[}\PYG{n}{i}\PYG{o}{]}\PYG{p}{.}\PYG{n+na}{start}\PYG{p}{(}\PYG{p}{)}\PYG{p}{;}
                \PYG{p}{\PYGZcb{}} \PYG{c+c1}{//Fin del for}

                \PYG{c+c1}{//Una vez arrancados esperamos a que todos terminen}
                \PYG{k}{for} \PYG{p}{(}\PYG{k+kt}{int} \PYG{n}{i}\PYG{o}{=}\PYG{l+m+mi}{0}\PYG{p}{;} \PYG{n}{i}\PYG{o}{\PYGZlt{}}\PYG{n}{NUM\PYGZus{}CLIENTES}\PYG{p}{;} \PYG{n}{i}\PYG{o}{+}\PYG{o}{+}\PYG{p}{)}\PYG{p}{\PYGZob{}}
                        \PYG{n}{hilosAsociados}\PYG{o}{[}\PYG{n}{i}\PYG{o}{]}\PYG{p}{.}\PYG{n+na}{join}\PYG{p}{(}\PYG{p}{)}\PYG{p}{;}
                \PYG{p}{\PYGZcb{}} \PYG{c+c1}{//Fin del for}
        \PYG{p}{\PYGZcb{}}
\PYG{p}{\PYGZcb{}}
\end{sphinxVerbatim}


\section{Simulación bancaria}
\label{\detokenize{textos/tema2:simulacion-bancaria}}
Un banco necesita controlar el acceso a cuentas bancarias y para ello desea hacer un programa de prueba en Java que permita lanzar procesos que ingresen y retiren dinero a la vez y comprobar así si el resultado final es el esperado.

Se parte de una cuenta con 100 euros y se pueden tener procesos que ingresen 100 euros, 50 o 20. También se pueden tener procesos que retiran 100, 50 o 20 euros euros. Se desean tener los siguientes procesos:
\begin{itemize}
\item {} 
40 procesos que ingresan 100

\item {} 
20 procesos que ingresan 50

\item {} 
60 que ingresen 20.

\end{itemize}

De la misma manera se desean lo siguientes procesos que retiran cantidades.
\begin{itemize}
\item {} 
40 procesos que retiran 100

\item {} 
20 procesos que retiran  50

\item {} 
60 que retiran 20.

\end{itemize}

Se desea comprobar que tras la ejecución la cuenta tiene exactamente 100 euros, que era la cantidad de la que se disponía al principio. Realizar el programa Java que demuestra dicho hecho.

En primer lugar la clase Cuenta podría ser así:

\begin{sphinxVerbatim}[commandchars=\\\{\}]
\PYG{k+kd}{public} \PYG{k+kd}{class} \PYG{n+nc}{Cuenta} \PYG{p}{\PYGZob{}}

    \PYG{k+kd}{private} \PYG{k+kt}{int} \PYG{n}{saldo}\PYG{p}{;}
    \PYG{k+kd}{private} \PYG{k+kt}{int} \PYG{n}{saldoInicial}\PYG{p}{;}

    \PYG{k+kd}{public} \PYG{n+nf}{Cuenta}\PYG{p}{(}\PYG{k+kt}{int} \PYG{n}{saldo}\PYG{p}{)}\PYG{p}{\PYGZob{}}
        \PYG{k}{this}\PYG{p}{.}\PYG{n+na}{saldoInicial}\PYG{o}{=}\PYG{n}{saldo}\PYG{p}{;}
        \PYG{k}{this}\PYG{p}{.}\PYG{n+na}{saldo}\PYG{o}{=}\PYG{n}{saldo}\PYG{p}{;}
    \PYG{p}{\PYGZcb{}}
    \PYG{k+kd}{public} \PYG{k+kd}{synchronized} \PYG{k+kt}{void} \PYG{n+nf}{hacerMovimiento}\PYG{p}{(}\PYG{k+kt}{int} \PYG{n}{cantidad}\PYG{p}{)}\PYG{p}{\PYGZob{}}
        \PYG{k}{this}\PYG{p}{.}\PYG{n+na}{saldo} \PYG{o}{=} \PYG{k}{this}\PYG{p}{.}\PYG{n+na}{saldo}\PYG{o}{+}\PYG{n}{cantidad}\PYG{p}{;}
    \PYG{p}{\PYGZcb{}}
    \PYG{k+kd}{public} \PYG{k+kt}{boolean} \PYG{n+nf}{esSimulacionCorrecta}\PYG{p}{(}\PYG{p}{)}\PYG{p}{\PYGZob{}}
        \PYG{k}{if} \PYG{p}{(}\PYG{k}{this}\PYG{p}{.}\PYG{n+na}{saldo}\PYG{o}{=}\PYG{o}{=}\PYG{k}{this}\PYG{p}{.}\PYG{n+na}{saldoInicial}\PYG{p}{)} \PYG{k}{return} \PYG{k+kc}{true}\PYG{p}{;}
        \PYG{k}{return} \PYG{k+kc}{false}\PYG{p}{;}
    \PYG{p}{\PYGZcb{}}
    \PYG{k+kd}{public} \PYG{k+kt}{int} \PYG{n+nf}{getSaldo}\PYG{p}{(}\PYG{p}{)}\PYG{p}{\PYGZob{}}
        \PYG{k}{return} \PYG{k}{this}\PYG{p}{.}\PYG{n+na}{saldo}\PYG{p}{;}
    \PYG{p}{\PYGZcb{}}
\PYG{p}{\PYGZcb{}}
\end{sphinxVerbatim}

La clase HiloCliente podría ser así:

\begin{sphinxVerbatim}[commandchars=\\\{\}]
\PYG{k+kd}{public} \PYG{k+kd}{class} \PYG{n+nc}{HiloCliente} \PYG{k+kd}{implements} \PYG{n}{Runnable}\PYG{p}{\PYGZob{}}
    \PYG{n}{Cuenta} \PYG{n}{cuenta}\PYG{p}{;}
    \PYG{k+kt}{int} \PYG{n}{cantidad}\PYG{p}{;}

    \PYG{k+kd}{public} \PYG{n+nf}{HiloCliente}\PYG{p}{(}\PYG{n}{Cuenta} \PYG{n}{cuenta}\PYG{p}{,} \PYG{k+kt}{int} \PYG{n}{cantidad}\PYG{p}{)} \PYG{p}{\PYGZob{}}
        \PYG{k}{this}\PYG{p}{.}\PYG{n+na}{cuenta} \PYG{o}{=} \PYG{n}{cuenta}\PYG{p}{;}
        \PYG{k}{this}\PYG{p}{.}\PYG{n+na}{cantidad} \PYG{o}{=} \PYG{n}{cantidad}\PYG{p}{;}
    \PYG{p}{\PYGZcb{}}

    \PYG{n+nd}{@Override}
    \PYG{k+kd}{public} \PYG{k+kt}{void} \PYG{n+nf}{run}\PYG{p}{(}\PYG{p}{)} \PYG{p}{\PYGZob{}}
        \PYG{c+cm}{/* Forzamos la maquinaria: repetimos}
\PYG{c+cm}{        la operación muchísimas veces para}
\PYG{c+cm}{        intentar verificar si la simulación es}
\PYG{c+cm}{        correcta}
\PYG{c+cm}{        */}
        \PYG{k}{for} \PYG{p}{(}\PYG{k+kt}{int} \PYG{n}{i}\PYG{o}{=}\PYG{l+m+mi}{0}\PYG{p}{;} \PYG{n}{i}\PYG{o}{\PYGZlt{}}\PYG{l+m+mi}{100}\PYG{p}{;} \PYG{n}{i}\PYG{o}{+}\PYG{o}{+}\PYG{p}{)}\PYG{p}{\PYGZob{}}
            \PYG{n}{cuenta}\PYG{p}{.}\PYG{n+na}{hacerMovimiento}\PYG{p}{(}\PYG{n}{cantidad}\PYG{p}{)}\PYG{p}{;}
        \PYG{p}{\PYGZcb{}}
    \PYG{p}{\PYGZcb{}}
\PYG{p}{\PYGZcb{}}
\end{sphinxVerbatim}

La clase Lanzador podría ser así:

\begin{sphinxVerbatim}[commandchars=\\\{\}]
\PYG{k+kd}{public} \PYG{k+kd}{class} \PYG{n+nc}{Lanzador} \PYG{p}{\PYGZob{}}
    \PYG{k+kd}{public} \PYG{k+kd}{static} \PYG{k+kt}{void} \PYG{n+nf}{main}\PYG{p}{(}\PYG{n}{String}\PYG{o}{[}\PYG{o}{]} \PYG{n}{args}\PYG{p}{)} \PYG{k+kd}{throws} \PYG{n}{InterruptedException}\PYG{p}{\PYGZob{}}
        \PYG{n}{Cuenta} \PYG{n}{cuenta} \PYG{o}{=} \PYG{k}{new} \PYG{n}{Cuenta} \PYG{p}{(}\PYG{l+m+mi}{100}\PYG{p}{)}\PYG{p}{;}

        \PYG{k+kd}{final} \PYG{k+kt}{int} \PYG{n}{NUM\PYGZus{}OPS\PYGZus{}CON\PYGZus{}100} \PYG{o}{=} \PYG{l+m+mi}{40}\PYG{p}{;}
        \PYG{k+kd}{final} \PYG{k+kt}{int} \PYG{n}{NUM\PYGZus{}OPS\PYGZus{}CON\PYGZus{}50}  \PYG{o}{=} \PYG{l+m+mi}{20}\PYG{p}{;}
        \PYG{k+kd}{final} \PYG{k+kt}{int} \PYG{n}{NUM\PYGZus{}OPS\PYGZus{}CON\PYGZus{}20}  \PYG{o}{=} \PYG{l+m+mi}{60}\PYG{p}{;}

        \PYG{n}{Thread}\PYG{o}{[}\PYG{o}{]} \PYG{n}{hilosIngresan100} \PYG{o}{=} \PYG{k}{new} \PYG{n}{Thread}\PYG{o}{[}\PYG{n}{NUM\PYGZus{}OPS\PYGZus{}CON\PYGZus{}100}\PYG{o}{]}\PYG{p}{;}
        \PYG{n}{Thread}\PYG{o}{[}\PYG{o}{]} \PYG{n}{hilosRetiran100}  \PYG{o}{=} \PYG{k}{new} \PYG{n}{Thread}\PYG{o}{[}\PYG{n}{NUM\PYGZus{}OPS\PYGZus{}CON\PYGZus{}100}\PYG{o}{]}\PYG{p}{;}
        \PYG{n}{Thread}\PYG{o}{[}\PYG{o}{]} \PYG{n}{hilosIngresan50}  \PYG{o}{=} \PYG{k}{new} \PYG{n}{Thread}\PYG{o}{[}\PYG{n}{NUM\PYGZus{}OPS\PYGZus{}CON\PYGZus{}50}\PYG{o}{]}\PYG{p}{;}
        \PYG{n}{Thread}\PYG{o}{[}\PYG{o}{]} \PYG{n}{hilosRetiran50}   \PYG{o}{=} \PYG{k}{new} \PYG{n}{Thread}\PYG{o}{[}\PYG{n}{NUM\PYGZus{}OPS\PYGZus{}CON\PYGZus{}50}\PYG{o}{]}\PYG{p}{;}
        \PYG{n}{Thread}\PYG{o}{[}\PYG{o}{]} \PYG{n}{hilosIngresan20}  \PYG{o}{=} \PYG{k}{new} \PYG{n}{Thread}\PYG{o}{[}\PYG{n}{NUM\PYGZus{}OPS\PYGZus{}CON\PYGZus{}20}\PYG{o}{]}\PYG{p}{;}
        \PYG{n}{Thread}\PYG{o}{[}\PYG{o}{]} \PYG{n}{hilosRetiran20}   \PYG{o}{=} \PYG{k}{new} \PYG{n}{Thread}\PYG{o}{[}\PYG{n}{NUM\PYGZus{}OPS\PYGZus{}CON\PYGZus{}20}\PYG{o}{]}\PYG{p}{;}

        \PYG{c+cm}{/* Arrancamos todos los hilos*/}
        \PYG{k}{for} \PYG{p}{(}\PYG{k+kt}{int} \PYG{n}{i}\PYG{o}{=}\PYG{l+m+mi}{0}\PYG{p}{;} \PYG{n}{i}\PYG{o}{\PYGZlt{}}\PYG{n}{NUM\PYGZus{}OPS\PYGZus{}CON\PYGZus{}100}\PYG{p}{;}\PYG{n}{i}\PYG{o}{+}\PYG{o}{+}\PYG{p}{)}\PYG{p}{\PYGZob{}}
            \PYG{n}{HiloCliente} \PYG{n}{ingresa} \PYG{o}{=} \PYG{k}{new} \PYG{n}{HiloCliente}\PYG{p}{(}\PYG{n}{cuenta}\PYG{p}{,} \PYG{l+m+mi}{100}\PYG{p}{)}\PYG{p}{;}
            \PYG{n}{HiloCliente} \PYG{n}{retira}  \PYG{o}{=} \PYG{k}{new} \PYG{n}{HiloCliente}\PYG{p}{(}\PYG{n}{cuenta}\PYG{p}{,} \PYG{o}{\PYGZhy{}}\PYG{l+m+mi}{100}\PYG{p}{)}\PYG{p}{;}

            \PYG{n}{hilosIngresan100}\PYG{o}{[}\PYG{n}{i}\PYG{o}{]}\PYG{o}{=} \PYG{k}{new} \PYG{n}{Thread}\PYG{p}{(}\PYG{n}{ingresa}\PYG{p}{)}\PYG{p}{;}
            \PYG{n}{hilosRetiran100}\PYG{o}{[}\PYG{n}{i}\PYG{o}{]} \PYG{o}{=} \PYG{k}{new} \PYG{n}{Thread}\PYG{p}{(}\PYG{n}{retira}\PYG{p}{)}\PYG{p}{;}

            \PYG{n}{hilosIngresan100}\PYG{o}{[}\PYG{n}{i}\PYG{o}{]}\PYG{p}{.}\PYG{n+na}{start}\PYG{p}{(}\PYG{p}{)}\PYG{p}{;}
            \PYG{n}{hilosRetiran100}\PYG{o}{[}\PYG{n}{i}\PYG{o}{]}\PYG{p}{.}\PYG{n+na}{start}\PYG{p}{(}\PYG{p}{)}\PYG{p}{;}
        \PYG{p}{\PYGZcb{}}

        \PYG{k}{for} \PYG{p}{(}\PYG{k+kt}{int} \PYG{n}{i}\PYG{o}{=}\PYG{l+m+mi}{0}\PYG{p}{;} \PYG{n}{i}\PYG{o}{\PYGZlt{}}\PYG{n}{NUM\PYGZus{}OPS\PYGZus{}CON\PYGZus{}50}\PYG{p}{;}\PYG{n}{i}\PYG{o}{+}\PYG{o}{+}\PYG{p}{)}\PYG{p}{\PYGZob{}}
            \PYG{n}{HiloCliente} \PYG{n}{ingresa} \PYG{o}{=} \PYG{k}{new} \PYG{n}{HiloCliente}\PYG{p}{(}\PYG{n}{cuenta}\PYG{p}{,} \PYG{l+m+mi}{50}\PYG{p}{)}\PYG{p}{;}
            \PYG{n}{HiloCliente} \PYG{n}{retira}  \PYG{o}{=} \PYG{k}{new} \PYG{n}{HiloCliente}\PYG{p}{(}\PYG{n}{cuenta}\PYG{p}{,} \PYG{o}{\PYGZhy{}}\PYG{l+m+mi}{50}\PYG{p}{)}\PYG{p}{;}

            \PYG{n}{hilosIngresan50}\PYG{o}{[}\PYG{n}{i}\PYG{o}{]}\PYG{o}{=} \PYG{k}{new} \PYG{n}{Thread}\PYG{p}{(}\PYG{n}{ingresa}\PYG{p}{)}\PYG{p}{;}
            \PYG{n}{hilosRetiran50}\PYG{o}{[}\PYG{n}{i}\PYG{o}{]} \PYG{o}{=} \PYG{k}{new} \PYG{n}{Thread}\PYG{p}{(}\PYG{n}{retira}\PYG{p}{)}\PYG{p}{;}

            \PYG{n}{hilosIngresan50}\PYG{o}{[}\PYG{n}{i}\PYG{o}{]}\PYG{p}{.}\PYG{n+na}{start}\PYG{p}{(}\PYG{p}{)}\PYG{p}{;}
            \PYG{n}{hilosRetiran50}\PYG{o}{[}\PYG{n}{i}\PYG{o}{]}\PYG{p}{.}\PYG{n+na}{start}\PYG{p}{(}\PYG{p}{)}\PYG{p}{;}
        \PYG{p}{\PYGZcb{}}

        \PYG{k}{for} \PYG{p}{(}\PYG{k+kt}{int} \PYG{n}{i}\PYG{o}{=}\PYG{l+m+mi}{0}\PYG{p}{;} \PYG{n}{i}\PYG{o}{\PYGZlt{}}\PYG{n}{NUM\PYGZus{}OPS\PYGZus{}CON\PYGZus{}20}\PYG{p}{;}\PYG{n}{i}\PYG{o}{+}\PYG{o}{+}\PYG{p}{)}\PYG{p}{\PYGZob{}}
            \PYG{n}{HiloCliente} \PYG{n}{ingresa} \PYG{o}{=} \PYG{k}{new} \PYG{n}{HiloCliente}\PYG{p}{(}\PYG{n}{cuenta}\PYG{p}{,} \PYG{l+m+mi}{20}\PYG{p}{)}\PYG{p}{;}
            \PYG{n}{HiloCliente} \PYG{n}{retira}  \PYG{o}{=} \PYG{k}{new} \PYG{n}{HiloCliente}\PYG{p}{(}\PYG{n}{cuenta}\PYG{p}{,} \PYG{o}{\PYGZhy{}}\PYG{l+m+mi}{20}\PYG{p}{)}\PYG{p}{;}

            \PYG{n}{hilosIngresan20}\PYG{o}{[}\PYG{n}{i}\PYG{o}{]}\PYG{o}{=} \PYG{k}{new} \PYG{n}{Thread}\PYG{p}{(}\PYG{n}{ingresa}\PYG{p}{)}\PYG{p}{;}
            \PYG{n}{hilosRetiran20}\PYG{o}{[}\PYG{n}{i}\PYG{o}{]} \PYG{o}{=} \PYG{k}{new} \PYG{n}{Thread}\PYG{p}{(}\PYG{n}{retira}\PYG{p}{)}\PYG{p}{;}

            \PYG{n}{hilosIngresan20}\PYG{o}{[}\PYG{n}{i}\PYG{o}{]}\PYG{p}{.}\PYG{n+na}{start}\PYG{p}{(}\PYG{p}{)}\PYG{p}{;}
            \PYG{n}{hilosRetiran20}\PYG{o}{[}\PYG{n}{i}\PYG{o}{]}\PYG{p}{.}\PYG{n+na}{start}\PYG{p}{(}\PYG{p}{)}\PYG{p}{;}
        \PYG{p}{\PYGZcb{}}

        \PYG{c+cm}{/* En este punto todos los hilos están arrancados,}
\PYG{c+cm}{        ahora toca esperarlos */}

        \PYG{k}{for} \PYG{p}{(}\PYG{k+kt}{int} \PYG{n}{i}\PYG{o}{=}\PYG{l+m+mi}{0}\PYG{p}{;} \PYG{n}{i}\PYG{o}{\PYGZlt{}}\PYG{n}{NUM\PYGZus{}OPS\PYGZus{}CON\PYGZus{}100}\PYG{p}{;}\PYG{n}{i}\PYG{o}{+}\PYG{o}{+}\PYG{p}{)}\PYG{p}{\PYGZob{}}
            \PYG{n}{hilosIngresan100}\PYG{o}{[}\PYG{n}{i}\PYG{o}{]}\PYG{p}{.}\PYG{n+na}{join}\PYG{p}{(}\PYG{p}{)}\PYG{p}{;}
            \PYG{n}{hilosRetiran100}\PYG{o}{[}\PYG{n}{i}\PYG{o}{]}\PYG{p}{.}\PYG{n+na}{join}\PYG{p}{(}\PYG{p}{)}\PYG{p}{;}
        \PYG{p}{\PYGZcb{}}

        \PYG{k}{for} \PYG{p}{(}\PYG{k+kt}{int} \PYG{n}{i}\PYG{o}{=}\PYG{l+m+mi}{0}\PYG{p}{;} \PYG{n}{i}\PYG{o}{\PYGZlt{}}\PYG{n}{NUM\PYGZus{}OPS\PYGZus{}CON\PYGZus{}50}\PYG{p}{;}\PYG{n}{i}\PYG{o}{+}\PYG{o}{+}\PYG{p}{)}\PYG{p}{\PYGZob{}}
            \PYG{n}{hilosIngresan50}\PYG{o}{[}\PYG{n}{i}\PYG{o}{]}\PYG{p}{.}\PYG{n+na}{join}\PYG{p}{(}\PYG{p}{)}\PYG{p}{;}
            \PYG{n}{hilosRetiran50}\PYG{o}{[}\PYG{n}{i}\PYG{o}{]}\PYG{p}{.}\PYG{n+na}{join}\PYG{p}{(}\PYG{p}{)}\PYG{p}{;}
        \PYG{p}{\PYGZcb{}}

        \PYG{k}{for} \PYG{p}{(}\PYG{k+kt}{int} \PYG{n}{i}\PYG{o}{=}\PYG{l+m+mi}{0}\PYG{p}{;} \PYG{n}{i}\PYG{o}{\PYGZlt{}}\PYG{n}{NUM\PYGZus{}OPS\PYGZus{}CON\PYGZus{}20}\PYG{p}{;}\PYG{n}{i}\PYG{o}{+}\PYG{o}{+}\PYG{p}{)}\PYG{p}{\PYGZob{}}
            \PYG{n}{hilosIngresan20}\PYG{o}{[}\PYG{n}{i}\PYG{o}{]}\PYG{p}{.}\PYG{n+na}{join}\PYG{p}{(}\PYG{p}{)}\PYG{p}{;}
            \PYG{n}{hilosRetiran20}\PYG{o}{[}\PYG{n}{i}\PYG{o}{]}\PYG{p}{.}\PYG{n+na}{join}\PYG{p}{(}\PYG{p}{)}\PYG{p}{;}
        \PYG{p}{\PYGZcb{}}
        \PYG{k}{if} \PYG{p}{(}\PYG{n}{cuenta}\PYG{p}{.}\PYG{n+na}{esSimulacionCorrecta}\PYG{p}{(}\PYG{p}{)}\PYG{p}{)}\PYG{p}{\PYGZob{}}
            \PYG{n}{System}\PYG{p}{.}\PYG{n+na}{out}\PYG{p}{.}\PYG{n+na}{println}\PYG{p}{(}\PYG{l+s}{\PYGZdq{}}\PYG{l+s}{La simulación fue correcta}\PYG{l+s}{\PYGZdq{}}\PYG{p}{)}\PYG{p}{;}
        \PYG{p}{\PYGZcb{}} \PYG{k}{else} \PYG{p}{\PYGZob{}}
            \PYG{n}{System}\PYG{p}{.}\PYG{n+na}{out}\PYG{p}{.}\PYG{n+na}{println}\PYG{p}{(}\PYG{l+s}{\PYGZdq{}}\PYG{l+s}{La simulación falló }\PYG{l+s}{\PYGZdq{}}\PYG{p}{)}\PYG{p}{;}
            \PYG{n}{System}\PYG{p}{.}\PYG{n+na}{out}\PYG{p}{.}\PYG{n+na}{println}\PYG{p}{(}\PYG{l+s}{\PYGZdq{}}\PYG{l+s}{La cuenta tiene:}\PYG{l+s}{\PYGZdq{}}\PYG{o}{+}
                    \PYG{n}{cuenta}\PYG{p}{.}\PYG{n+na}{getSaldo}\PYG{p}{(}\PYG{p}{)}\PYG{p}{)}\PYG{p}{;}
            \PYG{n}{System}\PYG{p}{.}\PYG{n+na}{out}\PYG{p}{.}\PYG{n+na}{println}\PYG{p}{(}\PYG{l+s}{\PYGZdq{}}\PYG{l+s}{Revise sus synchronized}\PYG{l+s}{\PYGZdq{}}\PYG{p}{)}\PYG{p}{;}
        \PYG{p}{\PYGZcb{}}

    \PYG{p}{\PYGZcb{}}
\PYG{p}{\PYGZcb{}}
\end{sphinxVerbatim}


\section{Clases de alto nivel: la clase \sphinxstyleliteralintitle{\sphinxupquote{RecursiveTask}}}
\label{\detokenize{textos/tema2:clases-de-alto-nivel-la-clase-recursivetask}}
En Java 8 y posteriores se puede utilizar una clase llamada \sphinxcode{\sphinxupquote{RecursiveTask}} que facilita un poco la tarea de crear aplicaciones con paralelismo. La idea básica es que podemos heredar de esta clase indicando qué tipo Java tendrá el resultado de la operación. Dentro de la clase deberemos implementar un método \sphinxcode{\sphinxupquote{compute}} que básicamente deberá trabajar recursivamente decidiendo si la tarea es lo bastante pequeña para ejecutarla o si la dividimos más.

Supongamos que queremos sumar todos los números de un vector. Por comodidad supongamos también que siempre nos pasan vectores con un tamaño que es potencia de 2. Podemos usar una clase Java como esta para hacer la operación de forma paralela.

\begin{sphinxVerbatim}[commandchars=\\\{\}]
\PYG{k+kn}{package} \PYG{n+nn}{io.github.oscarmaestre.paralelismo}\PYG{p}{;}

\PYG{k+kn}{import} \PYG{n+nn}{java.util.concurrent.RecursiveTask}\PYG{p}{;}

\PYG{k+kd}{public} \PYG{k+kd}{class} \PYG{n+nc}{Sumador} \PYG{k+kd}{extends} \PYG{n}{RecursiveTask}\PYG{o}{\PYGZlt{}}\PYG{n}{Long}\PYG{o}{\PYGZgt{}}\PYG{p}{\PYGZob{}}
    \PYG{k+kt}{int}\PYG{o}{[}\PYG{o}{]} \PYG{n}{numeros}\PYG{p}{;}
    \PYG{k+kt}{int} \PYG{n}{pos1}\PYG{p}{,} \PYG{n}{pos2}\PYG{p}{;}
    \PYG{k+kd}{final} \PYG{k+kt}{int} \PYG{n}{NUM\PYGZus{}MAX\PYGZus{}ELEMENTOS}\PYG{o}{=}\PYG{l+m+mi}{2}\PYG{p}{;}
    \PYG{k+kd}{private} \PYG{k+kt}{int} \PYG{n}{longitud}\PYG{p}{;}
    \PYG{k+kd}{private} \PYG{k+kt}{int} \PYG{n}{posMitad}\PYG{p}{;}
    \PYG{k+kd}{private} \PYG{k+kt}{int} \PYG{n}{limiteIzquierdo}\PYG{p}{;}
    \PYG{k+kd}{private} \PYG{k+kt}{int} \PYG{n}{limiteDerecho}\PYG{p}{;}
    \PYG{c+cm}{/*Por simplificar solo aceptamos vectores}
\PYG{c+cm}{    con un número de elementos que sea una}
\PYG{c+cm}{    potencia de 2*/}
    \PYG{k+kd}{public} \PYG{n+nf}{Sumador}\PYG{p}{(}\PYG{k+kt}{int}\PYG{o}{[}\PYG{o}{]} \PYG{n}{numeros}\PYG{p}{,} \PYG{k+kt}{int} \PYG{n}{pos1}\PYG{p}{,} \PYG{k+kt}{int} \PYG{n}{pos2}\PYG{p}{)}  \PYG{p}{\PYGZob{}}
        \PYG{k}{this}\PYG{p}{.}\PYG{n+na}{numeros}    \PYG{o}{=} \PYG{n}{numeros}\PYG{p}{;}
        \PYG{k}{this}\PYG{p}{.}\PYG{n+na}{pos1}       \PYG{o}{=} \PYG{n}{pos1}\PYG{p}{;}
        \PYG{k}{this}\PYG{p}{.}\PYG{n+na}{pos2}       \PYG{o}{=} \PYG{n}{pos2}\PYG{p}{;}
        \PYG{n}{longitud}        \PYG{o}{=} \PYG{l+m+mi}{1} \PYG{o}{+} \PYG{p}{(}\PYG{n}{pos2}\PYG{o}{\PYGZhy{}} \PYG{n}{pos1}\PYG{p}{)}\PYG{p}{;}
        \PYG{n}{posMitad}        \PYG{o}{=} \PYG{n}{longitud}\PYG{o}{/}\PYG{l+m+mi}{2}\PYG{p}{;}
        \PYG{n}{limiteIzquierdo} \PYG{o}{=} \PYG{n}{pos1}\PYG{o}{+}\PYG{p}{(}\PYG{n}{posMitad}\PYG{o}{\PYGZhy{}}\PYG{l+m+mi}{1}\PYG{p}{)}\PYG{p}{;}
        \PYG{n}{limiteDerecho}   \PYG{o}{=} \PYG{n}{limiteIzquierdo}\PYG{o}{+}\PYG{l+m+mi}{1}\PYG{p}{;}
    \PYG{p}{\PYGZcb{}}

    \PYG{n+nd}{@Override}
    \PYG{k+kd}{protected} \PYG{n}{Long} \PYG{n+nf}{compute}\PYG{p}{(}\PYG{p}{)} \PYG{p}{\PYGZob{}}
        \PYG{c+cm}{/* Si el vector tiene estos elementos,}
\PYG{c+cm}{        simplemente sumamos*/}
        \PYG{k}{if} \PYG{p}{(}\PYG{n}{longitud}\PYG{o}{=}\PYG{o}{=}\PYG{n}{NUM\PYGZus{}MAX\PYGZus{}ELEMENTOS}\PYG{p}{)}\PYG{p}{\PYGZob{}}
            \PYG{k+kt}{long} \PYG{n}{suma}\PYG{o}{=}\PYG{l+m+mi}{0}\PYG{p}{;}
            \PYG{k}{for} \PYG{p}{(}\PYG{k+kt}{int} \PYG{n}{i}\PYG{o}{=}\PYG{n}{pos1}\PYG{p}{;} \PYG{n}{i}\PYG{o}{\PYGZlt{}}\PYG{o}{=}\PYG{n}{pos2}\PYG{p}{;} \PYG{n}{i}\PYG{o}{+}\PYG{o}{+}\PYG{p}{)}\PYG{p}{\PYGZob{}}
                \PYG{n}{suma}\PYG{o}{=}\PYG{n}{suma} \PYG{o}{+} \PYG{n}{numeros}\PYG{o}{[}\PYG{n}{i}\PYG{o}{]}\PYG{p}{;}
            \PYG{p}{\PYGZcb{}}
            \PYG{k}{return} \PYG{n}{suma}\PYG{p}{;}
        \PYG{p}{\PYGZcb{}}

        \PYG{c+cm}{/*Pero si tiene más elementos,}
\PYG{c+cm}{        dividimos el vector en dos y sumamos}
\PYG{c+cm}{        las dos mitades de forma independiente}
\PYG{c+cm}{        y paralela*/}
        \PYG{n}{Sumador} \PYG{n}{sumadorIzq}\PYG{o}{=}
                \PYG{k}{new} \PYG{n}{Sumador}\PYG{p}{(}\PYG{n}{numeros}\PYG{p}{,} \PYG{n}{pos1}\PYG{p}{,} \PYG{n}{limiteIzquierdo}\PYG{p}{)}\PYG{p}{;}
        \PYG{n}{Sumador} \PYG{n}{sumadorDer}\PYG{o}{=}
                \PYG{k}{new} \PYG{n}{Sumador}\PYG{p}{(}\PYG{n}{numeros}\PYG{p}{,} \PYG{n}{limiteDerecho}\PYG{p}{,} \PYG{n}{pos2}\PYG{p}{)}\PYG{p}{;}

        \PYG{c+cm}{/*Lanzamos los sumadores...*/}
        \PYG{n}{sumadorIzq}\PYG{p}{.}\PYG{n+na}{fork}\PYG{p}{(}\PYG{p}{)}\PYG{p}{;}
        \PYG{n}{sumadorDer}\PYG{p}{.}\PYG{n+na}{fork}\PYG{p}{(}\PYG{p}{)}\PYG{p}{;}

        \PYG{c+cm}{/* Y recogemos sus resultados*/}
        \PYG{k+kt}{long} \PYG{n}{sumaIzq}\PYG{o}{=}\PYG{n}{sumadorIzq}\PYG{p}{.}\PYG{n+na}{join}\PYG{p}{(}\PYG{p}{)}\PYG{p}{;}
        \PYG{k+kt}{long} \PYG{n}{sumaDer}\PYG{o}{=}\PYG{n}{sumadorDer}\PYG{p}{.}\PYG{n+na}{join}\PYG{p}{(}\PYG{p}{)}\PYG{p}{;}

        \PYG{k}{return} \PYG{n}{sumaIzq} \PYG{o}{+} \PYG{n}{sumaDer}\PYG{p}{;}
    \PYG{p}{\PYGZcb{}} \PYG{c+cm}{/*Fin del compute*/}
\PYG{p}{\PYGZcb{}} \PYG{c+cm}{/*Fin de la clase*/}
\end{sphinxVerbatim}

Con esta clase como «lanzador de hilos», nuestro trabajo se simplifica bastante. Ahora podemos tener un \sphinxcode{\sphinxupquote{main}} que va a ser tan sencillo como el código siguiente:

\begin{sphinxVerbatim}[commandchars=\\\{\}]
\PYG{k+kn}{package} \PYG{n+nn}{io.github.oscarmaestre.paralelismo}\PYG{p}{;}

\PYG{k+kn}{import} \PYG{n+nn}{java.util.concurrent.ExecutionException}\PYG{p}{;}
\PYG{k+kn}{import} \PYG{n+nn}{java.util.concurrent.ForkJoinPool}\PYG{p}{;}

\PYG{k+kd}{public} \PYG{k+kd}{class} \PYG{n+nc}{SumadorParalelo} \PYG{p}{\PYGZob{}}

    \PYG{k+kd}{public} \PYG{k+kd}{static} \PYG{k+kt}{void} \PYG{n+nf}{main}\PYG{p}{(}\PYG{n}{String}\PYG{o}{[}\PYG{o}{]} \PYG{n}{args}\PYG{p}{)} \PYG{k+kd}{throws} \PYG{n}{InterruptedException}\PYG{p}{,} \PYG{n}{ExecutionException} \PYG{p}{\PYGZob{}}

        \PYG{c+cm}{/* Construimos un vector de ejemplo...*/}
        \PYG{k+kd}{final} \PYG{k+kt}{int} \PYG{n}{MAX\PYGZus{}NUMEROS}\PYG{o}{=}\PYG{l+m+mi}{1024}\PYG{p}{;}
        \PYG{k+kt}{int}\PYG{o}{[}\PYG{o}{]} \PYG{n}{numeros}\PYG{o}{=}\PYG{k}{new} \PYG{k+kt}{int}\PYG{o}{[}\PYG{n}{MAX\PYGZus{}NUMEROS}\PYG{o}{]}\PYG{p}{;}

        \PYG{c+cm}{/*Y lo rellenamos con números*/}
        \PYG{k}{for} \PYG{p}{(}\PYG{k+kt}{int} \PYG{n}{i}\PYG{o}{=}\PYG{l+m+mi}{0}\PYG{p}{;} \PYG{n}{i}\PYG{o}{\PYGZlt{}}\PYG{n}{MAX\PYGZus{}NUMEROS}\PYG{p}{;} \PYG{n}{i}\PYG{o}{+}\PYG{o}{+}\PYG{p}{)}\PYG{p}{\PYGZob{}}
            \PYG{n}{numeros}\PYG{o}{[}\PYG{n}{i}\PYG{o}{]}\PYG{o}{=}\PYG{n}{i}\PYG{p}{;}
        \PYG{p}{\PYGZcb{}}

        \PYG{c+cm}{/* Esta clase gestionará el paralelismo de las tareas*/}
        \PYG{n}{ForkJoinPool} \PYG{n}{pool}\PYG{o}{=}\PYG{k}{new} \PYG{n}{ForkJoinPool}\PYG{p}{(}\PYG{p}{)}\PYG{p}{;}

        \PYG{c+cm}{/* Fabricamos un sumador inicial que intente sumar todo*/}
        \PYG{n}{Sumador} \PYG{n}{sumador}\PYG{o}{=}\PYG{k}{new} \PYG{n}{Sumador}\PYG{p}{(}\PYG{n}{numeros}\PYG{p}{,} \PYG{l+m+mi}{0}\PYG{p}{,} \PYG{n}{MAX\PYGZus{}NUMEROS}\PYG{o}{\PYGZhy{}}\PYG{l+m+mi}{1}\PYG{p}{)}\PYG{p}{;}

        \PYG{c+cm}{/*Y la clase ForkJoinPool invocará a nuestro sumador lanzando}
\PYG{c+cm}{        los hilos, recogiendo los resultados y haciendo todo lo necesario}
\PYG{c+cm}{        para que al final solo tengamos que recoger el resultado*/}
        \PYG{n}{pool}\PYG{p}{.}\PYG{n+na}{invoke}\PYG{p}{(}\PYG{n}{sumador}\PYG{p}{)}\PYG{p}{;}

        \PYG{c+cm}{/* Resultado que podemos ver aquí*/}
        \PYG{n}{Long} \PYG{n}{resultado} \PYG{o}{=} \PYG{n}{sumador}\PYG{p}{.}\PYG{n+na}{get}\PYG{p}{(}\PYG{p}{)}\PYG{p}{;}
        \PYG{n}{System}\PYG{p}{.}\PYG{n+na}{out}\PYG{p}{.}\PYG{n+na}{println}\PYG{p}{(}\PYG{l+s}{\PYGZdq{}}\PYG{l+s}{La suma es:}\PYG{l+s}{\PYGZdq{}}\PYG{o}{+}\PYG{n}{resultado}\PYG{p}{)}\PYG{p}{;}
    \PYG{p}{\PYGZcb{}} \PYG{c+cm}{/*Fin del main*/}
\PYG{p}{\PYGZcb{}} \PYG{c+cm}{/*Fin de la clase*/}
\end{sphinxVerbatim}


\section{Documentación.}
\label{\detokenize{textos/tema2:documentacion}}
La estructura general de toda aplicación multihilo es algo similar a lo siguiente:
\begin{itemize}
\item {} 
Habrá tareas que se puedan paralelizar. Dichas tareas heredarán de \sphinxcode{\sphinxupquote{Runnable}}.

\item {} 
Habrá procesos que pueden ser accedidos por muchos hilos. Estos objetos \sphinxstylestrong{deberán usar synchronized}

\item {} 
Habrá que crear los recursos compartidos y crear los hilos. Esta operación se debe hacer en este orden.

\end{itemize}


\section{Depuración.}
\label{\detokenize{textos/tema2:depuracion}}
La depuración de programas multihilo debe hacerse exclusivamente trabajando con pruebas de unidad y las comprobaciones de tipos de Java.
\begin{itemize}
\item {} 
Por un lado, los depuradores solo funcionan vinculándose con el hilo principal («main») de la aplicación, por lo que si se desea depurar un hilo concreto se tendrá que averiguar el identificador de dicho hilo para poder conectar con él desde el depurador.

\item {} 
Los depuradores «interfieren» con la ejecución de los hilos así que pueden pasar cosas tan curiosas como que un programa \sphinxstyleemphasis{funcione cuando usamos el depurador pero vuelva a fallar al ejecutarlo solo}.

\item {} 
Los mensajes de traza pueden aparecer antes o despues de lo esperado, así que en realidad no sabemos cuando se escribió un mensaje en pantalla ni qué operaciones se efectuaron antes o despues.

\end{itemize}


\section{Ejercicio: empleado «Thread safe»}
\label{\detokenize{textos/tema2:ejercicio-empleado-thread-safe}}
Se dice que un método o clase es «Thread safe» (a prueba de hilos) cuando se ha programado con cuidado para que cualquier persona que lo utilice pueda hacerlo desde múltiples hilos de ejecución sin miedo a que se produzcan resultados indeseados. Evidentemente, la programación de código «Thread\sphinxhyphen{}safe» involucra muchísimo más cuidado de lo habitual. En el enunciado siguiente se pide desarrollar una clase que claramente debe ser «Thread safe».

Crear una clase multihilo llamada Empleado que permita cambiar la cantidad de horas trabajadas por parte del empleado y su sueldo permitiendo que haya muchos hilos a la vez que puedan cambiar ambos valores pero de manera que la ejecución dé siempre resultados coherentes.

En concreto, se desea tener un método \sphinxcode{\sphinxupquote{incrementarHoras(int numHoras)}} que acepte números positivos y negativos y que permita actualizar el número de horas trabajadas. También se desea un método \sphinxcode{\sphinxupquote{incrementarBonus (int bonus)}} que acepte positivos y negativos y que permita incrementar el bonus salarial del trabajador.

Si por ejemplo lanzamos 10 hilos con 20 ejecuciones cada uno y todos ellos llaman a \sphinxcode{\sphinxupquote{incrementarHoras(2)}} la ejecución de estos hilos debería reflejar que el empleado ha trabajado 400 horas este mes. De la misma manera, si hay 10 hilos que ejecuta cada uno 10 veces el método \sphinxcode{\sphinxupquote{incrementarBonus(10)}} el bonus salarial del empleado debería ser de 1000 euros.

\begin{sphinxVerbatim}[commandchars=\\\{\}]
\PYG{k+kd}{public} \PYG{k+kd}{class} \PYG{n+nc}{Empleado} \PYG{p}{\PYGZob{}}
        \PYG{k+kt}{int} \PYG{n}{bonus}\PYG{o}{=}\PYG{l+m+mi}{0}\PYG{p}{;}
        \PYG{k+kt}{int} \PYG{n}{horasTrabajadas}\PYG{o}{=}\PYG{l+m+mi}{0}\PYG{p}{;}
        \PYG{k+kd}{public} \PYG{k+kd}{synchronized} \PYG{k+kt}{void} \PYG{n+nf}{incrementarHoras}\PYG{p}{(}\PYG{k+kt}{int} \PYG{n}{numHoras}\PYG{p}{)}\PYG{p}{\PYGZob{}}
                \PYG{n}{horasTrabajadas}\PYG{o}{+}\PYG{o}{=}\PYG{n}{numHoras}\PYG{p}{;}
        \PYG{p}{\PYGZcb{}}
        \PYG{k+kd}{public} \PYG{k+kd}{synchronized} \PYG{k+kt}{void} \PYG{n+nf}{incrementarBonus}\PYG{p}{(}\PYG{k+kt}{int} \PYG{n}{bonusSalarial}\PYG{p}{)}\PYG{p}{\PYGZob{}}
                \PYG{n}{bonus}\PYG{o}{+}\PYG{o}{=}\PYG{n}{bonusSalarial}\PYG{p}{;}
        \PYG{p}{\PYGZcb{}}
\PYG{p}{\PYGZcb{}}
\end{sphinxVerbatim}


\chapter{Programación de comunicaciones en red}
\label{\detokenize{textos/tema3:programacion-de-comunicaciones-en-red}}\label{\detokenize{textos/tema3::doc}}

\section{Comunicación entre aplicaciones.}
\label{\detokenize{textos/tema3:comunicacion-entre-aplicaciones}}
En Java toda la comunicación vista en primer curso de DAM consiste en dos cosas
\begin{itemize}
\item {} 
Entrada/salida por consola: con las clases \sphinxcode{\sphinxupquote{System.in}} o \sphinxcode{\sphinxupquote{System.out}}.

\item {} 
Lectura/escritura en ficheros: con las clases File y similares.

\end{itemize}

Se puede avanzar un paso más utilizando Java para enviar datos a través de Internet a otro programa Java remoto, que es lo que haremos en este capítulo.


\section{Roles cliente y servidor.}
\label{\detokenize{textos/tema3:roles-cliente-y-servidor}}
Cuando se hacen programas Java que se comuniquen lo habitual es que uno o varios actúen de cliente y uno o varios actúen de servidores.
\begin{itemize}
\item {} 
Servidor: espera peticiones, recibe datos de entrada y devuelve respuestas.

\item {} 
Cliente: genera peticiones, las envía a un servidor y espera respuestas.

\end{itemize}

Un factor fundamental en los servidores es que tienen que ser capaces de procesar varias peticiones a la vez: \sphinxstylestrong{deben ser multihilo}.

Su arquitectura típica es la siguiente:

\begin{sphinxVerbatim}[commandchars=\\\{\}]
\PYG{k}{while} \PYG{p}{(}\PYG{k+kc}{true}\PYG{p}{)}\PYG{p}{\PYGZob{}}
        \PYG{n}{peticion}\PYG{o}{=}\PYG{n}{esperarPeticion}\PYG{p}{(}\PYG{p}{)}\PYG{p}{;}
        \PYG{n}{hiloAsociado}\PYG{o}{=}\PYG{k}{new} \PYG{n}{Hilo}\PYG{p}{(}\PYG{p}{)}\PYG{p}{;}
        \PYG{n}{hiloAsociado}\PYG{p}{.}\PYG{n+na}{atender}\PYG{p}{(}\PYG{n}{peticion}\PYG{p}{)}\PYG{p}{;}
\PYG{p}{\PYGZcb{}}
\end{sphinxVerbatim}


\section{Recordatorio de los flujos en Java}
\label{\detokenize{textos/tema3:recordatorio-de-los-flujos-en-java}}

\subsection{InputStreams y OutputStreams}
\label{\detokenize{textos/tema3:inputstreams-y-outputstreams}}
Manejan bytes a secas. Por ejemplo, si queremos leer un fichero byte a byte usaremos \sphinxcode{\sphinxupquote{FileInputStream}} y si queremos escribir usaremos \sphinxcode{\sphinxupquote{FileOutputStream}}.

Son operaciones a muy bajo nivel que usaremos muy pocas veces (por ejemplo, solo si quisiéramos cambiar el primer byte de un archivo). En general usaremos otras clases más cómodas de usar.


\subsection{Readers y Writers}
\label{\detokenize{textos/tema3:readers-y-writers}}
En lugar de manejar \sphinxstyleemphasis{bytes} manejan \sphinxstyleemphasis{caracteres} (recordemos que hoy en día y con Unicode una letra como la \sphinxstyleemphasis{ñ} en realidad podría ocupar más de un byte).

Así, cuando queramos leer letras de un archivo usaremos clases como \sphinxcode{\sphinxupquote{FileReader}} y \sphinxcode{\sphinxupquote{FileWriter}}.

Las clases \sphinxcode{\sphinxupquote{Readers}} y \sphinxcode{\sphinxupquote{Writers}} en realidad se apoyan sobre las \sphinxcode{\sphinxupquote{InputStreams}} y \sphinxcode{\sphinxupquote{OutputStreams}}.

A veces nos interesará mezclar conceptos y por ejemplo poder tener una clase que use caracteres cuando a lo mejor Java nos ha dado una clase que usa bytes. Así, por ejemplo \sphinxcode{\sphinxupquote{InputStreamReader}} puede coger un objeto que lea bytes y nos devolverá caracteres. De la misma forma \sphinxcode{\sphinxupquote{OutputStreamWriter}} coge letras y devuelve los bytes que la componen.


\subsection{\sphinxstyleliteralintitle{\sphinxupquote{BufferedReaders}} y \sphinxstyleliteralintitle{\sphinxupquote{PrintWriters}}}
\label{\detokenize{textos/tema3:bufferedreaders-y-printwriters}}
Cuando trabajamos con caracteres (que recordemos pueden tener varios bytes) normalmente no trabajamos de uno en uno. Es más frecuente usar \sphinxstylestrong{líneas} que se leen y escriben de una sola vez. Así por ejemplo, la clase \sphinxcode{\sphinxupquote{PrintWriter}} tiene un método \sphinxcode{\sphinxupquote{print(ln)}} que puede imprimir elementos complejos como \sphinxcode{\sphinxupquote{floats}} o cadenas largas.

Además, Java ofrece clases que gestionan automáticamente los \sphinxstyleemphasis{buffers} por nosotros lo que nos da más comodidad y eficiencia. Por ello es muy habitual hacer cosas como esta:

\begin{sphinxVerbatim}[commandchars=\\\{\}]
\PYG{n}{lectorEficiente} \PYG{o}{=} \PYG{k}{new}
        \PYG{n}{BufferedReader}\PYG{p}{(}\PYG{k}{new} \PYG{n}{FileReader}\PYG{p}{(}\PYG{l+s}{\PYGZdq{}}\PYG{l+s}{fich1.txt}\PYG{l+s}{\PYGZdq{}}\PYG{p}{)}\PYG{p}{)}\PYG{p}{;}
\PYG{n}{escritorEficiente} \PYG{o}{=} \PYG{k}{new}
        \PYG{n}{BufferedWriter}\PYG{p}{(}\PYG{k}{new} \PYG{n}{FileWriter}\PYG{p}{(}\PYG{l+s}{\PYGZdq{}}\PYG{l+s}{fich2.txt}\PYG{l+s}{\PYGZdq{}}\PYG{p}{)}\PYG{p}{)}\PYG{p}{;}
\end{sphinxVerbatim}

En el primer caso creamos un objeto \sphinxcode{\sphinxupquote{FileReader}} que es capaz de leer caracteres de \sphinxcode{\sphinxupquote{fich1.txt}}. Como esto nos parece poco práctico creamos otro objeto a partir del primero de tipo \sphinxcode{\sphinxupquote{BufferedReader}} que nos permitirá leer bloques enteros de texto.

De hecho, si se comprueba la ayuda de la clase \sphinxcode{\sphinxupquote{FileReader}} se verá que solo hay un método \sphinxcode{\sphinxupquote{read}} que devuelve un \sphinxcode{\sphinxupquote{int}}, es decir el siguiente \sphinxstylestrong{carácter} disponible, lo que hace que el método sea muy incómodo. Sin embargo \sphinxcode{\sphinxupquote{BufferedReader}} nos resuelve esta incomodidad permitiéndonos trabajar con líneas.


\section{Elementos de programación de aplicaciones en red. Librerías.}
\label{\detokenize{textos/tema3:elementos-de-programacion-de-aplicaciones-en-red-librerias}}
En Java toda la infraestructura de clases para trabajar con redes está en el paquete \sphinxcode{\sphinxupquote{java.net}}.

En muchos casos nuestros programas empezarán con la sentencia \sphinxcode{\sphinxupquote{import java.net.*}} pero muchos entornos (como Eclipse) son capaces de importar automáticamente las clases necesarias.


\subsection{La clase URL}
\label{\detokenize{textos/tema3:la-clase-url}}
La clase URL permite gestionar accesos a URLs del tipo \sphinxcode{\sphinxupquote{http://marca.com/fichero.html}} y descargar cosas con bastante sencillez.

Al crear un objeto URL se debe capturar la excepción \sphinxcode{\sphinxupquote{MalformedURLException}} que sucede cuando hay algún error en la URL, como por ejemplo escribir \sphinxcode{\sphinxupquote{htp://marca.com}} en lugar de \sphinxcode{\sphinxupquote{http://marca.com}} (obsérvese que el primero tiene un sola t en http en lugar de dos).

La clase URL nos ofrece un método \sphinxcode{\sphinxupquote{openStream}} que nos devuelve un flujo básico de bytes. Podemos crear objetos más sofisticados para leer bloques como muestra el programa siguiente:

\begin{sphinxVerbatim}[commandchars=\\\{\}]
\PYG{k+kd}{public} \PYG{k+kd}{class} \PYG{n+nc}{GestorDescargas} \PYG{p}{\PYGZob{}}
        \PYG{k+kd}{public} \PYG{k+kt}{void} \PYG{n+nf}{descargarArchivo}\PYG{p}{(}
                        \PYG{n}{String} \PYG{n}{url\PYGZus{}descargar}\PYG{p}{,}
                        \PYG{n}{String} \PYG{n}{nombreArchivo}\PYG{p}{)}\PYG{p}{\PYGZob{}}
                \PYG{n}{System}\PYG{p}{.}\PYG{n+na}{out}\PYG{p}{.}\PYG{n+na}{println}\PYG{p}{(}\PYG{l+s}{\PYGZdq{}}\PYG{l+s}{Descargando }\PYG{l+s}{\PYGZdq{}}
                                \PYG{o}{+}\PYG{n}{url\PYGZus{}descargar}\PYG{p}{)}\PYG{p}{;}
                \PYG{k}{try} \PYG{p}{\PYGZob{}}
                        \PYG{n}{URL} \PYG{n}{laUrl}\PYG{o}{=}\PYG{k}{new} \PYG{n}{URL}\PYG{p}{(}\PYG{n}{url\PYGZus{}descargar}\PYG{p}{)}\PYG{p}{;}
                        \PYG{n}{InputStream} \PYG{n}{is}\PYG{o}{=}\PYG{n}{laUrl}\PYG{p}{.}\PYG{n+na}{openStream}\PYG{p}{(}\PYG{p}{)}\PYG{p}{;}
                        \PYG{n}{InputStreamReader} \PYG{n}{reader}\PYG{o}{=}
                                        \PYG{k}{new} \PYG{n}{InputStreamReader}\PYG{p}{(}\PYG{n}{is}\PYG{p}{)}\PYG{p}{;}
                        \PYG{n}{BufferedReader} \PYG{n}{bReader}\PYG{o}{=}
                                        \PYG{k}{new} \PYG{n}{BufferedReader}\PYG{p}{(}\PYG{n}{reader}\PYG{p}{)}\PYG{p}{;}
                        \PYG{n}{FileWriter} \PYG{n}{escritorFichero}\PYG{o}{=}
                                \PYG{k}{new} \PYG{n}{FileWriter}\PYG{p}{(}\PYG{n}{nombreArchivo}\PYG{p}{)}\PYG{p}{;}
                        \PYG{n}{String} \PYG{n}{linea}\PYG{p}{;}
                        \PYG{k}{while} \PYG{p}{(}\PYG{p}{(}\PYG{n}{linea}\PYG{o}{=}\PYG{n}{bReader}\PYG{p}{.}\PYG{n+na}{readLine}\PYG{p}{(}\PYG{p}{)}\PYG{p}{)}\PYG{o}{!}\PYG{o}{=}\PYG{k+kc}{null}\PYG{p}{)}\PYG{p}{\PYGZob{}}
                                \PYG{n}{escritorFichero}\PYG{p}{.}\PYG{n+na}{write}\PYG{p}{(}\PYG{n}{linea}\PYG{p}{)}\PYG{p}{;}
                        \PYG{p}{\PYGZcb{}}
                        \PYG{n}{escritorFichero}\PYG{p}{.}\PYG{n+na}{close}\PYG{p}{(}\PYG{p}{)}\PYG{p}{;}
                        \PYG{n}{bReader}\PYG{p}{.}\PYG{n+na}{close}\PYG{p}{(}\PYG{p}{)}\PYG{p}{;}
                        \PYG{n}{reader}\PYG{p}{.}\PYG{n+na}{close}\PYG{p}{(}\PYG{p}{)}\PYG{p}{;}
                        \PYG{n}{is}\PYG{p}{.}\PYG{n+na}{close}\PYG{p}{(}\PYG{p}{)}\PYG{p}{;}
                \PYG{p}{\PYGZcb{}} \PYG{k}{catch} \PYG{p}{(}\PYG{n}{MalformedURLException} \PYG{n}{e}\PYG{p}{)} \PYG{p}{\PYGZob{}}
                        \PYG{n}{System}\PYG{p}{.}\PYG{n+na}{out}\PYG{p}{.}\PYG{n+na}{println}\PYG{p}{(}\PYG{l+s}{\PYGZdq{}}\PYG{l+s}{URL mal escrita!}\PYG{l+s}{\PYGZdq{}}\PYG{p}{)}\PYG{p}{;}
                        \PYG{k}{return} \PYG{p}{;}
                \PYG{p}{\PYGZcb{}} \PYG{k}{catch} \PYG{p}{(}\PYG{n}{IOException} \PYG{n}{e}\PYG{p}{)} \PYG{p}{\PYGZob{}}
                        \PYG{n}{System}\PYG{p}{.}\PYG{n+na}{out}\PYG{p}{.}\PYG{n+na}{println}\PYG{p}{(}
                                \PYG{l+s}{\PYGZdq{}}\PYG{l+s}{Fallo en la lectura del fichero}\PYG{l+s}{\PYGZdq{}}\PYG{p}{)}\PYG{p}{;}
                        \PYG{k}{return} \PYG{p}{;}
                \PYG{p}{\PYGZcb{}}
        \PYG{p}{\PYGZcb{}}
        \PYG{k+kd}{public} \PYG{k+kd}{static} \PYG{k+kt}{void} \PYG{n+nf}{main} \PYG{p}{(}\PYG{n}{String}\PYG{o}{[}\PYG{o}{]} \PYG{n}{argumentos}\PYG{p}{)}\PYG{p}{\PYGZob{}}
                \PYG{n}{GestorDescargas} \PYG{n}{gd}\PYG{o}{=}\PYG{k}{new} \PYG{n}{GestorDescargas}\PYG{p}{(}\PYG{p}{)}\PYG{p}{;}
                \PYG{n}{String} \PYG{n}{base}\PYG{o}{=}
                        \PYG{l+s}{\PYGZdq{}}\PYG{l+s}{http://10.13.0.20:8000}\PYG{l+s}{\PYGZdq{}}\PYG{o}{+}
                                        \PYG{l+s}{\PYGZdq{}}\PYG{l+s}{/ServiciosProcesos/textos/}\PYG{l+s}{\PYGZdq{}}\PYG{p}{;}
                \PYG{k}{for} \PYG{p}{(}\PYG{k+kt}{int} \PYG{n}{i}\PYG{o}{=}\PYG{l+m+mi}{1}\PYG{p}{;} \PYG{n}{i}\PYG{o}{\PYGZlt{}}\PYG{o}{=}\PYG{l+m+mi}{5}\PYG{p}{;} \PYG{n}{i}\PYG{o}{+}\PYG{o}{+}\PYG{p}{)}\PYG{p}{\PYGZob{}}
                        \PYG{n}{String} \PYG{n}{url}\PYG{o}{=}\PYG{n}{base}\PYG{o}{+}\PYG{l+s}{\PYGZdq{}}\PYG{l+s}{tema}\PYG{l+s}{\PYGZdq{}}\PYG{o}{+}\PYG{n}{i}\PYG{o}{+}\PYG{l+s}{\PYGZdq{}}\PYG{l+s}{.rst}\PYG{l+s}{\PYGZdq{}}\PYG{p}{;}
                        \PYG{n}{gd}\PYG{p}{.}\PYG{n+na}{descargarArchivo}\PYG{p}{(}\PYG{n}{url}\PYG{p}{)}\PYG{p}{;}
                \PYG{p}{\PYGZcb{}}
        \PYG{p}{\PYGZcb{}}
\PYG{p}{\PYGZcb{}}
\end{sphinxVerbatim}


\section{Repaso de redes}
\label{\detokenize{textos/tema3:repaso-de-redes}}
En redes el protocolo IP es el responsable de dos cuestiones fundamentales:
\begin{itemize}
\item {} 
Establecer un sistema de direcciones universal (direcciones IP)

\item {} 
Establecer los mecanismos de enrutado.

\end{itemize}

Como programadores el segundo no nos interesa, pero el primero será absolutamente fundamental para contactar con programas que estén en una ubicación remota.

Una ubicación remota \sphinxstyleemphasis{siempre} tendrá una dirección
IP pero \sphinxstyleemphasis{solo a veces tendrá un nombre DNS}. Para nosotros no habrá diferencia ya que si es necesario el sistema operativo traducirá de nombre DNS a IP.

Otro elemento necesario en la comunicación en redes es el uso de un puerto de un cierto protocolo:
\begin{itemize}
\item {} 
TCP: ofrece fiabilidad a los programas.

\item {} 
UDP: ofrece velocidad sacrificando la fiabilidad.

\end{itemize}

A partir de ahora cuando usemos un número de puerto habrá que comprobar si ese número ya está usado.

Por ejemplo, es mala idea que nuestros servidores usen el puerto 80 TCP para aceptar peticiones, probablemente ya esté en uso.
Antes de usar un puerto en una aplicación comercial deberíamos consultar la lista de «IANA assigned ports».

En líneas generales se pueden usar los puertos desde 1024 TCP a 49151 TCP, pero deberíamos comprobar que el número que elegimos no sea un número usado por un puerto de alguna aplicación que haya en la empresa.

En las prácticas de clase usaremos el 9876 TCP. Si se desea conectar desde el instituto con algún programa ejecutado en casa se deberá «abrir el puerto 9876 TCP». Abrir un puerto consiste en configurar el router para que \sphinxstylestrong{SÍ ACEPTE TRÁFICO INICIADO DESDE EL EXTERIOR} cosa que no hace nunca por motivos de protección.


\section{Funciones y objetos de las librerías.}
\label{\detokenize{textos/tema3:funciones-y-objetos-de-las-librerias}}
La clase URL proporciona un mecanismo muy sencillo pero por desgracia completamente atado al protocolo de las URL.

Java ofrece otros objetos que permiten tener un mayor control sobre lo que se envía o recibe a través de la red. Por desgracia esto implica que en muchos casos tendremos solo flujos de bajo nivel (streams).

En concreto Java ofrece dos elementos fundamentales para crear programas que usen redes
\begin{itemize}
\item {} 
Sockets

\item {} 
ServerSockets

\end{itemize}


\section{Sockets.}
\label{\detokenize{textos/tema3:sockets}}
Un \sphinxstyleemphasis{socket} es un objeto Java que nos permite contactar con un programa o servidor remoto. Dicho objeto nos proporcionará flujos de entrada y/o salida y podremos comunicarnos con dicho programa.

Existe otro tipo de sockets, los \sphinxstyleemphasis{ServerSocket}. Se utilizan para crear programas que acepten conexiones o peticiones.

Todos los objetos mencionados en este tema están en el paquete \sphinxcode{\sphinxupquote{java.net}}.


\section{Creación de sockets.}
\label{\detokenize{textos/tema3:creacion-de-sockets}}
En el siguiente código puede verse el proceso básico de creación de un socket. En los párrafos siguientes explicaremos el significado de los bloques de código.:

\begin{sphinxVerbatim}[commandchars=\\\{\}]
\PYG{k+kd}{public} \PYG{k+kd}{class} \PYG{n+nc}{Conector} \PYG{p}{\PYGZob{}}
        \PYG{k+kd}{public} \PYG{k+kd}{static} \PYG{k+kt}{void} \PYG{n+nf}{main}\PYG{p}{(}\PYG{n}{String}\PYG{o}{[}\PYG{o}{]} \PYG{n}{args}\PYG{p}{)} \PYG{p}{\PYGZob{}}
                \PYG{n}{String} \PYG{n}{destino}\PYG{o}{=}\PYG{l+s}{\PYGZdq{}}\PYG{l+s}{www.google.com}\PYG{l+s}{\PYGZdq{}}\PYG{p}{;}
                \PYG{k+kt}{int} \PYG{n}{puertoDestino}\PYG{o}{=}\PYG{l+m+mi}{80}\PYG{p}{;}
                \PYG{n}{Socket} \PYG{n}{socket}\PYG{o}{=}\PYG{k}{new} \PYG{n}{Socket}\PYG{p}{(}\PYG{p}{)}\PYG{p}{;}
                \PYG{n}{InetSocketAddress} \PYG{n}{direccion}\PYG{o}{=}\PYG{k}{new} \PYG{n}{InetSocketAddress}\PYG{p}{(}
                                \PYG{n}{destino}\PYG{p}{,} \PYG{n}{puertoDestino}\PYG{p}{)}\PYG{p}{;}
                \PYG{k}{try} \PYG{p}{\PYGZob{}}
                        \PYG{n}{socket}\PYG{p}{.}\PYG{n+na}{connect}\PYG{p}{(}\PYG{n}{direccion}\PYG{p}{)}\PYG{p}{;}
                        \PYG{c+c1}{//Si llegamos aquí es que la conexión}
                        \PYG{c+c1}{//sí se hizo.}

                        \PYG{n}{InputStream} \PYG{n}{is}\PYG{o}{=}\PYG{n}{socket}\PYG{p}{.}\PYG{n+na}{getInputStream}\PYG{p}{(}\PYG{p}{)}\PYG{p}{;}
                        \PYG{n}{OutputStream} \PYG{n}{os}\PYG{o}{=}\PYG{n}{socket}\PYG{p}{.}\PYG{n+na}{getOutputStream}\PYG{p}{(}\PYG{p}{)}\PYG{p}{;}


                \PYG{p}{\PYGZcb{}} \PYG{k}{catch} \PYG{p}{(}\PYG{n}{IOException} \PYG{n}{e}\PYG{p}{)} \PYG{p}{\PYGZob{}}
                        \PYG{n}{System}\PYG{p}{.}\PYG{n+na}{out}\PYG{p}{.}\PYG{n+na}{println}\PYG{p}{(}
                                \PYG{l+s}{\PYGZdq{}}\PYG{l+s}{No se pudo establecer la conexion }\PYG{l+s}{\PYGZdq{}}\PYG{o}{+}
                                \PYG{l+s}{\PYGZdq{}}\PYG{l+s}{ o hubo un fallo al leer datos.}\PYG{l+s}{\PYGZdq{}}
                        \PYG{p}{)}\PYG{p}{;}
                \PYG{p}{\PYGZcb{}}
        \PYG{p}{\PYGZcb{}}
\PYG{p}{\PYGZcb{}}
\end{sphinxVerbatim}

Para poder crear un socket primero necesitamos una dirección con la que contactar. Toda dirección está formada por dirección IP (o DNS) y un puerto. En nuestro caso intentaremos contactar con \sphinxcode{\sphinxupquote{www.google.com:80}}.

\begin{sphinxVerbatim}[commandchars=\\\{\}]
\PYG{n}{String} \PYG{n}{destino}\PYG{o}{=}\PYG{l+s}{\PYGZdq{}}\PYG{l+s}{www.google.com}\PYG{l+s}{\PYGZdq{}}\PYG{p}{;}
\PYG{k+kt}{int} \PYG{n}{puertoDestino}\PYG{o}{=}\PYG{l+m+mi}{80}\PYG{p}{;}
\PYG{n}{Socket} \PYG{n}{socket}\PYG{o}{=}\PYG{k}{new} \PYG{n}{Socket}\PYG{p}{(}\PYG{p}{)}\PYG{p}{;}
\PYG{n}{InetSocketAddress} \PYG{n}{direccion}\PYG{o}{=}\PYG{k}{new}
        \PYG{n}{InetSocketAddress}\PYG{p}{(}
                        \PYG{n}{destino}\PYG{p}{,} \PYG{n}{puertoDestino}\PYG{p}{)}\PYG{p}{;}
\end{sphinxVerbatim}


\section{Enlazado y establecimiento de conexiones.}
\label{\detokenize{textos/tema3:enlazado-y-establecimiento-de-conexiones}}
El paso crítico para iniciar la comunicación es llamar al método \sphinxcode{\sphinxupquote{connect}}. Este método puede disparar una excepción del tipo \sphinxcode{\sphinxupquote{IOException}} que puede significar dos cosas:
\begin{itemize}
\item {} 
La conexión no se pudo establecer.

\item {} 
Aunque la conexión se estableció no fue posible leer o escribir datos.

\end{itemize}

Así, la conexión debería realizarse así:

\begin{sphinxVerbatim}[commandchars=\\\{\}]
\PYG{k}{try} \PYG{p}{\PYGZob{}}
        \PYG{n}{socket}\PYG{p}{.}\PYG{n+na}{connect}\PYG{p}{(}\PYG{n}{direccion}\PYG{p}{)}\PYG{p}{;}
        \PYG{c+c1}{//Si llegamos aquí es que la conexión}
        \PYG{c+c1}{//sí se hizo.}

        \PYG{n}{InputStream} \PYG{n}{is}\PYG{o}{=}\PYG{n}{socket}\PYG{p}{.}\PYG{n+na}{getInputStream}\PYG{p}{(}\PYG{p}{)}\PYG{p}{;}
        \PYG{n}{OutputStream} \PYG{n}{os}\PYG{o}{=}\PYG{n}{socket}\PYG{p}{.}\PYG{n+na}{getOutputStream}\PYG{p}{(}\PYG{p}{)}\PYG{p}{;}

\PYG{p}{\PYGZcb{}}  \PYG{c+c1}{//Fin del try}
\PYG{k}{catch} \PYG{p}{(}\PYG{n}{IOException} \PYG{n}{e}\PYG{p}{)} \PYG{p}{\PYGZob{}}
        \PYG{n}{System}\PYG{p}{.}\PYG{n+na}{out}\PYG{p}{.}\PYG{n+na}{println}\PYG{p}{(}
                \PYG{l+s}{\PYGZdq{}}\PYG{l+s}{No se pudo establecer la conexion }\PYG{l+s}{\PYGZdq{}}\PYG{o}{+}
                \PYG{l+s}{\PYGZdq{}}\PYG{l+s}{ o hubo un fallo al leer datos.}\PYG{l+s}{\PYGZdq{}}
        \PYG{p}{)}\PYG{p}{;}
\PYG{p}{\PYGZcb{}} \PYG{c+c1}{//Fin del catch IOException}
\end{sphinxVerbatim}


\section{Utilización de sockets para la transmisión y recepción de información.}
\label{\detokenize{textos/tema3:utilizacion-de-sockets-para-la-transmision-y-recepcion-de-informacion}}
La clase \sphinxcode{\sphinxupquote{Socket}} tiene dos métodos llamados \sphinxcode{\sphinxupquote{getInputStream}} y \sphinxcode{\sphinxupquote{getOutputSream}} que nos permiten obtener \sphinxstyleemphasis{flujos orientados a bytes}. Recordemos que es posible crear nuestros propios flujos, con más métodos que ofrecen más comodidad.


\subsection{El ejemplo completo}
\label{\detokenize{textos/tema3:el-ejemplo-completo}}
Podemos contactar con un programa cualquiera escrito en cualquier lenguaje y enviar las peticiones de acuerdo a un protocolo. Nuestro programa podrá leer las respuestas independientemente de como fuera el servidor.

\begin{sphinxVerbatim}[commandchars=\\\{\}]
\PYG{k+kd}{public} \PYG{k+kd}{class} \PYG{n+nc}{Conector} \PYG{p}{\PYGZob{}}
        \PYG{k+kd}{public} \PYG{k+kd}{static} \PYG{k+kt}{void} \PYG{n+nf}{main}\PYG{p}{(}\PYG{n}{String}\PYG{o}{[}\PYG{o}{]} \PYG{n}{args}\PYG{p}{)} \PYG{p}{\PYGZob{}}
                \PYG{n}{System}\PYG{p}{.}\PYG{n+na}{out}\PYG{p}{.}\PYG{n+na}{println}\PYG{p}{(}\PYG{l+s}{\PYGZdq{}}\PYG{l+s}{Iniciando...}\PYG{l+s}{\PYGZdq{}}\PYG{p}{)}\PYG{p}{;}
                \PYG{n}{String} \PYG{n}{destino}\PYG{o}{=}\PYG{l+s}{\PYGZdq{}}\PYG{l+s}{10.8.0.253}\PYG{l+s}{\PYGZdq{}}\PYG{p}{;}
                \PYG{k+kt}{int} \PYG{n}{puertoDestino}\PYG{o}{=}\PYG{l+m+mi}{80}\PYG{p}{;}
                \PYG{n}{Socket} \PYG{n}{socket}\PYG{o}{=}\PYG{k}{new} \PYG{n}{Socket}\PYG{p}{(}\PYG{p}{)}\PYG{p}{;}
                \PYG{n}{InetSocketAddress} \PYG{n}{direccion}\PYG{o}{=}\PYG{k}{new} \PYG{n}{InetSocketAddress}\PYG{p}{(}
                                \PYG{n}{destino}\PYG{p}{,} \PYG{n}{puertoDestino}\PYG{p}{)}\PYG{p}{;}
                \PYG{k}{try} \PYG{p}{\PYGZob{}}
                        \PYG{n}{socket}\PYG{p}{.}\PYG{n+na}{connect}\PYG{p}{(}\PYG{n}{direccion}\PYG{p}{)}\PYG{p}{;}
                        \PYG{c+c1}{//Si llegamos aquí es que la conexión}
                        \PYG{c+c1}{//sí se hizo.}

                        \PYG{n}{InputStream} \PYG{n}{is}\PYG{o}{=}\PYG{n}{socket}\PYG{p}{.}\PYG{n+na}{getInputStream}\PYG{p}{(}\PYG{p}{)}\PYG{p}{;}
                        \PYG{n}{OutputStream} \PYG{n}{os}\PYG{o}{=}\PYG{n}{socket}\PYG{p}{.}\PYG{n+na}{getOutputStream}\PYG{p}{(}\PYG{p}{)}\PYG{p}{;}

                        \PYG{c+c1}{//Flujos que manejan caracteres}
                        \PYG{n}{InputStreamReader} \PYG{n}{isr}\PYG{o}{=}
                                        \PYG{k}{new} \PYG{n}{InputStreamReader}\PYG{p}{(}\PYG{n}{is}\PYG{p}{)}\PYG{p}{;}
                        \PYG{n}{OutputStreamWriter} \PYG{n}{osw}\PYG{o}{=}
                                        \PYG{k}{new} \PYG{n}{OutputStreamWriter}\PYG{p}{(}\PYG{n}{os}\PYG{p}{)}\PYG{p}{;}

                        \PYG{c+c1}{//Flujos de líneas}
                        \PYG{n}{BufferedReader} \PYG{n}{bReader}\PYG{o}{=}
                                        \PYG{k}{new} \PYG{n}{BufferedReader}\PYG{p}{(}\PYG{n}{isr}\PYG{p}{)}\PYG{p}{;}
                        \PYG{n}{PrintWriter} \PYG{n}{pWriter}\PYG{o}{=}
                                        \PYG{k}{new} \PYG{n}{PrintWriter}\PYG{p}{(}\PYG{n}{osw}\PYG{p}{)}\PYG{p}{;}


                        \PYG{n}{pWriter}\PYG{p}{.}\PYG{n+na}{println}\PYG{p}{(}\PYG{l+s}{\PYGZdq{}}\PYG{l+s}{GET /index.html}\PYG{l+s}{\PYGZdq{}}\PYG{p}{)}\PYG{p}{;}
                        \PYG{n}{pWriter}\PYG{p}{.}\PYG{n+na}{flush}\PYG{p}{(}\PYG{p}{)}\PYG{p}{;}
                        \PYG{n}{String} \PYG{n}{linea}\PYG{p}{;}
                        \PYG{n}{FileWriter} \PYG{n}{escritorArchivo}\PYG{o}{=}
                                        \PYG{k}{new} \PYG{n}{FileWriter}\PYG{p}{(}\PYG{l+s}{\PYGZdq{}}\PYG{l+s}{resultado.txt}\PYG{l+s}{\PYGZdq{}}\PYG{p}{)}\PYG{p}{;}
                        \PYG{k}{while} \PYG{p}{(}\PYG{p}{(}\PYG{n}{linea}\PYG{o}{=}\PYG{n}{bReader}\PYG{p}{.}\PYG{n+na}{readLine}\PYG{p}{(}\PYG{p}{)}\PYG{p}{)} \PYG{o}{!}\PYG{o}{=} \PYG{k+kc}{null} \PYG{p}{)}\PYG{p}{\PYGZob{}}
                                \PYG{n}{escritorArchivo}\PYG{p}{.}\PYG{n+na}{write}\PYG{p}{(}\PYG{n}{linea}\PYG{p}{)}\PYG{p}{;}
                        \PYG{p}{\PYGZcb{}}
                        \PYG{n}{escritorArchivo}\PYG{p}{.}\PYG{n+na}{close}\PYG{p}{(}\PYG{p}{)}\PYG{p}{;}
                        \PYG{n}{pWriter}\PYG{p}{.}\PYG{n+na}{close}\PYG{p}{(}\PYG{p}{)}\PYG{p}{;}
                        \PYG{n}{bReader}\PYG{p}{.}\PYG{n+na}{close}\PYG{p}{(}\PYG{p}{)}\PYG{p}{;}
                        \PYG{n}{isr}\PYG{p}{.}\PYG{n+na}{close}\PYG{p}{(}\PYG{p}{)}\PYG{p}{;}
                        \PYG{n}{osw}\PYG{p}{.}\PYG{n+na}{close}\PYG{p}{(}\PYG{p}{)}\PYG{p}{;}
                        \PYG{n}{is}\PYG{p}{.}\PYG{n+na}{close}\PYG{p}{(}\PYG{p}{)}\PYG{p}{;}
                        \PYG{n}{os}\PYG{p}{.}\PYG{n+na}{close}\PYG{p}{(}\PYG{p}{)}\PYG{p}{;}
                \PYG{p}{\PYGZcb{}} \PYG{k}{catch} \PYG{p}{(}\PYG{n}{IOException} \PYG{n}{e}\PYG{p}{)} \PYG{p}{\PYGZob{}}
                        \PYG{n}{System}\PYG{p}{.}\PYG{n+na}{out}\PYG{p}{.}\PYG{n+na}{println}\PYG{p}{(}
                                \PYG{l+s}{\PYGZdq{}}\PYG{l+s}{No se pudo establecer la conexion }\PYG{l+s}{\PYGZdq{}}\PYG{o}{+}
                                \PYG{l+s}{\PYGZdq{}}\PYG{l+s}{ o hubo un fallo al leer datos.}\PYG{l+s}{\PYGZdq{}}
                        \PYG{p}{)}\PYG{p}{;}
                \PYG{p}{\PYGZcb{}} \PYG{c+c1}{//Fin del catch}
        \PYG{p}{\PYGZcb{}} \PYG{c+c1}{//Fin del main}
\PYG{p}{\PYGZcb{}} \PYG{c+c1}{//Fin de la clase Conector}
\end{sphinxVerbatim}


\section{Programación de aplicaciones cliente y servidor.}
\label{\detokenize{textos/tema3:programacion-de-aplicaciones-cliente-y-servidor}}
Al crear aplicaciones cliente y servidor puede ocurrir que tengamos que implementar varias operaciones:
\begin{itemize}
\item {} 
Si tenemos que programar el servidor \sphinxstylestrong{deberemos definir un protocolo} de acceso a ese servidor.

\item {} 
Si tenemos que programar solo el cliente \sphinxstylestrong{necesitaremos conocer el protocolo de acceso} a ese servidor.

\item {} 
Si tenemos que programar los dos tendremos que empezar por \sphinxstylestrong{definir el protocolo de comunicación entre ambos}.

\end{itemize}

En el ejemplo siguiente puede verse un ejemplo para Python 3 que implementa un servidor de cálculo. El servidor tiene un protocolo muy rígido (demasiado) que consiste en lo siguiente:
\begin{enumerate}
\sphinxsetlistlabels{\arabic}{enumi}{enumii}{}{.}%
\item {} 
El servidor espera que primero envíen la operación que puede ser \sphinxcode{\sphinxupquote{+}} o \sphinxcode{\sphinxupquote{\sphinxhyphen{}}}. La operación debe terminar con un fin de línea UNIX (\sphinxcode{\sphinxupquote{\textbackslash{}n}})

\item {} 
Despues acepta un número de dos cifras (ni una ni tres) terminado en un fín de línea UNIX.

\item {} 
Despues acepta un segundo número de dos cifras terminado en un fin de línea UNIX.

\end{enumerate}

\begin{sphinxVerbatim}[commandchars=\\\{\}]
\PYG{k+kn}{import} \PYG{n+nn}{socketserver}
\PYG{n}{TAM\PYGZus{}MAXIMO\PYGZus{}PARAMETROS}\PYG{o}{=}\PYG{l+m+mi}{64}
\PYG{n}{PUERTO}\PYG{o}{=}\PYG{l+m+mi}{9876}
\PYG{k}{class} \PYG{n+nc}{GestorConexion}\PYG{p}{(}
        \PYG{n}{socketserver}\PYG{o}{.}\PYG{n}{BaseRequestHandler}\PYG{p}{)}\PYG{p}{:}

        \PYG{k}{def} \PYG{n+nf}{leer\PYGZus{}cadena}\PYG{p}{(}\PYG{n+nb+bp}{self}\PYG{p}{,} \PYG{n}{LONGITUD}\PYG{p}{)}\PYG{p}{:}
                \PYG{n}{cadena}\PYG{o}{=}\PYG{n+nb+bp}{self}\PYG{o}{.}\PYG{n}{request}\PYG{o}{.}\PYG{n}{recv}\PYG{p}{(}\PYG{n}{LONGITUD}\PYG{p}{)}
                \PYG{k}{return} \PYG{n}{cadena}\PYG{o}{.}\PYG{n}{strip}\PYG{p}{(}\PYG{p}{)}

        \PYG{k}{def} \PYG{n+nf}{convertir\PYGZus{}a\PYGZus{}cadena}\PYG{p}{(}\PYG{n+nb+bp}{self}\PYG{p}{,} \PYG{n+nb}{bytes}\PYG{p}{)}\PYG{p}{:}
                \PYG{k}{return} \PYG{n+nb}{bytes}\PYG{o}{.}\PYG{n}{decode}\PYG{p}{(}\PYG{l+s+s2}{\PYGZdq{}}\PYG{l+s+s2}{utf\PYGZhy{}8}\PYG{l+s+s2}{\PYGZdq{}}\PYG{p}{)}

        \PYG{k}{def} \PYG{n+nf}{calcular\PYGZus{}resultado}\PYG{p}{(}
                \PYG{n+nb+bp}{self}\PYG{p}{,} \PYG{n}{n1}\PYG{p}{,} \PYG{n}{op}\PYG{p}{,} \PYG{n}{n2}\PYG{p}{)}\PYG{p}{:}
                \PYG{n}{n1}\PYG{o}{=}\PYG{n+nb}{int}\PYG{p}{(}\PYG{n}{n1}\PYG{p}{)}
                \PYG{n}{n2}\PYG{o}{=}\PYG{n+nb}{int}\PYG{p}{(}\PYG{n}{n2}\PYG{p}{)}

                \PYG{n}{op}\PYG{o}{=}\PYG{n+nb+bp}{self}\PYG{o}{.}\PYG{n}{convertir\PYGZus{}a\PYGZus{}cadena}\PYG{p}{(}\PYG{n}{op}\PYG{p}{)}
                \PYG{k}{if} \PYG{p}{(}\PYG{n}{op}\PYG{o}{==}\PYG{l+s+s2}{\PYGZdq{}}\PYG{l+s+s2}{+}\PYG{l+s+s2}{\PYGZdq{}}\PYG{p}{)}\PYG{p}{:}
                        \PYG{k}{return} \PYG{n}{n1}\PYG{o}{+}\PYG{n}{n2}
                \PYG{k}{if} \PYG{p}{(}\PYG{n}{op}\PYG{o}{==}\PYG{l+s+s2}{\PYGZdq{}}\PYG{l+s+s2}{\PYGZhy{}}\PYG{l+s+s2}{\PYGZdq{}}\PYG{p}{)}\PYG{p}{:}
                        \PYG{k}{return} \PYG{n}{n1}\PYG{o}{\PYGZhy{}}\PYG{n}{n2}
                \PYG{k}{return} \PYG{l+m+mi}{0}
        \PYG{l+s+sd}{\PYGZdq{}\PYGZdq{}\PYGZdq{}Controlador de evento \PYGZsq{}NuevaConexion\PYGZdq{}\PYGZdq{}\PYGZdq{}}
        \PYG{k}{def} \PYG{n+nf}{handle}\PYG{p}{(}\PYG{n+nb+bp}{self}\PYG{p}{)}\PYG{p}{:}
                \PYG{n}{direccion}\PYG{o}{=}\PYG{n+nb+bp}{self}\PYG{o}{.}\PYG{n}{client\PYGZus{}address}\PYG{p}{[}\PYG{l+m+mi}{0}\PYG{p}{]}
                \PYG{n}{operacion}   \PYG{o}{=}   \PYG{n+nb+bp}{self}\PYG{o}{.}\PYG{n}{leer\PYGZus{}cadena}\PYG{p}{(}\PYG{l+m+mi}{2}\PYG{p}{)}
                \PYG{n}{num1}        \PYG{o}{=}   \PYG{n+nb+bp}{self}\PYG{o}{.}\PYG{n}{leer\PYGZus{}cadena}\PYG{p}{(}\PYG{l+m+mi}{3}\PYG{p}{)}
                \PYG{n}{num2}        \PYG{o}{=}   \PYG{n+nb+bp}{self}\PYG{o}{.}\PYG{n}{leer\PYGZus{}cadena}\PYG{p}{(}\PYG{l+m+mi}{3}\PYG{p}{)}
                \PYG{n+nb}{print} \PYG{p}{(}\PYG{n}{direccion}\PYG{o}{+}\PYG{l+s+s2}{\PYGZdq{}}\PYG{l+s+s2}{ pregunta:}\PYG{l+s+s2}{\PYGZdq{}}\PYG{o}{+}\PYG{n+nb}{str}\PYG{p}{(}\PYG{n}{num1}\PYG{p}{)}\PYG{o}{+}\PYG{l+s+s2}{\PYGZdq{}}\PYG{l+s+s2}{ }\PYG{l+s+s2}{\PYGZdq{}}\PYG{o}{+}\PYG{n+nb}{str}\PYG{p}{(}\PYG{n}{operacion}\PYG{p}{)}\PYG{o}{+}\PYG{l+s+s2}{\PYGZdq{}}\PYG{l+s+s2}{ }\PYG{l+s+s2}{\PYGZdq{}}\PYG{o}{+}\PYG{n+nb}{str}\PYG{p}{(}\PYG{n}{num2}\PYG{p}{)}\PYG{p}{)}

                \PYG{n}{resultado}\PYG{o}{=}\PYG{n+nb+bp}{self}\PYG{o}{.}\PYG{n}{calcular\PYGZus{}resultado}\PYG{p}{(}\PYG{n}{num1}\PYG{p}{,} \PYG{n}{operacion}\PYG{p}{,} \PYG{n}{num2}\PYG{p}{)}
                \PYG{n+nb}{print} \PYG{p}{(}\PYG{l+s+s2}{\PYGZdq{}}\PYG{l+s+s2}{Devolviendo a }\PYG{l+s+s2}{\PYGZdq{}} \PYG{o}{+} \PYG{n}{direccion}\PYG{o}{+}\PYG{l+s+s2}{\PYGZdq{}}\PYG{l+s+s2}{ el resultado }\PYG{l+s+s2}{\PYGZdq{}}\PYG{o}{+}\PYG{n+nb}{str}\PYG{p}{(}\PYG{n}{resultado}\PYG{p}{)}\PYG{p}{)}
                \PYG{n}{bytes\PYGZus{}resultado}\PYG{o}{=}\PYG{n+nb}{bytearray}\PYG{p}{(}\PYG{n+nb}{str}\PYG{p}{(}\PYG{n}{resultado}\PYG{p}{)}\PYG{p}{,} \PYG{l+s+s2}{\PYGZdq{}}\PYG{l+s+s2}{utf\PYGZhy{}8}\PYG{l+s+s2}{\PYGZdq{}}\PYG{p}{)}\PYG{p}{;}
                \PYG{n+nb+bp}{self}\PYG{o}{.}\PYG{n}{request}\PYG{o}{.}\PYG{n}{send}\PYG{p}{(}\PYG{n}{bytes\PYGZus{}resultado}\PYG{p}{)}
\PYG{n}{servidor}\PYG{o}{=}\PYG{n}{socketserver}\PYG{o}{.}\PYG{n}{TCPServer}\PYG{p}{(}\PYG{p}{(}\PYG{l+s+s2}{\PYGZdq{}}\PYG{l+s+s2}{10.13.0.20}\PYG{l+s+s2}{\PYGZdq{}}\PYG{p}{,} \PYG{l+m+mi}{9876}\PYG{p}{)}\PYG{p}{,} \PYG{n}{GestorConexion}\PYG{p}{)}
\PYG{n+nb}{print} \PYG{p}{(}\PYG{l+s+s2}{\PYGZdq{}}\PYG{l+s+s2}{Servidor en marcha.}\PYG{l+s+s2}{\PYGZdq{}}\PYG{p}{)}
\PYG{n}{servidor}\PYG{o}{.}\PYG{n}{serve\PYGZus{}forever}\PYG{p}{(}\PYG{p}{)}
\end{sphinxVerbatim}

La comunicación Java con el servidor sería algo así:

\begin{sphinxVerbatim}[commandchars=\\\{\}]
\PYG{k+kt}{byte}\PYG{o}{[}\PYG{o}{]} \PYG{n}{bSuma}\PYG{o}{=}\PYG{l+s}{\PYGZdq{}}\PYG{l+s}{+}\PYG{l+s}{\PYGZbs{}}\PYG{l+s}{n}\PYG{l+s}{\PYGZdq{}}\PYG{p}{.}\PYG{n+na}{getBytes}\PYG{p}{(}\PYG{p}{)}\PYG{p}{;}
\PYG{k+kt}{byte}\PYG{o}{[}\PYG{o}{]} \PYG{n}{bOp1}\PYG{o}{=}\PYG{l+s}{\PYGZdq{}}\PYG{l+s}{42}\PYG{l+s}{\PYGZbs{}}\PYG{l+s}{n}\PYG{l+s}{\PYGZdq{}}\PYG{p}{.}\PYG{n+na}{getBytes}\PYG{p}{(}\PYG{p}{)}\PYG{p}{;}
\PYG{k+kt}{byte}\PYG{o}{[}\PYG{o}{]} \PYG{n}{bOp2}\PYG{o}{=}\PYG{l+s}{\PYGZdq{}}\PYG{l+s}{34}\PYG{l+s}{\PYGZbs{}}\PYG{l+s}{n}\PYG{l+s}{\PYGZdq{}}\PYG{p}{.}\PYG{n+na}{getBytes}\PYG{p}{(}\PYG{p}{)}\PYG{p}{;}

\PYG{n}{os}\PYG{p}{.}\PYG{n+na}{write}\PYG{p}{(}\PYG{n}{bSuma}\PYG{p}{)}\PYG{p}{;}
\PYG{n}{os}\PYG{p}{.}\PYG{n+na}{write}\PYG{p}{(}\PYG{n}{bOp1}\PYG{p}{)}\PYG{p}{;}
\PYG{n}{os}\PYG{p}{.}\PYG{n+na}{write}\PYG{p}{(}\PYG{n}{bOp2}\PYG{p}{)}\PYG{p}{;}
\PYG{n}{os}\PYG{p}{.}\PYG{n+na}{flush}\PYG{p}{(}\PYG{p}{)}\PYG{p}{;}

\PYG{n}{InputStreamReader} \PYG{n}{isr}\PYG{o}{=}
        \PYG{k}{new} \PYG{n}{InputStreamReader}\PYG{p}{(}\PYG{n}{is}\PYG{p}{)}\PYG{p}{;}
\PYG{n}{BufferedReader} \PYG{n}{br}\PYG{o}{=}
        \PYG{k}{new} \PYG{n}{BufferedReader}\PYG{p}{(}\PYG{n}{isr}\PYG{p}{)}\PYG{p}{;}
\PYG{n}{String} \PYG{n}{cadenaRecibida}\PYG{o}{=}\PYG{n}{br}\PYG{p}{.}\PYG{n+na}{readLine}\PYG{p}{(}\PYG{p}{)}\PYG{p}{;}
\PYG{n}{System}\PYG{p}{.}\PYG{n+na}{out}\PYG{p}{.}\PYG{n+na}{println}\PYG{p}{(}\PYG{l+s}{\PYGZdq{}}\PYG{l+s}{Recibido:}\PYG{l+s}{\PYGZdq{}}\PYG{o}{+}
                \PYG{n}{cadenaRecibida}\PYG{p}{)}\PYG{p}{;}

\PYG{n}{is}\PYG{p}{.}\PYG{n+na}{close}\PYG{p}{(}\PYG{p}{)}\PYG{p}{;}
\PYG{n}{os}\PYG{p}{.}\PYG{n+na}{close}\PYG{p}{(}\PYG{p}{)}\PYG{p}{;}
\PYG{n}{socket}\PYG{p}{.}\PYG{n+na}{close}\PYG{p}{(}\PYG{p}{)}\PYG{p}{;}
\end{sphinxVerbatim}


\subsection{Ejemplo de servidor Java}
\label{\detokenize{textos/tema3:ejemplo-de-servidor-java}}
Supongamos que se nos pide crear un servidor de operaciones de cálculo que sea menos estricto que el anterior:
\begin{itemize}
\item {} 
Cualquier parámetro que envíe el usuario debe ir terminado en un fin de línea UNIX (\sphinxcode{\sphinxupquote{\textbackslash{}n}}).

\item {} 
El usuario enviará primero un símbolo «+», «\sphinxhyphen{}«, «*» o «/».

\item {} 
Despues se puede enviar un positivo de 1 a 8 cifras. El usuario podría equivocarse y enviar en vez de «3762» algo como «37a62». En ese caso se asume que el parámetro es 0.

\item {} 
Despues se envía un segundo positivo de 1 a 8 cifras igual que el anterior.

\item {} 
Cuando el servidor haya recogido todos los parámetros contestará al cliente con un positivo de 1 a 16 cifras.

\end{itemize}

Antes de empezar crear el código que permita procesar estos parámetros complejos.

\begin{sphinxVerbatim}[commandchars=\\\{\}]
\PYG{k+kd}{public} \PYG{k+kd}{class} \PYG{n+nc}{ServidorCalculo} \PYG{p}{\PYGZob{}}
        \PYG{k+kd}{public} \PYG{k+kt}{int} \PYG{n+nf}{extraerNumero}\PYG{p}{(}\PYG{n}{String} \PYG{n}{linea}\PYG{p}{)}\PYG{p}{\PYGZob{}}
                \PYG{c+cm}{/* 1. Comprobar si es un número}
\PYG{c+cm}{                 * 2. Ver si el número es correcto (32a75)}
\PYG{c+cm}{                 * 3. Ver si tiene de 1 a 8 cifras}
\PYG{c+cm}{                 */}
                \PYG{k+kt}{int} \PYG{n}{numero}\PYG{p}{;}
                \PYG{k}{try}\PYG{p}{\PYGZob{}}
                        \PYG{n}{numero}\PYG{o}{=}\PYG{n}{Integer}\PYG{p}{.}\PYG{n+na}{parseInt}\PYG{p}{(}\PYG{n}{linea}\PYG{p}{)}\PYG{p}{;}
                \PYG{p}{\PYGZcb{}}
                \PYG{k}{catch} \PYG{p}{(}\PYG{n}{NumberFormatException} \PYG{n}{e}\PYG{p}{)}\PYG{p}{\PYGZob{}}
                        \PYG{n}{numero}\PYG{o}{=}\PYG{l+m+mi}{0}\PYG{p}{;}
                \PYG{p}{\PYGZcb{}}
                \PYG{c+cm}{/* Si el número es mayor de 100 millones no}
\PYG{c+cm}{                 * es válido tampoco}
\PYG{c+cm}{                 */}
                \PYG{k}{if} \PYG{p}{(}\PYG{n}{numero}\PYG{o}{\PYGZgt{}}\PYG{o}{=}\PYG{l+m+mi}{100000000}\PYG{p}{)}\PYG{p}{\PYGZob{}}
                        \PYG{n}{numero}\PYG{o}{=}\PYG{l+m+mi}{0}\PYG{p}{;}
                \PYG{p}{\PYGZcb{}}
                \PYG{k}{return} \PYG{n}{numero}\PYG{p}{;}

        \PYG{p}{\PYGZcb{}}
        \PYG{k+kd}{public} \PYG{k+kt}{void} \PYG{n+nf}{escuchar}\PYG{p}{(}\PYG{p}{)}\PYG{p}{\PYGZob{}}
                \PYG{n}{System}\PYG{p}{.}\PYG{n+na}{out}\PYG{p}{.}\PYG{n+na}{println}\PYG{p}{(}\PYG{l+s}{\PYGZdq{}}\PYG{l+s}{Arrancado el servidor}\PYG{l+s}{\PYGZdq{}}\PYG{p}{)}\PYG{p}{;}

                \PYG{k}{while} \PYG{p}{(}\PYG{k+kc}{true}\PYG{p}{)}\PYG{p}{\PYGZob{}}

                \PYG{p}{\PYGZcb{}}
        \PYG{p}{\PYGZcb{}}
\PYG{p}{\PYGZcb{}}
\end{sphinxVerbatim}

Así, el código completo del servidor sería:

\begin{sphinxVerbatim}[commandchars=\\\{\}]
\PYG{k+kd}{public} \PYG{k+kd}{class} \PYG{n+nc}{ServidorCalculo} \PYG{p}{\PYGZob{}}
        \PYG{k+kd}{public} \PYG{k+kt}{int} \PYG{n+nf}{extraerNumero}\PYG{p}{(}\PYG{n}{String} \PYG{n}{linea}\PYG{p}{)}\PYG{p}{\PYGZob{}}
                \PYG{c+cm}{/* 1. Comprobar si es un número}
\PYG{c+cm}{                 * 2. Ver si el número es correcto (32a75)}
\PYG{c+cm}{                 * 3. Ver si tiene de 1 a 8 cifras}
\PYG{c+cm}{                 */}
                \PYG{k+kt}{int} \PYG{n}{numero}\PYG{p}{;}
                \PYG{k}{try}\PYG{p}{\PYGZob{}}
                        \PYG{n}{numero}\PYG{o}{=}\PYG{n}{Integer}\PYG{p}{.}\PYG{n+na}{parseInt}\PYG{p}{(}\PYG{n}{linea}\PYG{p}{)}\PYG{p}{;}
                \PYG{p}{\PYGZcb{}}
                \PYG{k}{catch} \PYG{p}{(}\PYG{n}{NumberFormatException} \PYG{n}{e}\PYG{p}{)}\PYG{p}{\PYGZob{}}
                        \PYG{n}{numero}\PYG{o}{=}\PYG{l+m+mi}{0}\PYG{p}{;}
                \PYG{p}{\PYGZcb{}}
                \PYG{c+cm}{/* Si el número es mayor de 100 millones no}
\PYG{c+cm}{                 * es válido tampoco}
\PYG{c+cm}{                 */}
                \PYG{k}{if} \PYG{p}{(}\PYG{n}{numero}\PYG{o}{\PYGZgt{}}\PYG{o}{=}\PYG{l+m+mi}{100000000}\PYG{p}{)}\PYG{p}{\PYGZob{}}
                        \PYG{n}{numero}\PYG{o}{=}\PYG{l+m+mi}{0}\PYG{p}{;}
                \PYG{p}{\PYGZcb{}}
                \PYG{k}{return} \PYG{n}{numero}\PYG{p}{;}
        \PYG{p}{\PYGZcb{}}

        \PYG{k+kd}{public} \PYG{k+kt}{int} \PYG{n+nf}{calcular}\PYG{p}{(}\PYG{n}{String} \PYG{n}{op}\PYG{p}{,} \PYG{n}{String} \PYG{n}{n1}\PYG{p}{,} \PYG{n}{String} \PYG{n}{n2}\PYG{p}{)}\PYG{p}{\PYGZob{}}
                \PYG{k+kt}{int} \PYG{n}{resultado}\PYG{o}{=}\PYG{l+m+mi}{0}\PYG{p}{;}
                \PYG{k+kt}{char} \PYG{n}{simbolo}\PYG{o}{=}\PYG{n}{op}\PYG{p}{.}\PYG{n+na}{charAt}\PYG{p}{(}\PYG{l+m+mi}{0}\PYG{p}{)}\PYG{p}{;}
                \PYG{k+kt}{int} \PYG{n}{num1}\PYG{o}{=}\PYG{k}{this}\PYG{p}{.}\PYG{n+na}{extraerNumero}\PYG{p}{(}\PYG{n}{n1}\PYG{p}{)}\PYG{p}{;}
                \PYG{k+kt}{int} \PYG{n}{num2}\PYG{o}{=}\PYG{k}{this}\PYG{p}{.}\PYG{n+na}{extraerNumero}\PYG{p}{(}\PYG{n}{n2}\PYG{p}{)}\PYG{p}{;}
                \PYG{k}{if} \PYG{p}{(}\PYG{n}{simbolo}\PYG{o}{=}\PYG{o}{=}\PYG{l+s+sc}{\PYGZsq{}+\PYGZsq{}}\PYG{p}{)}\PYG{p}{\PYGZob{}}
                        \PYG{n}{resultado}\PYG{o}{=}\PYG{n}{num1}\PYG{o}{+}\PYG{n}{num2}\PYG{p}{;}
                \PYG{p}{\PYGZcb{}}
                \PYG{k}{return} \PYG{n}{resultado}\PYG{p}{;}
        \PYG{p}{\PYGZcb{}}

        \PYG{k+kd}{public} \PYG{k+kt}{void} \PYG{n+nf}{escuchar}\PYG{p}{(}\PYG{p}{)} \PYG{k+kd}{throws} \PYG{n}{IOException}\PYG{p}{\PYGZob{}}
                \PYG{n}{System}\PYG{p}{.}\PYG{n+na}{out}\PYG{p}{.}\PYG{n+na}{println}\PYG{p}{(}\PYG{l+s}{\PYGZdq{}}\PYG{l+s}{Arrancado el servidor}\PYG{l+s}{\PYGZdq{}}\PYG{p}{)}\PYG{p}{;}
                \PYG{n}{ServerSocket} \PYG{n}{socketEscucha}\PYG{o}{=}\PYG{k+kc}{null}\PYG{p}{;}
                \PYG{k}{try} \PYG{p}{\PYGZob{}}
                        \PYG{n}{socketEscucha}\PYG{o}{=}\PYG{k}{new} \PYG{n}{ServerSocket}\PYG{p}{(}\PYG{l+m+mi}{9876}\PYG{p}{)}\PYG{p}{;}
                \PYG{p}{\PYGZcb{}} \PYG{k}{catch} \PYG{p}{(}\PYG{n}{IOException} \PYG{n}{e}\PYG{p}{)} \PYG{p}{\PYGZob{}}
                        \PYG{n}{System}\PYG{p}{.}\PYG{n+na}{out}\PYG{p}{.}\PYG{n+na}{println}\PYG{p}{(}
                                        \PYG{l+s}{\PYGZdq{}}\PYG{l+s}{No se pudo poner un socket }\PYG{l+s}{\PYGZdq{}}\PYG{o}{+}
                                        \PYG{l+s}{\PYGZdq{}}\PYG{l+s}{a escuchar en TCP 9876}\PYG{l+s}{\PYGZdq{}}\PYG{p}{)}\PYG{p}{;}
                        \PYG{k}{return}\PYG{p}{;}
                \PYG{p}{\PYGZcb{}}
                \PYG{k}{while} \PYG{p}{(}\PYG{k+kc}{true}\PYG{p}{)}\PYG{p}{\PYGZob{}}
                        \PYG{n}{Socket} \PYG{n}{conexion}\PYG{o}{=}\PYG{n}{socketEscucha}\PYG{p}{.}\PYG{n+na}{accept}\PYG{p}{(}\PYG{p}{)}\PYG{p}{;}
                        \PYG{n}{System}\PYG{p}{.}\PYG{n+na}{out}\PYG{p}{.}\PYG{n+na}{println}\PYG{p}{(}\PYG{l+s}{\PYGZdq{}}\PYG{l+s}{Conexion recibida!}\PYG{l+s}{\PYGZdq{}}\PYG{p}{)}\PYG{p}{;}
                        \PYG{n}{InputStream} \PYG{n}{is}\PYG{o}{=}\PYG{n}{conexion}\PYG{p}{.}\PYG{n+na}{getInputStream}\PYG{p}{(}\PYG{p}{)}\PYG{p}{;}
                        \PYG{n}{InputStreamReader} \PYG{n}{isr}\PYG{o}{=}
                                        \PYG{k}{new} \PYG{n}{InputStreamReader}\PYG{p}{(}\PYG{n}{is}\PYG{p}{)}\PYG{p}{;}
                        \PYG{n}{BufferedReader} \PYG{n}{bf}\PYG{o}{=}
                                        \PYG{k}{new} \PYG{n}{BufferedReader}\PYG{p}{(}\PYG{n}{isr}\PYG{p}{)}\PYG{p}{;}
                        \PYG{n}{String} \PYG{n}{linea}\PYG{o}{=}\PYG{n}{bf}\PYG{p}{.}\PYG{n+na}{readLine}\PYG{p}{(}\PYG{p}{)}\PYG{p}{;}
                        \PYG{n}{String} \PYG{n}{num1}\PYG{o}{=}\PYG{n}{bf}\PYG{p}{.}\PYG{n+na}{readLine}\PYG{p}{(}\PYG{p}{)}\PYG{p}{;}
                        \PYG{n}{String} \PYG{n}{num2}\PYG{o}{=}\PYG{n}{bf}\PYG{p}{.}\PYG{n+na}{readLine}\PYG{p}{(}\PYG{p}{)}\PYG{p}{;}
                        \PYG{c+cm}{/* Calculamos el resultado*/}
                        \PYG{n}{Integer} \PYG{n}{result}\PYG{o}{=}\PYG{k}{this}\PYG{p}{.}\PYG{n+na}{calcular}\PYG{p}{(}\PYG{n}{linea}\PYG{p}{,} \PYG{n}{num1}\PYG{p}{,} \PYG{n}{num2}\PYG{p}{)}\PYG{p}{;}
                        \PYG{n}{OutputStream} \PYG{n}{os}\PYG{o}{=}\PYG{n}{conexion}\PYG{p}{.}\PYG{n+na}{getOutputStream}\PYG{p}{(}\PYG{p}{)}\PYG{p}{;}
                        \PYG{n}{PrintWriter} \PYG{n}{pw}\PYG{o}{=}\PYG{k}{new} \PYG{n}{PrintWriter}\PYG{p}{(}\PYG{n}{os}\PYG{p}{)}\PYG{p}{;}
                        \PYG{n}{pw}\PYG{p}{.}\PYG{n+na}{write}\PYG{p}{(}\PYG{n}{result}\PYG{p}{.}\PYG{n+na}{toString}\PYG{p}{(}\PYG{p}{)}\PYG{o}{+}\PYG{l+s}{\PYGZdq{}}\PYG{l+s}{\PYGZbs{}}\PYG{l+s}{n}\PYG{l+s}{\PYGZdq{}}\PYG{p}{)}\PYG{p}{;}
                        \PYG{n}{pw}\PYG{p}{.}\PYG{n+na}{flush}\PYG{p}{(}\PYG{p}{)}\PYG{p}{;}
                \PYG{p}{\PYGZcb{}}
        \PYG{p}{\PYGZcb{}}
\PYG{p}{\PYGZcb{}}
\end{sphinxVerbatim}

Y el cliente sería:

\begin{sphinxVerbatim}[commandchars=\\\{\}]
\PYG{k+kd}{public} \PYG{k+kd}{class} \PYG{n+nc}{ClienteCalculo} \PYG{p}{\PYGZob{}}
        \PYG{k+kd}{public} \PYG{k+kd}{static} \PYG{n}{BufferedReader} \PYG{n+nf}{getFlujo}\PYG{p}{(}\PYG{n}{InputStream} \PYG{n}{is}\PYG{p}{)}\PYG{p}{\PYGZob{}}
                \PYG{n}{InputStreamReader} \PYG{n}{isr}\PYG{o}{=}
                                \PYG{k}{new} \PYG{n}{InputStreamReader}\PYG{p}{(}\PYG{n}{is}\PYG{p}{)}\PYG{p}{;}
                \PYG{n}{BufferedReader} \PYG{n}{bfr}\PYG{o}{=}
                                \PYG{k}{new} \PYG{n}{BufferedReader}\PYG{p}{(}\PYG{n}{isr}\PYG{p}{)}\PYG{p}{;}
                \PYG{k}{return} \PYG{n}{bfr}\PYG{p}{;}
        \PYG{p}{\PYGZcb{}}
        \PYG{c+cm}{/**}
\PYG{c+cm}{         * @param args}
\PYG{c+cm}{         * @throws IOException}
\PYG{c+cm}{         */}
        \PYG{k+kd}{public} \PYG{k+kd}{static} \PYG{k+kt}{void} \PYG{n+nf}{main}\PYG{p}{(}\PYG{n}{String}\PYG{o}{[}\PYG{o}{]} \PYG{n}{args}\PYG{p}{)} \PYG{k+kd}{throws} \PYG{n}{IOException} \PYG{p}{\PYGZob{}}
                \PYG{n}{InetSocketAddress} \PYG{n}{direccion}\PYG{o}{=}\PYG{k}{new}
                                \PYG{n}{InetSocketAddress}\PYG{p}{(}\PYG{l+s}{\PYGZdq{}}\PYG{l+s}{10.13.0.20}\PYG{l+s}{\PYGZdq{}}\PYG{p}{,} \PYG{l+m+mi}{9876}\PYG{p}{)}\PYG{p}{;}
                \PYG{n}{Socket} \PYG{n}{socket}\PYG{o}{=}\PYG{k}{new} \PYG{n}{Socket}\PYG{p}{(}\PYG{p}{)}\PYG{p}{;}
                \PYG{n}{socket}\PYG{p}{.}\PYG{n+na}{connect}\PYG{p}{(}\PYG{n}{direccion}\PYG{p}{)}\PYG{p}{;}
                \PYG{n}{BufferedReader} \PYG{n}{bfr}\PYG{o}{=}
                                \PYG{n}{ClienteCalculo}\PYG{p}{.}\PYG{n+na}{getFlujo}\PYG{p}{(}
                                                \PYG{n}{socket}\PYG{p}{.}\PYG{n+na}{getInputStream}\PYG{p}{(}\PYG{p}{)}\PYG{p}{)}\PYG{p}{;}
                \PYG{n}{PrintWriter} \PYG{n}{pw}\PYG{o}{=}\PYG{k}{new}
                                \PYG{n}{PrintWriter}\PYG{p}{(}\PYG{n}{socket}\PYG{p}{.}\PYG{n+na}{getOutputStream}\PYG{p}{(}\PYG{p}{)}\PYG{p}{)}\PYG{p}{;}
                \PYG{n}{pw}\PYG{p}{.}\PYG{n+na}{print}\PYG{p}{(}\PYG{l+s}{\PYGZdq{}}\PYG{l+s}{+}\PYG{l+s}{\PYGZbs{}}\PYG{l+s}{n}\PYG{l+s}{\PYGZdq{}}\PYG{p}{)}\PYG{p}{;}
                \PYG{n}{pw}\PYG{p}{.}\PYG{n+na}{print}\PYG{p}{(}\PYG{l+s}{\PYGZdq{}}\PYG{l+s}{42}\PYG{l+s}{\PYGZbs{}}\PYG{l+s}{n}\PYG{l+s}{\PYGZdq{}}\PYG{p}{)}\PYG{p}{;}
                \PYG{n}{pw}\PYG{p}{.}\PYG{n+na}{print}\PYG{p}{(}\PYG{l+s}{\PYGZdq{}}\PYG{l+s}{84}\PYG{l+s}{\PYGZbs{}}\PYG{l+s}{n}\PYG{l+s}{\PYGZdq{}}\PYG{p}{)}\PYG{p}{;}
                \PYG{n}{pw}\PYG{p}{.}\PYG{n+na}{flush}\PYG{p}{(}\PYG{p}{)}\PYG{p}{;}
                \PYG{n}{String} \PYG{n}{resultado}\PYG{o}{=}\PYG{n}{bfr}\PYG{p}{.}\PYG{n+na}{readLine}\PYG{p}{(}\PYG{p}{)}\PYG{p}{;}
                \PYG{n}{System}\PYG{p}{.}\PYG{n+na}{out}\PYG{p}{.}\PYG{n+na}{println}
                        \PYG{p}{(}\PYG{l+s}{\PYGZdq{}}\PYG{l+s}{El resultado fue:}\PYG{l+s}{\PYGZdq{}}\PYG{o}{+}\PYG{n}{resultado}\PYG{p}{)}\PYG{p}{;}
        \PYG{p}{\PYGZcb{}}
\PYG{p}{\PYGZcb{}}
\end{sphinxVerbatim}


\section{Utilización de hilos en la programación de aplicaciones en red.}
\label{\detokenize{textos/tema3:utilizacion-de-hilos-en-la-programacion-de-aplicaciones-en-red}}
En el caso de aplicaciones que necesiten aceptar varias conexiones \sphinxstylestrong{habrá que mover todo el código de gestión de peticiones a una clase que implemente Runnable}

Ahora el servidor será así:

\begin{sphinxVerbatim}[commandchars=\\\{\}]
\PYG{k}{while} \PYG{p}{(}\PYG{k+kc}{true}\PYG{p}{)}\PYG{p}{\PYGZob{}}
        \PYG{n}{Socket} \PYG{n}{conexion}\PYG{o}{=}\PYG{n}{socketEscucha}\PYG{p}{.}\PYG{n+na}{accept}\PYG{p}{(}\PYG{p}{)}\PYG{p}{;}
        \PYG{n}{System}\PYG{p}{.}\PYG{n+na}{out}\PYG{p}{.}\PYG{n+na}{println}\PYG{p}{(}\PYG{l+s}{\PYGZdq{}}\PYG{l+s}{Conexion recibida}\PYG{l+s}{\PYGZdq{}}\PYG{p}{)}\PYG{p}{;}
        \PYG{n}{Peticion} \PYG{n}{p}\PYG{o}{=}\PYG{k}{new} \PYG{n}{Peticion}\PYG{p}{(}\PYG{n}{conexion}\PYG{p}{)}\PYG{p}{;}
        \PYG{n}{Thread} \PYG{n}{hilo}\PYG{o}{=}\PYG{k}{new} \PYG{n}{Thread}\PYG{p}{(}\PYG{n}{p}\PYG{p}{)}\PYG{p}{;}
        \PYG{n}{hilo}\PYG{p}{.}\PYG{n+na}{start}\PYG{p}{(}\PYG{p}{)}\PYG{p}{;}
\PYG{p}{\PYGZcb{}}
\end{sphinxVerbatim}

Pero ahora tendremos una clase Petición como esta:

\begin{sphinxVerbatim}[commandchars=\\\{\}]
\PYG{k+kd}{public} \PYG{k+kd}{class} \PYG{n+nc}{Peticion} \PYG{k+kd}{implements} \PYG{n}{Runnable}\PYG{p}{\PYGZob{}}
        \PYG{n}{BufferedReader} \PYG{n}{bfr}\PYG{p}{;}
        \PYG{n}{PrintWriter} \PYG{n}{pw}\PYG{p}{;}
        \PYG{n}{Socket} \PYG{n}{socket}\PYG{p}{;}
        \PYG{k+kd}{public} \PYG{n+nf}{Peticion}\PYG{p}{(}\PYG{n}{Socket} \PYG{n}{socket}\PYG{p}{)}\PYG{p}{\PYGZob{}}
                \PYG{k}{this}\PYG{p}{.}\PYG{n+na}{socket}\PYG{o}{=}\PYG{n}{socket}\PYG{p}{;}
        \PYG{p}{\PYGZcb{}}
        \PYG{k+kd}{public} \PYG{k+kt}{int} \PYG{n+nf}{extraerNumero}\PYG{p}{(}\PYG{n}{String} \PYG{n}{linea}\PYG{p}{)}\PYG{p}{\PYGZob{}}
                \PYG{c+cm}{/* 1. Comprobar si es un número}
\PYG{c+cm}{                 * 2. Ver si el número es correcto (32a75)}
\PYG{c+cm}{                 * 3. Ver si tiene de 1 a 8 cifras}
\PYG{c+cm}{                 */}
                \PYG{k+kt}{int} \PYG{n}{numero}\PYG{p}{;}
                \PYG{k}{try}\PYG{p}{\PYGZob{}}
                        \PYG{n}{numero}\PYG{o}{=}\PYG{n}{Integer}\PYG{p}{.}\PYG{n+na}{parseInt}\PYG{p}{(}\PYG{n}{linea}\PYG{p}{)}\PYG{p}{;}
                \PYG{p}{\PYGZcb{}}
                \PYG{k}{catch} \PYG{p}{(}\PYG{n}{NumberFormatException} \PYG{n}{e}\PYG{p}{)}\PYG{p}{\PYGZob{}}
                        \PYG{n}{numero}\PYG{o}{=}\PYG{l+m+mi}{0}\PYG{p}{;}
                \PYG{p}{\PYGZcb{}}
                \PYG{c+cm}{/* Si el número es mayor de 100 millones no}
\PYG{c+cm}{                 * es válido tampoco}
\PYG{c+cm}{                 */}
                \PYG{k}{if} \PYG{p}{(}\PYG{n}{numero}\PYG{o}{\PYGZgt{}}\PYG{o}{=}\PYG{l+m+mi}{100000000}\PYG{p}{)}\PYG{p}{\PYGZob{}}
                        \PYG{n}{numero}\PYG{o}{=}\PYG{l+m+mi}{0}\PYG{p}{;}
                \PYG{p}{\PYGZcb{}}
                \PYG{k}{return} \PYG{n}{numero}\PYG{p}{;}

        \PYG{p}{\PYGZcb{}}

        \PYG{k+kd}{public} \PYG{k+kt}{int} \PYG{n+nf}{calcular}\PYG{p}{(}\PYG{n}{String} \PYG{n}{op}\PYG{p}{,} \PYG{n}{String} \PYG{n}{n1}\PYG{p}{,} \PYG{n}{String} \PYG{n}{n2}\PYG{p}{)}\PYG{p}{\PYGZob{}}
                \PYG{k+kt}{int} \PYG{n}{resultado}\PYG{o}{=}\PYG{l+m+mi}{0}\PYG{p}{;}
                \PYG{k+kt}{char} \PYG{n}{simbolo}\PYG{o}{=}\PYG{n}{op}\PYG{p}{.}\PYG{n+na}{charAt}\PYG{p}{(}\PYG{l+m+mi}{0}\PYG{p}{)}\PYG{p}{;}
                \PYG{k+kt}{int} \PYG{n}{num1}\PYG{o}{=}\PYG{k}{this}\PYG{p}{.}\PYG{n+na}{extraerNumero}\PYG{p}{(}\PYG{n}{n1}\PYG{p}{)}\PYG{p}{;}
                \PYG{k+kt}{int} \PYG{n}{num2}\PYG{o}{=}\PYG{k}{this}\PYG{p}{.}\PYG{n+na}{extraerNumero}\PYG{p}{(}\PYG{n}{n2}\PYG{p}{)}\PYG{p}{;}
                \PYG{k}{if} \PYG{p}{(}\PYG{n}{simbolo}\PYG{o}{=}\PYG{o}{=}\PYG{l+s+sc}{\PYGZsq{}+\PYGZsq{}}\PYG{p}{)}\PYG{p}{\PYGZob{}}
                        \PYG{n}{resultado}\PYG{o}{=}\PYG{n}{num1}\PYG{o}{+}\PYG{n}{num2}\PYG{p}{;}
                \PYG{p}{\PYGZcb{}}
                \PYG{k}{return} \PYG{n}{resultado}\PYG{p}{;}
        \PYG{p}{\PYGZcb{}}
        \PYG{k+kd}{public} \PYG{k+kt}{void} \PYG{n+nf}{run}\PYG{p}{(}\PYG{p}{)}\PYG{p}{\PYGZob{}}
                \PYG{k}{try} \PYG{p}{\PYGZob{}}
                        \PYG{n}{InputStream} \PYG{n}{is}\PYG{o}{=}\PYG{n}{socket}\PYG{p}{.}\PYG{n+na}{getInputStream}\PYG{p}{(}\PYG{p}{)}\PYG{p}{;}
                        \PYG{n}{InputStreamReader} \PYG{n}{isr}\PYG{o}{=}
                                        \PYG{k}{new} \PYG{n}{InputStreamReader}\PYG{p}{(}\PYG{n}{is}\PYG{p}{)}\PYG{p}{;}
                        \PYG{n}{bfr}\PYG{o}{=}\PYG{k}{new} \PYG{n}{BufferedReader}\PYG{p}{(}\PYG{n}{isr}\PYG{p}{)}\PYG{p}{;}
                        \PYG{n}{OutputStream} \PYG{n}{os}\PYG{o}{=}\PYG{n}{socket}\PYG{p}{.}\PYG{n+na}{getOutputStream}\PYG{p}{(}\PYG{p}{)}\PYG{p}{;}
                        \PYG{n}{pw}\PYG{o}{=}\PYG{k}{new} \PYG{n}{PrintWriter}\PYG{p}{(}\PYG{n}{os}\PYG{p}{)}\PYG{p}{;}
                        \PYG{n}{String} \PYG{n}{linea}\PYG{p}{;}
                        \PYG{k}{while} \PYG{p}{(}\PYG{k+kc}{true}\PYG{p}{)}\PYG{p}{\PYGZob{}}
                                \PYG{n}{linea} \PYG{o}{=} \PYG{n}{bfr}\PYG{p}{.}\PYG{n+na}{readLine}\PYG{p}{(}\PYG{p}{)}\PYG{p}{;}
                                \PYG{n}{String} \PYG{n}{num1}\PYG{o}{=}\PYG{n}{bfr}\PYG{p}{.}\PYG{n+na}{readLine}\PYG{p}{(}\PYG{p}{)}\PYG{p}{;}
                                \PYG{n}{String} \PYG{n}{num2}\PYG{o}{=}\PYG{n}{bfr}\PYG{p}{.}\PYG{n+na}{readLine}\PYG{p}{(}\PYG{p}{)}\PYG{p}{;}
                                \PYG{c+cm}{/* Calculamos el resultado*/}
                                \PYG{n}{Integer} \PYG{n}{result}\PYG{o}{=}\PYG{k}{this}\PYG{p}{.}\PYG{n+na}{calcular}\PYG{p}{(}\PYG{n}{linea}\PYG{p}{,} \PYG{n}{num1}\PYG{p}{,} \PYG{n}{num2}\PYG{p}{)}\PYG{p}{;}
                                \PYG{n}{System}\PYG{p}{.}\PYG{n+na}{out}\PYG{p}{.}\PYG{n+na}{println}\PYG{p}{(}\PYG{l+s}{\PYGZdq{}}\PYG{l+s}{El servidor dio resultado:}\PYG{l+s}{\PYGZdq{}}\PYG{o}{+}\PYG{n}{result}\PYG{p}{)}\PYG{p}{;}
                                \PYG{n}{pw}\PYG{p}{.}\PYG{n+na}{write}\PYG{p}{(}\PYG{n}{result}\PYG{p}{.}\PYG{n+na}{toString}\PYG{p}{(}\PYG{p}{)}\PYG{o}{+}\PYG{l+s}{\PYGZdq{}}\PYG{l+s}{\PYGZbs{}}\PYG{l+s}{n}\PYG{l+s}{\PYGZdq{}}\PYG{p}{)}\PYG{p}{;}
                                \PYG{n}{pw}\PYG{p}{.}\PYG{n+na}{flush}\PYG{p}{(}\PYG{p}{)}\PYG{p}{;}
                        \PYG{p}{\PYGZcb{}}
                \PYG{p}{\PYGZcb{}} \PYG{k}{catch} \PYG{p}{(}\PYG{n}{IOException} \PYG{n}{e}\PYG{p}{)} \PYG{p}{\PYGZob{}}
                \PYG{p}{\PYGZcb{}}
        \PYG{p}{\PYGZcb{}}
\PYG{p}{\PYGZcb{}}
\end{sphinxVerbatim}


\subsection{Ejercicio: servicio de ordenación}
\label{\detokenize{textos/tema3:ejercicio-servicio-de-ordenacion}}
Crear una arquitectura cliente/servidor que permita a un cliente, enviar dos cadenas a un servidor para saber cual de ellas va antes que otra:
\begin{itemize}
\item {} 
Un cliente puede enviar las cadenas «hola», «mundo». El servidor comprobará que en el diccionario la primera va antes que la segunda, por lo cual contestará «hola», «mundo».

\item {} 
Si el cliente enviase «mundo», «hola» el servidor debe devolver la respuesta «hola», «mundo».

\end{itemize}

Debido a posibles mejoras futuras, se espera que el servidor sea capaz de saber qué versión del protocolo se maneja. Esto es debido a que en el futuro se espera lanzar una versión 2 del protocolo en la que se puedan enviar varias cadenas seguidas.

Crear el protocolo, el código Java del cliente y el código Java del servidor con capacidad para procesar muchas peticiones a la vez (multihilo).

Se debe aceptar que un cliente que ya tenga un socket abierto envíe todas las parejas de cadenas que desee.


\subsection{Una clase Protocolo}
\label{\detokenize{textos/tema3:una-clase-protocolo}}
Dado que los protocolos pueden ser variables puede ser útil encapsular el comportamiento del protocolo en una pequeña clase separada:

\begin{sphinxVerbatim}[commandchars=\\\{\}]
\PYG{k+kd}{public} \PYG{k+kd}{class} \PYG{n+nc}{Protocolo} \PYG{p}{\PYGZob{}}
        \PYG{k+kd}{private} \PYG{k+kd}{final} \PYG{n}{String} \PYG{n}{terminador}\PYG{o}{=}\PYG{l+s}{\PYGZdq{}}\PYG{l+s}{\PYGZbs{}}\PYG{l+s}{n}\PYG{l+s}{\PYGZdq{}}\PYG{p}{;}
        \PYG{k+kd}{public} \PYG{n}{String} \PYG{n+nf}{getMensajeVersion}\PYG{p}{(}\PYG{k+kt}{int} \PYG{n}{version}\PYG{p}{)}\PYG{p}{\PYGZob{}}
                \PYG{n}{Integer} \PYG{n}{i}\PYG{o}{=}\PYG{n}{version}\PYG{p}{;}
                \PYG{k}{return} \PYG{n}{i}\PYG{p}{.}\PYG{n+na}{toString}\PYG{p}{(}\PYG{p}{)}\PYG{o}{+}\PYG{n}{terminador}\PYG{p}{;}
        \PYG{p}{\PYGZcb{}}
        \PYG{k+kd}{public} \PYG{k+kt}{int} \PYG{n+nf}{getNumVersion}\PYG{p}{(}\PYG{n}{String} \PYG{n}{mensaje}\PYG{p}{)}\PYG{p}{\PYGZob{}}
                \PYG{n}{Integer} \PYG{n}{num}\PYG{o}{=}\PYG{n}{Integer}\PYG{p}{.}\PYG{n+na}{parseInt}\PYG{p}{(}\PYG{n}{mensaje}\PYG{p}{)}\PYG{p}{;}
                \PYG{k}{return} \PYG{n}{num}\PYG{p}{;}
        \PYG{p}{\PYGZcb{}}
        \PYG{k+kd}{public} \PYG{n}{String} \PYG{n+nf}{getMensaje}\PYG{p}{(}\PYG{n}{String} \PYG{n}{cadena}\PYG{p}{)}\PYG{p}{\PYGZob{}}
                \PYG{k}{return} \PYG{n}{cadena}\PYG{o}{+}\PYG{n}{terminador}\PYG{p}{;}
        \PYG{p}{\PYGZcb{}}
\PYG{p}{\PYGZcb{}}
\end{sphinxVerbatim}


\subsection{Una clase con funciones de utilidad}
\label{\detokenize{textos/tema3:una-clase-con-funciones-de-utilidad}}
Algunas operaciones son muy sencillas, pero muy engorrosas. Alargan el código innecesariamente y lo hacen más difícil de entender. Si además se realizan a menudo puede ser interesante empaquetar toda la funcionalidad en una clase.

\begin{sphinxVerbatim}[commandchars=\\\{\}]
\PYG{k+kd}{public} \PYG{k+kd}{class} \PYG{n+nc}{Utilidades} \PYG{p}{\PYGZob{}}
        \PYG{c+cm}{/* Obtiene un flujo de escritura}
\PYG{c+cm}{        a partir de un socket*/}
        \PYG{k+kd}{public} \PYG{n}{PrintWriter} \PYG{n+nf}{getFlujoEscritura}
                \PYG{p}{(}\PYG{n}{Socket} \PYG{n}{s}\PYG{p}{)} \PYG{k+kd}{throws} \PYG{n}{IOException}\PYG{p}{\PYGZob{}}
                \PYG{n}{OutputStream} \PYG{n}{os}\PYG{o}{=}\PYG{n}{s}\PYG{p}{.}\PYG{n+na}{getOutputStream}\PYG{p}{(}\PYG{p}{)}\PYG{p}{;}
                \PYG{n}{PrintWriter} \PYG{n}{pw}\PYG{o}{=}\PYG{k}{new} \PYG{n}{PrintWriter}\PYG{p}{(}\PYG{n}{os}\PYG{p}{)}\PYG{p}{;}
                \PYG{k}{return} \PYG{n}{pw}\PYG{p}{;}
        \PYG{p}{\PYGZcb{}}
        \PYG{c+cm}{/* Obtiene un flujo de lectura}
\PYG{c+cm}{        a partir de un socket*/}
        \PYG{k+kd}{public} \PYG{n}{BufferedReader}
                \PYG{n+nf}{getFlujoLectura}\PYG{p}{(}\PYG{n}{Socket} \PYG{n}{s}\PYG{p}{)}
                                \PYG{k+kd}{throws} \PYG{n}{IOException}\PYG{p}{\PYGZob{}}
                \PYG{n}{InputStream} \PYG{n}{is}\PYG{o}{=}\PYG{n}{s}\PYG{p}{.}\PYG{n+na}{getInputStream}\PYG{p}{(}\PYG{p}{)}\PYG{p}{;}
                \PYG{n}{InputStreamReader} \PYG{n}{isr}\PYG{o}{=}
                                \PYG{k}{new} \PYG{n}{InputStreamReader}\PYG{p}{(}\PYG{n}{is}\PYG{p}{)}\PYG{p}{;}
                \PYG{n}{BufferedReader} \PYG{n}{bfr}\PYG{o}{=}\PYG{k}{new} \PYG{n}{BufferedReader}\PYG{p}{(}\PYG{n}{isr}\PYG{p}{)}\PYG{p}{;}
                \PYG{k}{return} \PYG{n}{bfr}\PYG{p}{;}
        \PYG{p}{\PYGZcb{}}
\PYG{p}{\PYGZcb{}}
\end{sphinxVerbatim}


\subsection{La clase Petición}
\label{\detokenize{textos/tema3:la-clase-peticion}}
\begin{sphinxVerbatim}[commandchars=\\\{\}]
\PYG{k+kd}{public} \PYG{k+kd}{class} \PYG{n+nc}{Peticion} \PYG{k+kd}{implements} \PYG{n}{Runnable} \PYG{p}{\PYGZob{}}
        \PYG{n}{Socket} \PYG{n}{conexionParaAtender}\PYG{p}{;}

        \PYG{k+kd}{public} \PYG{n+nf}{Peticion} \PYG{p}{(} \PYG{n}{Socket} \PYG{n}{s} \PYG{p}{)}\PYG{p}{\PYGZob{}}
                \PYG{k}{this}\PYG{p}{.}\PYG{n+na}{conexionParaAtender}\PYG{o}{=}\PYG{n}{s}\PYG{p}{;}
        \PYG{p}{\PYGZcb{}}
        \PYG{n+nd}{@Override}
        \PYG{k+kd}{public} \PYG{k+kt}{void} \PYG{n+nf}{run}\PYG{p}{(}\PYG{p}{)} \PYG{p}{\PYGZob{}}
                \PYG{k}{try}\PYG{p}{\PYGZob{}}
                        \PYG{n}{PrintWriter} \PYG{n}{flujoEscritura}\PYG{o}{=}
                                \PYG{n}{Utilidades}\PYG{p}{.}\PYG{n+na}{getFlujoEscritura}\PYG{p}{(}
                                                \PYG{k}{this}\PYG{p}{.}\PYG{n+na}{conexionParaAtender}
                                                \PYG{p}{)}\PYG{p}{;}
                        \PYG{n}{BufferedReader} \PYG{n}{flujoLectura}\PYG{o}{=}
                                \PYG{n}{Utilidades}\PYG{p}{.}\PYG{n+na}{getFlujoLectura}\PYG{p}{(}
                                                \PYG{n}{conexionParaAtender}\PYG{p}{)}\PYG{p}{;}
                        \PYG{n}{String} \PYG{n}{protocolo}\PYG{o}{=}
                                        \PYG{n}{flujoLectura}\PYG{p}{.}\PYG{n+na}{readLine}\PYG{p}{(}\PYG{p}{)}\PYG{p}{;}
                        \PYG{k+kt}{int} \PYG{n}{numVersion}\PYG{o}{=}
                                        \PYG{n}{Protocolo}\PYG{p}{.}\PYG{n+na}{getNumVersion}\PYG{p}{(}\PYG{n}{protocolo}\PYG{p}{)}\PYG{p}{;}
                        \PYG{k}{if} \PYG{p}{(}\PYG{n}{numVersion}\PYG{o}{=}\PYG{o}{=}\PYG{l+m+mi}{1}\PYG{p}{)}\PYG{p}{\PYGZob{}}
                                \PYG{n}{String} \PYG{n}{linea1}\PYG{o}{=}
                                                \PYG{n}{flujoLectura}\PYG{p}{.}\PYG{n+na}{readLine}\PYG{p}{(}\PYG{p}{)}\PYG{p}{;}
                                \PYG{n}{String} \PYG{n}{linea2}\PYG{o}{=}
                                                \PYG{n}{flujoLectura}\PYG{p}{.}\PYG{n+na}{readLine}\PYG{p}{(}\PYG{p}{)}\PYG{p}{;}
                                \PYG{c+c1}{//Linea 1 va despues en el}
                                \PYG{k}{if} \PYG{p}{(}\PYG{n}{linea1}\PYG{p}{.}\PYG{n+na}{compareTo}\PYG{p}{(}\PYG{n}{linea2}\PYG{p}{)}\PYG{o}{\PYGZgt{}}\PYG{l+m+mi}{0}\PYG{p}{)}\PYG{p}{\PYGZob{}}
                                         \PYG{n}{dicc}
                                        \PYG{n}{flujoEscritura}\PYG{p}{.}\PYG{n+na}{println}\PYG{p}{(}\PYG{n}{linea2}\PYG{p}{)}\PYG{p}{;}
                                        \PYG{n}{flujoEscritura}\PYG{p}{.}\PYG{n+na}{println}\PYG{p}{(}\PYG{n}{linea1}\PYG{p}{)}\PYG{p}{;}
                                        \PYG{n}{flujoEscritura}\PYG{p}{.}\PYG{n+na}{flush}\PYG{p}{(}\PYG{p}{)}\PYG{p}{;}
                                \PYG{p}{\PYGZcb{}} \PYG{k}{else} \PYG{p}{\PYGZob{}}
                                        \PYG{n}{flujoEscritura}\PYG{p}{.}\PYG{n+na}{println}\PYG{p}{(}\PYG{n}{linea1}\PYG{p}{)}\PYG{p}{;}
                                        \PYG{n}{flujoEscritura}\PYG{p}{.}\PYG{n+na}{println}\PYG{p}{(}\PYG{n}{linea2}\PYG{p}{)}\PYG{p}{;}
                                        \PYG{n}{flujoEscritura}\PYG{p}{.}\PYG{n+na}{flush}\PYG{p}{(}\PYG{p}{)}\PYG{p}{;}
                                \PYG{p}{\PYGZcb{}}
                        \PYG{p}{\PYGZcb{}}
                \PYG{p}{\PYGZcb{}}
                \PYG{k}{catch} \PYG{p}{(}\PYG{n}{IOException} \PYG{n}{e}\PYG{p}{)}\PYG{p}{\PYGZob{}}
                        \PYG{n}{System}\PYG{p}{.}\PYG{n+na}{out}\PYG{p}{.}\PYG{n+na}{println}\PYG{p}{(}
                                        \PYG{l+s}{\PYGZdq{}}\PYG{l+s}{No se pudo crear algún flujo}\PYG{l+s}{\PYGZdq{}}\PYG{p}{)}\PYG{p}{;}
                        \PYG{k}{return} \PYG{p}{;}
                \PYG{p}{\PYGZcb{}}
        \PYG{p}{\PYGZcb{}}
\PYG{p}{\PYGZcb{}}
\end{sphinxVerbatim}


\subsection{La clase Servidor}
\label{\detokenize{textos/tema3:la-clase-servidor}}
\begin{sphinxVerbatim}[commandchars=\\\{\}]
\PYG{k+kd}{public} \PYG{k+kd}{class} \PYG{n+nc}{ServidorOrdenacion} \PYG{p}{\PYGZob{}}
        \PYG{k+kd}{public} \PYG{k+kt}{void} \PYG{n+nf}{escuchar}\PYG{p}{(}\PYG{p}{)} \PYG{k+kd}{throws} \PYG{n}{IOException}\PYG{p}{\PYGZob{}}
                \PYG{n}{ServerSocket} \PYG{n}{socket}\PYG{p}{;}
                \PYG{k}{try}\PYG{p}{\PYGZob{}}
                        \PYG{n}{socket}\PYG{o}{=}\PYG{k}{new} \PYG{n}{ServerSocket}\PYG{p}{(}\PYG{l+m+mi}{9876}\PYG{p}{)}\PYG{p}{;}
                \PYG{p}{\PYGZcb{}}
                \PYG{k}{catch}\PYG{p}{(}\PYG{n}{Exception} \PYG{n}{e}\PYG{p}{)}\PYG{p}{\PYGZob{}}
                        \PYG{n}{System}\PYG{p}{.}\PYG{n+na}{out}\PYG{p}{.}\PYG{n+na}{println}\PYG{p}{(}\PYG{l+s}{\PYGZdq{}}\PYG{l+s}{No se pudo arrancar}\PYG{l+s}{\PYGZdq{}}\PYG{p}{)}\PYG{p}{;}
                        \PYG{k}{return} \PYG{p}{;}
                \PYG{p}{\PYGZcb{}}
                \PYG{k}{while} \PYG{p}{(}\PYG{k+kc}{true}\PYG{p}{)}\PYG{p}{\PYGZob{}}
                        \PYG{n}{System}\PYG{p}{.}\PYG{n+na}{out}\PYG{p}{.}\PYG{n+na}{println}\PYG{p}{(}\PYG{l+s}{\PYGZdq{}}\PYG{l+s}{Servidor esperando}\PYG{l+s}{\PYGZdq{}}\PYG{p}{)}\PYG{p}{;}
                        \PYG{n}{Socket} \PYG{n}{conexionCliente}\PYG{o}{=}
                                        \PYG{n}{socket}\PYG{p}{.}\PYG{n+na}{accept}\PYG{p}{(}\PYG{p}{)}\PYG{p}{;}
                        \PYG{n}{System}\PYG{p}{.}\PYG{n+na}{out}\PYG{p}{.}\PYG{n+na}{println}\PYG{p}{(}\PYG{l+s}{\PYGZdq{}}\PYG{l+s}{Alguien conectó}\PYG{l+s}{\PYGZdq{}}\PYG{p}{)}\PYG{p}{;}
                        \PYG{n}{Peticion} \PYG{n}{p}\PYG{o}{=}
                                        \PYG{k}{new} \PYG{n}{Peticion}\PYG{p}{(}\PYG{n}{conexionCliente}\PYG{p}{)}\PYG{p}{;}
                        \PYG{n}{Thread} \PYG{n}{hiloAsociado}\PYG{o}{=}
                                        \PYG{k}{new} \PYG{n}{Thread}\PYG{p}{(}\PYG{n}{p}\PYG{p}{)}\PYG{p}{;}
                        \PYG{n}{hiloAsociado}\PYG{p}{.}\PYG{n+na}{start}\PYG{p}{(}\PYG{p}{)}\PYG{p}{;}
                \PYG{p}{\PYGZcb{}}
        \PYG{p}{\PYGZcb{}} \PYG{c+c1}{// Fin del método escuchar}
        \PYG{k+kd}{public} \PYG{k+kd}{static} \PYG{k+kt}{void}  \PYG{n+nf}{main}\PYG{p}{(}\PYG{n}{String}\PYG{o}{[}\PYG{o}{]} \PYG{n}{argumentos}\PYG{p}{)}\PYG{p}{\PYGZob{}}
                \PYG{n}{ServidorOrdenacion} \PYG{n}{s}\PYG{o}{=}
                                \PYG{k}{new} \PYG{n}{ServidorOrdenacion}\PYG{p}{(}\PYG{p}{)}\PYG{p}{;}
                \PYG{k}{try} \PYG{p}{\PYGZob{}}
                        \PYG{n}{s}\PYG{p}{.}\PYG{n+na}{escuchar}\PYG{p}{(}\PYG{p}{)}\PYG{p}{;}
                \PYG{p}{\PYGZcb{}} \PYG{k}{catch} \PYG{p}{(}\PYG{n}{Exception} \PYG{n}{e}\PYG{p}{)}\PYG{p}{\PYGZob{}}
                        \PYG{n}{System}\PYG{p}{.}\PYG{n+na}{out}\PYG{p}{.}\PYG{n+na}{println}\PYG{p}{(}\PYG{l+s}{\PYGZdq{}}\PYG{l+s}{No se pudo arrancar}\PYG{l+s}{\PYGZdq{}}\PYG{p}{)}\PYG{p}{;}
                        \PYG{n}{System}\PYG{p}{.}\PYG{n+na}{out}\PYG{p}{.}\PYG{n+na}{println}\PYG{p}{(}\PYG{l+s}{\PYGZdq{}}\PYG{l+s}{ el cliente o el serv}\PYG{l+s}{\PYGZdq{}}\PYG{p}{)}\PYG{p}{;}
                \PYG{p}{\PYGZcb{}}
        \PYG{p}{\PYGZcb{}}
\PYG{p}{\PYGZcb{}}
\end{sphinxVerbatim}


\subsection{La clase Cliente}
\label{\detokenize{textos/tema3:la-clase-cliente}}
\begin{sphinxVerbatim}[commandchars=\\\{\}]
\PYG{k+kd}{public} \PYG{k+kd}{class} \PYG{n+nc}{Cliente} \PYG{p}{\PYGZob{}}
        \PYG{k+kd}{public} \PYG{k+kt}{void} \PYG{n+nf}{ordenar}\PYG{p}{(}\PYG{n}{String} \PYG{n}{s1}\PYG{p}{,} \PYG{n}{String} \PYG{n}{s2}\PYG{p}{)} \PYG{k+kd}{throws} \PYG{n}{IOException}\PYG{p}{\PYGZob{}}
                \PYG{n}{InetSocketAddress} \PYG{n}{direccion}\PYG{o}{=}
                                \PYG{k}{new} \PYG{n}{InetSocketAddress}\PYG{p}{(}\PYG{l+s}{\PYGZdq{}}\PYG{l+s}{10.13.0.20}\PYG{l+s}{\PYGZdq{}}\PYG{p}{,} \PYG{l+m+mi}{9876}\PYG{p}{)}\PYG{p}{;}
                \PYG{n}{Socket} \PYG{n}{conexion}\PYG{o}{=}
                                \PYG{k}{new} \PYG{n}{Socket}\PYG{p}{(}\PYG{p}{)}\PYG{p}{;}
                \PYG{n}{conexion}\PYG{p}{.}\PYG{n+na}{connect}\PYG{p}{(}\PYG{n}{direccion}\PYG{p}{)}\PYG{p}{;}
                \PYG{n}{System}\PYG{p}{.}\PYG{n+na}{out}\PYG{p}{.}\PYG{n+na}{println}\PYG{p}{(}\PYG{l+s}{\PYGZdq{}}\PYG{l+s}{Conexion establecida}\PYG{l+s}{\PYGZdq{}}\PYG{p}{)}\PYG{p}{;}
                \PYG{c+cm}{/* Ahora hay que crear flujos de salida, enviar}
\PYG{c+cm}{                 * cadenas por allí y esperar los resultados.}
\PYG{c+cm}{                 */}
                \PYG{k}{try}\PYG{p}{\PYGZob{}}

                        \PYG{n}{BufferedReader} \PYG{n}{flujoLectura}\PYG{o}{=}
                                \PYG{n}{Utilidades}\PYG{p}{.}\PYG{n+na}{getFlujoLectura}\PYG{p}{(}\PYG{n}{conexion}\PYG{p}{)}\PYG{p}{;}
                        \PYG{n}{PrintWriter} \PYG{n}{flujoEscritura}\PYG{o}{=}
                                \PYG{n}{Utilidades}\PYG{p}{.}\PYG{n+na}{getFlujoEscritura}\PYG{p}{(}\PYG{n}{conexion}\PYG{p}{)}\PYG{p}{;}

                        \PYG{n}{flujoEscritura}\PYG{p}{.}\PYG{n+na}{println}\PYG{p}{(}\PYG{l+s}{\PYGZdq{}}\PYG{l+s}{1}\PYG{l+s}{\PYGZdq{}}\PYG{p}{)}\PYG{p}{;}
                        \PYG{n}{flujoEscritura}\PYG{p}{.}\PYG{n+na}{println}\PYG{p}{(}\PYG{n}{s1}\PYG{p}{)}\PYG{p}{;}
                        \PYG{n}{flujoEscritura}\PYG{p}{.}\PYG{n+na}{println}\PYG{p}{(}\PYG{n}{s2}\PYG{p}{)}\PYG{p}{;}
                        \PYG{n}{flujoEscritura}\PYG{p}{.}\PYG{n+na}{flush}\PYG{p}{(}\PYG{p}{)}\PYG{p}{;}
                        \PYG{n}{String} \PYG{n}{linea1}\PYG{o}{=}\PYG{n}{flujoLectura}\PYG{p}{.}\PYG{n+na}{readLine}\PYG{p}{(}\PYG{p}{)}\PYG{p}{;}
                        \PYG{n}{String} \PYG{n}{linea2}\PYG{o}{=}\PYG{n}{flujoLectura}\PYG{p}{.}\PYG{n+na}{readLine}\PYG{p}{(}\PYG{p}{)}\PYG{p}{;}
                        \PYG{n}{System}\PYG{p}{.}\PYG{n+na}{out}\PYG{p}{.}\PYG{n+na}{println}\PYG{p}{(}\PYG{l+s}{\PYGZdq{}}\PYG{l+s}{El servidor devolvió }\PYG{l+s}{\PYGZdq{}}\PYG{o}{+}
                                        \PYG{n}{linea1} \PYG{o}{+} \PYG{l+s}{\PYGZdq{}}\PYG{l+s}{ y }\PYG{l+s}{\PYGZdq{}}\PYG{o}{+}\PYG{n}{linea2}\PYG{p}{)}\PYG{p}{;}
                \PYG{p}{\PYGZcb{}} \PYG{k}{catch} \PYG{p}{(}\PYG{n}{IOException} \PYG{n}{e}\PYG{p}{)}\PYG{p}{\PYGZob{}}

                \PYG{p}{\PYGZcb{}}
        \PYG{p}{\PYGZcb{}}
        \PYG{k+kd}{public} \PYG{k+kd}{static} \PYG{k+kt}{void} \PYG{n+nf}{main}\PYG{p}{(}\PYG{n}{String}\PYG{o}{[}\PYG{o}{]} \PYG{n}{args}\PYG{p}{)} \PYG{p}{\PYGZob{}}
                \PYG{n}{Cliente} \PYG{n}{c}\PYG{o}{=}\PYG{k}{new} \PYG{n}{Cliente}\PYG{p}{(}\PYG{p}{)}\PYG{p}{;}
                \PYG{k}{try} \PYG{p}{\PYGZob{}}
                        \PYG{n}{c}\PYG{p}{.}\PYG{n+na}{ordenar}\PYG{p}{(}\PYG{l+s}{\PYGZdq{}}\PYG{l+s}{aaa}\PYG{l+s}{\PYGZdq{}}\PYG{p}{,} \PYG{l+s}{\PYGZdq{}}\PYG{l+s}{bbb}\PYG{l+s}{\PYGZdq{}}\PYG{p}{)}\PYG{p}{;}
                \PYG{p}{\PYGZcb{}} \PYG{k}{catch} \PYG{p}{(}\PYG{n}{IOException} \PYG{n}{e}\PYG{p}{)} \PYG{p}{\PYGZob{}}
                        \PYG{n}{System}\PYG{p}{.}\PYG{n+na}{out}\PYG{p}{.}\PYG{n+na}{println}\PYG{p}{(}\PYG{l+s}{\PYGZdq{}}\PYG{l+s}{Fallo la conexion o }\PYG{l+s}{\PYGZdq{}}\PYG{p}{)}\PYG{p}{;}
                        \PYG{n}{System}\PYG{p}{.}\PYG{n+na}{out}\PYG{p}{.}\PYG{n+na}{println}\PYG{p}{(}\PYG{l+s}{\PYGZdq{}}\PYG{l+s}{los flujos}\PYG{l+s}{\PYGZdq{}}\PYG{p}{)}\PYG{p}{;}
                \PYG{p}{\PYGZcb{}} \PYG{c+c1}{//Fin del catch}
        \PYG{p}{\PYGZcb{}} \PYG{c+c1}{//Fin del main}
\PYG{p}{\PYGZcb{}} \PYG{c+c1}{//Fin de la clase}
\end{sphinxVerbatim}


\subsection{Ampliación}
\label{\detokenize{textos/tema3:ampliacion}}
Finalmente la empresa va a necesitar una versión mejorada del servidor que permita a otros cliente enviar un número de palabras y luego las palabras. Se desea hacer todo sin romper la compatibilidad con los clientes viejos. Mostrar el código Java del servidor y del cliente.

En el servidor se añade este código extra a la hora de comprobar el protocolo:

\begin{sphinxVerbatim}[commandchars=\\\{\}]
\PYG{k}{if} \PYG{p}{(}\PYG{n}{numVersion}\PYG{o}{=}\PYG{o}{=}\PYG{l+m+mi}{2}\PYG{p}{)}\PYG{p}{\PYGZob{}}
        \PYG{n}{System}\PYG{p}{.}\PYG{n+na}{out}\PYG{p}{.}\PYG{n+na}{println}\PYG{p}{(}\PYG{l+s}{\PYGZdq{}}\PYG{l+s}{Llegó un v2}\PYG{l+s}{\PYGZdq{}}\PYG{p}{)}\PYG{p}{;}
        \PYG{n}{String} \PYG{n}{lineaCantidadPalabras}\PYG{o}{=}
        \PYG{n}{flujoLectura}\PYG{p}{.}\PYG{n+na}{readLine}\PYG{p}{(}\PYG{p}{)}\PYG{p}{;}
        \PYG{k+kt}{int} \PYG{n}{numPalabras}\PYG{o}{=}
                \PYG{n}{Integer}\PYG{p}{.}\PYG{n+na}{parseInt}
                \PYG{p}{(}\PYG{n}{lineaCantidadPalabras}\PYG{p}{)}\PYG{p}{;}
        \PYG{n}{String}\PYG{o}{[}\PYG{o}{]} \PYG{n}{palabras}\PYG{o}{=}
                                \PYG{k}{new} \PYG{n}{String}\PYG{o}{[}\PYG{n}{numPalabras}\PYG{o}{]}\PYG{p}{;}
        \PYG{k}{for} \PYG{p}{(}\PYG{k+kt}{int} \PYG{n}{i}\PYG{o}{=}\PYG{l+m+mi}{0}\PYG{p}{;}\PYG{n}{i}\PYG{o}{\PYGZlt{}}\PYG{n}{numPalabras}\PYG{p}{;}\PYG{n}{i}\PYG{o}{+}\PYG{o}{+}\PYG{p}{)}\PYG{p}{\PYGZob{}}
                \PYG{n}{palabras}\PYG{o}{[}\PYG{n}{i}\PYG{o}{]}\PYG{o}{=}
                        \PYG{n}{flujoLectura}\PYG{p}{.}\PYG{n+na}{readLine}\PYG{p}{(}\PYG{p}{)}\PYG{p}{;}
        \PYG{p}{\PYGZcb{}}
        \PYG{n}{palabras}\PYG{o}{=}\PYG{k}{this}\PYG{p}{.}\PYG{n+na}{ordenar}\PYG{p}{(}\PYG{n}{palabras}\PYG{p}{)}\PYG{p}{;}
        \PYG{k}{for} \PYG{p}{(}\PYG{k+kt}{int} \PYG{n}{i}\PYG{o}{=}\PYG{l+m+mi}{0}\PYG{p}{;} \PYG{n}{i}\PYG{o}{\PYGZlt{}}\PYG{n}{palabras}\PYG{p}{.}\PYG{n+na}{length}\PYG{p}{;} \PYG{n}{i}\PYG{o}{+}\PYG{o}{+}\PYG{p}{)}\PYG{p}{\PYGZob{}}
                \PYG{n}{flujoEscritura}\PYG{p}{.}\PYG{n+na}{println}\PYG{p}{(}\PYG{n}{palabras}\PYG{o}{[}\PYG{n}{i}\PYG{o}{]}\PYG{p}{)}\PYG{p}{;}
        \PYG{p}{\PYGZcb{}}
        \PYG{n}{flujoEscritura}\PYG{p}{.}\PYG{n+na}{flush}\PYG{p}{(}\PYG{p}{)}\PYG{p}{;}
\PYG{p}{\PYGZcb{}}
\end{sphinxVerbatim}

Y finalmente solo habría que implementar un método en la petición que reciba un vector de \sphinxcode{\sphinxupquote{String}} (las palabras) y devuelva el mismo vector pero ordenado.


\section{Ejercicios}
\label{\detokenize{textos/tema3:ejercicios}}

\subsection{Sumas de verificación}
\label{\detokenize{textos/tema3:sumas-de-verificacion}}
Crear un servidor multihilo que permita a los clientes calcular «sumas de verificación» compuesta por la suma de los valores ASCII de los carácteres.

Así, la suma de verificación de la cadena «ABC» es 65+66+67=198.

El protocolo será el siguiente:
\begin{itemize}
\item {} 
El servidor esperá recibir una línea con solo un número positivo. Dicho número es la cantidad de líneas que nos va a enviar el cliente. Supongamos que envía 3.

\item {} 
Despues se deben recibir 3 líneas. En cada línea hay una sola palabra, de la cual el servidor calculará las sumas.

\item {} 
Despues el servidor contesta al cliente, y enviará 3 líneas separadas. En cada línea estará la suma de verificación. Las sumas se envían en el mismo orden que se recibieron.

\end{itemize}

A modo de ejemplo si el cliente envió estas líneas
\begin{itemize}
\item {} 
2

\item {} 
ABC

\item {} 
ZZ

\end{itemize}

Entonces el servidor luego enviará
\begin{itemize}
\item {} 
198 (que es la suma para ABC)

\item {} 
180 (que es la suma para ZZ)

\end{itemize}


\subsection{Solución a las sumas de verificación}
\label{\detokenize{textos/tema3:solucion-a-las-sumas-de-verificacion}}
El código siguiente ilustra la clase que hace sumas de verificación.

\begin{sphinxVerbatim}[commandchars=\\\{\}]
\PYG{k+kd}{public} \PYG{k+kd}{class} \PYG{n+nc}{Sumador} \PYG{p}{\PYGZob{}}

    \PYG{k+kd}{public} \PYG{k+kd}{static} \PYG{k+kt}{int} \PYG{n+nf}{sumaSimple}\PYG{p}{(}\PYG{n}{String} \PYG{n}{cad}\PYG{p}{)} \PYG{p}{\PYGZob{}}
        \PYG{k+kt}{int} \PYG{n}{suma} \PYG{o}{=} \PYG{l+m+mi}{0}\PYG{p}{;}
        \PYG{k}{for} \PYG{p}{(}\PYG{k+kt}{int} \PYG{n}{i} \PYG{o}{=} \PYG{l+m+mi}{0}\PYG{p}{;} \PYG{n}{i} \PYG{o}{\PYGZlt{}} \PYG{n}{cad}\PYG{p}{.}\PYG{n+na}{length}\PYG{p}{(}\PYG{p}{)}\PYG{p}{;} \PYG{n}{i}\PYG{o}{+}\PYG{o}{+}\PYG{p}{)} \PYG{p}{\PYGZob{}}
            \PYG{n}{suma} \PYG{o}{+}\PYG{o}{=} \PYG{n}{cad}\PYG{p}{.}\PYG{n+na}{codePointAt}\PYG{p}{(}\PYG{n}{i}\PYG{p}{)}\PYG{p}{;}
        \PYG{p}{\PYGZcb{}}
        \PYG{k}{return} \PYG{n}{suma}\PYG{p}{;}
    \PYG{p}{\PYGZcb{}}
\PYG{p}{\PYGZcb{}}
\end{sphinxVerbatim}

A continuación se muestra el servidor.

\begin{sphinxVerbatim}[commandchars=\\\{\}]
\PYG{k+kd}{public} \PYG{k+kd}{class} \PYG{n+nc}{Servidor} \PYG{p}{\PYGZob{}}

    \PYG{k+kd}{public} \PYG{k+kt}{void} \PYG{n+nf}{servir}\PYG{p}{(}\PYG{p}{)} \PYG{p}{\PYGZob{}}
        \PYG{n}{ServerSocket} \PYG{n}{serverSocket}\PYG{p}{;}
        \PYG{k+kd}{final} \PYG{k+kt}{int} \PYG{n}{PUERTO} \PYG{o}{=} \PYG{l+m+mi}{9876}\PYG{p}{;}
        \PYG{k}{try} \PYG{p}{\PYGZob{}}
            \PYG{n}{serverSocket} \PYG{o}{=} \PYG{k}{new} \PYG{n}{ServerSocket}\PYG{p}{(}\PYG{n}{PUERTO}\PYG{p}{)}\PYG{p}{;}
            \PYG{k}{while} \PYG{p}{(}\PYG{k+kc}{true}\PYG{p}{)} \PYG{p}{\PYGZob{}}
                \PYG{n}{Socket} \PYG{n}{conexion}\PYG{p}{;}
                \PYG{n}{conexion} \PYG{o}{=} \PYG{n}{serverSocket}\PYG{p}{.}\PYG{n+na}{accept}\PYG{p}{(}\PYG{p}{)}\PYG{p}{;}
                \PYG{n}{HiloConexion} \PYG{n}{hiloConexion}\PYG{p}{;}
                \PYG{n}{hiloConexion} \PYG{o}{=} \PYG{k}{new} \PYG{n}{HiloConexion}\PYG{p}{(}\PYG{n}{conexion}\PYG{p}{)}\PYG{p}{;}
                \PYG{n}{Thread} \PYG{n}{hilo} \PYG{o}{=} \PYG{k}{new} \PYG{n}{Thread}\PYG{p}{(}\PYG{n}{hiloConexion}\PYG{p}{)}\PYG{p}{;}
                \PYG{n}{hilo}\PYG{p}{.}\PYG{n+na}{start}\PYG{p}{(}\PYG{p}{)}\PYG{p}{;}
            \PYG{p}{\PYGZcb{}}
        \PYG{p}{\PYGZcb{}} \PYG{k}{catch} \PYG{p}{(}\PYG{n}{IOException} \PYG{n}{e}\PYG{p}{)} \PYG{p}{\PYGZob{}}
            \PYG{c+c1}{//No se pudo crear el server}
            \PYG{c+c1}{//socket porque no tenemos permisos}
            \PYG{c+c1}{//Se pudo crear pero no fuimos}
            \PYG{c+c1}{//capaces de enviar o recibir nada}
            \PYG{c+c1}{//Todo funcionaba pero el usuario}
            \PYG{c+c1}{//interrumpió}
        \PYG{p}{\PYGZcb{}}
    \PYG{p}{\PYGZcb{}}

    \PYG{k+kd}{public} \PYG{k+kd}{static} \PYG{k+kt}{void} \PYG{n+nf}{main}\PYG{p}{(}\PYG{n}{String}\PYG{o}{[}\PYG{o}{]} \PYG{n}{args}\PYG{p}{)} \PYG{p}{\PYGZob{}}
        \PYG{n}{Servidor} \PYG{n}{servidor}\PYG{p}{;}
        \PYG{n}{servidor} \PYG{o}{=} \PYG{k}{new} \PYG{n}{Servidor}\PYG{p}{(}\PYG{p}{)}\PYG{p}{;}
        \PYG{n}{servidor}\PYG{p}{.}\PYG{n+na}{servir}\PYG{p}{(}\PYG{p}{)}\PYG{p}{;}
    \PYG{p}{\PYGZcb{}}
\PYG{p}{\PYGZcb{}}
\end{sphinxVerbatim}

Y a  continuación el cliente:

\begin{sphinxVerbatim}[commandchars=\\\{\}]
\PYG{k+kd}{public} \PYG{k+kd}{class} \PYG{n+nc}{Cliente} \PYG{p}{\PYGZob{}}

    \PYG{k+kd}{public} \PYG{k+kt}{void} \PYG{n+nf}{verificarCadenas}\PYG{p}{(}\PYG{n}{BufferedReader} \PYG{n}{bfr}\PYG{p}{,}
                                 \PYG{n}{PrintWriter} \PYG{n}{pw}\PYG{p}{)} \PYG{k+kd}{throws} \PYG{n}{IOException} \PYG{p}{\PYGZob{}}
        \PYG{n}{pw}\PYG{p}{.}\PYG{n+na}{println}\PYG{p}{(}\PYG{l+m+mi}{2}\PYG{p}{)}\PYG{p}{;}
        \PYG{n}{pw}\PYG{p}{.}\PYG{n+na}{println}\PYG{p}{(}\PYG{l+s}{\PYGZdq{}}\PYG{l+s}{ABC}\PYG{l+s}{\PYGZdq{}}\PYG{p}{)}\PYG{p}{;}
        \PYG{n}{pw}\PYG{p}{.}\PYG{n+na}{println}\PYG{p}{(}\PYG{l+s}{\PYGZdq{}}\PYG{l+s}{ZZ}\PYG{l+s}{\PYGZdq{}}\PYG{p}{)}\PYG{p}{;}
        \PYG{n}{pw}\PYG{p}{.}\PYG{n+na}{flush}\PYG{p}{(}\PYG{p}{)}\PYG{p}{;}
        \PYG{n}{String} \PYG{n}{suma1} \PYG{o}{=} \PYG{n}{bfr}\PYG{p}{.}\PYG{n+na}{readLine}\PYG{p}{(}\PYG{p}{)}\PYG{p}{;}
        \PYG{n}{String} \PYG{n}{suma2} \PYG{o}{=} \PYG{n}{bfr}\PYG{p}{.}\PYG{n+na}{readLine}\PYG{p}{(}\PYG{p}{)}\PYG{p}{;}
        \PYG{n}{System}\PYG{p}{.}\PYG{n+na}{out}\PYG{p}{.}\PYG{n+na}{println}\PYG{p}{(}\PYG{n}{suma1}\PYG{p}{)}\PYG{p}{;}
        \PYG{n}{System}\PYG{p}{.}\PYG{n+na}{out}\PYG{p}{.}\PYG{n+na}{println}\PYG{p}{(}\PYG{n}{suma2}\PYG{p}{)}\PYG{p}{;}
    \PYG{p}{\PYGZcb{}}

    \PYG{k+kd}{public} \PYG{k+kd}{static} \PYG{k+kt}{void} \PYG{n+nf}{main}\PYG{p}{(}\PYG{n}{String}\PYG{o}{[}\PYG{o}{]} \PYG{n}{args}\PYG{p}{)} \PYG{p}{\PYGZob{}}
        \PYG{n}{Cliente} \PYG{n}{cliente}\PYG{p}{;}
        \PYG{n}{cliente} \PYG{o}{=} \PYG{k}{new} \PYG{n}{Cliente}\PYG{p}{(}\PYG{p}{)}\PYG{p}{;}
        \PYG{n}{InetSocketAddress} \PYG{n}{direccion}\PYG{p}{;}
        \PYG{n}{direccion} \PYG{o}{=} \PYG{k}{new} \PYG{n}{InetSocketAddress}\PYG{p}{(}\PYG{l+s}{\PYGZdq{}}\PYG{l+s}{10.13.0.100}\PYG{l+s}{\PYGZdq{}}\PYG{p}{,}
                                          \PYG{l+m+mi}{9876}\PYG{p}{)}\PYG{p}{;}
        \PYG{n}{Socket} \PYG{n}{conexion}\PYG{p}{;}
        \PYG{n}{conexion} \PYG{o}{=} \PYG{k}{new} \PYG{n}{Socket}\PYG{p}{(}\PYG{p}{)}\PYG{p}{;}
        \PYG{k}{try} \PYG{p}{\PYGZob{}}
            \PYG{n}{conexion}\PYG{p}{.}\PYG{n+na}{connect}\PYG{p}{(}\PYG{n}{direccion}\PYG{p}{)}\PYG{p}{;}
            \PYG{n}{BufferedReader} \PYG{n}{bfr}\PYG{p}{;}
            \PYG{n}{bfr} \PYG{o}{=} \PYG{n}{Utilidades}\PYG{p}{.}\PYG{n+na}{getFlujoLectura}\PYG{p}{(}\PYG{n}{conexion}\PYG{p}{)}\PYG{p}{;}
            \PYG{n}{PrintWriter} \PYG{n}{pw}\PYG{p}{;}
            \PYG{n}{pw} \PYG{o}{=} \PYG{n}{Utilidades}\PYG{p}{.}\PYG{n+na}{getFlujoEscritura}\PYG{p}{(}\PYG{n}{conexion}\PYG{p}{)}\PYG{p}{;}
            \PYG{n}{cliente}\PYG{p}{.}\PYG{n+na}{verificarCadenas}\PYG{p}{(}\PYG{n}{bfr}\PYG{p}{,} \PYG{n}{pw}\PYG{p}{)}\PYG{p}{;}
            \PYG{n}{pw}\PYG{p}{.}\PYG{n+na}{close}\PYG{p}{(}\PYG{p}{)}\PYG{p}{;}
            \PYG{n}{bfr}\PYG{p}{.}\PYG{n+na}{close}\PYG{p}{(}\PYG{p}{)}\PYG{p}{;}
            \PYG{n}{conexion}\PYG{p}{.}\PYG{n+na}{close}\PYG{p}{(}\PYG{p}{)}\PYG{p}{;}
        \PYG{p}{\PYGZcb{}} \PYG{k}{catch} \PYG{p}{(}\PYG{n}{IOException} \PYG{n}{e}\PYG{p}{)} \PYG{p}{\PYGZob{}}
            \PYG{c+c1}{//Quiza el servidor no está encendido}
            \PYG{c+c1}{//Quizá lo esté pero su cortafuegos}
            \PYG{c+c1}{//impide conexiones}
            \PYG{c+c1}{//...}
        \PYG{p}{\PYGZcb{}}
    \PYG{p}{\PYGZcb{}}
\PYG{p}{\PYGZcb{}}
\end{sphinxVerbatim}


\subsection{Servidor de bases de datos}
\label{\detokenize{textos/tema3:servidor-de-bases-de-datos}}
Se desea crear un pequeño servidor de datos en el cual los cliente puedan almacenar información de empleados. La unica información que se almacena es su código y su nombre (tampoco hay que preocuparse de qué hacer si se repite un código)

Un cliente puede enviar al servidor varias cosas:
\begin{itemize}
\item {} 
Puede enviar una linea con el texto «CREAR». En ese caso el cliente debe enviar despues dos líneas.  La primera línea contiene el codigo y la segunda el nombre.

\item {} 
Un cliente puede enviar la cadena «SELECT». En ese caso el servidor contesta con una línea numérica que indica cuantos empleados hay. Si contesta por ejemplo 2 significa que va a enviar 2 parejas de líneas (codigo, nombre). Si contesta 3 significa que va a enviar 6 líneas indicando los codigos y los nombres de los empleados.

\end{itemize}


\subsection{Servidor de eco}
\label{\detokenize{textos/tema3:servidor-de-eco}}
Se desea crear una aplicación cliente/servidor que consista de las siguientes
partes:
\begin{itemize}
\item {} 
Por un lado habrá un servidor multihilo que implementará la capacidad de «hacer el eco» de las cadenas que reciba. Así, el servidor escuchará una cadena y cuando la reciba devolverá exactamente la misma cadena al cliente. Se da por sentado que se recibirá una sola línea y que el servidor siempre tiene memoria suficiente para leer dicha línea.

\item {} 
Por otro lado, se desea tener un cliente para hacer pruebas conectándose a  dicha servidor. Un cliente simplemente se conecta al servidor, le envía una cadena y espera recibir la misma cadena.

\item {} 
En último lugar se desea tener una clase que compruebe la capacidad del servidor. Esta clase empezará lanzando muchos clientes a la vez y comprobará si ha habido alguna excepción. Si no la hay es que el servidor es capaz de manejar dicha cantidad de clientes y se incrementará la cantidad total de clientes simultáneos. El cliente irá incrementando la cantidad de clientes hasta lograr una excepción momento en el cual mostrará por pantalla la cantidad total de clientes que pudo manejar el servidor sin generar ninguna excepción.

\end{itemize}


\subsection{Comentarios generales a la solución}
\label{\detokenize{textos/tema3:comentarios-generales-a-la-solucion}}
Cuando probamos el servidor y el cliente en la misma máquina es muy difícil saber si los fallos se produjeron por culpa del servidor o del cliente, ya que ambos comparten una máquina con una misma RAM.

Por otro lado, incluso probando en máquinas distintas pueden obtenerse fallos por ejemplo al lanzar 3000 hilos y otro día al lanzar 4000 hilos. Recordemos que cuando lanzamos nuestro servidor puede haber otros procesos activos que consuman RAM y tiempo de CPU por lo que averiguar un valor real es \sphinxstylestrong{absolutamente imposible}


\subsection{Solución al servidor}
\label{\detokenize{textos/tema3:solucion-al-servidor}}
\begin{sphinxVerbatim}[commandchars=\\\{\}]
\PYG{k+kd}{public} \PYG{k+kd}{class} \PYG{n+nc}{Servidor} \PYG{p}{\PYGZob{}}

    \PYG{k+kd}{public} \PYG{k+kt}{void} \PYG{n+nf}{servir}\PYG{p}{(}\PYG{p}{)} \PYG{p}{\PYGZob{}}
        \PYG{n}{System}\PYG{p}{.}\PYG{n+na}{out}\PYG{p}{.}\PYG{n+na}{println}\PYG{p}{(}\PYG{l+s}{\PYGZdq{}}\PYG{l+s}{Servidor activo!}\PYG{l+s}{\PYGZdq{}}\PYG{p}{)}\PYG{p}{;}
        \PYG{n}{ServerSocket} \PYG{n}{serverSocket}\PYG{p}{;}
        \PYG{k+kd}{final} \PYG{k+kt}{int} \PYG{n}{PUERTO} \PYG{o}{=} \PYG{l+m+mi}{9876}\PYG{p}{;}
        \PYG{k}{try} \PYG{p}{\PYGZob{}}
            \PYG{n}{serverSocket} \PYG{o}{=} \PYG{k}{new} \PYG{n}{ServerSocket}\PYG{p}{(}\PYG{n}{PUERTO}\PYG{p}{)}\PYG{p}{;}
            \PYG{k}{while} \PYG{p}{(}\PYG{k+kc}{true}\PYG{p}{)} \PYG{p}{\PYGZob{}}
                \PYG{n}{Socket} \PYG{n}{conexion}\PYG{p}{;}
                \PYG{n}{conexion} \PYG{o}{=} \PYG{n}{serverSocket}\PYG{p}{.}\PYG{n+na}{accept}\PYG{p}{(}\PYG{p}{)}\PYG{p}{;}
                \PYG{n}{HiloConexion} \PYG{n}{hiloConexion}\PYG{p}{;}
                \PYG{n}{hiloConexion} \PYG{o}{=} \PYG{k}{new} \PYG{n}{HiloConexion}\PYG{p}{(}\PYG{n}{conexion}\PYG{p}{)}\PYG{p}{;}
                \PYG{n}{Thread} \PYG{n}{hilo} \PYG{o}{=} \PYG{k}{new} \PYG{n}{Thread}\PYG{p}{(}\PYG{n}{hiloConexion}\PYG{p}{)}\PYG{p}{;}
                \PYG{n}{hilo}\PYG{p}{.}\PYG{n+na}{start}\PYG{p}{(}\PYG{p}{)}\PYG{p}{;}
            \PYG{p}{\PYGZcb{}}
        \PYG{p}{\PYGZcb{}} \PYG{k}{catch} \PYG{p}{(}\PYG{n}{IOException} \PYG{n}{e}\PYG{p}{)} \PYG{p}{\PYGZob{}}
            \PYG{c+c1}{//No se pudo crear el server}
            \PYG{c+c1}{//socket porque no tenemos permisos}
            \PYG{c+c1}{//Se pudo crear pero no fuimos}
            \PYG{c+c1}{//capaces de enviar o recibir nada}
            \PYG{c+c1}{//Todo funcionaba pero el usuario}
            \PYG{c+c1}{//interrumpió}
            \PYG{n}{System}\PYG{p}{.}\PYG{n+na}{out}\PYG{p}{.}\PYG{n+na}{println}\PYG{p}{(}\PYG{l+s}{\PYGZdq{}}\PYG{l+s}{Error en conexion }\PYG{l+s}{\PYGZdq{}} \PYG{o}{+}
                               \PYG{l+s}{\PYGZdq{}}\PYG{l+s}{o al crear los hilos o al procesar E/S}\PYG{l+s}{\PYGZdq{}}\PYG{p}{)}\PYG{p}{;}
        \PYG{p}{\PYGZcb{}}
    \PYG{p}{\PYGZcb{}}

    \PYG{k+kd}{public} \PYG{k+kd}{static} \PYG{k+kt}{void} \PYG{n+nf}{main}\PYG{p}{(}\PYG{n}{String}\PYG{o}{[}\PYG{o}{]} \PYG{n}{args}\PYG{p}{)} \PYG{p}{\PYGZob{}}
        \PYG{n}{Servidor} \PYG{n}{servidor}\PYG{p}{;}
        \PYG{n}{servidor} \PYG{o}{=} \PYG{k}{new} \PYG{n}{Servidor}\PYG{p}{(}\PYG{p}{)}\PYG{p}{;}
        \PYG{n}{servidor}\PYG{p}{.}\PYG{n+na}{servir}\PYG{p}{(}\PYG{p}{)}\PYG{p}{;}
    \PYG{p}{\PYGZcb{}}
\PYG{p}{\PYGZcb{}}
\end{sphinxVerbatim}


\subsection{Solución HiloConexion}
\label{\detokenize{textos/tema3:solucion-hiloconexion}}
\begin{sphinxVerbatim}[commandchars=\\\{\}]
\PYG{k+kd}{public} \PYG{k+kd}{class} \PYG{n+nc}{HiloConexion} \PYG{k+kd}{implements} \PYG{n}{Runnable} \PYG{p}{\PYGZob{}}

    \PYG{n}{BufferedReader} \PYG{n}{bfr}\PYG{p}{;}

    \PYG{n}{PrintWriter} \PYG{n}{pw}\PYG{p}{;}

    \PYG{n}{Socket} \PYG{n}{socket}\PYG{p}{;}

    \PYG{k+kd}{public} \PYG{n+nf}{HiloConexion}\PYG{p}{(}\PYG{n}{Socket} \PYG{n}{socket}\PYG{p}{)} \PYG{p}{\PYGZob{}}
        \PYG{k}{this}\PYG{p}{.}\PYG{n+na}{socket} \PYG{o}{=} \PYG{n}{socket}\PYG{p}{;}
    \PYG{p}{\PYGZcb{}}

    \PYG{k+kd}{public} \PYG{k+kt}{void} \PYG{n+nf}{run}\PYG{p}{(}\PYG{p}{)} \PYG{p}{\PYGZob{}}
        \PYG{k}{try} \PYG{p}{\PYGZob{}}
            \PYG{c+cm}{/* Nuestro hilo se limita a}
\PYG{c+cm}{             * recibir una línea y reenviarla}
\PYG{c+cm}{             * de vuelta al cliente}
\PYG{c+cm}{             */}
            \PYG{n}{bfr} \PYG{o}{=} \PYG{n}{Utilidades}\PYG{p}{.}\PYG{n+na}{getFlujoLectura}\PYG{p}{(}\PYG{k}{this}\PYG{p}{.}\PYG{n+na}{socket}\PYG{p}{)}\PYG{p}{;}
            \PYG{n}{pw} \PYG{o}{=} \PYG{n}{Utilidades}\PYG{p}{.}\PYG{n+na}{getFlujoEscritura}\PYG{p}{(}\PYG{k}{this}\PYG{p}{.}\PYG{n+na}{socket}\PYG{p}{)}\PYG{p}{;}
            \PYG{n}{String} \PYG{n}{lineaRecibida}\PYG{p}{;}
            \PYG{n}{lineaRecibida} \PYG{o}{=} \PYG{n}{bfr}\PYG{p}{.}\PYG{n+na}{readLine}\PYG{p}{(}\PYG{p}{)}\PYG{p}{;}
            \PYG{n}{System}\PYG{p}{.}\PYG{n+na}{out}\PYG{p}{.}\PYG{n+na}{print}\PYG{p}{(}
                \PYG{n}{Thread}\PYG{p}{.}\PYG{n+na}{currentThread}\PYG{p}{(}\PYG{p}{)}\PYG{p}{.}\PYG{n+na}{getName}\PYG{p}{(}\PYG{p}{)}\PYG{p}{)}\PYG{p}{;}
            \PYG{n}{System}\PYG{p}{.}\PYG{n+na}{out}\PYG{p}{.}\PYG{n+na}{println}\PYG{p}{(}\PYG{l+s}{\PYGZdq{}}\PYG{l+s}{ recibio:}\PYG{l+s}{\PYGZdq{}} \PYG{o}{+} \PYG{n}{lineaRecibida}\PYG{p}{)}\PYG{p}{;}
            \PYG{n}{pw}\PYG{p}{.}\PYG{n+na}{println}\PYG{p}{(}\PYG{n}{lineaRecibida}\PYG{p}{)}\PYG{p}{;}
            \PYG{n}{pw}\PYG{p}{.}\PYG{n+na}{flush}\PYG{p}{(}\PYG{p}{)}\PYG{p}{;}
        \PYG{p}{\PYGZcb{}} \PYG{k}{catch} \PYG{p}{(}\PYG{n}{IOException} \PYG{n}{e}\PYG{p}{)} \PYG{p}{\PYGZob{}}
            \PYG{n}{System}\PYG{p}{.}\PYG{n+na}{out}\PYG{p}{.}\PYG{n+na}{println}\PYG{p}{(}\PYG{l+s}{\PYGZdq{}}\PYG{l+s}{Hubo un fallo al enviar/recibir datos}\PYG{l+s}{\PYGZdq{}}\PYG{p}{)}\PYG{p}{;}
        \PYG{p}{\PYGZcb{}}
    \PYG{p}{\PYGZcb{}}
    \PYG{c+c1}{//Fin del run}
\PYG{p}{\PYGZcb{}}
\end{sphinxVerbatim}


\subsection{Solución Cliente}
\label{\detokenize{textos/tema3:solucion-cliente}}
\begin{sphinxVerbatim}[commandchars=\\\{\}]
\PYG{k+kd}{public} \PYG{k+kd}{class} \PYG{n+nc}{Cliente} \PYG{k+kd}{implements} \PYG{n}{Runnable} \PYG{p}{\PYGZob{}}

    \PYG{n}{Socket} \PYG{n}{conexion}\PYG{p}{;}

    \PYG{n}{Random} \PYG{n}{generador}\PYG{p}{;}

    \PYG{n}{BufferedReader} \PYG{n}{bfr}\PYG{p}{;}

    \PYG{n}{PrintWriter} \PYG{n}{pw}\PYG{p}{;}

    \PYG{n}{String}\PYG{o}{[}\PYG{o}{]} \PYG{n}{palabras} \PYG{o}{=} \PYG{p}{\PYGZob{}} \PYG{l+s}{\PYGZdq{}}\PYG{l+s}{Hola}\PYG{l+s}{\PYGZdq{}}\PYG{p}{,} \PYG{l+s}{\PYGZdq{}}\PYG{l+s}{mundo}\PYG{l+s}{\PYGZdq{}}\PYG{p}{,} \PYG{l+s}{\PYGZdq{}}\PYG{l+s}{Java}\PYG{l+s}{\PYGZdq{}}\PYG{p}{,} \PYG{l+s}{\PYGZdq{}}\PYG{l+s}{hilo}\PYG{l+s}{\PYGZdq{}} \PYG{p}{\PYGZcb{}}\PYG{p}{;}

    \PYG{k+kt}{int} \PYG{n}{numHilo}\PYG{p}{;}

    \PYG{c+cm}{/* Esta variable se puede comprobar por}
\PYG{c+cm}{     * parte del lanzador para ver si hubo un fallo}
\PYG{c+cm}{     */}
    \PYG{k+kt}{boolean} \PYG{n}{algoFallo} \PYG{o}{=} \PYG{k+kc}{false}\PYG{p}{;}

    \PYG{k+kd}{public} \PYG{n+nf}{Cliente}\PYG{p}{(}\PYG{p}{)} \PYG{p}{\PYGZob{}}
        \PYG{n}{generador} \PYG{o}{=} \PYG{k}{new} \PYG{n}{Random}\PYG{p}{(}\PYG{p}{)}\PYG{p}{;}
        \PYG{n}{InetSocketAddress} \PYG{n}{direccion}\PYG{p}{;}
        \PYG{n}{direccion} \PYG{o}{=} \PYG{k}{new} \PYG{n}{InetSocketAddress}\PYG{p}{(}\PYG{l+s}{\PYGZdq{}}\PYG{l+s}{localhost}\PYG{l+s}{\PYGZdq{}}\PYG{p}{,}
                                          \PYG{l+m+mi}{9876}\PYG{p}{)}\PYG{p}{;}
        \PYG{n}{Socket} \PYG{n}{conexion}\PYG{p}{;}
        \PYG{n}{conexion} \PYG{o}{=} \PYG{k}{new} \PYG{n}{Socket}\PYG{p}{(}\PYG{p}{)}\PYG{p}{;}
        \PYG{k}{try} \PYG{p}{\PYGZob{}}
            \PYG{n}{conexion}\PYG{p}{.}\PYG{n+na}{connect}\PYG{p}{(}\PYG{n}{direccion}\PYG{p}{)}\PYG{p}{;}
            \PYG{n}{bfr} \PYG{o}{=} \PYG{n}{Utilidades}\PYG{p}{.}\PYG{n+na}{getFlujoLectura}\PYG{p}{(}\PYG{n}{conexion}\PYG{p}{)}\PYG{p}{;}
            \PYG{n}{pw} \PYG{o}{=} \PYG{n}{Utilidades}\PYG{p}{.}\PYG{n+na}{getFlujoEscritura}\PYG{p}{(}\PYG{n}{conexion}\PYG{p}{)}\PYG{p}{;}
        \PYG{p}{\PYGZcb{}} \PYG{k}{catch} \PYG{p}{(}\PYG{n}{IOException} \PYG{n}{e}\PYG{p}{)} \PYG{p}{\PYGZob{}}
            \PYG{n}{algoFallo} \PYG{o}{=} \PYG{k+kc}{true}\PYG{p}{;}
        \PYG{p}{\PYGZcb{}}
        \PYG{c+c1}{//Fin del catch}
    \PYG{p}{\PYGZcb{}}

    \PYG{k+kd}{public} \PYG{k+kt}{int} \PYG{n+nf}{getNumHilo}\PYG{p}{(}\PYG{p}{)} \PYG{p}{\PYGZob{}}
        \PYG{k}{return} \PYG{n}{numHilo}\PYG{p}{;}
    \PYG{p}{\PYGZcb{}}

    \PYG{k+kd}{public} \PYG{k+kt}{void} \PYG{n+nf}{setNumHilo}\PYG{p}{(}\PYG{k+kt}{int} \PYG{n}{numHilo}\PYG{p}{)} \PYG{p}{\PYGZob{}}
        \PYG{k}{this}\PYG{p}{.}\PYG{n+na}{numHilo} \PYG{o}{=} \PYG{n}{numHilo}\PYG{p}{;}
    \PYG{p}{\PYGZcb{}}

    \PYG{k+kd}{public} \PYG{k+kt}{boolean} \PYG{n+nf}{servidorFunciona}\PYG{p}{(}\PYG{p}{)} \PYG{p}{\PYGZob{}}
        \PYG{c+cm}{/* Elegimos una palabra al azar*/}
        \PYG{n}{String} \PYG{n}{palabra} \PYG{o}{=} \PYG{n}{palabras}\PYG{o}{[}\PYG{n}{generador}\PYG{p}{.}\PYG{n+na}{nextInt}\PYG{p}{(}
                                      \PYG{n}{palabras}\PYG{p}{.}\PYG{n+na}{length}\PYG{p}{)}\PYG{o}{]}\PYG{p}{;}
        \PYG{n}{String} \PYG{n}{eco}\PYG{p}{;}
        \PYG{k}{try} \PYG{p}{\PYGZob{}}
            \PYG{c+cm}{/* Si no pudimos obtener un flujo}
\PYG{c+cm}{             * de lectura o escritura con}
\PYG{c+cm}{             * el servidor es que estaba}
\PYG{c+cm}{             * colapsado}
\PYG{c+cm}{             */}
            \PYG{k}{if} \PYG{p}{(}\PYG{p}{(}\PYG{n}{bfr} \PYG{o}{=}\PYG{o}{=} \PYG{k+kc}{null}\PYG{p}{)} \PYG{o}{|}\PYG{o}{|} \PYG{p}{(}\PYG{n}{pw} \PYG{o}{=}\PYG{o}{=} \PYG{k+kc}{null}\PYG{p}{)}\PYG{p}{)} \PYG{p}{\PYGZob{}}
                \PYG{c+cm}{/* Indicamos que algo fallo*/}
                \PYG{n}{algoFallo} \PYG{o}{=} \PYG{k+kc}{true}\PYG{p}{;}
                \PYG{c+cm}{/* Y decimos que este}
\PYG{c+cm}{                 * metodo no ha funcionado}
\PYG{c+cm}{                 */}
                \PYG{k}{return} \PYG{k+kc}{false}\PYG{p}{;}
            \PYG{p}{\PYGZcb{}}
            \PYG{n}{pw}\PYG{p}{.}\PYG{n+na}{println}\PYG{p}{(}\PYG{n}{palabra}\PYG{p}{)}\PYG{p}{;}
            \PYG{n}{pw}\PYG{p}{.}\PYG{n+na}{flush}\PYG{p}{(}\PYG{p}{)}\PYG{p}{;}
            \PYG{n}{eco} \PYG{o}{=} \PYG{n}{bfr}\PYG{p}{.}\PYG{n+na}{readLine}\PYG{p}{(}\PYG{p}{)}\PYG{p}{;}
            \PYG{k}{if} \PYG{p}{(}\PYG{n}{eco}\PYG{p}{.}\PYG{n+na}{equals}\PYG{p}{(}\PYG{n}{palabra}\PYG{p}{)}\PYG{p}{)} \PYG{p}{\PYGZob{}}
                \PYG{n}{System}\PYG{p}{.}\PYG{n+na}{out}\PYG{p}{.}\PYG{n+na}{println}\PYG{p}{(}\PYG{l+s}{\PYGZdq{}}\PYG{l+s}{Hilo }\PYG{l+s}{\PYGZdq{}} \PYG{o}{+} \PYG{n}{numHilo} \PYG{o}{+}
                                   \PYG{l+s}{\PYGZdq{}}\PYG{l+s}{ recibio bien:}\PYG{l+s}{\PYGZdq{}} \PYG{o}{+} \PYG{n}{eco}\PYG{p}{)}\PYG{p}{;}
                \PYG{k}{return} \PYG{k+kc}{true}\PYG{p}{;}
            \PYG{p}{\PYGZcb{}}
            \PYG{c+c1}{//Fin del if}
        \PYG{p}{\PYGZcb{}} \PYG{k}{catch} \PYG{p}{(}\PYG{n}{IOException} \PYG{n}{e}\PYG{p}{)} \PYG{p}{\PYGZob{}}
            \PYG{k}{return} \PYG{k+kc}{false}\PYG{p}{;}
        \PYG{p}{\PYGZcb{}}
        \PYG{c+cm}{/*Si se llega a este punto es porque}
\PYG{c+cm}{         *la palabra devuelta no fue la}
\PYG{c+cm}{         *que enviamos, o sea que el servidor falló}
\PYG{c+cm}{        */}
        \PYG{k}{return} \PYG{k+kc}{false}\PYG{p}{;}
    \PYG{p}{\PYGZcb{}}

    \PYG{n+nd}{@Override}
    \PYG{k+kd}{public} \PYG{k+kt}{void} \PYG{n+nf}{run}\PYG{p}{(}\PYG{p}{)} \PYG{p}{\PYGZob{}}
        \PYG{k}{if} \PYG{p}{(}\PYG{o}{!}\PYG{n}{servidorFunciona}\PYG{p}{(}\PYG{p}{)}\PYG{p}{)} \PYG{p}{\PYGZob{}}
            \PYG{c+cm}{/* Imprimimos un mensaje y}
\PYG{c+cm}{             * además cambiamos la variable}
\PYG{c+cm}{             * que indica que hubo un fallo}
\PYG{c+cm}{             */}
            \PYG{n}{System}\PYG{p}{.}\PYG{n+na}{out}\PYG{p}{.}\PYG{n+na}{println}\PYG{p}{(}\PYG{l+s}{\PYGZdq{}}\PYG{l+s}{Fallo en el hilo }\PYG{l+s}{\PYGZdq{}} \PYG{o}{+} \PYG{n}{numHilo}\PYG{p}{)}\PYG{p}{;}
            \PYG{n}{algoFallo} \PYG{o}{=} \PYG{k+kc}{true}\PYG{p}{;}
        \PYG{p}{\PYGZcb{}}
    \PYG{p}{\PYGZcb{}}

    \PYG{c+c1}{//Fin del run}
    \PYG{k+kd}{public} \PYG{k+kt}{boolean} \PYG{n+nf}{huboFallo}\PYG{p}{(}\PYG{p}{)} \PYG{p}{\PYGZob{}}
        \PYG{k}{return} \PYG{n}{algoFallo}\PYG{p}{;}
    \PYG{p}{\PYGZcb{}}
\PYG{p}{\PYGZcb{}}
\end{sphinxVerbatim}


\subsection{Solución LanzadorClientes}
\label{\detokenize{textos/tema3:solucion-lanzadorclientes}}
\begin{sphinxVerbatim}[commandchars=\\\{\}]
\PYG{k+kd}{public} \PYG{k+kd}{class} \PYG{n+nc}{LanzadorClientes} \PYG{p}{\PYGZob{}}

    \PYG{k+kd}{public} \PYG{k+kt}{boolean} \PYG{n+nf}{servidorAtendioClientes}\PYG{p}{(}
        \PYG{k+kt}{int} \PYG{n}{numClientes}\PYG{p}{)} \PYG{p}{\PYGZob{}}
        \PYG{c+c1}{//En este metodo lanzamos x clientes y luego comprobamos}
        \PYG{c+c1}{//si todos funcionaron bien. Si todo fue bien}
        \PYG{c+c1}{//devolvemos true y si no devolveremos false}
        \PYG{n}{Thread}\PYG{o}{[}\PYG{o}{]} \PYG{n}{hilos} \PYG{o}{=} \PYG{k}{new} \PYG{n}{Thread}\PYG{o}{[}\PYG{n}{numClientes}\PYG{o}{]}\PYG{p}{;}
        \PYG{n}{Cliente}\PYG{o}{[}\PYG{o}{]} \PYG{n}{clientes} \PYG{o}{=} \PYG{k}{new} \PYG{n}{Cliente}\PYG{o}{[}\PYG{n}{numClientes}\PYG{o}{]}\PYG{p}{;}
        \PYG{k}{for} \PYG{p}{(}\PYG{k+kt}{int} \PYG{n}{i} \PYG{o}{=} \PYG{l+m+mi}{0}\PYG{p}{;} \PYG{n}{i} \PYG{o}{\PYGZlt{}} \PYG{n}{numClientes}\PYG{p}{;} \PYG{n}{i}\PYG{o}{+}\PYG{o}{+}\PYG{p}{)} \PYG{p}{\PYGZob{}}
            \PYG{n}{Cliente} \PYG{n}{cliente} \PYG{o}{=} \PYG{k}{new} \PYG{n}{Cliente}\PYG{p}{(}\PYG{p}{)}\PYG{p}{;}
            \PYG{n}{cliente}\PYG{p}{.}\PYG{n+na}{setNumHilo}\PYG{p}{(}\PYG{n}{i}\PYG{p}{)}\PYG{p}{;}
            \PYG{n}{Thread} \PYG{n}{hiloAsociado} \PYG{o}{=} \PYG{k}{new} \PYG{n}{Thread}\PYG{p}{(}\PYG{n}{cliente}\PYG{p}{)}\PYG{p}{;}
            \PYG{n}{hilos}\PYG{o}{[}\PYG{n}{i}\PYG{o}{]} \PYG{o}{=} \PYG{n}{hiloAsociado}\PYG{p}{;}
            \PYG{n}{hiloAsociado}\PYG{p}{.}\PYG{n+na}{start}\PYG{p}{(}\PYG{p}{)}\PYG{p}{;}
            \PYG{n}{clientes}\PYG{o}{[}\PYG{n}{i}\PYG{o}{]} \PYG{o}{=} \PYG{n}{cliente}\PYG{p}{;}
        \PYG{p}{\PYGZcb{}}
        \PYG{n}{System}\PYG{p}{.}\PYG{n+na}{out}\PYG{p}{.}\PYG{n+na}{println}\PYG{p}{(}\PYG{l+s}{\PYGZdq{}}\PYG{l+s}{Lanzados!}\PYG{l+s}{\PYGZdq{}}\PYG{p}{)}\PYG{p}{;}
        \PYG{c+cm}{/* Esperamos que todos los hilos acaben}
\PYG{c+cm}{         * dandoles un plazo maximo mas bien pequeño}
\PYG{c+cm}{         * Si en ese tiempo no se completó}
\PYG{c+cm}{         * una operación tan simple, probablemente}
\PYG{c+cm}{         * el servidor falló*/}
        \PYG{k}{for} \PYG{p}{(}\PYG{k+kt}{int} \PYG{n}{i} \PYG{o}{=} \PYG{l+m+mi}{0}\PYG{p}{;} \PYG{n}{i} \PYG{o}{\PYGZlt{}} \PYG{n}{numClientes}\PYG{p}{;} \PYG{n}{i}\PYG{o}{+}\PYG{o}{+}\PYG{p}{)} \PYG{p}{\PYGZob{}}
            \PYG{k}{try} \PYG{p}{\PYGZob{}}
                \PYG{n}{hilos}\PYG{o}{[}\PYG{n}{i}\PYG{o}{]}\PYG{p}{.}\PYG{n+na}{join}\PYG{p}{(}\PYG{p}{)}\PYG{p}{;}
            \PYG{p}{\PYGZcb{}} \PYG{k}{catch} \PYG{p}{(}\PYG{n}{InterruptedException} \PYG{n}{e}\PYG{p}{)} \PYG{p}{\PYGZob{}}
                \PYG{n}{System}\PYG{p}{.}\PYG{n+na}{out}\PYG{p}{.}\PYG{n+na}{println}\PYG{p}{(}\PYG{l+s}{\PYGZdq{}}\PYG{l+s}{Se interrumpió un hilo por parte }\PYG{l+s}{\PYGZdq{}}
                                   \PYG{o}{+} \PYG{l+s}{\PYGZdq{}}\PYG{l+s}{de alguna clase del cliente }\PYG{l+s}{\PYGZdq{}}\PYG{p}{)}\PYG{p}{;}
            \PYG{p}{\PYGZcb{}}
        \PYG{p}{\PYGZcb{}}
        \PYG{c+cm}{/* Comprobamos si todos los hilos están bien}
\PYG{c+cm}{         * y en cuanto uno sufra un fallo podemos}
\PYG{c+cm}{         * asumir que el servidor no pudo atender tantos}
\PYG{c+cm}{         * clientes*/}
        \PYG{k}{for} \PYG{p}{(}\PYG{k+kt}{int} \PYG{n}{i} \PYG{o}{=} \PYG{l+m+mi}{0}\PYG{p}{;} \PYG{n}{i} \PYG{o}{\PYGZlt{}} \PYG{n}{numClientes}\PYG{p}{;} \PYG{n}{i}\PYG{o}{+}\PYG{o}{+}\PYG{p}{)} \PYG{p}{\PYGZob{}}
            \PYG{k}{if} \PYG{p}{(}\PYG{n}{clientes}\PYG{o}{[}\PYG{n}{i}\PYG{o}{]}\PYG{p}{.}\PYG{n+na}{huboFallo}\PYG{p}{(}\PYG{p}{)}\PYG{p}{)} \PYG{p}{\PYGZob{}}
                \PYG{k}{return} \PYG{k+kc}{false}\PYG{p}{;}
            \PYG{p}{\PYGZcb{}}
        \PYG{p}{\PYGZcb{}}
        \PYG{c+c1}{//pudo atender a todos los clientes a la vez}
        \PYG{k}{return} \PYG{k+kc}{true}\PYG{p}{;}
    \PYG{p}{\PYGZcb{}}

    \PYG{k+kd}{public} \PYG{k+kd}{static} \PYG{k+kt}{void} \PYG{n+nf}{main}\PYG{p}{(}\PYG{n}{String}\PYG{o}{[}\PYG{o}{]} \PYG{n}{args}\PYG{p}{)} \PYG{p}{\PYGZob{}}
        \PYG{n}{LanzadorClientes} \PYG{n}{lanzador} \PYG{o}{=} \PYG{k}{new}
        \PYG{n}{LanzadorClientes}\PYG{p}{(}\PYG{p}{)}\PYG{p}{;}
        \PYG{c+cm}{/* Vamos probando a lanzar muchos clientes}
\PYG{c+cm}{         * hasta que forcemos un fallo}
\PYG{c+cm}{         */}
        \PYG{k}{for} \PYG{p}{(}\PYG{k+kt}{int} \PYG{n}{i} \PYG{o}{=} \PYG{l+m+mi}{1}\PYG{p}{;} \PYG{n}{i} \PYG{o}{\PYGZlt{}} \PYG{l+m+mi}{1000}\PYG{p}{;} \PYG{n}{i}\PYG{o}{+}\PYG{o}{+}\PYG{p}{)} \PYG{p}{\PYGZob{}}
            \PYG{k+kt}{boolean} \PYG{n}{todoOK}\PYG{p}{;}
            \PYG{k+kt}{int} \PYG{n}{numClientes} \PYG{o}{=} \PYG{n}{i} \PYG{o}{*} \PYG{l+m+mi}{1000}\PYG{p}{;}
            \PYG{n}{System}\PYG{p}{.}\PYG{n+na}{out}\PYG{p}{.}\PYG{n+na}{println}\PYG{p}{(}\PYG{l+s}{\PYGZdq{}}\PYG{l+s}{Lanzando }\PYG{l+s}{\PYGZdq{}} \PYG{o}{+} \PYG{n}{numClientes} \PYG{o}{+}
                               \PYG{l+s}{\PYGZdq{}}\PYG{l+s}{ clientes...}\PYG{l+s}{\PYGZdq{}}\PYG{p}{)}\PYG{p}{;}
            \PYG{n}{todoOK} \PYG{o}{=} \PYG{n}{lanzador}\PYG{p}{.}\PYG{n+na}{servidorAtendioClientes}\PYG{p}{(}
                         \PYG{n}{numClientes}\PYG{p}{)}\PYG{p}{;}
            \PYG{c+cm}{/* Si algo no fue bien, indicamos la cantidad}
\PYG{c+cm}{             * de clientes con que se produjo el fallo}
\PYG{c+cm}{             */}
            \PYG{k}{if} \PYG{p}{(}\PYG{o}{!}\PYG{n}{todoOK}\PYG{p}{)} \PYG{p}{\PYGZob{}}
                \PYG{n}{System}\PYG{p}{.}\PYG{n+na}{out}\PYG{p}{.}\PYG{n+na}{println}\PYG{p}{(}\PYG{l+s}{\PYGZdq{}}\PYG{l+s}{El servidor pareció fallar con:}\PYG{l+s}{\PYGZdq{}}
                                   \PYG{o}{+} \PYG{n}{numClientes}\PYG{p}{)}\PYG{p}{;}
                \PYG{k}{return}\PYG{p}{;}
            \PYG{p}{\PYGZcb{}}
            \PYG{c+c1}{//Fin del if}
        \PYG{p}{\PYGZcb{}}
        \PYG{c+c1}{//Fin del main}
    \PYG{p}{\PYGZcb{}}
\PYG{p}{\PYGZcb{}}
\end{sphinxVerbatim}


\chapter{Generación de servicios en red}
\label{\detokenize{textos/tema4:generacion-de-servicios-en-red}}\label{\detokenize{textos/tema4::doc}}
Una vez vistas las comunicaciones en Java a través de sockets podemos utilizar dicho conocimiento para dar un paso más allá y acceder a servicios de red habituales. En las siguientes secciones desglosamos el funcionamiento de los protocoles y proporcionamos algunas clases básicas para realizar dichos accesos.


\section{Telnet/SSH}
\label{\detokenize{textos/tema4:telnet-ssh}}
Aunque hoy está prácticamente obsoleto, Telnet fue un servicio de administración remota que permitía conectarse a máquinas UNIX desde un sitio remoto, permitiendo enviar comandos y ver el resultado de dichos comandos en nuestro terminal \sphinxstyleemphasis{como si estuviésemos sentados en la máquina administrada}

El gran problema de Telnet fue \sphinxstylestrong{la seguridad}. Telnet no enviaba datos cifrados por lo que cualquier persona con un \sphinxstyleemphasis{sniffer} podía capturar el tráfico de una sesión y ver no solamente los comandos sino también los usuarios y contraseñas enviados a través de la red. Por ello, Telnet prácticamente no se usa hoy en día y ha sido sustituido «de facto» por SSH, que hace lo mismo pero cifrando la comunicación con criptografía de clave pública.


\section{FTP}
\label{\detokenize{textos/tema4:ftp}}
FTP significa «File Transfer Protocol» y es un protocolo orientado a comandos pensado para descargar ficheros. Podría decirse que también tiende a desaparecer, ya que cada vez más a menudo se usa el navegador (con HTTP) para descargar ficheros. En realidad, algunos navegadores, como Mozilla Firefox siguen implementando el protocolo FTP, permitiendo así el descargar ficheros como si en realidad estuviésemos usando comandos.

Cabe destacar que FTP tiene dos modalidades de uso: pasivo y activo. La diferencia entre activo y pasivo está en quien inicia las conexiones. Por defecto, los router no suelen permitir la entrada de conexiones iniciadas desde el exterior (aunque sí permiten que entren datos del exterior si desde el interior hemos iniciado previamente la conexion). Cuando se definió FTP como protocolo, se daba por sentado que todo el mundo aceptaría conexiones siempre, por lo cual en el FTP normal (el activo) se permite que el servidor inicie una conexión con nosotros para enviarnos un fichero. En FTP hay dos conexiones, una para enviar comandos (que inician los cliente) y otra (que inicia el servidor) para recibir el archivo. Podría decirse que en el FTP normal/activo «no se descarga» sino que «el servidor envía un fichero».

Cuando la seguridad empezó a preocupar a todo el mundo los fabricantes y compañías telefónicas empezaron a distribuir routers en los que por defecto no se aceptaban conexiones empezadas fuera de nuestra red, así que de repente el FTP empezó mostrar un comportamiento peculiar que podría resumirse en la frase  «se pueden enviar comandos pero no descargar ficheros». Ese error tiene su lógica porque la conexión de comandos la inicia el cliente pero el servidor inicia la conexión (mejor dicho, lo intenta) para enviar el fichero al cliente. Debido a estos problemas  se desarrolló el  «modo pasivo», que consiste en que ahora el cliente inicia una conexión para enviar comandos e inicia otra conexión para descargar el fichero. Como la conexión de descarga se inicia «dentro de la red», el router autoriza la salida, toma nota de la conexión y cuando llega la respuesta, la deja pasar (porque observa que la conexión no se ha iniciado fuera, sino que es la respuesta a una conexión iniciada dentro). Como puede deducirse se llama pasivo porque el servidor ya no es activo, sino que se limita a recibir conexiones.


\section{HTTP}
\label{\detokenize{textos/tema4:http}}
El protocolo para la transferencia de hipertexto (HyperText Transfer Protocol) es un protocolo que en su momento se diseñó para que los navegadores (clientes web) se conectasen a servidores y descargasen archivos HTML. Sin embargo, su uso se ha popularizado en otros ámbitos como son la creación de aplicaciones web, es decir aplicaciones pensadas para ser manejadas desde un navegador.

Desde mayo de 2015, HTTP va por su versión 2.0 que incluye mejoras en la latencia de los tiempos de respuesta y mejoras en el empaquetado de los datos.

HTTP es un protocolo basado en texto.  A modo de ejemplo se incluye a continuación un ejemplo de petición (copiada de Wikipedia):

\begin{sphinxVerbatim}[commandchars=\\\{\}]
GET /index.html HTTP/1.1
Host: www.example.com
Referer: www.google.com
User\PYGZhy{}Agent: Mozilla/5.0 (X11; Linux x86\PYGZus{}64; rv:45.0) Gecko/20100101 Firefox/45.0
Connection: keep\PYGZhy{}alive
[Línea en blanco]
\end{sphinxVerbatim}

Y a continuación se muestra una respuesta (ejemplo copiado también de Wikipedia):

\begin{sphinxVerbatim}[commandchars=\\\{\}]
HTTP/1.1 200 OK
Date: Fri, 31 Dec 2003 23:59:59 GMT
Content\PYGZhy{}Type: text/html
Content\PYGZhy{}Length: 1221

\PYGZlt{}html lang=\PYGZdq{}es\PYGZdq{}\PYGZgt{}
\PYGZlt{}head\PYGZgt{}
\PYGZlt{}meta charset=\PYGZdq{}utf\PYGZhy{}8\PYGZdq{}\PYGZgt{}
\PYGZlt{}title\PYGZgt{}Título del sitio\PYGZlt{}/title\PYGZgt{}
\PYGZlt{}/head\PYGZgt{}
\PYGZlt{}body\PYGZgt{}
\PYGZlt{}h1\PYGZgt{}Página \PYGZlt{}/h1\PYGZgt{}
    ...
\PYGZlt{}/body\PYGZgt{}
\PYGZlt{}/html\PYGZgt{}
\end{sphinxVerbatim}


\section{POP3}
\label{\detokenize{textos/tema4:pop3}}
POP3 significa «Post Office Protocol version 3» y \sphinxstylestrong{está pensando fundamentalmente} para descargar cliente desde un servidor de correo al ordenador de un cliente que habitualmente utiliza Outlook, Mozilla Thunderbird o algún otro cliente de correo. Las versiones más antiguas no usaban mecanismos de cifrado, pero al igual que Telnet, POP3 ha necesitado cambiar para manejar correctamente la seguridad (aunque Telnet ha sido sustituido por otro protocolo y POP3 lo que ha hecho ha sido incorporar extensiones).


\section{SMTP}
\label{\detokenize{textos/tema4:smtp}}
Al contrario que POP3, está pensado sobre todo para \sphinxstylestrong{enviar correo}. Esto significa que quienes usan SMTP son máquinas que están en una de estas situaciones.
\begin{itemize}
\item {} 
Cliente de correo (que supongamos que tiene el usuario «pepe») de un servidor de correo (por ejemplo «hotmail.es») y que quiere enviar un mensaje a otro usuario «john» que tiene su cuenta en «gmail.com». Esto significa que si «pepe» escribe un mensaje en su Outlook y quiere enviarlo a «\sphinxhref{mailto:john@gmail.com}{john@gmail.com}» primero tiene que enviar el mensaje desde su ordenador a «hotmail.es» y pedirle que lo entregue. La subida del correo desde el ordenador de «pepe» a «hotmail.es» \sphinxstylestrong{se hace mediante SMTP}.

\item {} 
Servidor de correo que quiere enviar mensaje a otro servidor. En el ejemplo anterior «hotmail.es» recibe un correo que debe entregar a su destinatario, pero como dicho destinatario está en otro servidor debe entregarlo a «gmail.com» que comprobará si tiene un usuario «john» y si es así recibirá el mensaje. \sphinxstylestrong{Todo este proceso también se hace mediante SMTP}

\end{itemize}

Ejemplo de base: supongamos que deseamos crear un programa Java que al ser lanzado permita enviar un mensaje con un fichero del cual nos dan la ruta y que en el cuerpo del mensaje indique el nombre del archivo. El programa debe ser capaz de enviar el email a varios destinatarios y debe ser capaz también de enviar varios adjuntos a la vez.

\begin{sphinxadmonition}{warning}{Advertencia:}
Si utilizamos este código con un servidor público (GMail, Yahoo, Hotmail) debemos asegurarnos de que configuramos nuestra cuenta para permitir el acceso a aplicaciones desconocidas por dichos servidores. La forma de configurarlo varía de unos servicios a otros
\end{sphinxadmonition}

\begin{sphinxadmonition}{danger}{Peligro:}
Una vez activado el acceso a aplicaciones desconocidas GMail permitirá a nuestra aplicación enviar y recibir emails con nuestra cuenta de correo, pero existe el riesgo de que nuestra cuenta se marque como «generadora de spam». Se recomienda usar cuentas de prueba con estos programas.
\end{sphinxadmonition}


\subsection{Librerías de clases y componentes.}
\label{\detokenize{textos/tema4:librerias-de-clases-y-componentes}}
Para crear este programa vamos a usar las siguientes bibliotecas de libre distribución:
\begin{itemize}
\item {} 
La biblioteca JavaMail. Curiosamente, no se distribuye con el JDK, sino que debe descargarse el JAR por separado.

\item {} 
La biblioteca Apache Commons Email. Es una biblioteca de libre distribución que facilita el desarrollo de aplicaciones que necesiten enviar o recibir email.

\end{itemize}


\subsection{Utilización de objetos predefinidos.}
\label{\detokenize{textos/tema4:utilizacion-de-objetos-predefinidos}}
El objeto fundamental es la clase \sphinxcode{\sphinxupquote{Email}}, proporcionado por Apache Commons Email. Esta clase no tiene propiedades de interés y no está orientada a eventos, sin embargo tiene algunos métodos interesantes que además son bastante autoexplicativos.
\begin{itemize}
\item {} 
El método \sphinxcode{\sphinxupquote{setHostname}} permite configurar el servidor SMTP de nuestro proveedor de correo.

\item {} 
El método \sphinxcode{\sphinxupquote{setSmtpPort}} permite indicar el número de puerto en el que el servidor de correo escucha. Para conexiones SMTPS (SMTP seguro) el puerto es el 465.

\item {} 
El método \sphinxcode{\sphinxupquote{setAuthenticator}} permite indicar el objeto que se encargará de configurar la autenticación. Suele usar un objeto básico de la clase \sphinxcode{\sphinxupquote{DefaultAuthenticator}}.

\item {} 
El método \sphinxcode{\sphinxupquote{setSslOnConect}} se usa para indicar si se establece una conexión SSL (o TLS) en el momento de conectar

\end{itemize}


\subsection{Establecimiento y finalización de conexiones.}
\label{\detokenize{textos/tema4:establecimiento-y-finalizacion-de-conexiones}}
La biblioteca proporciona de manera transparente el proceso correcto para el establecimiento y finalización de conexiones.


\subsection{Transmisión de información.}
\label{\detokenize{textos/tema4:transmision-de-informacion}}
Toda la transmisión de información es controlada por la biblioteca, liberando al programador de tareas de control.


\subsection{Programación de aplicaciones cliente.}
\label{\detokenize{textos/tema4:programacion-de-aplicaciones-cliente}}
La clase \sphinxcode{\sphinxupquote{Cliente.java}} que se adjunta a continuación presenta un interfaz con dos métodos que permiten enviar correos electrónicos con adjuntos. El código es mejorable pero se ha primado la legibilidad e inteligibilidad del mismo.

\begin{sphinxVerbatim}[commandchars=\\\{\}]
\PYG{c+cm}{/* Para poder usar esta clase debes disponer}
\PYG{c+cm}{del JAR de la biblioteca Apache Commons EMail y}
\PYG{c+cm}{las extensiones JavaMail así como configurar tu entorno}
\PYG{c+cm}{para que encuentre correctamente dichas bibliotecas.}
\PYG{c+cm}{Puedes encontrar el JAR en https://commons.apache.org/email/}
\PYG{c+cm}{*/}
\PYG{k+kd}{public} \PYG{k+kd}{class} \PYG{n+nc}{RemitenteCorreo} \PYG{p}{\PYGZob{}}
    \PYG{k+kd}{private} \PYG{n}{String}          \PYG{n}{servidorSMTP}\PYG{p}{;}
    \PYG{k+kd}{private} \PYG{n}{String}          \PYG{n}{usuarioRemitente}\PYG{p}{;}
    \PYG{k+kd}{private} \PYG{n}{String}          \PYG{n}{claveRemitente}\PYG{p}{;}
    \PYG{k+kd}{private} \PYG{n}{MultiPartEmail}  \PYG{n}{email}\PYG{p}{;}

    \PYG{k+kd}{public} \PYG{n+nf}{RemitenteCorreo}\PYG{p}{(}\PYG{n}{String} \PYG{n}{servidorSMTP}\PYG{p}{,}
            \PYG{n}{String} \PYG{n}{usuarioRemitente}\PYG{p}{,} \PYG{n}{String} \PYG{n}{claveRemitente}\PYG{p}{)} \PYG{p}{\PYGZob{}}
        \PYG{k}{this}\PYG{p}{.}\PYG{n+na}{servidorSMTP}       \PYG{o}{=}   \PYG{n}{servidorSMTP}\PYG{p}{;}
        \PYG{k}{this}\PYG{p}{.}\PYG{n+na}{usuarioRemitente}   \PYG{o}{=}   \PYG{n}{usuarioRemitente}\PYG{p}{;}
        \PYG{k}{this}\PYG{p}{.}\PYG{n+na}{claveRemitente}     \PYG{o}{=}   \PYG{n}{claveRemitente}\PYG{p}{;}
        \PYG{k}{this}\PYG{p}{.}\PYG{n+na}{email}              \PYG{o}{=}   \PYG{k+kc}{null}\PYG{p}{;}
    \PYG{p}{\PYGZcb{}} \PYG{c+c1}{//Fin del constructor}

    \PYG{c+c1}{//Método usado solo en el interior de la clase}
    \PYG{k+kd}{private} \PYG{k+kt}{void} \PYG{n+nf}{iniciarConexionEmail}\PYG{p}{(}\PYG{p}{)}\PYG{p}{\PYGZob{}}
        \PYG{n}{email}\PYG{o}{=}\PYG{k}{new} \PYG{n}{MultiPartEmail}\PYG{p}{(}\PYG{p}{)}\PYG{p}{;}
        \PYG{c+cm}{/*Se indica el servidor del remitente*/}
        \PYG{n}{email}\PYG{p}{.}\PYG{n+na}{setHostName}\PYG{p}{(}\PYG{n}{servidorSMTP}\PYG{p}{)}\PYG{p}{;}
        \PYG{c+cm}{/*Habitualmente el puerto 465 se usa para SMTPS,}
\PYG{c+cm}{        en el que la encriptación se inicia antes de enviar nada*/}
        \PYG{n}{email}\PYG{p}{.}\PYG{n+na}{setSmtpPort}\PYG{p}{(}\PYG{l+m+mi}{465}\PYG{p}{)}\PYG{p}{;}

        \PYG{c+cm}{/*Se configura la autenticación*/}
        \PYG{n}{DefaultAuthenticator} \PYG{n}{sistemaAutenticacion}\PYG{p}{;}
        \PYG{n}{sistemaAutenticacion}\PYG{o}{=}\PYG{k}{new} \PYG{n}{DefaultAuthenticator}\PYG{p}{(}
                \PYG{k}{this}\PYG{p}{.}\PYG{n+na}{usuarioRemitente}\PYG{p}{,}
                \PYG{k}{this}\PYG{p}{.}\PYG{n+na}{claveRemitente} \PYG{p}{)}\PYG{p}{;}
        \PYG{n}{email}\PYG{p}{.}\PYG{n+na}{setAuthenticator}\PYG{p}{(}\PYG{n}{sistemaAutenticacion}\PYG{p}{)}\PYG{p}{;}

        \PYG{c+cm}{/*Se indica que vamos a usar el cifrado}
\PYG{c+cm}{        al inicio de la conexión    */}
        \PYG{n}{email}\PYG{p}{.}\PYG{n+na}{setSSLOnConnect}\PYG{p}{(}\PYG{k+kc}{true}\PYG{p}{)}\PYG{p}{;}
    \PYG{p}{\PYGZcb{}}

    \PYG{k+kd}{private}  \PYG{k+kt}{void} \PYG{n+nf}{configurarParametrosBasicos} \PYG{p}{(}
        \PYG{n}{String} \PYG{n}{asunto}\PYG{p}{,} \PYG{n}{String} \PYG{n}{textoEmail}\PYG{p}{,} \PYG{n}{String} \PYG{n}{destinatario}\PYG{p}{,}
            \PYG{n}{String}\PYG{o}{[}\PYG{o}{]} \PYG{n}{destinatariosCC}\PYG{p}{,} \PYG{n}{String}\PYG{o}{[}\PYG{o}{]} \PYG{n}{destinatariosBCC}\PYG{p}{)} \PYG{k+kd}{throws} \PYG{n}{EmailException}
    \PYG{p}{\PYGZob{}}
        \PYG{c+cm}{/*Se indica el asunto*/}
        \PYG{n}{email}\PYG{p}{.}\PYG{n+na}{setSubject}\PYG{p}{(}\PYG{n}{asunto}\PYG{p}{)}\PYG{p}{;}
        \PYG{c+cm}{/*Se indica el remitente*/}
        \PYG{n}{email}\PYG{p}{.}\PYG{n+na}{setFrom}\PYG{p}{(}\PYG{k}{this}\PYG{p}{.}\PYG{n+na}{usuarioRemitente}\PYG{o}{+}\PYG{l+s}{\PYGZdq{}}\PYG{l+s}{@}\PYG{l+s}{\PYGZdq{}}\PYG{o}{+}\PYG{k}{this}\PYG{p}{.}\PYG{n+na}{servidorSMTP}\PYG{p}{)}\PYG{p}{;}
        \PYG{c+cm}{/*Se pasa el texto*/}
        \PYG{n}{email}\PYG{p}{.}\PYG{n+na}{setMsg}\PYG{p}{(}\PYG{n}{textoEmail}\PYG{p}{)}\PYG{p}{;}
        \PYG{c+cm}{/*Se configura el destinatario principal*/}
        \PYG{n}{email}\PYG{p}{.}\PYG{n+na}{addTo}\PYG{p}{(}\PYG{n}{destinatario}\PYG{p}{)}\PYG{p}{;}
        \PYG{c+cm}{/*Y se configuran otros posibles destinatarios*/}
        \PYG{k}{if} \PYG{p}{(}\PYG{n}{destinatariosCC}\PYG{o}{!}\PYG{o}{=}\PYG{k+kc}{null}\PYG{p}{)}\PYG{p}{\PYGZob{}}
            \PYG{n}{email}\PYG{p}{.}\PYG{n+na}{addCc}\PYG{p}{(}\PYG{n}{destinatariosCC}\PYG{p}{)}\PYG{p}{;}
        \PYG{p}{\PYGZcb{}}
        \PYG{k}{if} \PYG{p}{(}\PYG{n}{destinatariosBCC}\PYG{o}{!}\PYG{o}{=}\PYG{k+kc}{null}\PYG{p}{)}\PYG{p}{\PYGZob{}}
            \PYG{n}{email}\PYG{p}{.}\PYG{n+na}{addBcc}\PYG{p}{(}\PYG{n}{destinatariosBCC}\PYG{p}{)}\PYG{p}{;}
        \PYG{p}{\PYGZcb{}}
    \PYG{p}{\PYGZcb{}}
    \PYG{k+kd}{public} \PYG{k+kt}{void} \PYG{n+nf}{enviarMensaje}\PYG{p}{(}\PYG{n}{String} \PYG{n}{asunto}\PYG{p}{,} \PYG{n}{String} \PYG{n}{textoEmail}\PYG{p}{,} \PYG{n}{String} \PYG{n}{destinatario}\PYG{p}{,}
            \PYG{n}{String}\PYG{o}{[}\PYG{o}{]} \PYG{n}{destinatariosCC}\PYG{p}{,} \PYG{n}{String}\PYG{o}{[}\PYG{o}{]} \PYG{n}{destinatariosBCC}\PYG{p}{)}
            \PYG{k+kd}{throws} \PYG{n}{EmailException}
    \PYG{p}{\PYGZob{}}
        \PYG{n}{iniciarConexionEmail}\PYG{p}{(}\PYG{p}{)}\PYG{p}{;}
        \PYG{k}{this}\PYG{p}{.}\PYG{n+na}{configurarParametrosBasicos}\PYG{p}{(}\PYG{n}{asunto}\PYG{p}{,} \PYG{n}{textoEmail}\PYG{p}{,}
                \PYG{n}{destinatario}\PYG{p}{,} \PYG{n}{destinatariosCC}\PYG{p}{,} \PYG{n}{destinatariosBCC}\PYG{p}{)}\PYG{p}{;}
        \PYG{n}{email}\PYG{p}{.}\PYG{n+na}{send}\PYG{p}{(}\PYG{p}{)}\PYG{p}{;}
    \PYG{p}{\PYGZcb{}}

    \PYG{k+kd}{public} \PYG{k+kt}{void} \PYG{n+nf}{enviarMensajeConAdjuntos}\PYG{p}{(}\PYG{n}{String} \PYG{n}{asunto}\PYG{p}{,} \PYG{n}{String} \PYG{n}{textoEmail}\PYG{p}{,} \PYG{n}{String} \PYG{n}{destinatario}\PYG{p}{,}
            \PYG{n}{String}\PYG{o}{[}\PYG{o}{]} \PYG{n}{destinatariosCC}\PYG{p}{,} \PYG{n}{String}\PYG{o}{[}\PYG{o}{]} \PYG{n}{destinatariosBCC}\PYG{p}{,}
            \PYG{n}{String}\PYG{o}{[}\PYG{o}{]} \PYG{n}{listaRutasArchivo}\PYG{p}{)}
            \PYG{k+kd}{throws} \PYG{n}{EmailException}\PYG{p}{,} \PYG{n}{FileNotFoundException}
    \PYG{p}{\PYGZob{}}
        \PYG{c+cm}{/*Se configura lo basico*/}
        \PYG{n}{iniciarConexionEmail}\PYG{p}{(}\PYG{p}{)}\PYG{p}{;}
        \PYG{k}{this}\PYG{p}{.}\PYG{n+na}{configurarParametrosBasicos}\PYG{p}{(}\PYG{n}{asunto}\PYG{p}{,} \PYG{n}{textoEmail}\PYG{p}{,}
                \PYG{n}{destinatario}\PYG{p}{,} \PYG{n}{destinatariosCC}\PYG{p}{,} \PYG{n}{destinatariosBCC}\PYG{p}{)}\PYG{p}{;}
        \PYG{c+cm}{/*Y añadimos los adjuntos*/}
        \PYG{k}{for} \PYG{p}{(}\PYG{n}{String} \PYG{n}{ruta} \PYG{p}{:} \PYG{n}{listaRutasArchivo}\PYG{p}{)}\PYG{p}{\PYGZob{}}
            \PYG{n}{File} \PYG{n}{fichero}\PYG{o}{=}\PYG{k}{new} \PYG{n}{File}\PYG{p}{(}\PYG{n}{ruta}\PYG{p}{)}\PYG{p}{;}
            \PYG{k}{this}\PYG{p}{.}\PYG{n+na}{email}\PYG{p}{.}\PYG{n+na}{attach}\PYG{p}{(}\PYG{n}{fichero}\PYG{p}{)}\PYG{p}{;}
        \PYG{p}{\PYGZcb{}}
        \PYG{c+c1}{// Y se envía el mensaje}
        \PYG{n}{email}\PYG{p}{.}\PYG{n+na}{send}\PYG{p}{(}\PYG{p}{)}\PYG{p}{;}
    \PYG{p}{\PYGZcb{}}
\PYG{p}{\PYGZcb{}}
\end{sphinxVerbatim}


\section{Programación de servidores.}
\label{\detokenize{textos/tema4:programacion-de-servidores}}
Sabemos que un servidor es simplemente un programa. En concreto un programa que ofrece sus servicios a otros programas. Este concepto permite estructurar las aplicaciones utilizando un mecanismo que se conoce como «programación cliente/servidor». En estos programas \sphinxstylestrong{se determina un protocolo de comunicaciones} entre cliente y servidor y despues el cliente «pide» operaciones al servidor y este le devuelve resultados. Lo más interesante de este sistema que:
\begin{itemize}
\item {} 
El cliente y el servidor pueden estar en sitios distintos y comunicarse a través de protocolos de red.

\item {} 
El cliente y el servidor \sphinxstyleemphasis{pueden estar programados en distintos lenguajes}

\end{itemize}

Supongamos un servidor que ofrece operaciones de cálculo a cualquier programa cliente. Un protocolo de comunicaciones muy simple sería:
\begin{itemize}
\item {} 
Cuando el cliente se conecta al servidor envía una primera línea con uno de estos cuatro símbolos: + \sphinxhyphen{} * /

\item {} 
Despues el cliente envía dos números, cada uno en una línea separada.

\item {} 
Cuando el cliente haya enviado estas 3 líneas debe ponerse en modo de escucha y esperar que el servidor le devuelva un resultado que será un único número en una única línea.

\end{itemize}

Aunque para programas pequeños esto funciona perfectamente, existen estándares ya especificados que nos ayudan a construir aplicaciones. A continuación veremos algunos.


\subsection{SOAP}
\label{\detokenize{textos/tema4:soap}}

\subsection{WCF}
\label{\detokenize{textos/tema4:wcf}}
En el tutorial siguiente se explica como crear un servicio web muy básico que implemente algunas operaciones con cadenas. La construcción de un servicio WCF tiene dos operaciones básicas:
\begin{enumerate}
\sphinxsetlistlabels{\arabic}{enumi}{enumii}{}{.}%
\item {} 
Establecer las operaciones que vamos a «ofrecer al exterior». Esto se hará con un \sphinxstyleemphasis{interfaz} que etiquetaremos con algunos atributos.

\item {} 
Crear una clase que implemente el interfaz que hayamos decidido en el paso 1.

\end{enumerate}

Así, por ejemplo, si consideramos que algunas operaciones muy complejas son «convertir una cadena en mayúsculas» e «invertir una cadena» definiremos un interfaz que tenga dos operaciones públicas que reciben una cadena y devuelven una cadena. Despues en el siguiente paso crearemos una clase que implemente ese interfaz.

En primer lugar crearemos un proyecto en Visual Studio del tipo «Biblioteca de Servicios WCF» y crearemos un proyecto llamado ServicioCadenas. El editor creará dos ficheros, uno para el interfaz y otro para la implementación.

\begin{figure}[htbp]
\centering
\capstart

\noindent\sphinxincludegraphics{{wcf1}.png}
\caption{Creación de un servicio web.}\label{\detokenize{textos/tema4:id1}}\end{figure}

En el fichero de interfaz pondremos este código para definir el interfaz.

\begin{sphinxVerbatim}[commandchars=\\\{\}]
\PYG{k}{using} \PYG{n+nn}{System}\PYG{p}{;}
\PYG{k}{using} \PYG{n+nn}{System.Collections.Generic}\PYG{p}{;}
\PYG{k}{using} \PYG{n+nn}{System.Linq}\PYG{p}{;}
\PYG{k}{using} \PYG{n+nn}{System.Runtime.Serialization}\PYG{p}{;}
\PYG{k}{using} \PYG{n+nn}{System.ServiceModel}\PYG{p}{;}
\PYG{k}{using} \PYG{n+nn}{System.Text}\PYG{p}{;}

\PYG{k}{namespace} \PYG{n+nn}{ServicioCadenas}
\PYG{p}{\PYGZob{}}
\PYG{n+na}{    [ServiceContract]}
    \PYG{k}{public} \PYG{k}{interface} \PYG{n}{IServicioCadenas}

    \PYG{p}{\PYGZob{}}
\PYG{n+na}{        [OperationContract]}
        \PYG{k+kt}{string} \PYG{n+nf}{PasarAMayusculas}\PYG{p}{(}\PYG{k+kt}{string} \PYG{n}{cadena}\PYG{p}{)}\PYG{p}{;}
\PYG{n+na}{        [OperationContract]}
        \PYG{k+kt}{string} \PYG{n+nf}{Invertir}\PYG{p}{(}\PYG{k+kt}{string} \PYG{n}{cadena}\PYG{p}{)}\PYG{p}{;}
    \PYG{p}{\PYGZcb{}}
\PYG{p}{\PYGZcb{}}
\end{sphinxVerbatim}

Como puede verse hemos usado diversos atributos:
\begin{itemize}
\item {} 
El atributo \sphinxcode{\sphinxupquote{{[}ServiceContract{]}}} indica que este interfaz \sphinxstyleemphasis{regula el contrato por el que se rige este servicio}. Se llama «contrato» al conjunto de operaciones y parámetros que un servidor ofrece.

\item {} 
El atributo \sphinxcode{\sphinxupquote{{[}OperationContract{]}}} indica que el método que hay a continuación es un método que se ofrece al exterior y por lo tanto los clientes podrán invocarlo y deberán hacerlo pasando los parámetros adecuados y deberán recoger un valor del tipo devuelto.

\end{itemize}

El siguiente paso es \sphinxstyleemphasis{implementar el interfaz definido}, es decir añadir una clase que implemente esos métodos. En concreto ahora pondremos este código en el fichero de implementación  (por brevedad se han omitido las sentencias \sphinxcode{\sphinxupquote{using}}):

\begin{sphinxVerbatim}[commandchars=\\\{\}]
\PYG{k}{namespace} \PYG{n+nn}{ServicioCadenas}
\PYG{p}{\PYGZob{}}
    \PYG{k}{public} \PYG{k}{class} \PYG{n+nc}{ServicioCadenas} \PYG{p}{:} \PYG{n}{IServicioCadenas}
    \PYG{p}{\PYGZob{}}
        \PYG{k}{public} \PYG{k+kt}{string} \PYG{n+nf}{Invertir}\PYG{p}{(}\PYG{k+kt}{string} \PYG{n}{cadena}\PYG{p}{)}
        \PYG{p}{\PYGZob{}}
            \PYG{k+kt}{char}\PYG{p}{[}\PYG{p}{]} \PYG{n}{caracteres} \PYG{p}{=} \PYG{n}{cadena}\PYG{p}{.}\PYG{n}{ToCharArray}\PYG{p}{(}\PYG{p}{)}\PYG{p}{;}
            \PYG{n}{Array}\PYG{p}{.}\PYG{n}{Reverse}\PYG{p}{(}\PYG{n}{caracteres}\PYG{p}{)}\PYG{p}{;}
            \PYG{k}{return} \PYG{k}{new} \PYG{n+nf}{string}\PYG{p}{(}\PYG{n}{caracteres}\PYG{p}{)}\PYG{p}{;}
        \PYG{p}{\PYGZcb{}}

        \PYG{k}{public} \PYG{k+kt}{string} \PYG{n+nf}{PasarAMayusculas}\PYG{p}{(}\PYG{k+kt}{string} \PYG{n}{cadena}\PYG{p}{)}
        \PYG{p}{\PYGZob{}}
            \PYG{k}{return} \PYG{n}{cadena}\PYG{p}{.}\PYG{n}{ToUpper}\PYG{p}{(}\PYG{p}{)}\PYG{p}{;}
        \PYG{p}{\PYGZcb{}}
    \PYG{p}{\PYGZcb{}}
\PYG{p}{\PYGZcb{}}
\end{sphinxVerbatim}

\begin{sphinxadmonition}{warning}{Advertencia:}
Debe recordarse modificar el archivo App.config. Hemos cambiado el nombre de la clase y del interfaz así que deberemos asegurarnos de que los nuevos nombres aparezcan en dicho archivo.
\end{sphinxadmonition}

Con este código tan simple y el apoyo de Visual Studio (y las clases WCF) ya se dispone de un servicio. Para probarlo no hace falta implementar un cliente, sino que puede usarse el propio Visual Studio, que implementa una especie de cuadro de diálogo que permite hacer algunas pruebas. Si ejecutamos este código (usando el botón Iniciar o pulsando F5). Al hacerlo veremos esto:

\begin{figure}[htbp]
\centering
\capstart

\noindent\sphinxincludegraphics{{wcf2}.png}
\caption{Probando el servicio web.}\label{\detokenize{textos/tema4:id2}}\end{figure}

Como puede verse en la figura, hemos elegido un método y hemos puesto algún valor en la cadena. Al invocar el servicio, el método se ejecuta correctamente y obtenemos el resultado esperado. Además, en la ventana podemos ver la referencia URL del servicio \sphinxstylestrong{que podremos usar en otro ordenador por remoto que sea}


\subsection{Implementación de comunicaciones simultáneas.}
\label{\detokenize{textos/tema4:implementacion-de-comunicaciones-simultaneas}}

\subsection{Documentación.}
\label{\detokenize{textos/tema4:documentacion}}

\subsection{Depuración.}
\label{\detokenize{textos/tema4:depuracion}}

\subsection{Monitorización de tiempos de respuesta.}
\label{\detokenize{textos/tema4:monitorizacion-de-tiempos-de-respuesta}}

\chapter{Utilización de técnicas de programación segura}
\label{\detokenize{textos/tema5:utilizacion-de-tecnicas-de-programacion-segura}}\label{\detokenize{textos/tema5::doc}}

\section{Introducción}
\label{\detokenize{textos/tema5:introduccion}}
En general, cuando se envía algo a través de sockets se envía como «texto plano», es decir, no sabemos si hay alguien usando un sniffer en la red y por tanto no sabemos si alguien está capturando los datos.

En general, cualquier sistema que pretenda ser seguro necesitará usar cifrado.


\section{Prácticas de programación segura.}
\label{\detokenize{textos/tema5:practicas-de-programacion-segura}}
Para enviar mensajes cifrados se necesita algún mecanismo o algoritmo para convertir un texto normal en uno más difícil de comprender.

El esquema general de todos los métodos es tener código como el siguiente:

\begin{sphinxVerbatim}[commandchars=\\\{\}]
\PYG{k+kd}{public} \PYG{n}{String} \PYG{n+nf}{cifrar} \PYG{p}{(}\PYG{n}{String} \PYG{n}{texto}\PYG{p}{,} \PYG{n}{String} \PYG{n}{clave}\PYG{p}{)}
\PYG{p}{\PYGZob{}}
\PYG{p}{\PYGZcb{}}

\PYG{k+kd}{public} \PYG{n}{String} \PYG{n+nf}{descifrar}\PYG{p}{(}\PYG{n}{String} \PYG{n}{texto}\PYG{p}{,}\PYG{n}{String} \PYG{n}{clave}\PYG{p}{)}
\PYG{p}{\PYGZob{}}
\PYG{p}{\PYGZcb{}}
\end{sphinxVerbatim}


\subsection{Método César}
\label{\detokenize{textos/tema5:metodo-cesar}}
Si el alfabeto es el siguiente:

ABCDEFGHIJKLMNÑOPQRSTUVWXYZ0123456789\sphinxhyphen{}
BCDEFGHIJKLMNÑOPQRSTUVWXYZ0123456789\sphinxhyphen{}A

El mensaje HOLA MUNDO, con clave 1 sería así
\begin{quote}

HOLAMUNDO
IPMBNVÑEP
\end{quote}

El descifrado simplemente implicaría el método inverso. Si el desplazamiento es un valor distinto de 1, lo único que hay que hacer es construir el alfabeto rotado tantas veces como el desplazamiento

Esta clase implementa un sistema de rotado básico para poder efectuar

\begin{sphinxVerbatim}[commandchars=\\\{\}]
\PYG{k+kd}{public} \PYG{k+kd}{class} \PYG{n+nc}{Cifrador} \PYG{p}{\PYGZob{}}
        \PYG{k+kd}{private} \PYG{n}{String} \PYG{n}{alfabeto}\PYG{o}{=}
                        \PYG{l+s}{\PYGZdq{}}\PYG{l+s}{ABCDEFGHIJKLMNÑOPQRSTUVWXYZ}\PYG{l+s}{\PYGZdq{}}\PYG{o}{+}
                        \PYG{l+s}{\PYGZdq{}}\PYG{l+s}{0123456789 }\PYG{l+s}{\PYGZdq{}}\PYG{p}{;}
        \PYG{k+kd}{private} \PYG{n}{String} \PYG{n}{alfabetoCifrado}\PYG{p}{;}
        \PYG{c+cm}{/* Dada una cadena como}
\PYG{c+cm}{         * \PYGZdq{}ABC\PYGZdq{} y un número (p.ej 2)}
\PYG{c+cm}{         * devuelve la cadena rotada a la izq}
\PYG{c+cm}{         * tantas veces como indique el numero,}
\PYG{c+cm}{         * en este caso \PYGZdq{}CAB\PYGZdq{}}
\PYG{c+cm}{         */}
        \PYG{k+kd}{public} \PYG{n}{String} \PYG{n+nf}{rotar}\PYG{p}{(}\PYG{n}{String} \PYG{n}{cad}\PYG{p}{,}\PYG{k+kt}{int} \PYG{n}{numVeces}\PYG{p}{)}\PYG{p}{\PYGZob{}}
                \PYG{k+kt}{char}\PYG{o}{[}\PYG{o}{]} \PYG{n}{resultado}\PYG{o}{=}\PYG{k}{new} \PYG{k+kt}{char}\PYG{o}{[}\PYG{n}{cad}\PYG{p}{.}\PYG{n+na}{length}\PYG{p}{(}\PYG{p}{)}\PYG{o}{]}\PYG{p}{;}
                \PYG{k}{for} \PYG{p}{(}\PYG{k+kt}{int} \PYG{n}{i}\PYG{o}{=}\PYG{l+m+mi}{0}\PYG{p}{;} \PYG{n}{i}\PYG{o}{\PYGZlt{}}\PYG{n}{cad}\PYG{p}{.}\PYG{n+na}{length}\PYG{p}{(}\PYG{p}{)}\PYG{p}{;}\PYG{n}{i}\PYG{o}{+}\PYG{o}{+}\PYG{p}{)}\PYG{p}{\PYGZob{}}
                        \PYG{k+kt}{int} \PYG{n}{posParaExtraer}\PYG{o}{=}\PYG{p}{(}\PYG{n}{i}\PYG{o}{+}\PYG{n}{numVeces}\PYG{p}{)}\PYG{o}{\PYGZpc{}}\PYG{n}{cad}\PYG{p}{.}\PYG{n+na}{length}\PYG{p}{(}\PYG{p}{)}\PYG{p}{;}
                        \PYG{n}{resultado}\PYG{o}{[}\PYG{n}{i}\PYG{o}{]}\PYG{o}{=}\PYG{n}{cad}\PYG{p}{.}\PYG{n+na}{charAt}\PYG{p}{(}\PYG{n}{posParaExtraer}\PYG{p}{)}\PYG{p}{;}
                \PYG{p}{\PYGZcb{}}
                \PYG{n}{String} \PYG{n}{cadResultado}\PYG{o}{=}\PYG{n}{String}\PYG{p}{.}\PYG{n+na}{copyValueOf}\PYG{p}{(}\PYG{n}{resultado}\PYG{p}{)}\PYG{p}{;}
                \PYG{k}{return} \PYG{n}{cadResultado}\PYG{p}{;}
        \PYG{p}{\PYGZcb{}}
        \PYG{k+kd}{public} \PYG{n}{String} \PYG{n+nf}{cifrar}
                \PYG{p}{(}\PYG{n}{String} \PYG{n}{mensaje}\PYG{p}{,} \PYG{n}{String} \PYG{n}{clave}\PYG{p}{)}\PYG{p}{\PYGZob{}}
                \PYG{n}{String} \PYG{n}{mensajeCifrado}\PYG{o}{=}\PYG{l+s}{\PYGZdq{}}\PYG{l+s}{\PYGZdq{}}\PYG{p}{;}

                \PYG{k}{return} \PYG{n}{mensajeCifrado}\PYG{p}{;}
        \PYG{p}{\PYGZcb{}}
        \PYG{k+kd}{public} \PYG{n}{String} \PYG{n+nf}{descifrar}
                \PYG{p}{(}\PYG{n}{String} \PYG{n}{mensajeCifrado}\PYG{p}{,} \PYG{n}{String} \PYG{n}{clave}\PYG{p}{)}\PYG{p}{\PYGZob{}}
                \PYG{n}{String} \PYG{n}{mensajeDescifrado}\PYG{o}{=}\PYG{l+s}{\PYGZdq{}}\PYG{l+s}{\PYGZdq{}}\PYG{p}{;}

                \PYG{k}{return} \PYG{n}{mensajeDescifrado}\PYG{p}{;}
        \PYG{p}{\PYGZcb{}}
        \PYG{k+kd}{public} \PYG{k+kd}{static} \PYG{k+kt}{void} \PYG{n+nf}{main}\PYG{p}{(}\PYG{n}{String}\PYG{o}{[}\PYG{o}{]} \PYG{n}{args}\PYG{p}{)}\PYG{p}{\PYGZob{}}
                \PYG{n}{Cifrador} \PYG{n}{c}\PYG{o}{=}\PYG{k}{new} \PYG{n}{Cifrador}\PYG{p}{(}\PYG{p}{)}\PYG{p}{;}
                \PYG{n}{String} \PYG{n}{cad}\PYG{o}{=}\PYG{n}{c}\PYG{p}{.}\PYG{n+na}{rotar}\PYG{p}{(}\PYG{l+s}{\PYGZdq{}}\PYG{l+s}{ABCDEFG}\PYG{l+s}{\PYGZdq{}}\PYG{p}{,} \PYG{l+m+mi}{0}\PYG{p}{)}\PYG{p}{;}
                \PYG{n}{System}\PYG{p}{.}\PYG{n+na}{out}\PYG{p}{.}\PYG{n+na}{println}\PYG{p}{(}\PYG{n}{cad}\PYG{p}{)}\PYG{p}{;}
        \PYG{p}{\PYGZcb{}}
\PYG{p}{\PYGZcb{}}
\end{sphinxVerbatim}


\section{Criptografía de clave pública y clave privada.}
\label{\detokenize{textos/tema5:criptografia-de-clave-publica-y-clave-privada}}
Los principales sistemas modernos de seguridad utilizan dos claves ,una para cifrar y otra para descifrar. Esto se puede usar de diversas formas.


\section{Principales aplicaciones de la criptografía.}
\label{\detokenize{textos/tema5:principales-aplicaciones-de-la-criptografia}}\begin{itemize}
\item {} 
Mensajería segura: todo el mundo da su clave de cifrado pero conserva la de descifrado. Si queremos enviar un mensaje a alguien cogemos su clave de cifrado y ciframos el mensaje que le enviamos. Solo él podrá descifrarlo.

\item {} 
Firma digital: pilar del comercio electrónico. Permite verificar que un archivo no ha sido modificado.

\item {} 
Mensajería segura: en este tipo de mensajería se intenta evitar que un atacante (quizá con un \sphinxstyleemphasis{sniffer}) consiga descifrar nuestros mensajes.

\item {} 
Autenticación: los sistemas de autenticación intentan resolver una cuestión clave en la informática: \sphinxstylestrong{verificar que una máquina es quien dice ser}

\end{itemize}


\section{Protocolos criptográficos.}
\label{\detokenize{textos/tema5:protocolos-criptograficos}}
En realidad protocolos criptográficos hay muchos, y suelen dividirse en sistemas simétricos o asimétricos.
\begin{itemize}
\item {} 
Los sistemas simétricos son aquellos basados en una función que convierte un mensaje en otro mensaje cifrado. Si se desea descifrar algo se aplica el proceso inverso con la misma clave que se usó.

\item {} 
Los sistemas asimétricos utilizan una clave de cifrado y otra de descifrado. Aunque se tenga una clave es matemáticamente imposible averiguar la otra clave por lo que se puede dar a todo el mundo una de las claves (llamada habitualmente \sphinxstylestrong{clave pública}) y conservar la otra (llamada \sphinxstylestrong{clave privada}). Además, podemos usar las claves para lo que queramos y por ejemplo en unos casos cifraremos con la clave pública y en otros tal vez cifremos con la clave privada.

\end{itemize}

Hoy por hoy, las mayores garantías las ofrecen los asimétricos, de los cuales hay varios sistemas. El inconveniente que pueden tener los asimétricos es que son más lentos computacionalmente.

En este curso usaremos el cifrado asimétrico RSA.


\section{Encriptación de información.}
\label{\detokenize{textos/tema5:encriptacion-de-informacion}}
El siguiente código muestra como crear una clase que permita cifrar y descifrar textos.

\begin{sphinxVerbatim}[commandchars=\\\{\}]
\PYG{k+kd}{public} \PYG{k+kd}{class} \PYG{n+nc}{GestorCifrado} \PYG{p}{\PYGZob{}}
        \PYG{n}{KeyPair} \PYG{n}{claves}\PYG{p}{;}
        \PYG{n}{KeyPairGenerator} \PYG{n}{generadorClaves}\PYG{p}{;}
        \PYG{n}{Cipher} \PYG{n}{cifrador}\PYG{p}{;}
        \PYG{k+kd}{public} \PYG{n+nf}{GestorCifrado}\PYG{p}{(}\PYG{p}{)}
                        \PYG{k+kd}{throws} \PYG{n}{NoSuchAlgorithmException}\PYG{p}{,}
                        \PYG{n}{NoSuchPaddingException}\PYG{p}{\PYGZob{}}
                \PYG{n}{generadorClaves}\PYG{o}{=}
                                \PYG{n}{KeyPairGenerator}\PYG{p}{.}\PYG{n+na}{getInstance}\PYG{p}{(}\PYG{l+s}{\PYGZdq{}}\PYG{l+s}{RSA}\PYG{l+s}{\PYGZdq{}}\PYG{p}{)}\PYG{p}{;}
                \PYG{c+cm}{/*Usaremos una longitud de clave}
\PYG{c+cm}{                 * de 1024 bits */}
                \PYG{n}{generadorClaves}\PYG{p}{.}\PYG{n+na}{initialize}\PYG{p}{(}\PYG{l+m+mi}{1024}\PYG{p}{)}\PYG{p}{;}
                \PYG{n}{claves}\PYG{o}{=}\PYG{n}{generadorClaves}\PYG{p}{.}\PYG{n+na}{generateKeyPair}\PYG{p}{(}\PYG{p}{)}\PYG{p}{;}
                \PYG{n}{cifrador}\PYG{o}{=}\PYG{n}{Cipher}\PYG{p}{.}\PYG{n+na}{getInstance}\PYG{p}{(}\PYG{l+s}{\PYGZdq{}}\PYG{l+s}{RSA}\PYG{l+s}{\PYGZdq{}}\PYG{p}{)}\PYG{p}{;}
        \PYG{p}{\PYGZcb{}}
        \PYG{k+kd}{public} \PYG{n}{PublicKey} \PYG{n+nf}{getPublica}\PYG{p}{(}\PYG{p}{)}\PYG{p}{\PYGZob{}}
                \PYG{k}{return} \PYG{n}{claves}\PYG{p}{.}\PYG{n+na}{getPublic}\PYG{p}{(}\PYG{p}{)}\PYG{p}{;}
        \PYG{p}{\PYGZcb{}}
        \PYG{k+kd}{public} \PYG{n}{PrivateKey} \PYG{n+nf}{getPrivada}\PYG{p}{(}\PYG{p}{)}\PYG{p}{\PYGZob{}}
                \PYG{k}{return} \PYG{n}{claves}\PYG{p}{.}\PYG{n+na}{getPrivate}\PYG{p}{(}\PYG{p}{)}\PYG{p}{;}
        \PYG{p}{\PYGZcb{}}

        \PYG{k+kd}{public} \PYG{k+kt}{byte}\PYG{o}{[}\PYG{o}{]} \PYG{n+nf}{cifrar}\PYG{p}{(}\PYG{k+kt}{byte}\PYG{o}{[}\PYG{o}{]} \PYG{n}{paraCifrar}\PYG{p}{,}
                        \PYG{n}{Key} \PYG{n}{claveCifrado}
                        \PYG{p}{)} \PYG{k+kd}{throws} \PYG{n}{InvalidKeyException}\PYG{p}{,}
                        \PYG{n}{IllegalBlockSizeException}\PYG{p}{,}
                        \PYG{n}{BadPaddingException}\PYG{p}{\PYGZob{}}
                \PYG{k+kt}{byte}\PYG{o}{[}\PYG{o}{]} \PYG{n}{resultado}\PYG{p}{;}
                \PYG{c+cm}{/* Se pone el cifrador en modo cifrado*/}
                \PYG{n}{cifrador}\PYG{p}{.}\PYG{n+na}{init}\PYG{p}{(}\PYG{n}{Cipher}\PYG{p}{.}\PYG{n+na}{ENCRYPT\PYGZus{}MODE}\PYG{p}{,}
                                \PYG{n}{claveCifrado}\PYG{p}{)}\PYG{p}{;}
                \PYG{n}{resultado}\PYG{o}{=}\PYG{n}{cifrador}\PYG{p}{.}\PYG{n+na}{doFinal}\PYG{p}{(}\PYG{n}{paraCifrar}\PYG{p}{)}\PYG{p}{;}
                \PYG{k}{return} \PYG{n}{resultado}\PYG{p}{;}
        \PYG{p}{\PYGZcb{}}

        \PYG{k+kd}{public} \PYG{k+kt}{byte}\PYG{o}{[}\PYG{o}{]} \PYG{n+nf}{descifrar}\PYG{p}{(}
                        \PYG{k+kt}{byte}\PYG{o}{[}\PYG{o}{]} \PYG{n}{paraDescifrar}\PYG{p}{,}
                        \PYG{n}{Key} \PYG{n}{claveDescifrado}\PYG{p}{)}
                                        \PYG{k+kd}{throws} \PYG{n}{InvalidKeyException}\PYG{p}{,}
                                        \PYG{n}{IllegalBlockSizeException}\PYG{p}{,}
                                        \PYG{n}{BadPaddingException}\PYG{p}{\PYGZob{}}
                \PYG{k+kt}{byte}\PYG{o}{[}\PYG{o}{]} \PYG{n}{resultado}\PYG{p}{;}
                \PYG{c+cm}{/* Se pone el cifrador en modo descifrado*/}
                \PYG{n}{cifrador}\PYG{p}{.}\PYG{n+na}{init}\PYG{p}{(}\PYG{n}{Cipher}\PYG{p}{.}\PYG{n+na}{DECRYPT\PYGZus{}MODE}\PYG{p}{,}
                                \PYG{n}{claveDescifrado}\PYG{p}{)}\PYG{p}{;}
                \PYG{n}{resultado}\PYG{o}{=}\PYG{n}{cifrador}\PYG{p}{.}\PYG{n+na}{doFinal}\PYG{p}{(}\PYG{n}{paraDescifrar}\PYG{p}{)}\PYG{p}{;}
                \PYG{k}{return} \PYG{n}{resultado}\PYG{p}{;}
        \PYG{p}{\PYGZcb{}}



        \PYG{k+kd}{public} \PYG{k+kd}{static} \PYG{k+kt}{void} \PYG{n+nf}{main}\PYG{p}{(}\PYG{n}{String}\PYG{o}{[}\PYG{o}{]} \PYG{n}{args}\PYG{p}{)}
                        \PYG{k+kd}{throws} \PYG{n}{NoSuchAlgorithmException}\PYG{p}{,}
                        \PYG{n}{NoSuchPaddingException}\PYG{p}{,}
                        \PYG{n}{InvalidKeyException}\PYG{p}{,}
                        \PYG{n}{IllegalBlockSizeException}\PYG{p}{,}
                        \PYG{n}{BadPaddingException}\PYG{p}{,}
                        \PYG{n}{UnsupportedEncodingException}
        \PYG{p}{\PYGZob{}}
                \PYG{n}{GestorCifrado} \PYG{n}{gestorCifrado}\PYG{o}{=}
                                \PYG{k}{new} \PYG{n}{GestorCifrado}\PYG{p}{(}\PYG{p}{)}\PYG{p}{;}
                \PYG{n}{String} \PYG{n}{mensajeOriginal}\PYG{o}{=}\PYG{l+s}{\PYGZdq{}}\PYG{l+s}{Hola mundo}\PYG{l+s}{\PYGZdq{}}\PYG{p}{;}
                \PYG{n}{Key} \PYG{n}{clavePublica}\PYG{o}{=}\PYG{n}{gestorCifrado}\PYG{p}{.}\PYG{n+na}{getPublica}\PYG{p}{(}\PYG{p}{)}\PYG{p}{;}

                \PYG{k+kt}{byte}\PYG{o}{[}\PYG{o}{]} \PYG{n}{mensajeCifrado}\PYG{o}{=}
                                \PYG{n}{gestorCifrado}\PYG{p}{.}\PYG{n+na}{cifrar}\PYG{p}{(}
                                                \PYG{n}{mensajeOriginal}\PYG{p}{.}\PYG{n+na}{getBytes}\PYG{p}{(}\PYG{p}{)}\PYG{p}{,}
                                                \PYG{n}{clavePublica}
                \PYG{p}{)}\PYG{p}{;}
                \PYG{n}{String} \PYG{n}{cadCifrada}\PYG{o}{=}
                                \PYG{k}{new} \PYG{n}{String}\PYG{p}{(}\PYG{n}{mensajeCifrado}\PYG{p}{,} \PYG{l+s}{\PYGZdq{}}\PYG{l+s}{UTF\PYGZhy{}8}\PYG{l+s}{\PYGZdq{}}\PYG{p}{)}\PYG{p}{;}

                \PYG{n}{System}\PYG{p}{.}\PYG{n+na}{out}\PYG{p}{.}\PYG{n+na}{println}
                        \PYG{p}{(}\PYG{l+s}{\PYGZdq{}}\PYG{l+s}{Cadena original:}\PYG{l+s}{\PYGZdq{}}\PYG{o}{+}\PYG{n}{mensajeOriginal}\PYG{p}{)}\PYG{p}{;}
                \PYG{n}{System}\PYG{p}{.}\PYG{n+na}{out}\PYG{p}{.}\PYG{n+na}{println}
                        \PYG{p}{(}\PYG{l+s}{\PYGZdq{}}\PYG{l+s}{Cadena cifrada:}\PYG{l+s}{\PYGZdq{}}\PYG{o}{+}\PYG{n}{cadCifrada}\PYG{p}{)}\PYG{p}{;}

                \PYG{c+cm}{/* Cogemos la cadCifrada y la desciframos}
\PYG{c+cm}{                 * con la otra clave */}
                \PYG{n}{Key} \PYG{n}{clavePrivada}\PYG{p}{;}
                \PYG{n}{clavePrivada}\PYG{o}{=}\PYG{n}{gestorCifrado}\PYG{p}{.}\PYG{n+na}{getPrivada}\PYG{p}{(}\PYG{p}{)}\PYG{p}{;}
                \PYG{k+kt}{byte}\PYG{o}{[}\PYG{o}{]} \PYG{n}{descifrada}\PYG{o}{=}
                                \PYG{n}{gestorCifrado}\PYG{p}{.}\PYG{n+na}{descifrar}\PYG{p}{(}
                                                \PYG{n}{mensajeCifrado}\PYG{p}{,}\PYG{n}{clavePrivada}\PYG{p}{)}\PYG{p}{;}
                \PYG{n}{String} \PYG{n}{mensajeDescifrado}\PYG{p}{;}
    \PYG{c+cm}{/* E imprimimos el mensaje*/}
                \PYG{n}{mensajeDescifrado}\PYG{o}{=}
                                \PYG{k}{new} \PYG{n}{String}\PYG{p}{(}\PYG{n}{descifrada}\PYG{p}{,} \PYG{l+s}{\PYGZdq{}}\PYG{l+s}{UTF\PYGZhy{}8}\PYG{l+s}{\PYGZdq{}}\PYG{p}{)}\PYG{p}{;}
                \PYG{n}{System}\PYG{p}{.}\PYG{n+na}{out}\PYG{p}{.}\PYG{n+na}{println}\PYG{p}{(}
                                \PYG{l+s}{\PYGZdq{}}\PYG{l+s}{El mensaje descifrado es:}\PYG{l+s}{\PYGZdq{}}\PYG{o}{+}
                                                \PYG{n}{mensajeDescifrado}\PYG{p}{)}\PYG{p}{;}
        \PYG{p}{\PYGZcb{}}
\PYG{p}{\PYGZcb{}}
\end{sphinxVerbatim}

\begin{sphinxadmonition}{warning}{Advertencia:}
Los objetos que cifran y descifran en Java utilizan estrictamente objetos \sphinxcode{\sphinxupquote{byte{[}{]}}}, que
son los que debemos manejar siempre. Las conversiones a \sphinxcode{\sphinxupquote{String}} las hacemos nosotros para poder visualizar resultados.
\end{sphinxadmonition}


\section{Trabajando con bloques largos}
\label{\detokenize{textos/tema5:trabajando-con-bloques-largos}}
En general los distintos métodos criptográficos no siempre pueden trabajar con bloques todo lo largos que queramos, así que si se necesita «cifrar una cadena larga» es muy probable que tengamos que ir «cifrando bloque a bloque».


\section{Protocolos seguros de comunicaciones.}
\label{\detokenize{textos/tema5:protocolos-seguros-de-comunicaciones}}
En general, ahora que ya conocemos sockets, el uso de servidores y clientes y el uso de la criptografía de clave asimétrica ya es posible crear aplicaciones que se comuniquen de forma muy segura.

En general, todo protocolo que queramos implementar dará estos pasos.
\begin{enumerate}
\sphinxsetlistlabels{\arabic}{enumi}{enumii}{}{.}%
\item {} 
Todo cliente genera su pareja de claves.

\item {} 
Todo servidor genera su pareja de claves.

\item {} 
Cuando un cliente se conecte a un servidor, le envía su clave de cifrado y conserva la de descifrado.

\item {} 
Cuando un servidor recibe la conexión de un cliente recibe la clave de cifrado de dicho cliente.

\item {} 
El servidor envía su clave pública al cliente.

\item {} 
Ahora cliente y servidor pueden enviar mensajes al otro con la garantía de que solo servidor y cliente respectivamente pueden descifrar.

\end{enumerate}

En realidad se puede asegurar más el proceso haciendo que en el paso 5 el servidor cifre su propia clave pública con la clave pública del cliente. De esta forma, incluso aunque alguien robara la clave privada del cliente tampoco tendría demasiado, ya que tendría que robar la clave privada del servidor.


\section{Infraestructura de clave pública (PKI)}
\label{\detokenize{textos/tema5:infraestructura-de-clave-publica-pki}}
Para garantizar la seguridad es necesario que entre un tercer jugador en el intercambio de claves entre clientes y servidores. Este tercer individuo son las \sphinxstyleemphasis{autoridades de certificación} .

\begin{figure}[htbp]
\centering
\capstart

\noindent\sphinxincludegraphics{{Paso01}.png}
\caption{Autoridades de certificación.}\label{\detokenize{textos/tema5:id1}}\end{figure}


\section{Programación de aplicaciones con comunicaciones seguras.}
\label{\detokenize{textos/tema5:programacion-de-aplicaciones-con-comunicaciones-seguras}}
Por fortuna Java dispone de clases ya prefabricadas que facilitan enormemente el que dos aplicaciones intercambios datos de forma segura a través de una red. Se deben considerar los siguientes puntos:
\begin{itemize}
\item {} 
El servidor debe tener su propio certificado. Si no lo tenemos, se puede generar primero una pareja de claves con la herramienta \sphinxcode{\sphinxupquote{keytool}}, como se muestra en la figura adjunta. La herramienta guardará la pareja de claves en un almacén (el cual tiene su propia clave). Despues generaremos un certificado a partir de esa pareja con \sphinxcode{\sphinxupquote{keytool \sphinxhyphen{}export \sphinxhyphen{}file certificadoservidor.cer \sphinxhyphen{}keystore almacenclaves}}.

\item {} 
El código del servidor necesitará indicar el fichero donde se almacenan las claves y la clave para acceder a ese almacén.

\item {} 
El cliente necesita indicar que confía en el certificado del servidor. Dicho certificado del servidor puede estar guardado (por ejemplo) en el almacén de claves del cliente.

\item {} 
Aunque no suele hacerse también podría hacerse a la inversa y obligar al cliente a tener un certificado que el servidor pudiera importar, lo que aumentaría la seguridad.

\end{itemize}

\begin{figure}[htbp]
\centering
\capstart

\noindent\sphinxincludegraphics{{generacion_clave}.png}
\caption{Generando la pareja de claves del servidor.}\label{\detokenize{textos/tema5:id2}}\end{figure}

Los pasos desglosados implican ejecutar estos comandos en el servidor:

\begin{sphinxVerbatim}[commandchars=\\\{\}]
\PYG{c+c1}{\PYGZsh{} El servidor genera una pareja de claves que se almacena en un}
\PYG{c+c1}{\PYGZsh{}fichero llamado \PYGZdq{}clavesservidor\PYGZdq{}. Dentro del fichero se indica}
\PYG{c+c1}{\PYGZsh{}un alias para poder referirnos a esa clave fácilmente}
\PYG{n}{keytool} \PYG{o}{\PYGZhy{}}\PYG{n}{genkeypair} \PYG{o}{\PYGZhy{}}\PYG{n}{keyalg} \PYG{n}{RSA}
     \PYG{o}{\PYGZhy{}}\PYG{n}{alias} \PYG{n}{servidor} \PYG{o}{\PYGZhy{}}\PYG{n}{keystore} \PYG{n}{clavesservidor}

\PYG{c+c1}{\PYGZsh{}El servidor genera su \PYGZdq{}certificado\PYGZdq{}, es decir un fichero que}
\PYG{c+c1}{\PYGZsh{}de alguna forma indica quien es él. El certificado se almacena}
\PYG{c+c1}{\PYGZsh{}en un fichero llamado clavesservidor y a partir de él queremos}
\PYG{c+c1}{\PYGZsh{}generar el certificado de un alias que tiene que haber llamado servidor}
\PYG{n}{keytool} \PYG{o}{\PYGZhy{}}\PYG{o}{\PYGZhy{}}\PYG{n}{exportcert} \PYG{o}{\PYGZhy{}}\PYG{n}{alias} \PYG{n}{servidor}
     \PYG{o}{\PYGZhy{}}\PYG{n}{file} \PYG{n}{servidor}\PYG{o}{.}\PYG{n}{cer} \PYG{o}{\PYGZhy{}}\PYG{n}{keystore} \PYG{n}{clavesservidor}
\end{sphinxVerbatim}

En el cliente daremos estos pasos:

\begin{sphinxVerbatim}[commandchars=\\\{\}]
\PYG{c+c1}{\PYGZsh{}Se genera una pareja de claves (en realidad no nos hace falta solo}
\PYG{c+c1}{\PYGZsh{}queremos tener un almacén de claves.}
\PYG{n}{keytool} \PYG{o}{\PYGZhy{}}\PYG{n}{genkeypair} \PYG{o}{\PYGZhy{}}\PYG{n}{keyalg} \PYG{n}{RSA} \PYG{o}{\PYGZhy{}}\PYG{n}{alias} \PYG{n}{cliente} \PYG{o}{\PYGZhy{}}\PYG{n}{keystore} \PYG{n}{clavescliente}

\PYG{c+c1}{\PYGZsh{}Se importa el certificado del servidor indicando que pertenece a}
\PYG{c+c1}{\PYGZsh{}la lista de certificados confiables.}
\PYG{n}{keytool} \PYG{o}{\PYGZhy{}}\PYG{n}{importcert} \PYG{o}{\PYGZhy{}}\PYG{n}{trustcacerts} \PYG{o}{\PYGZhy{}}\PYG{n}{alias} \PYG{n}{servidor} \PYG{o}{\PYGZhy{}}\PYG{n}{file} \PYG{n}{servidor}\PYG{o}{.}\PYG{n}{cer} \PYG{o}{\PYGZhy{}}\PYG{n}{keystore} \PYG{n}{clavescliente}
\end{sphinxVerbatim}

Una vez creados los ficheros iniciales se deben dar los siguientes pasos en Java (servidor y cliente van por separado):
\begin{enumerate}
\sphinxsetlistlabels{\arabic}{enumi}{enumii}{}{.}%
\item {} 
El servidor debe cargar su almacén de claves (el fichero \sphinxcode{\sphinxupquote{clavesservidor}})

\item {} 
Ese almacén (cargado en un objeto Java llamado \sphinxcode{\sphinxupquote{KeyStore}}), se usará para crear un gestor de claves (objeto \sphinxcode{\sphinxupquote{KeyManager}}), el cual se obtiene a partir de una «fábrica» llamada \sphinxcode{\sphinxupquote{KeyManagerFactory}}.

\item {} 
Se creará un contexto SSL (objeto \sphinxcode{\sphinxupquote{SSLContext}}) a partir de la fábrica comentada.

\item {} 
El objeto \sphinxcode{\sphinxupquote{SSLContext}} permitirá crear una fábrica de sockets que será la que finalmente nos permita tener un \sphinxcode{\sphinxupquote{SSLServerSocket}}, es decir un socket de servidor que usará cifrado.

\end{enumerate}

El código Java del servidor sería algo así:

\begin{sphinxVerbatim}[commandchars=\\\{\}]
\PYG{k+kd}{public} \PYG{n+nf}{OtroServidor} \PYG{p}{(}\PYG{n}{String} \PYG{n}{rutaAlmacen}\PYG{p}{,} \PYG{n}{String} \PYG{n}{claveAlmacen}\PYG{p}{)}\PYG{p}{\PYGZob{}}
            \PYG{k}{this}\PYG{p}{.}\PYG{n+na}{rutaAlmacen}\PYG{o}{=}\PYG{n}{rutaAlmacen}\PYG{p}{;}
            \PYG{k}{this}\PYG{p}{.}\PYG{n+na}{claveAlmacen}\PYG{o}{=}\PYG{n}{claveAlmacen}\PYG{p}{;}
    \PYG{p}{\PYGZcb{}}

    \PYG{k+kd}{public} \PYG{n}{SSLServerSocket} \PYG{n+nf}{getServerSocketSeguro}\PYG{p}{(}\PYG{p}{)}
                    \PYG{k+kd}{throws} \PYG{n}{KeyStoreException}\PYG{p}{,} \PYG{n}{NoSuchAlgorithmException}\PYG{p}{,}
                    \PYG{n}{CertificateException}\PYG{p}{,} \PYG{n}{IOException}\PYG{p}{,}
                    \PYG{n}{KeyManagementException}\PYG{p}{,} \PYG{n}{UnrecoverableKeyException}
    \PYG{p}{\PYGZob{}}
            \PYG{n}{SSLServerSocket} \PYG{n}{serverSocket}\PYG{o}{=}\PYG{k+kc}{null}\PYG{p}{;}
            \PYG{c+cm}{/* Paso 1, se carga el almacén de claves*/}
            \PYG{n}{FileInputStream} \PYG{n}{fichAlmacen}\PYG{o}{=}
                            \PYG{k}{new} \PYG{n}{FileInputStream}\PYG{p}{(}\PYG{k}{this}\PYG{p}{.}\PYG{n+na}{rutaAlmacen}\PYG{p}{)}\PYG{p}{;}
            \PYG{c+cm}{/* Paso 1.1, se crea un almacén del tipo por defecto}
\PYG{c+cm}{             * que es un JKS (Java Key Store), a día de hoy*/}
            \PYG{n}{KeyStore} \PYG{n}{almacen}\PYG{o}{=}\PYG{n}{KeyStore}\PYG{p}{.}\PYG{n+na}{getInstance}\PYG{p}{(}\PYG{n}{KeyStore}\PYG{p}{.}\PYG{n+na}{getDefaultType}\PYG{p}{(}\PYG{p}{)}\PYG{p}{)}\PYG{p}{;}
            \PYG{n}{almacen}\PYG{p}{.}\PYG{n+na}{load}\PYG{p}{(}\PYG{n}{fichAlmacen}\PYG{p}{,} \PYG{n}{claveAlmacen}\PYG{p}{.}\PYG{n+na}{toCharArray}\PYG{p}{(}\PYG{p}{)}\PYG{p}{)}\PYG{p}{;}
            \PYG{c+cm}{/* Paso 2: obtener una fábrica de KeyManagers que ofrezcan}
\PYG{c+cm}{             * soporte al algoritmo por defecto*/}
            \PYG{n}{KeyManagerFactory} \PYG{n}{fabrica}\PYG{o}{=}
                            \PYG{n}{KeyManagerFactory}\PYG{p}{.}\PYG{n+na}{getInstance}\PYG{p}{(}
                                \PYG{n}{KeyManagerFactory}\PYG{p}{.}\PYG{n+na}{getDefaultAlgorithm}\PYG{p}{(}\PYG{p}{)}\PYG{p}{)}\PYG{p}{;}
            \PYG{n}{fabrica}\PYG{p}{.}\PYG{n+na}{init}\PYG{p}{(}\PYG{n}{almacen}\PYG{p}{,} \PYG{n}{claveAlmacen}\PYG{p}{.}\PYG{n+na}{toCharArray}\PYG{p}{(}\PYG{p}{)}\PYG{p}{)}\PYG{p}{;}
            \PYG{c+cm}{/* Paso 3:Intentamos obtener un contexto SSL}
\PYG{c+cm}{             * que ofrezca soporte a TLS (el sistema más}
\PYG{c+cm}{             * seguro hoy día) */}
            \PYG{n}{SSLContext} \PYG{n}{contextoSSL}\PYG{o}{=}\PYG{n}{SSLContext}\PYG{p}{.}\PYG{n+na}{getInstance}\PYG{p}{(}\PYG{l+s}{\PYGZdq{}}\PYG{l+s}{TLS}\PYG{l+s}{\PYGZdq{}}\PYG{p}{)}\PYG{p}{;}
            \PYG{n}{contextoSSL}\PYG{p}{.}\PYG{n+na}{init}\PYG{p}{(}\PYG{n}{fabrica}\PYG{p}{.}\PYG{n+na}{getKeyManagers}\PYG{p}{(}\PYG{p}{)}\PYG{p}{,} \PYG{k+kc}{null}\PYG{p}{,} \PYG{k+kc}{null}\PYG{p}{)}\PYG{p}{;}
            \PYG{c+cm}{/* Paso 4: Se obtiene una fábrica de sockets que permita}
\PYG{c+cm}{             * obtener un SSLServerSocket */}
            \PYG{n}{SSLServerSocketFactory} \PYG{n}{fabricaSockets}\PYG{o}{=}
                            \PYG{n}{contextoSSL}\PYG{p}{.}\PYG{n+na}{getServerSocketFactory}\PYG{p}{(}\PYG{p}{)}\PYG{p}{;}
            \PYG{n}{serverSocket}\PYG{o}{=}
                    \PYG{p}{(}\PYG{n}{SSLServerSocket}\PYG{p}{)}
                            \PYG{n}{fabricaSockets}\PYG{p}{.}\PYG{n+na}{createServerSocket}\PYG{p}{(}\PYG{n}{puerto}\PYG{p}{)}\PYG{p}{;}
            \PYG{k}{return} \PYG{n}{serverSocket}\PYG{p}{;}
    \PYG{p}{\PYGZcb{}}
    \PYG{k+kd}{public} \PYG{k+kt}{void} \PYG{n+nf}{escuchar}\PYG{p}{(}\PYG{p}{)}
        \PYG{k+kd}{throws} \PYG{n}{KeyManagementException}\PYG{p}{,} \PYG{n}{UnrecoverableKeyException}\PYG{p}{,}
            \PYG{n}{KeyStoreException}\PYG{p}{,} \PYG{n}{NoSuchAlgorithmException}\PYG{p}{,}
            \PYG{n}{CertificateException}\PYG{p}{,} \PYG{n}{IOException}
    \PYG{p}{\PYGZob{}}
            \PYG{n}{SSLServerSocket} \PYG{n}{socketServidor}\PYG{o}{=}\PYG{k}{this}\PYG{p}{.}\PYG{n+na}{getServerSocketSeguro}\PYG{p}{(}\PYG{p}{)}\PYG{p}{;}
            \PYG{n}{BufferedReader} \PYG{n}{entrada}\PYG{p}{;}
            \PYG{n}{PrintWriter} \PYG{n}{salida}\PYG{p}{;}
            \PYG{k}{while} \PYG{p}{(}\PYG{k+kc}{true}\PYG{p}{)}\PYG{p}{\PYGZob{}}
                    \PYG{n}{Socket} \PYG{n}{connRecibida}\PYG{o}{=}\PYG{n}{socketServidor}\PYG{p}{.}\PYG{n+na}{accept}\PYG{p}{(}\PYG{p}{)}\PYG{p}{;}
                    \PYG{n}{System}\PYG{p}{.}\PYG{n+na}{out}\PYG{p}{.}\PYG{n+na}{println}\PYG{p}{(}\PYG{l+s}{\PYGZdq{}}\PYG{l+s}{Conexion segura recibida}\PYG{l+s}{\PYGZdq{}}\PYG{p}{)}\PYG{p}{;}
                    \PYG{n}{entrada}\PYG{o}{=}
                        \PYG{k}{new} \PYG{n}{BufferedReader}\PYG{p}{(}
                        \PYG{k}{new} \PYG{n}{InputStreamReader}\PYG{p}{(}\PYG{n}{connRecibida}\PYG{p}{.}\PYG{n+na}{getInputStream}\PYG{p}{(}\PYG{p}{)}\PYG{p}{)}\PYG{p}{)}\PYG{p}{;}
                    \PYG{n}{salida}\PYG{o}{=}
                        \PYG{k}{new} \PYG{n}{PrintWriter}\PYG{p}{(}
                            \PYG{k}{new} \PYG{n}{OutputStreamWriter}\PYG{p}{(}
                            \PYG{n}{connRecibida}\PYG{p}{.}\PYG{n+na}{getOutputStream}\PYG{p}{(}\PYG{p}{)}\PYG{p}{)}\PYG{p}{)}\PYG{p}{;}
                    \PYG{n}{String} \PYG{n}{linea}\PYG{o}{=}\PYG{n}{entrada}\PYG{p}{.}\PYG{n+na}{readLine}\PYG{p}{(}\PYG{p}{)}\PYG{p}{;}
                    \PYG{n}{salida}\PYG{p}{.}\PYG{n+na}{println}\PYG{p}{(}\PYG{n}{linea}\PYG{p}{.}\PYG{n+na}{length}\PYG{p}{(}\PYG{p}{)}\PYG{p}{)}\PYG{p}{;}
                    \PYG{n}{salida}\PYG{p}{.}\PYG{n+na}{flush}\PYG{p}{(}\PYG{p}{)}\PYG{p}{;}
            \PYG{p}{\PYGZcb{}}
    \PYG{p}{\PYGZcb{}}
\end{sphinxVerbatim}

En el cliente se tienen que dar algunos pasos parecidos:
\begin{enumerate}
\sphinxsetlistlabels{\arabic}{enumi}{enumii}{}{.}%
\item {} 
En primer lugar se carga el almacén de claves del cliente (que contiene el certificado del servidor y que es la clave para poder «autenticar» el servidor)

\item {} 
El almacén del cliente se usará para crear un «gestor de confianza» (\sphinxcode{\sphinxupquote{TrustManager}}) que Java usará para determinar si puede confiar o no en una conexión. Usaremos un \sphinxcode{\sphinxupquote{TrustManagerFactory}} que usará el almacén del cliente para crear objetos que puedan gestionar la confianza.

\item {} 
Se creará un contexto SSL (\sphinxcode{\sphinxupquote{SSLContext}}) que se basará en los \sphinxcode{\sphinxupquote{TrustManager}} que pueda crear la fábrica.

\item {} 
A partir del contexto SSL el cliente ya puede crear un socket seguro (\sphinxcode{\sphinxupquote{SSLSocket}}) que puede usar para conectar con el servidor de forma segura.

\end{enumerate}

El código del cliente sería algo así:

\begin{sphinxVerbatim}[commandchars=\\\{\}]
\PYG{k+kd}{public} \PYG{k+kd}{class} \PYG{n+nc}{OtroCliente} \PYG{p}{\PYGZob{}}
    \PYG{n}{String} \PYG{n}{almacen}\PYG{o}{=}\PYG{l+s}{\PYGZdq{}}\PYG{l+s}{/home/usuario/clavescliente}\PYG{l+s}{\PYGZdq{}}\PYG{p}{;}
    \PYG{n}{String} \PYG{n}{clave}\PYG{o}{=}\PYG{l+s}{\PYGZdq{}}\PYG{l+s}{abcdabcd}\PYG{l+s}{\PYGZdq{}}\PYG{p}{;}
    \PYG{n}{SSLSocket} \PYG{n}{conexion}\PYG{p}{;}
    \PYG{k+kd}{public} \PYG{n+nf}{OtroCliente}\PYG{p}{(}\PYG{n}{String} \PYG{n}{ip}\PYG{p}{,} \PYG{k+kt}{int} \PYG{n}{puerto}\PYG{p}{)}
                    \PYG{k+kd}{throws} \PYG{n}{UnknownHostException}\PYG{p}{,} \PYG{n}{IOException}\PYG{p}{,}
                    \PYG{n}{KeyManagementException}\PYG{p}{,} \PYG{n}{NoSuchAlgorithmException}\PYG{p}{,}
                    \PYG{n}{KeyStoreException}\PYG{p}{,} \PYG{n}{CertificateException}\PYG{p}{\PYGZob{}}

            \PYG{n}{conexion}\PYG{o}{=}\PYG{k}{this}\PYG{p}{.}\PYG{n+na}{obtenerSocket}\PYG{p}{(}\PYG{n}{ip}\PYG{p}{,}\PYG{n}{puerto}\PYG{p}{)}\PYG{p}{;}
    \PYG{p}{\PYGZcb{}}
    \PYG{c+cm}{/* Envía un mensaje de prueba para verificar que la conexión}
\PYG{c+cm}{     * SSL es correcta */}
    \PYG{k+kd}{public} \PYG{k+kt}{void} \PYG{n+nf}{conectar}\PYG{p}{(}\PYG{p}{)} \PYG{k+kd}{throws} \PYG{n}{IOException}\PYG{p}{\PYGZob{}}
            \PYG{n}{System}\PYG{p}{.}\PYG{n+na}{out}\PYG{p}{.}\PYG{n+na}{println}\PYG{p}{(}\PYG{l+s}{\PYGZdq{}}\PYG{l+s}{Iniciando..}\PYG{l+s}{\PYGZdq{}}\PYG{p}{)}\PYG{p}{;}
            \PYG{n}{BufferedReader} \PYG{n}{entrada}\PYG{p}{;}
            \PYG{n}{PrintWriter} \PYG{n}{salida}\PYG{p}{;}
            \PYG{n}{entrada}\PYG{o}{=}\PYG{k}{new} \PYG{n}{BufferedReader}\PYG{p}{(}\PYG{k}{new} \PYG{n}{InputStreamReader}\PYG{p}{(}\PYG{n}{conexion}\PYG{p}{.}\PYG{n+na}{getInputStream}\PYG{p}{(}\PYG{p}{)}\PYG{p}{)}\PYG{p}{)}\PYG{p}{;}
            \PYG{n}{salida}\PYG{o}{=}\PYG{k}{new} \PYG{n}{PrintWriter}\PYG{p}{(}\PYG{k}{new} \PYG{n}{OutputStreamWriter}\PYG{p}{(}\PYG{n}{conexion}\PYG{p}{.}\PYG{n+na}{getOutputStream}\PYG{p}{(}\PYG{p}{)}\PYG{p}{)}\PYG{p}{)}\PYG{p}{;}
            \PYG{c+cm}{/* De esta linea se intenta averiguar la longitud*/}
            \PYG{n}{salida}\PYG{p}{.}\PYG{n+na}{println}\PYG{p}{(}\PYG{l+s}{\PYGZdq{}}\PYG{l+s}{1234567890}\PYG{l+s}{\PYGZdq{}}\PYG{p}{)}\PYG{p}{;}
            \PYG{n}{salida}\PYG{p}{.}\PYG{n+na}{flush}\PYG{p}{(}\PYG{p}{)}\PYG{p}{;}
            \PYG{c+cm}{/* Si todo va bien, el servidor nos contesta el numero*/}
            \PYG{n}{String} \PYG{n}{num}\PYG{o}{=}\PYG{n}{entrada}\PYG{p}{.}\PYG{n+na}{readLine}\PYG{p}{(}\PYG{p}{)}\PYG{p}{;}
            \PYG{k+kt}{int} \PYG{n}{longitud}\PYG{o}{=}\PYG{n}{Integer}\PYG{p}{.}\PYG{n+na}{parseInt}\PYG{p}{(}\PYG{n}{num}\PYG{p}{)}\PYG{p}{;}
            \PYG{n}{System}\PYG{p}{.}\PYG{n+na}{out}\PYG{p}{.}\PYG{n+na}{println}\PYG{p}{(}\PYG{l+s}{\PYGZdq{}}\PYG{l+s}{La longitud devuelta es:}\PYG{l+s}{\PYGZdq{}}\PYG{o}{+}\PYG{n}{longitud}\PYG{p}{)}\PYG{p}{;}

    \PYG{p}{\PYGZcb{}}
    \PYG{k+kd}{public} \PYG{n}{SSLSocket} \PYG{n+nf}{obtenerSocket}\PYG{p}{(}\PYG{n}{String} \PYG{n}{ip}\PYG{p}{,} \PYG{k+kt}{int} \PYG{n}{puerto}\PYG{p}{)}
                    \PYG{k+kd}{throws} \PYG{n}{KeyStoreException}\PYG{p}{,} \PYG{n}{NoSuchAlgorithmException}\PYG{p}{,}
                    \PYG{n}{CertificateException}\PYG{p}{,} \PYG{n}{IOException}\PYG{p}{,} \PYG{n}{KeyManagementException}
    \PYG{p}{\PYGZob{}}
            \PYG{n}{System}\PYG{p}{.}\PYG{n+na}{out}\PYG{p}{.}\PYG{n+na}{println}\PYG{p}{(}\PYG{l+s}{\PYGZdq{}}\PYG{l+s}{Obteniendo socket}\PYG{l+s}{\PYGZdq{}}\PYG{p}{)}\PYG{p}{;}
            \PYG{n}{SSLSocket} \PYG{n}{socket}\PYG{o}{=}\PYG{k+kc}{null}\PYG{p}{;}
            \PYG{c+cm}{/* Paso 1: se carga al almacén de claves}
\PYG{c+cm}{             * (que recordemos debe contener el}
\PYG{c+cm}{             * certificado del servidor)*/}
            \PYG{n}{KeyStore} \PYG{n}{almacenCliente}\PYG{o}{=}\PYG{n}{KeyStore}\PYG{p}{.}\PYG{n+na}{getInstance}\PYG{p}{(}\PYG{n}{KeyStore}\PYG{p}{.}\PYG{n+na}{getDefaultType}\PYG{p}{(}\PYG{p}{)}\PYG{p}{)}\PYG{p}{;}
            \PYG{n}{FileInputStream} \PYG{n}{ficheroAlmacenClaves}\PYG{o}{=}
                            \PYG{k}{new} \PYG{n}{FileInputStream}\PYG{p}{(} \PYG{k}{this}\PYG{p}{.}\PYG{n+na}{almacen} \PYG{p}{)}\PYG{p}{;}
            \PYG{n}{almacenCliente}\PYG{p}{.}\PYG{n+na}{load}\PYG{p}{(}\PYG{n}{ficheroAlmacenClaves}\PYG{p}{,} \PYG{n}{clave}\PYG{p}{.}\PYG{n+na}{toCharArray}\PYG{p}{(}\PYG{p}{)}\PYG{p}{)}\PYG{p}{;}
            \PYG{n}{System}\PYG{p}{.}\PYG{n+na}{out}\PYG{p}{.}\PYG{n+na}{println}\PYG{p}{(}\PYG{l+s}{\PYGZdq{}}\PYG{l+s}{Almacen cargado}\PYG{l+s}{\PYGZdq{}}\PYG{p}{)}\PYG{p}{;}
            \PYG{c+cm}{/* Paso 2, crearemos una fabrica de gestores de confianza}
\PYG{c+cm}{             * que use el almacén cargado antes (que contiene el}
\PYG{c+cm}{             * certificado del servidor)}
\PYG{c+cm}{             */}
            \PYG{n}{TrustManagerFactory} \PYG{n}{fabricaGestoresConfianza}\PYG{o}{=}
                            \PYG{n}{TrustManagerFactory}\PYG{p}{.}\PYG{n+na}{getInstance}\PYG{p}{(}
                                            \PYG{n}{TrustManagerFactory}\PYG{p}{.}\PYG{n+na}{getDefaultAlgorithm}\PYG{p}{(}\PYG{p}{)}\PYG{p}{)}\PYG{p}{;}
            \PYG{n}{fabricaGestoresConfianza}\PYG{p}{.}\PYG{n+na}{init}\PYG{p}{(}\PYG{n}{almacenCliente}\PYG{p}{)}\PYG{p}{;}
            \PYG{n}{System}\PYG{p}{.}\PYG{n+na}{out}\PYG{p}{.}\PYG{n+na}{println}\PYG{p}{(}\PYG{l+s}{\PYGZdq{}}\PYG{l+s}{Fabrica Trust creada}\PYG{l+s}{\PYGZdq{}}\PYG{p}{)}\PYG{p}{;}
            \PYG{c+cm}{/*Paso 3: se crea el contexto SSL, que ofrezca}
\PYG{c+cm}{             * soporte al algoritmo TLS*/}
            \PYG{n}{SSLContext} \PYG{n}{contexto}\PYG{o}{=}\PYG{n}{SSLContext}\PYG{p}{.}\PYG{n+na}{getInstance}\PYG{p}{(}\PYG{l+s}{\PYGZdq{}}\PYG{l+s}{TLS}\PYG{l+s}{\PYGZdq{}}\PYG{p}{)}\PYG{p}{;}
            \PYG{n}{contexto}\PYG{p}{.}\PYG{n+na}{init}\PYG{p}{(}
                            \PYG{k+kc}{null}\PYG{p}{,} \PYG{n}{fabricaGestoresConfianza}\PYG{p}{.}\PYG{n+na}{getTrustManagers}\PYG{p}{(}\PYG{p}{)}\PYG{p}{,} \PYG{k+kc}{null}\PYG{p}{)}\PYG{p}{;}
            \PYG{c+cm}{/* Paso 4: Se crea un socket que conecte con el servidor*/}
            \PYG{n}{System}\PYG{p}{.}\PYG{n+na}{out}\PYG{p}{.}\PYG{n+na}{println}\PYG{p}{(}\PYG{l+s}{\PYGZdq{}}\PYG{l+s}{Contexto creado}\PYG{l+s}{\PYGZdq{}}\PYG{p}{)}\PYG{p}{;}
            \PYG{n}{SSLSocketFactory} \PYG{n}{fabricaSockets}\PYG{o}{=}
                            \PYG{n}{contexto}\PYG{p}{.}\PYG{n+na}{getSocketFactory}\PYG{p}{(}\PYG{p}{)}\PYG{p}{;}
            \PYG{n}{socket}\PYG{o}{=}\PYG{p}{(}\PYG{n}{SSLSocket}\PYG{p}{)} \PYG{n}{fabricaSockets}\PYG{p}{.}\PYG{n+na}{createSocket}\PYG{p}{(}\PYG{n}{ip}\PYG{p}{,} \PYG{n}{puerto}\PYG{p}{)}\PYG{p}{;}
            \PYG{c+cm}{/* Y devolvemos el socket*/}
            \PYG{n}{System}\PYG{p}{.}\PYG{n+na}{out}\PYG{p}{.}\PYG{n+na}{println}\PYG{p}{(}\PYG{l+s}{\PYGZdq{}}\PYG{l+s}{Socket creado}\PYG{l+s}{\PYGZdq{}}\PYG{p}{)}\PYG{p}{;}
            \PYG{k}{return} \PYG{n}{socket}\PYG{p}{;}
    \PYG{p}{\PYGZcb{}}
\PYG{p}{\PYGZcb{}}
\end{sphinxVerbatim}


\section{Firmado de aplicaciones}
\label{\detokenize{textos/tema5:firmado-de-aplicaciones}}
Utilizando la criptografía de clave pública es posible «firmar» aplicaciones. El firmado es un mecanismo que permite al usuario de una aplicación el verificar que la aplicación no ha sido alterada desde que el programador la creó (virus o programas malignos, personal descontento con la empresa, etc…).

Antes de efectuar el firmado se debe disponer de un par de claves generadas con la herramienta \sphinxcode{\sphinxupquote{keytool}} mencionada anteriormente. Supongamos que el almacén de claves está creado y que en él hay uno o varios \sphinxstyleemphasis{alias} creados. El proceso de firmado es el siguiente:
\begin{enumerate}
\sphinxsetlistlabels{\arabic}{enumi}{enumii}{}{.}%
\item {} 
Crear la aplicación, que puede estar formada por un conjunto de clases pero que en última instancia tendrá un \sphinxcode{\sphinxupquote{main}}.

\item {} 
Empaquetar la aplicación con \sphinxcode{\sphinxupquote{jar cfe Aplicacion.jar com.ies.Aplicacion *}}. Este comando crea un fichero (\sphinxcode{\sphinxupquote{f}}) JAR en el cual el punto de entrada (\sphinxcode{\sphinxupquote{e}}) es la clase \sphinxcode{\sphinxupquote{com.ies.Aplicacion}} (que es la que tendrá el \sphinxcode{\sphinxupquote{main}}).

\item {} 
Puede comprobarse que la aplicación dentro del JAR se ejecuta correctamente con \sphinxcode{\sphinxupquote{java \sphinxhyphen{}jar Aplicacion.jar}}.

\item {} 
Ahora se puede ejecutar \sphinxcode{\sphinxupquote{jarsigner \sphinxhyphen{}keystore \textless{}ruta\sphinxhyphen{}almacen\textgreater{} Aplicacion.jar \textless{}alias\textgreater{}}}.

\end{enumerate}

Con estos pasos se tiene un aplicación firmada que el usuario puede verificar si así lo desea. De hecho, si se extrae el contenido del JAR con \sphinxcode{\sphinxupquote{jar \sphinxhyphen{}xf Aplicacion.jar}} se extraen los archivos \sphinxcode{\sphinxupquote{.class}} y un fichero \sphinxcode{\sphinxupquote{META\sphinxhyphen{}INF/Manifest}} que se puede abrir con un editor para ver que realmente está firmado.

Para que otras personas puedan comprobar que nuestra aplicacion es correcta los programadores deberemos exportar un certificado que los usuarios puedan importar para hacer el verificado. Recordemos que el comando es:

\begin{sphinxVerbatim}[commandchars=\\\{\}]
\PYG{n}{keytool} \PYG{o}{\PYGZhy{}}\PYG{n}{exportcert} \PYG{o}{\PYGZhy{}}\PYG{n}{keystore} \PYG{o}{.}\PYG{o}{.}\PYGZbs{}\PYG{n}{Almacen}\PYG{o}{.}\PYG{n}{store} \PYG{o}{\PYGZhy{}}\PYG{n}{file} \PYG{n}{Programador}\PYG{o}{.}\PYG{n}{cer} \PYG{o}{\PYGZhy{}}\PYG{n}{alias} \PYG{n}{Programador}
\end{sphinxVerbatim}


\section{Verificado de aplicaciones}
\label{\detokenize{textos/tema5:verificado-de-aplicaciones}}
Si ahora otro usuario desea ejecutar nuestra aplicación deberá importar nuestro certificado. El proceso de verificado es simple:
\begin{enumerate}
\sphinxsetlistlabels{\arabic}{enumi}{enumii}{}{.}%
\item {} 
El usuario importa el certificado.

\item {} 
Ahora que tiene el certificado puede comprobar la aplicación con \sphinxcode{\sphinxupquote{jarsigner \sphinxhyphen{}verify \sphinxhyphen{}keystore \textless{}ruta\sphinxhyphen{}almacen\textgreater{} Aplicacion.jar \textless{}alias\_del\_programador\textgreater{}}}

\end{enumerate}

El comando deberá responder con algo como \sphinxcode{\sphinxupquote{jar verified}}. Sin embargo si no tenemos un certificado firmado por alguna autoridad de certificación (CA) la herramienta se quejará de que algunos criterios de seguridad no se cumplen.


\section{Ejercicio}
\label{\detokenize{textos/tema5:ejercicio}}
Intenta extraer el archivo JAR y reemplaza el \sphinxcode{\sphinxupquote{.class}} por alguna otra clase. Vuelve a crear el archivo .JAR y vuelve a intentar verificarlo, ¿qué ocurre?


\section{Recordatorio}
\label{\detokenize{textos/tema5:recordatorio}}
Hemos hecho el proceso de firmado y verificado con \sphinxstylestrong{certificados autofirmados}, lo cual es útil para practicar pero \sphinxstylestrong{completamente inútil desde el punto de vista de la seguridad}. Para que un certificado sea seguro debemos hacer que previamente alguna autoridad de certificación nos lo firme primero (para lo cual suele ser habitual el tener que pagar).


\section{Política de seguridad.}
\label{\detokenize{textos/tema5:politica-de-seguridad}}
Java incluye un mecanismo para definir \sphinxstyleemphasis{políticas de seguridad}. La definición oficial de Java para una política es \sphinxstyleemphasis{objeto que especifica qué permisos están disponibles para el código en función de su origen y del usuario con el que se ejecutan}. Este origen puede ser los diversos directorios del sistema operativo o incluso direcciones URL.

Cabe destacar que \sphinxstylestrong{todo lo que se menciona aquí no funciona si el usuario tiene acceso al sistema y puede ejecutar el intérprete de Java sin restricciones}. Es decir, se necesita el trabajo de un administrador de sistemas para restringir la manera en la que el usuario ejecuta el código.

Supongamos entonces que estamos en un entorno seguro donde los programas Java se ejecutan utilizando un gestor de seguridad, es decir, se lanzan ejecutando \sphinxcode{\sphinxupquote{java \sphinxhyphen{}Djava.security.manager Clase}}. En principio, los programas no podrán hacer muchas cosas, como por ejemplo, conectarse a Internet o leer un fichero que no esté en el mismo directorio de la clase.


\section{Programación de mecanismos de control de acceso.}
\label{\detokenize{textos/tema5:programacion-de-mecanismos-de-control-de-acceso}}

\section{Pruebas y depuración.}
\label{\detokenize{textos/tema5:pruebas-y-depuracion}}

\chapter{Servicios web}
\label{\detokenize{textos/tema6:servicios-web}}\label{\detokenize{textos/tema6::doc}}

\section{Introducción}
\label{\detokenize{textos/tema6:introduccion}}
Un problema que ocurre a menudo es el siguiente: \sphinxstyleemphasis{¿como se puede conseguir
que aplicaciones escritas en distintos lenguajes se intercambien datos
entre sí?}

Supongamos que en un departamento de la empresa se dispone de una
aplicación escrita en C\# a la que se le puede suministrar un nombre de mes
y que devuelve los mejores clientes de ese mes. Y supongamos
por otro lado que se dispone de una aplicación Java de escritorio que
permite a los clientes mostrar gráficos de una manera muy cómoda. ¿Como
conseguir que la aplicación Java se integre con la aplicación en C\#?
\begin{itemize}
\item {} 
Una posibilidad sería reescribir la funcionalidad. Tomar la tarea hecha
en C\# y reescribirla en la aplicación Java. Evidentemente, esto llevaría bastante tiempo.

\item {} 
Otra posibilidad sería inventar algún protocolo que permitiese llevar los
datos entre distintas aplicaciones, lo cual también puede ser farragoso y
propenso a errores.

\end{itemize}

Sin embargo, los servicios web nos ofrecen una mejora sobre la segunda
posibilidad, al proporcionar \sphinxstylestrong{un mecanismo estandarizado para intercambiar
datos entre aplicaciones} Los servicios web facilitan la tarea de
construir programas que puedan intercambiarse datos incluso entre distintos
lenguajes así como facilitar dicho intercambio entre aplicaciones que ya se
hubiesen creado sin pensar en esta integración.


\section{Prerrequisitos}
\label{\detokenize{textos/tema6:prerrequisitos}}
Necesitaremos el siguiente software:

{\color{red}\bfseries{}*}JDK
{\color{red}\bfseries{}*}Apache ANT.
{\color{red}\bfseries{}*}Apache Axis


\chapter{Anexo: hilos con Swing}
\label{\detokenize{textos/anexo_hilos_con_swing:anexo-hilos-con-swing}}\label{\detokenize{textos/anexo_hilos_con_swing::doc}}

\section{Introducción}
\label{\detokenize{textos/anexo_hilos_con_swing:introduccion}}
Cuando creamos una aplicación de escritorio con Swing utilizaremos botones, cuadros de texto, etc… Sin embargo, \sphinxstylestrong{toda aplicación Swing utiliza hilos internamente} . Esto significa que si necesitamos resolver una tarea larga \sphinxstylestrong{podríamos dejar «bloqueada» la ventana principal mientras resolvemos la tarea} . Esto produce una sensación muy pobre en el usuario y además puede impulsarle a cerrar la aplicación creyendo que \sphinxstyleemphasis{«se ha colgado»} . Para evitarlo podemos crear aplicaciones Swing que utilicen hilos de manera que el interfaz principal se mantenga activo aunque se esté ejecutando una tarea larga en segundo plano. A modo de ejemplo, vamos a crear una aplicación que sume todos los números de un intervalo dado y que tenga un interfaz como el siguiente:

\begin{figure}[htbp]
\centering

\noindent\sphinxincludegraphics[scale=0.5]{{HilosSwing01}.png}
\end{figure}


\section{Paso 1: el interfaz}
\label{\detokenize{textos/anexo_hilos_con_swing:paso-1-el-interfaz}}
Empieza por crear un proyecto nuevo y diseña una interfaz como la anterior. En este interfaz tenemos los siguientes controles:
\begin{itemize}
\item {} 
El boton \sphinxcode{\sphinxupquote{btnSinHilos}}.

\item {} 
El boton \sphinxcode{\sphinxupquote{btnConHilos}}.

\item {} 
Una barra de progreso que llamaremos \sphinxcode{\sphinxupquote{pbProgreso}}.

\item {} 
Un área de texto que llamaremos \sphinxcode{\sphinxupquote{txtAreaTexto}}.

\end{itemize}


\section{Paso 2: el bloque del interfaz}
\label{\detokenize{textos/anexo_hilos_con_swing:paso-2-el-bloque-del-interfaz}}
Vamos a simular que al pulsar el boton \sphinxcode{\sphinxupquote{btnSinHilos}} se ejecuta una tarea larga. Para ello vamos a asociarle a dicho botón este código de gestion de eventos:

\begin{sphinxVerbatim}[commandchars=\\\{\}]
\PYG{k+kd}{private} \PYG{k+kt}{void} \PYG{n+nf}{btnSinHilosMouseClicked}\PYG{p}{(}\PYG{n}{java}\PYG{p}{.}\PYG{n+na}{awt}\PYG{p}{.}\PYG{n+na}{event}\PYG{p}{.}\PYG{n+na}{MouseEvent} \PYG{n}{evt}\PYG{p}{)} \PYG{p}{\PYGZob{}}
    \PYG{k+kt}{int} \PYG{n}{suma}\PYG{o}{=}\PYG{l+m+mi}{0}\PYG{p}{;}
    \PYG{k}{for} \PYG{p}{(}\PYG{k+kt}{int} \PYG{n}{i}\PYG{o}{=}\PYG{l+m+mi}{0}\PYG{p}{;} \PYG{n}{i}\PYG{o}{\PYGZlt{}}\PYG{l+m+mi}{2000}\PYG{p}{;} \PYG{n}{i}\PYG{o}{+}\PYG{o}{+}\PYG{p}{)}\PYG{p}{\PYGZob{}}
        \PYG{n}{suma}\PYG{o}{=}\PYG{n}{suma}\PYG{o}{+}\PYG{n}{i}\PYG{p}{;}
        \PYG{k}{try} \PYG{p}{\PYGZob{}}
            \PYG{n}{Thread}\PYG{p}{.}\PYG{n+na}{sleep}\PYG{p}{(}\PYG{l+m+mi}{1}\PYG{p}{)}\PYG{p}{;}
        \PYG{p}{\PYGZcb{}} \PYG{k}{catch} \PYG{p}{(}\PYG{n}{InterruptedException} \PYG{n}{ex}\PYG{p}{)} \PYG{p}{\PYGZob{}}
            \PYG{n}{System}\PYG{p}{.}\PYG{n+na}{out}\PYG{p}{.}\PYG{n+na}{println}\PYG{p}{(}\PYG{l+s}{\PYGZdq{}}\PYG{l+s}{Error, hilo interrumpido}\PYG{l+s}{\PYGZdq{}}\PYG{p}{)}\PYG{p}{;}
        \PYG{p}{\PYGZcb{}} \PYG{c+cm}{/*Fin del catch*/}
    \PYG{p}{\PYGZcb{}} \PYG{c+cm}{/* Fin del for*/}
\PYG{p}{\PYGZcb{}}
\end{sphinxVerbatim}

Si ejecutamos nuestra aplicación,despues pulsamos el botón \sphinxcode{\sphinxupquote{btnSinHilos}} e intentamos escribir en el cuadro de texto veremos que \sphinxstylestrong{NO OCURRE NADA} . El programa no está mostrando ningún progreso aunque en realidad si «toma nota de las teclas pulsadas» y nos las mostrará tan pronto acabe la suma. Sin embargo es evidente que este comportamiento no es muy recomendable.


\section{Paso 3: la clase \sphinxstyleliteralintitle{\sphinxupquote{SwingWorker}}}
\label{\detokenize{textos/anexo_hilos_con_swing:paso-3-la-clase-swingworker}}
La clase \sphinxcode{\sphinxupquote{SwingWorker\textless{}Tipo1, Tipo2\textgreater{}}} es una clase genérica diseñada para heredar de ella y ejecutar tareas en segundo plano. En ella
\begin{itemize}
\item {} 
El \sphinxcode{\sphinxupquote{Tipo1}} es el tipo del resultado que devolveremos.

\item {} 
El \sphinxcode{\sphinxupquote{Tipo2}} es el tipo que se utilizará para medir el progreso.

\end{itemize}

Supongamos que deseamos hacer la misma suma, ir mostrando el progreso y al terminar devolver un mensaje con el resultado en forma de cadena, podemos crear una clase como la siguiente:

\begin{sphinxVerbatim}[commandchars=\\\{\}]
\PYG{k+kd}{public} \PYG{k+kd}{class} \PYG{n+nc}{TareaParalelizada} \PYG{k+kd}{extends} \PYG{n}{SwingWorker}\PYG{o}{\PYGZlt{}}\PYG{n}{String}\PYG{p}{,} \PYG{n}{Integer}\PYG{o}{\PYGZgt{}}\PYG{p}{\PYGZob{}}
    \PYG{c+cm}{/* Barra de progreso que la tarea irá actualizando a medida}
\PYG{c+cm}{    que los cálculos progresen */}
    \PYG{n}{JProgressBar} \PYG{n}{pbBarraProgreso}\PYG{p}{;}

    \PYG{c+cm}{/*Intervalo de números que se van a sumar*/}
    \PYG{k+kt}{int} \PYG{n}{min}\PYG{p}{,} \PYG{n}{max}\PYG{p}{;}

    \PYG{c+cm}{/*Constructor*/}
    \PYG{k+kd}{public} \PYG{n+nf}{TareaParalelizada}\PYG{p}{(}\PYG{n}{JProgressBar} \PYG{n}{pbBarraProgreso}\PYG{p}{,} \PYG{k+kt}{int} \PYG{n}{min}\PYG{p}{,} \PYG{k+kt}{int} \PYG{n}{max}\PYG{p}{)} \PYG{p}{\PYGZob{}}
        \PYG{k}{this}\PYG{p}{.}\PYG{n+na}{pbBarraProgreso} \PYG{o}{=} \PYG{n}{pbBarraProgreso}\PYG{p}{;}
        \PYG{k}{this}\PYG{p}{.}\PYG{n+na}{min} \PYG{o}{=} \PYG{n}{min}\PYG{p}{;}
        \PYG{k}{this}\PYG{p}{.}\PYG{n+na}{max} \PYG{o}{=} \PYG{n}{max}\PYG{p}{;}
    \PYG{p}{\PYGZcb{}}

    \PYG{n+nd}{@Override}
    \PYG{k+kd}{protected} \PYG{n}{String} \PYG{n+nf}{doInBackground}\PYG{p}{(}\PYG{p}{)} \PYG{k+kd}{throws} \PYG{n}{Exception} \PYG{p}{\PYGZob{}}
        \PYG{c+cm}{/*Avisamos a la barra de cuales son los valores más pequeños}
\PYG{c+cm}{        y más grandes que se van a recorrer */}
        \PYG{k}{this}\PYG{p}{.}\PYG{n+na}{pbBarraProgreso}\PYG{p}{.}\PYG{n+na}{setMaximum}\PYG{p}{(}\PYG{n}{max}\PYG{p}{)}\PYG{p}{;}
        \PYG{k}{this}\PYG{p}{.}\PYG{n+na}{pbBarraProgreso}\PYG{p}{.}\PYG{n+na}{setMinimum}\PYG{p}{(}\PYG{n}{min}\PYG{p}{)}\PYG{p}{;}
        \PYG{k+kt}{int} \PYG{n}{suma}\PYG{o}{=}\PYG{l+m+mi}{0}\PYG{p}{;}

        \PYG{c+cm}{/*Recorremos los números...*/}
        \PYG{k}{for} \PYG{p}{(}\PYG{k+kt}{int} \PYG{n}{i}\PYG{o}{=}\PYG{l+m+mi}{0}\PYG{p}{;} \PYG{n}{i}\PYG{o}{\PYGZlt{}}\PYG{l+m+mi}{2000}\PYG{p}{;} \PYG{n}{i}\PYG{o}{+}\PYG{o}{+}\PYG{p}{)}\PYG{p}{\PYGZob{}}
            \PYG{n}{suma}\PYG{o}{=}\PYG{n}{suma}\PYG{o}{+}\PYG{n}{i}\PYG{p}{;}
            \PYG{n}{Thread}\PYG{p}{.}\PYG{n+na}{sleep}\PYG{p}{(}\PYG{l+m+mi}{1}\PYG{p}{)}\PYG{p}{;}
            \PYG{c+cm}{/*...y actualizamos la barra para que el usuario pueda}
\PYG{c+cm}{            ir viendo el progreso*/}
            \PYG{k}{this}\PYG{p}{.}\PYG{n+na}{pbBarraProgreso}\PYG{p}{.}\PYG{n+na}{setValue}\PYG{p}{(}\PYG{n}{i}\PYG{p}{)}\PYG{p}{;}
        \PYG{p}{\PYGZcb{}}
        \PYG{k}{return} \PYG{l+s}{\PYGZdq{}}\PYG{l+s}{Sumado:}\PYG{l+s}{\PYGZdq{}}\PYG{o}{+}\PYG{n}{suma}\PYG{p}{;}
    \PYG{p}{\PYGZcb{}}
\PYG{p}{\PYGZcb{}}
\end{sphinxVerbatim}

Y despues asociemos el siguiente código al gestor del evento \sphinxcode{\sphinxupquote{click}} para el \sphinxcode{\sphinxupquote{btnBotonConHilos}}.

\begin{sphinxVerbatim}[commandchars=\\\{\}]
\PYG{k+kd}{private} \PYG{k+kt}{void} \PYG{n+nf}{btnConHilosMouseClicked}\PYG{p}{(}\PYG{n}{java}\PYG{p}{.}\PYG{n+na}{awt}\PYG{p}{.}\PYG{n+na}{event}\PYG{p}{.}\PYG{n+na}{MouseEvent} \PYG{n}{evt}\PYG{p}{)} \PYG{p}{\PYGZob{}}
    \PYG{n}{TareaParalelizada} \PYG{n}{t}\PYG{o}{=}\PYG{k}{new} \PYG{n}{TareaParalelizada}\PYG{p}{(}\PYG{k}{this}\PYG{p}{.}\PYG{n+na}{pbProgreso}\PYG{p}{,} \PYG{l+m+mi}{1}\PYG{p}{,} \PYG{l+m+mi}{2000}\PYG{p}{)}\PYG{p}{;}
    \PYG{n}{t}\PYG{p}{.}\PYG{n+na}{execute}\PYG{p}{(}\PYG{p}{)}\PYG{p}{;}
\PYG{p}{\PYGZcb{}}
\end{sphinxVerbatim}

si ahora repetimos la ejecución de la aplicación pero ahora pulsamos el boton «Con hilos» veremos que \sphinxstylestrong{SÍ PODEMOS INTERACTUAR CON EL ÁREA DE TEXTO} . De hecho podríamos hacer cualquier otra cosa y la vez ir visualizando el progreso de la tarea, lo cual es claramente mucho más recomendable.


\chapter{Anexos diversos.}
\label{\detokenize{textos/anexos:anexos-diversos}}\label{\detokenize{textos/anexos::doc}}

\section{Compartición de código entre proyectos Gradle de Netbeans}
\label{\detokenize{textos/anexos:comparticion-de-codigo-entre-proyectos-gradle-de-netbeans}}

\section{Compartición de código entre proyectos Maven en Netbeans}
\label{\detokenize{textos/anexos:comparticion-de-codigo-entre-proyectos-maven-en-netbeans}}

\section{La clase \sphinxstyleliteralintitle{\sphinxupquote{UtilidadesFicheros}}}
\label{\detokenize{textos/anexos:la-clase-utilidadesficheros}}
A continuación se muestra el código completo de la clase \sphinxcode{\sphinxupquote{UtilidadesFicheros}}:

\begin{sphinxVerbatim}[commandchars=\\\{\}]
\PYG{k+kd}{public} \PYG{k+kd}{class} \PYG{n+nc}{UtilidadesFicheros} \PYG{p}{\PYGZob{}}

    \PYG{k+kd}{public} \PYG{k+kd}{static} \PYG{n}{BufferedReader} \PYG{n+nf}{getBufferedReader}\PYG{p}{(}
        \PYG{n}{String} \PYG{n}{nombreFichero}\PYG{p}{)} \PYG{k+kd}{throws}
    \PYG{n}{FileNotFoundException} \PYG{p}{\PYGZob{}}
        \PYG{n}{FileReader} \PYG{n}{lector}\PYG{p}{;}
        \PYG{n}{lector} \PYG{o}{=} \PYG{k}{new} \PYG{n}{FileReader}\PYG{p}{(}\PYG{n}{nombreFichero}\PYG{p}{)}\PYG{p}{;}
        \PYG{n}{BufferedReader} \PYG{n}{bufferedReader}\PYG{p}{;}
        \PYG{n}{bufferedReader} \PYG{o}{=} \PYG{k}{new} \PYG{n}{BufferedReader}\PYG{p}{(}\PYG{n}{lector}\PYG{p}{)}\PYG{p}{;}
        \PYG{k}{return} \PYG{n}{bufferedReader}\PYG{p}{;}
    \PYG{p}{\PYGZcb{}}

    \PYG{k+kd}{public} \PYG{k+kd}{static} \PYG{n}{PrintWriter} \PYG{n+nf}{getPrintWriter}\PYG{p}{(}
        \PYG{n}{String} \PYG{n}{nombreFichero}\PYG{p}{)} \PYG{k+kd}{throws} \PYG{n}{IOException} \PYG{p}{\PYGZob{}}
        \PYG{n}{PrintWriter} \PYG{n}{printWriter}\PYG{p}{;}
        \PYG{n}{FileWriter} \PYG{n}{fileWriter}\PYG{p}{;}
        \PYG{n}{fileWriter} \PYG{o}{=} \PYG{k}{new} \PYG{n}{FileWriter}\PYG{p}{(}\PYG{n}{nombreFichero}\PYG{p}{)}\PYG{p}{;}
        \PYG{n}{printWriter} \PYG{o}{=} \PYG{k}{new} \PYG{n}{PrintWriter}\PYG{p}{(}\PYG{n}{fileWriter}\PYG{p}{)}\PYG{p}{;}
        \PYG{k}{return} \PYG{n}{printWriter}\PYG{p}{;}
        \PYG{c+c1}{//Fin de getPrintWriter}
    \PYG{p}{\PYGZcb{}}

    \PYG{k+kd}{public} \PYG{k+kd}{static} \PYG{n}{ArrayList}\PYG{o}{\PYGZlt{}}\PYG{n}{String}\PYG{o}{\PYGZgt{}} \PYG{n+nf}{getLineasFichero}\PYG{p}{(}
        \PYG{n}{String} \PYG{n}{nombreFichero}\PYG{p}{)} \PYG{k+kd}{throws} \PYG{n}{IOException} \PYG{p}{\PYGZob{}}
        \PYG{n}{ArrayList}\PYG{o}{\PYGZlt{}}\PYG{n}{String}\PYG{o}{\PYGZgt{}} \PYG{n}{lineas} \PYG{o}{=} \PYG{k}{new} \PYG{n}{ArrayList}\PYG{o}{\PYGZlt{}}\PYG{n}{String}\PYG{o}{\PYGZgt{}}\PYG{p}{(}\PYG{p}{)}\PYG{p}{;}
        \PYG{n}{BufferedReader} \PYG{n}{bfr} \PYG{o}{=} \PYG{n}{getBufferedReader}\PYG{p}{(}\PYG{n}{nombreFichero}\PYG{p}{)}\PYG{p}{;}
        \PYG{c+c1}{//Leemos líneas del fichero...}
        \PYG{n}{String} \PYG{n}{linea} \PYG{o}{=} \PYG{n}{bfr}\PYG{p}{.}\PYG{n+na}{readLine}\PYG{p}{(}\PYG{p}{)}\PYG{p}{;}
        \PYG{k}{while} \PYG{p}{(}\PYG{n}{linea} \PYG{o}{!}\PYG{o}{=} \PYG{k+kc}{null}\PYG{p}{)} \PYG{p}{\PYGZob{}}
            \PYG{c+c1}{//Y las añadimos al array}
            \PYG{n}{lineas}\PYG{p}{.}\PYG{n+na}{add}\PYG{p}{(}\PYG{n}{linea}\PYG{p}{)}\PYG{p}{;}
            \PYG{n}{linea} \PYG{o}{=} \PYG{n}{bfr}\PYG{p}{.}\PYG{n+na}{readLine}\PYG{p}{(}\PYG{p}{)}\PYG{p}{;}
        \PYG{p}{\PYGZcb{}}
        \PYG{c+c1}{//Fin del bucle que lee líneas}
        \PYG{k}{return} \PYG{n}{lineas}\PYG{p}{;}
    \PYG{p}{\PYGZcb{}}

    \PYG{c+c1}{//Fin de getLineasFichero}
    \PYG{k+kd}{public} \PYG{k+kd}{static} \PYG{k+kt}{long} \PYG{n+nf}{getSuma}\PYG{p}{(}\PYG{n}{String}\PYG{o}{[}\PYG{o}{]}
                               \PYG{n}{listaNombresFichero}\PYG{p}{)} \PYG{p}{\PYGZob{}}
        \PYG{k+kt}{long} \PYG{n}{suma} \PYG{o}{=} \PYG{l+m+mi}{0}\PYG{p}{;}
        \PYG{n}{ArrayList}\PYG{o}{\PYGZlt{}}\PYG{n}{String}\PYG{o}{\PYGZgt{}} \PYG{n}{lineas}\PYG{p}{;}
        \PYG{n}{String} \PYG{n}{lineaCantidad}\PYG{p}{;}
        \PYG{k+kt}{long} \PYG{n}{cantidad}\PYG{p}{;}
        \PYG{k}{for} \PYG{p}{(}\PYG{n}{String} \PYG{n}{nombreFichero} \PYG{p}{:} \PYG{n}{listaNombresFichero}\PYG{p}{)} \PYG{p}{\PYGZob{}}
            \PYG{k}{try} \PYG{p}{\PYGZob{}}
                \PYG{c+c1}{//Recuperamos todas las lineas}
                \PYG{n}{lineas} \PYG{o}{=} \PYG{n}{getLineasFichero}\PYG{p}{(}\PYG{n}{nombreFichero}\PYG{p}{)}\PYG{p}{;}
                \PYG{c+c1}{//Pero solo nos interesa la primera}
                \PYG{n}{lineaCantidad} \PYG{o}{=} \PYG{n}{lineas}\PYG{p}{.}\PYG{n+na}{get}\PYG{p}{(}\PYG{l+m+mi}{0}\PYG{p}{)}\PYG{p}{;}
                \PYG{c+c1}{//Convertimos la linea a número}
                \PYG{n}{cantidad} \PYG{o}{=} \PYG{n}{Long}\PYG{p}{.}\PYG{n+na}{parseLong}\PYG{p}{(}\PYG{n}{lineaCantidad}\PYG{p}{)}\PYG{p}{;}
                \PYG{c+c1}{//Y se incrementa la suma total}
                \PYG{n}{suma} \PYG{o}{=} \PYG{n}{suma} \PYG{o}{+} \PYG{n}{cantidad}\PYG{p}{;}
            \PYG{p}{\PYGZcb{}} \PYG{k}{catch} \PYG{p}{(}\PYG{n}{IOException} \PYG{n}{e}\PYG{p}{)} \PYG{p}{\PYGZob{}}
                \PYG{n}{System}\PYG{p}{.}\PYG{n+na}{err}\PYG{p}{.}\PYG{n+na}{println}\PYG{p}{(}\PYG{l+s}{\PYGZdq{}}\PYG{l+s}{Fallo al procesar el fichero }\PYG{l+s}{\PYGZdq{}}
                                   \PYG{o}{+} \PYG{n}{nombreFichero}\PYG{p}{)}\PYG{p}{;}
                \PYG{c+c1}{//Fin del catch}
            \PYG{p}{\PYGZcb{}}
            \PYG{c+c1}{//Fin del for que recorre los nombres de fichero}
        \PYG{p}{\PYGZcb{}}
        \PYG{k}{return} \PYG{n}{suma}\PYG{p}{;}
    \PYG{p}{\PYGZcb{}}
\PYG{p}{\PYGZcb{}}
\end{sphinxVerbatim}


\section{Solución al problema del simulador de Casino}
\label{\detokenize{textos/anexos:solucion-al-problema-del-simulador-de-casino}}
Se ha optado por un diseño muy simple y con un acoplamiento muy alto. En pocas palabras hay solo dos clases:
\begin{enumerate}
\sphinxsetlistlabels{\arabic}{enumi}{enumii}{}{.}%
\item {} 
La clase \sphinxcode{\sphinxupquote{Banca}}: ejecuta toda la simulación y acepta apuestas de jugadores. La banca crea los Jugadores (que serán hilos), y luego pasará a ejecutar la simulación, que puede pasar por diversos estados. Algunos métodos de la banca deben ser \sphinxcode{\sphinxupquote{synchronized}}, ya que todos los hilos Jugador tienen una referencia a la clase \sphinxcode{\sphinxupquote{Banca}} y por lo tanto \sphinxstylestrong{puede haber varios hilos intentando entrar en un cierto método}.

\item {} 
La clase \sphinxcode{\sphinxupquote{Jugador}}: no necesita métodos \sphinxcode{\sphinxupquote{synchronized}}, ya que cada \sphinxcode{\sphinxupquote{Jugador}} accede solo a su propia información. \sphinxstyleemphasis{Sin embargo, si hubiese más de una banca (quizá porque un Jugador jugase a varias ruletas a la vez) la situación sería muy distinta}.

\item {} 
En realidad, la clase \sphinxcode{\sphinxupquote{Jugador}} es solo una base para poder crear distintos tipos de Jugador que siguen distintas estrategias.

\end{enumerate}

En las secciones siguientes se ilustra el código de las distintas clases.


\subsection{La clase Banca}
\label{\detokenize{textos/anexos:la-clase-banca}}
\begin{sphinxVerbatim}[commandchars=\\\{\}]
\PYG{k+kd}{public} \PYG{k+kd}{class} \PYG{n+nc}{Banca} \PYG{p}{\PYGZob{}}

    \PYG{k+kd}{protected} \PYG{k+kt}{long} \PYG{n}{saldo}\PYG{p}{;}

    \PYG{k+kd}{protected} \PYG{k+kt}{boolean} \PYG{n}{enBancarrota}\PYG{p}{;}

    \PYG{k+kd}{protected} \PYG{n}{Random} \PYG{n}{generador}\PYG{p}{;}

    \PYG{k+kd}{protected} \PYG{k+kt}{boolean} \PYG{n}{sePuedenHacerApuestas}\PYG{p}{;}

    \PYG{k+kd}{protected} \PYG{k+kt}{int} \PYG{n}{numeroGanador}\PYG{p}{;}

    \PYG{k+kd}{public} \PYG{k+kd}{enum} \PYG{n}{Estado} \PYG{p}{\PYGZob{}}

        \PYG{n}{INICIO}\PYG{p}{,} \PYG{n}{ACEPTANDO\PYGZus{}APUESTAS}\PYG{p}{,} \PYG{n}{RULETA\PYGZus{}GIRANDO}\PYG{p}{,} \PYG{n}{PAGANDO\PYGZus{}APUESTAS}\PYG{p}{,} \PYG{n}{EN\PYGZus{}BANCARROTA}
    \PYG{p}{\PYGZcb{}}

    \PYG{p}{;}

    \PYG{k+kd}{private} \PYG{n}{Estado} \PYG{n}{estadoRuleta}\PYG{p}{;}

    \PYG{k+kd}{private} \PYG{n}{ArrayList}\PYG{o}{\PYGZlt{}}\PYG{n}{Jugador}\PYG{o}{\PYGZgt{}} \PYG{n}{apostadores}\PYG{p}{;}

    \PYG{k+kd}{public} \PYG{n+nf}{Banca}\PYG{p}{(}\PYG{k+kt}{long} \PYG{n}{saldoInicial}\PYG{p}{)} \PYG{p}{\PYGZob{}}
        \PYG{n}{saldo} \PYG{o}{=} \PYG{n}{saldoInicial}\PYG{p}{;}
        \PYG{n}{enBancarrota} \PYG{o}{=} \PYG{k+kc}{false}\PYG{p}{;}
        \PYG{n}{estadoRuleta} \PYG{o}{=} \PYG{n}{Estado}\PYG{p}{.}\PYG{n+na}{INICIO}\PYG{p}{;}
        \PYG{n}{generador} \PYG{o}{=} \PYG{k}{new} \PYG{n}{Random}\PYG{p}{(}\PYG{p}{)}\PYG{p}{;}
        \PYG{n}{apostadores} \PYG{o}{=} \PYG{k}{new} \PYG{n}{ArrayList}\PYG{o}{\PYGZlt{}}\PYG{n}{Jugador}\PYG{o}{\PYGZgt{}}\PYG{p}{(}\PYG{p}{)}\PYG{p}{;}
    \PYG{p}{\PYGZcb{}}

    \PYG{k+kd}{public} \PYG{k+kd}{synchronized} \PYG{k+kt}{boolean} \PYG{n+nf}{enBancarrota}\PYG{p}{(}\PYG{p}{)} \PYG{p}{\PYGZob{}}
        \PYG{k}{return} \PYG{n}{enBancarrota}\PYG{p}{;}
    \PYG{p}{\PYGZcb{}}

    \PYG{k+kd}{public} \PYG{k+kd}{synchronized} \PYG{k+kt}{void} \PYG{n+nf}{sumarSaldo}\PYG{p}{(}
        \PYG{k+kt}{long} \PYG{n}{cantidad}\PYG{p}{)} \PYG{p}{\PYGZob{}}
        \PYG{n}{saldo} \PYG{o}{=} \PYG{n}{saldo} \PYG{o}{+} \PYG{n}{cantidad}\PYG{p}{;}
    \PYG{p}{\PYGZcb{}}

    \PYG{k+kd}{public} \PYG{k+kd}{synchronized} \PYG{k+kt}{void} \PYG{n+nf}{restarSaldo}\PYG{p}{(}
        \PYG{k+kt}{long} \PYG{n}{cantidad}\PYG{p}{)} \PYG{p}{\PYGZob{}}
        \PYG{k}{if} \PYG{p}{(}\PYG{n}{saldo} \PYG{o}{\PYGZhy{}} \PYG{n}{cantidad} \PYG{o}{\PYGZlt{}}\PYG{o}{=} \PYG{l+m+mi}{0}\PYG{p}{)} \PYG{p}{\PYGZob{}}
            \PYG{n}{saldo} \PYG{o}{=} \PYG{l+m+mi}{0}\PYG{p}{;}
            \PYG{n}{estadoRuleta} \PYG{o}{=} \PYG{n}{Estado}\PYG{p}{.}\PYG{n+na}{EN\PYGZus{}BANCARROTA}\PYG{p}{;}
            \PYG{k}{return}\PYG{p}{;}
        \PYG{p}{\PYGZcb{}}
        \PYG{n}{saldo} \PYG{o}{=} \PYG{n}{saldo} \PYG{o}{\PYGZhy{}} \PYG{n}{cantidad}\PYG{p}{;}
    \PYG{p}{\PYGZcb{}}

    \PYG{k+kd}{public} \PYG{k+kd}{synchronized} \PYG{k+kt}{void} \PYG{n+nf}{aceptarApuesta}\PYG{p}{(}
        \PYG{n}{Jugador} \PYG{n}{jugador}\PYG{p}{)} \PYG{p}{\PYGZob{}}
        \PYG{k}{if} \PYG{p}{(}\PYG{n}{estadoRuleta} \PYG{o}{=}\PYG{o}{=} \PYG{n}{Estado}\PYG{p}{.}\PYG{n+na}{ACEPTANDO\PYGZus{}APUESTAS}\PYG{p}{)} \PYG{p}{\PYGZob{}}
            \PYG{n}{apostadores}\PYG{p}{.}\PYG{n+na}{add}\PYG{p}{(}\PYG{n}{jugador}\PYG{p}{)}\PYG{p}{;}
        \PYG{p}{\PYGZcb{}}
    \PYG{p}{\PYGZcb{}}

    \PYG{k+kd}{public} \PYG{k+kd}{synchronized} \PYG{k+kt}{boolean} \PYG{n+nf}{aceptaApuestas}\PYG{p}{(}\PYG{p}{)} \PYG{p}{\PYGZob{}}
        \PYG{k}{if} \PYG{p}{(}\PYG{n}{estadoRuleta} \PYG{o}{=}\PYG{o}{=} \PYG{n}{Estado}\PYG{p}{.}\PYG{n+na}{ACEPTANDO\PYGZus{}APUESTAS}\PYG{p}{)} \PYG{p}{\PYGZob{}}
            \PYG{k}{return} \PYG{k+kc}{true}\PYG{p}{;}
        \PYG{p}{\PYGZcb{}}
        \PYG{k}{return} \PYG{k+kc}{false}\PYG{p}{;}
    \PYG{p}{\PYGZcb{}}

    \PYG{k+kd}{public} \PYG{k+kt}{void} \PYG{n+nf}{comunicarNumeroGanador}\PYG{p}{(}\PYG{k+kt}{int} \PYG{n}{numero}\PYG{p}{)} \PYG{p}{\PYGZob{}}
        \PYG{c+cm}{/* Al pasar el número a los jugadores, ellos nos}
\PYG{c+cm}{         * irán restando el saldo que les corresponda por haber ganado */}
        \PYG{k+kt}{int} \PYG{n}{numApostadores} \PYG{o}{=} \PYG{n}{apostadores}\PYG{p}{.}\PYG{n+na}{size}\PYG{p}{(}\PYG{p}{)}\PYG{p}{;}
        \PYG{k}{for} \PYG{p}{(}\PYG{n}{Jugador} \PYG{n}{apostador} \PYG{p}{:} \PYG{n}{apostadores}\PYG{p}{)} \PYG{p}{\PYGZob{}}
            \PYG{n}{apostador}\PYG{p}{.}\PYG{n+na}{comunicarNumero}\PYG{p}{(}\PYG{n}{numeroGanador}\PYG{p}{)}\PYG{p}{;}
        \PYG{p}{\PYGZcb{}}
        \PYG{c+cm}{/* Una vez comunicadas todas las apuestas, borramos}
\PYG{c+cm}{         * el vector. La partida va a volver a empezar */}
        \PYG{n}{apostadores}\PYG{p}{.}\PYG{n+na}{clear}\PYG{p}{(}\PYG{p}{)}\PYG{p}{;}
    \PYG{p}{\PYGZcb{}}

    \PYG{k+kd}{public} \PYG{k+kt}{void} \PYG{n+nf}{girarRuleta}\PYG{p}{(}\PYG{p}{)} \PYG{k+kd}{throws}
    \PYG{n}{InterruptedException} \PYG{p}{\PYGZob{}}
        \PYG{k+kt}{int} \PYG{n}{segundosAzar}\PYG{p}{;}
        \PYG{n}{System}\PYG{p}{.}\PYG{n+na}{out}\PYG{p}{.}\PYG{n+na}{println}\PYG{p}{(}\PYG{l+s}{\PYGZdq{}}\PYG{l+s}{Empieza el juego!}\PYG{l+s}{\PYGZdq{}}\PYG{p}{)}\PYG{p}{;}
        \PYG{k}{while} \PYG{p}{(}\PYG{n}{estadoRuleta} \PYG{o}{!}\PYG{o}{=} \PYG{n}{Estado}\PYG{p}{.}\PYG{n+na}{EN\PYGZus{}BANCARROTA}\PYG{p}{)} \PYG{p}{\PYGZob{}}
            \PYG{n}{estadoRuleta} \PYG{o}{=} \PYG{n}{Estado}\PYG{p}{.}\PYG{n+na}{ACEPTANDO\PYGZus{}APUESTAS}\PYG{p}{;}
            \PYG{c+cm}{/* Se eligen unos milisegundos al azar para que los jugadores}
\PYG{c+cm}{             * elijan, aunque quizá no todos puedan llegar a apostar}
\PYG{c+cm}{             */}
            \PYG{n}{segundosAzar} \PYG{o}{=} \PYG{l+m+mi}{1} \PYG{o}{+} \PYG{n}{generador}\PYG{p}{.}\PYG{n+na}{nextInt}\PYG{p}{(}\PYG{l+m+mi}{3}\PYG{p}{)}\PYG{p}{;}
            \PYG{n}{System}\PYG{p}{.}\PYG{n+na}{out}\PYG{p}{.}\PYG{n+na}{println}\PYG{p}{(}\PYG{l+s}{\PYGZdq{}}\PYG{l+s}{Hagan juego, tienen Vds }\PYG{l+s}{\PYGZdq{}} \PYG{o}{+}
                               \PYG{n}{segundosAzar} \PYG{o}{+} \PYG{l+s}{\PYGZdq{}}\PYG{l+s}{ segundos}\PYG{l+s}{\PYGZdq{}}\PYG{p}{)}\PYG{p}{;}
            \PYG{n}{Thread}\PYG{p}{.}\PYG{n+na}{sleep}\PYG{p}{(}\PYG{l+m+mi}{1000} \PYG{o}{*} \PYG{n}{segundosAzar}\PYG{p}{)}\PYG{p}{;}
            \PYG{n}{System}\PYG{p}{.}\PYG{n+na}{out}\PYG{p}{.}\PYG{n+na}{println}\PYG{p}{(}\PYG{l+s}{\PYGZdq{}}\PYG{l+s}{Ya no va más, señores. ¡Girando!}\PYG{l+s}{\PYGZdq{}}\PYG{p}{)}\PYG{p}{;}
            \PYG{n}{estadoRuleta} \PYG{o}{=} \PYG{n}{Estado}\PYG{p}{.}\PYG{n+na}{RULETA\PYGZus{}GIRANDO}\PYG{p}{;}
            \PYG{n}{Thread}\PYG{p}{.}\PYG{n+na}{sleep}\PYG{p}{(}\PYG{l+m+mi}{3000}\PYG{p}{)}\PYG{p}{;}
            \PYG{n}{numeroGanador} \PYG{o}{=} \PYG{n}{generador}\PYG{p}{.}\PYG{n+na}{nextInt}\PYG{p}{(}\PYG{l+m+mi}{37}\PYG{p}{)}\PYG{p}{;}
            \PYG{n}{System}\PYG{p}{.}\PYG{n+na}{out}\PYG{p}{.}\PYG{n+na}{println}\PYG{p}{(}\PYG{l+s}{\PYGZdq{}}\PYG{l+s}{El número ganador es el :}\PYG{l+s}{\PYGZdq{}} \PYG{o}{+}
                               \PYG{n}{numeroGanador}\PYG{p}{)}\PYG{p}{;}
            \PYG{n}{estadoRuleta} \PYG{o}{=} \PYG{n}{Estado}\PYG{p}{.}\PYG{n+na}{PAGANDO\PYGZus{}APUESTAS}\PYG{p}{;}
            \PYG{k}{this}\PYG{p}{.}\PYG{n+na}{comunicarNumeroGanador}\PYG{p}{(}\PYG{n}{numeroGanador}\PYG{p}{)}\PYG{p}{;}
            \PYG{n}{System}\PYG{p}{.}\PYG{n+na}{out}\PYG{p}{.}\PYG{n+na}{println}\PYG{p}{(}\PYG{l+s}{\PYGZdq{}}\PYG{l+s}{El saldo de la banca es ahora:}\PYG{l+s}{\PYGZdq{}}
                               \PYG{o}{+} \PYG{n}{saldo}\PYG{p}{)}\PYG{p}{;}
        \PYG{p}{\PYGZcb{}}
    \PYG{p}{\PYGZcb{}}

    \PYG{k+kd}{public} \PYG{k+kt}{void} \PYG{n+nf}{simular}\PYG{p}{(}\PYG{k+kt}{int} \PYG{n}{jugadoresPar}\PYG{p}{,}
                        \PYG{k+kt}{int} \PYG{n}{jugadoresMartingala}\PYG{p}{,}
                        \PYG{k+kt}{int} \PYG{n}{jugadoresClasicos}\PYG{p}{)} \PYG{k+kd}{throws}
    \PYG{n}{InterruptedException} \PYG{p}{\PYGZob{}}
        \PYG{n}{Thread}\PYG{o}{[}\PYG{o}{]} \PYG{n}{hilosJugadoresPares} \PYG{o}{=} \PYG{k}{new} \PYG{n}{Thread}\PYG{o}{[}\PYG{n}{jugadoresPar}\PYG{o}{]}\PYG{p}{;}
        \PYG{k}{for} \PYG{p}{(}\PYG{k+kt}{int} \PYG{n}{i} \PYG{o}{=} \PYG{l+m+mi}{0}\PYG{p}{;} \PYG{n}{i} \PYG{o}{\PYGZlt{}} \PYG{n}{jugadoresPar}\PYG{p}{;} \PYG{n}{i}\PYG{o}{+}\PYG{o}{+}\PYG{p}{)} \PYG{p}{\PYGZob{}}
            \PYG{n}{JugadorParImpar} \PYG{n}{jugador} \PYG{o}{=} \PYG{k}{new} \PYG{n}{JugadorParImpar}\PYG{p}{(}
                \PYG{l+m+mi}{1000}\PYG{p}{,} \PYG{k}{this}\PYG{p}{)}\PYG{p}{;}
            \PYG{n}{hilosJugadoresPares}\PYG{o}{[}\PYG{n}{i}\PYG{o}{]} \PYG{o}{=} \PYG{k}{new} \PYG{n}{Thread}\PYG{p}{(}\PYG{n}{jugador}\PYG{p}{)}\PYG{p}{;}
            \PYG{n}{hilosJugadoresPares}\PYG{o}{[}\PYG{n}{i}\PYG{o}{]}\PYG{p}{.}\PYG{n+na}{setName}\PYG{p}{(}\PYG{l+s}{\PYGZdq{}}\PYG{l+s}{Apostador par/impar }\PYG{l+s}{\PYGZdq{}}
                                           \PYG{o}{+} \PYG{n}{i}\PYG{p}{)}\PYG{p}{;}
            \PYG{n}{hilosJugadoresPares}\PYG{o}{[}\PYG{n}{i}\PYG{o}{]}\PYG{p}{.}\PYG{n+na}{start}\PYG{p}{(}\PYG{p}{)}\PYG{p}{;}
        \PYG{p}{\PYGZcb{}}
        \PYG{n}{Thread}\PYG{o}{[}\PYG{o}{]} \PYG{n}{hilosJugadoresMartingala} \PYG{o}{=} \PYG{k}{new} \PYG{n}{Thread}\PYG{o}{[}\PYG{n}{jugadoresMartingala}\PYG{o}{]}\PYG{p}{;}
        \PYG{k}{for} \PYG{p}{(}\PYG{k+kt}{int} \PYG{n}{i} \PYG{o}{=} \PYG{l+m+mi}{0}\PYG{p}{;} \PYG{n}{i} \PYG{o}{\PYGZlt{}} \PYG{n}{jugadoresMartingala}\PYG{p}{;} \PYG{n}{i}\PYG{o}{+}\PYG{o}{+}\PYG{p}{)} \PYG{p}{\PYGZob{}}
            \PYG{n}{JugadorMartingala} \PYG{n}{jugador} \PYG{o}{=} \PYG{k}{new} \PYG{n}{JugadorMartingala}\PYG{p}{(}
                \PYG{l+m+mi}{1000}\PYG{p}{,} \PYG{k}{this}\PYG{p}{)}\PYG{p}{;}
            \PYG{n}{hilosJugadoresMartingala}\PYG{o}{[}\PYG{n}{i}\PYG{o}{]} \PYG{o}{=} \PYG{k}{new} \PYG{n}{Thread}\PYG{p}{(}\PYG{n}{jugador}\PYG{p}{)}\PYG{p}{;}
            \PYG{n}{hilosJugadoresMartingala}\PYG{o}{[}\PYG{n}{i}\PYG{o}{]}\PYG{p}{.}\PYG{n+na}{setName}\PYG{p}{(}\PYG{l+s}{\PYGZdq{}}\PYG{l+s}{Apostador martingala }\PYG{l+s}{\PYGZdq{}}
                                                \PYG{o}{+} \PYG{n}{i}\PYG{p}{)}\PYG{p}{;}
            \PYG{n}{hilosJugadoresMartingala}\PYG{o}{[}\PYG{n}{i}\PYG{o}{]}\PYG{p}{.}\PYG{n+na}{start}\PYG{p}{(}\PYG{p}{)}\PYG{p}{;}
        \PYG{p}{\PYGZcb{}}
        \PYG{n}{Thread}\PYG{o}{[}\PYG{o}{]} \PYG{n}{hilosJugadoresClasico} \PYG{o}{=} \PYG{k}{new} \PYG{n}{Thread}\PYG{o}{[}\PYG{n}{jugadoresClasicos}\PYG{o}{]}\PYG{p}{;}
        \PYG{k}{for} \PYG{p}{(}\PYG{k+kt}{int} \PYG{n}{i} \PYG{o}{=} \PYG{l+m+mi}{0}\PYG{p}{;} \PYG{n}{i} \PYG{o}{\PYGZlt{}} \PYG{n}{jugadoresClasicos}\PYG{p}{;} \PYG{n}{i}\PYG{o}{+}\PYG{o}{+}\PYG{p}{)} \PYG{p}{\PYGZob{}}
            \PYG{n}{JugadorClasico} \PYG{n}{jugador} \PYG{o}{=} \PYG{k}{new} \PYG{n}{JugadorClasico}\PYG{p}{(}\PYG{l+m+mi}{1000}\PYG{p}{,}
                    \PYG{k}{this}\PYG{p}{)}\PYG{p}{;}
            \PYG{n}{hilosJugadoresClasico}\PYG{o}{[}\PYG{n}{i}\PYG{o}{]} \PYG{o}{=} \PYG{k}{new} \PYG{n}{Thread}\PYG{p}{(}\PYG{n}{jugador}\PYG{p}{)}\PYG{p}{;}
            \PYG{n}{hilosJugadoresClasico}\PYG{o}{[}\PYG{n}{i}\PYG{o}{]}\PYG{p}{.}\PYG{n+na}{setName}\PYG{p}{(}\PYG{l+s}{\PYGZdq{}}\PYG{l+s}{Apostador clasico }\PYG{l+s}{\PYGZdq{}}
                                             \PYG{o}{+} \PYG{n}{i}\PYG{p}{)}\PYG{p}{;}
            \PYG{n}{hilosJugadoresClasico}\PYG{o}{[}\PYG{n}{i}\PYG{o}{]}\PYG{p}{.}\PYG{n+na}{start}\PYG{p}{(}\PYG{p}{)}\PYG{p}{;}
        \PYG{p}{\PYGZcb{}}
        \PYG{k}{this}\PYG{p}{.}\PYG{n+na}{girarRuleta}\PYG{p}{(}\PYG{p}{)}\PYG{p}{;}
    \PYG{p}{\PYGZcb{}}

    \PYG{k+kd}{public} \PYG{k+kd}{static} \PYG{k+kt}{void} \PYG{n+nf}{main}\PYG{p}{(}\PYG{n}{String}\PYG{o}{[}\PYG{o}{]} \PYG{n}{args}\PYG{p}{)} \PYG{k+kd}{throws}
    \PYG{n}{InterruptedException} \PYG{p}{\PYGZob{}}
        \PYG{n}{Banca} \PYG{n}{b} \PYG{o}{=} \PYG{k}{new} \PYG{n}{Banca}\PYG{p}{(}\PYG{l+m+mi}{50000}\PYG{p}{)}\PYG{p}{;}
        \PYG{n}{b}\PYG{p}{.}\PYG{n+na}{simular}\PYG{p}{(}\PYG{l+m+mi}{5}\PYG{p}{,} \PYG{l+m+mi}{5}\PYG{p}{,} \PYG{l+m+mi}{5}\PYG{p}{)}\PYG{p}{;}
    \PYG{p}{\PYGZcb{}}
\PYG{p}{\PYGZcb{}}
\end{sphinxVerbatim}


\subsection{La clase Jugador}
\label{\detokenize{textos/anexos:la-clase-jugador}}
\begin{sphinxVerbatim}[commandchars=\\\{\}]
\PYG{k+kd}{public} \PYG{k+kd}{abstract} \PYG{k+kd}{class} \PYG{n+nc}{Jugador} \PYG{k+kd}{implements}
    \PYG{n}{Runnable} \PYG{p}{\PYGZob{}}

    \PYG{k+kd}{protected} \PYG{k+kt}{long} \PYG{n}{saldo}\PYG{p}{;}

    \PYG{k+kd}{protected} \PYG{k+kt}{boolean} \PYG{n}{enBancarrota}\PYG{p}{;}

    \PYG{k+kd}{protected} \PYG{k+kt}{long} \PYG{n}{cantidadApostada}\PYG{p}{;}

    \PYG{k+kd}{protected} \PYG{k+kt}{boolean} \PYG{n}{apuestaRealizada}\PYG{p}{;}

    \PYG{k+kd}{protected} \PYG{n}{Banca} \PYG{n}{banca}\PYG{p}{;}

    \PYG{k+kd}{protected} \PYG{n}{String} \PYG{n}{nombreHilo}\PYG{p}{;}

    \PYG{k+kd}{protected} \PYG{n}{Random} \PYG{n}{generador}\PYG{p}{;}

    \PYG{k+kd}{public} \PYG{n+nf}{Jugador}\PYG{p}{(}\PYG{k+kt}{long} \PYG{n}{saldoInicial}\PYG{p}{,} \PYG{n}{Banca} \PYG{n}{b}\PYG{p}{)} \PYG{p}{\PYGZob{}}
        \PYG{n}{saldo} \PYG{o}{=} \PYG{n}{saldoInicial}\PYG{p}{;}
        \PYG{n}{banca} \PYG{o}{=} \PYG{n}{b}\PYG{p}{;}
        \PYG{n}{apuestaRealizada} \PYG{o}{=} \PYG{k+kc}{false}\PYG{p}{;}
        \PYG{n}{generador} \PYG{o}{=} \PYG{k}{new} \PYG{n}{Random}\PYG{p}{(}\PYG{p}{)}\PYG{p}{;}
    \PYG{p}{\PYGZcb{}}

    \PYG{k+kd}{public} \PYG{k+kt}{void} \PYG{n+nf}{sumarSaldo}\PYG{p}{(}\PYG{k+kt}{long} \PYG{n}{cantidad}\PYG{p}{)} \PYG{p}{\PYGZob{}}
        \PYG{n}{saldo} \PYG{o}{=} \PYG{n}{saldo} \PYG{o}{+} \PYG{n}{cantidad}\PYG{p}{;}
    \PYG{p}{\PYGZcb{}}

    \PYG{k+kd}{public} \PYG{k+kt}{void} \PYG{n+nf}{restarSaldo}\PYG{p}{(}\PYG{k+kt}{long} \PYG{n}{cantidad}\PYG{p}{)} \PYG{p}{\PYGZob{}}
        \PYG{k}{if} \PYG{p}{(}\PYG{n}{saldo} \PYG{o}{\PYGZhy{}} \PYG{n}{cantidad} \PYG{o}{\PYGZlt{}}\PYG{o}{=} \PYG{l+m+mi}{0}\PYG{p}{)} \PYG{p}{\PYGZob{}}
            \PYG{n}{saldo} \PYG{o}{=} \PYG{l+m+mi}{0}\PYG{p}{;}
            \PYG{n}{enBancarrota} \PYG{o}{=} \PYG{k+kc}{true}\PYG{p}{;}
            \PYG{k}{return}\PYG{p}{;}
        \PYG{p}{\PYGZcb{}}
        \PYG{n}{saldo} \PYG{o}{=} \PYG{n}{saldo} \PYG{o}{\PYGZhy{}} \PYG{n}{cantidad}\PYG{p}{;}
    \PYG{p}{\PYGZcb{}}

    \PYG{k+kd}{public} \PYG{k+kt}{boolean} \PYG{n+nf}{enBancarrota}\PYG{p}{(}\PYG{p}{)} \PYG{p}{\PYGZob{}}
        \PYG{k}{return} \PYG{n}{enBancarrota}\PYG{p}{;}
    \PYG{p}{\PYGZcb{}}

    \PYG{c+cm}{/* Lo usa la banca para comunicarnos el número*/}
    \PYG{k+kd}{public} \PYG{k+kd}{abstract} \PYG{k+kt}{void} \PYG{n+nf}{comunicarNumero}\PYG{p}{(}\PYG{k+kt}{int} \PYG{n}{numero}\PYG{p}{)}\PYG{p}{;}

    \PYG{k+kd}{public} \PYG{k+kd}{abstract} \PYG{k+kt}{void} \PYG{n+nf}{hacerApuesta}\PYG{p}{(}\PYG{p}{)}\PYG{p}{;}

    \PYG{c+cm}{/* Todos los jugadores hacen lo mismo:}
\PYG{c+cm}{     * Mientras no estemos en bancarrota ni la}
\PYG{c+cm}{     * banca tampoco, hacemos apuestas. La banca}
\PYG{c+cm}{     * nos dirá el número que haya salido y en}
\PYG{c+cm}{     * ese momento (y si procede) incrementaremos}
\PYG{c+cm}{     * nuestro saldo}
\PYG{c+cm}{     */}
    \PYG{n+nd}{@Override}
    \PYG{k+kd}{public} \PYG{k+kt}{void} \PYG{n+nf}{run}\PYG{p}{(}\PYG{p}{)} \PYG{p}{\PYGZob{}}
        \PYG{n}{nombreHilo} \PYG{o}{=} \PYG{n}{Thread}\PYG{p}{.}\PYG{n+na}{currentThread}\PYG{p}{(}\PYG{p}{)}\PYG{p}{.}\PYG{n+na}{getName}\PYG{p}{(}\PYG{p}{)}\PYG{p}{;}
        \PYG{k}{while} \PYG{p}{(}\PYG{p}{(}\PYG{o}{!}\PYG{n}{enBancarrota}\PYG{p}{)}
                \PYG{o}{\PYGZam{}}\PYG{o}{\PYGZam{}} \PYG{p}{(}\PYG{o}{!}\PYG{n}{banca}\PYG{p}{.}\PYG{n+na}{enBancarrota}\PYG{p}{(}\PYG{p}{)}\PYG{p}{)}\PYG{p}{)} \PYG{p}{\PYGZob{}}
            \PYG{k+kt}{int} \PYG{n}{msAzar}\PYG{p}{;}
            \PYG{c+cm}{/* Mientras la ruleta no acepte apuestas, dormimos un}
\PYG{c+cm}{             * periodo al azar */}
            \PYG{k}{while} \PYG{p}{(}\PYG{o}{!}\PYG{n}{banca}\PYG{p}{.}\PYG{n+na}{aceptaApuestas}\PYG{p}{(}\PYG{p}{)}\PYG{p}{)} \PYG{p}{\PYGZob{}}
                \PYG{n}{msAzar} \PYG{o}{=} \PYG{k}{this}\PYG{p}{.}\PYG{n+na}{generador}\PYG{p}{.}\PYG{n+na}{nextInt}\PYG{p}{(}\PYG{l+m+mi}{500}\PYG{p}{)}\PYG{p}{;}
                \PYG{k}{try} \PYG{p}{\PYGZob{}}
                    \PYG{c+c1}{//System.out.println(nombreHilo+\PYGZdq{}:banca ocupada, durmiendo...\PYGZdq{});}
                    \PYG{n}{Thread}\PYG{p}{.}\PYG{n+na}{sleep}\PYG{p}{(}\PYG{n}{msAzar}\PYG{p}{)}\PYG{p}{;}
                \PYG{p}{\PYGZcb{}} \PYG{k}{catch} \PYG{p}{(}\PYG{n}{InterruptedException} \PYG{n}{e}\PYG{p}{)} \PYG{p}{\PYGZob{}}
                    \PYG{k}{return}\PYG{p}{;}
                \PYG{p}{\PYGZcb{}}
                \PYG{k}{if} \PYG{p}{(}\PYG{n}{banca}\PYG{p}{.}\PYG{n+na}{enBancarrota}\PYG{p}{(}\PYG{p}{)}\PYG{p}{)} \PYG{p}{\PYGZob{}}
                    \PYG{k}{return}\PYG{p}{;}
                \PYG{p}{\PYGZcb{}}
            \PYG{p}{\PYGZcb{}}
            \PYG{n}{hacerApuesta}\PYG{p}{(}\PYG{p}{)}\PYG{p}{;}
        \PYG{p}{\PYGZcb{}}
        \PYG{n}{String} \PYG{n}{nombre} \PYG{o}{=} \PYG{n}{Thread}\PYG{p}{.}\PYG{n+na}{currentThread}\PYG{p}{(}\PYG{p}{)}\PYG{p}{.}\PYG{n+na}{getName}\PYG{p}{(}\PYG{p}{)}\PYG{p}{;}
        \PYG{k}{if} \PYG{p}{(}\PYG{n}{enBancarrota}\PYG{p}{)} \PYG{p}{\PYGZob{}}
            \PYG{n}{System}\PYG{p}{.}\PYG{n+na}{out}\PYG{p}{.}\PYG{n+na}{println}\PYG{p}{(}\PYG{n}{nombre} \PYG{o}{+}
                               \PYG{l+s}{\PYGZdq{}}\PYG{l+s}{: ¡¡Me arruiné!!}\PYG{l+s}{\PYGZdq{}}\PYG{p}{)}\PYG{p}{;}
            \PYG{k}{return}\PYG{p}{;}
        \PYG{p}{\PYGZcb{}}
        \PYG{k}{if} \PYG{p}{(}\PYG{n}{banca}\PYG{p}{.}\PYG{n+na}{enBancarrota}\PYG{p}{(}\PYG{p}{)}\PYG{p}{)} \PYG{p}{\PYGZob{}}
            \PYG{n}{System}\PYG{p}{.}\PYG{n+na}{out}\PYG{p}{.}\PYG{n+na}{println}\PYG{p}{(}\PYG{n}{nombre} \PYG{o}{+}
                               \PYG{l+s}{\PYGZdq{}}\PYG{l+s}{ hizo saltar la banca}\PYG{l+s}{\PYGZdq{}}\PYG{p}{)}\PYG{p}{;}
        \PYG{p}{\PYGZcb{}}
    \PYG{p}{\PYGZcb{}}
\PYG{p}{\PYGZcb{}}
\end{sphinxVerbatim}


\subsection{La clase JugadorClasico}
\label{\detokenize{textos/anexos:la-clase-jugadorclasico}}
\begin{sphinxVerbatim}[commandchars=\\\{\}]
\PYG{k+kd}{public} \PYG{k+kd}{class} \PYG{n+nc}{JugadorClasico} \PYG{k+kd}{extends} \PYG{n}{Jugador} \PYG{p}{\PYGZob{}}

    \PYG{k+kt}{int} \PYG{n}{numeroElegido}\PYG{p}{;}

    \PYG{k+kd}{public} \PYG{n+nf}{JugadorClasico}\PYG{p}{(}\PYG{k+kt}{long} \PYG{n}{saldoInicial}\PYG{p}{,}
                          \PYG{n}{Banca} \PYG{n}{b}\PYG{p}{)} \PYG{p}{\PYGZob{}}
        \PYG{k+kd}{super}\PYG{p}{(}\PYG{n}{saldoInicial}\PYG{p}{,} \PYG{n}{b}\PYG{p}{)}\PYG{p}{;}
    \PYG{p}{\PYGZcb{}}

    \PYG{n+nd}{@Override}
    \PYG{k+kd}{public} \PYG{k+kt}{void} \PYG{n+nf}{comunicarNumero}\PYG{p}{(}\PYG{k+kt}{int} \PYG{n}{numero}\PYG{p}{)} \PYG{p}{\PYGZob{}}
        \PYG{c+cm}{/* No hace falta comprobar si el numero es}
\PYG{c+cm}{         * 0, ya que este jugador no lo elige nunca */}
        \PYG{k}{if} \PYG{p}{(}\PYG{n}{numero} \PYG{o}{=}\PYG{o}{=} \PYG{n}{numeroElegido}\PYG{p}{)} \PYG{p}{\PYGZob{}}
            \PYG{n}{System}\PYG{p}{.}\PYG{n+na}{out}\PYG{p}{.}\PYG{n+na}{println}\PYG{p}{(}\PYG{n}{nombreHilo} \PYG{o}{+}
                               \PYG{l+s}{\PYGZdq{}}\PYG{l+s}{: ¡Gana 36 veces lo jugado, 360 euros}\PYG{l+s}{\PYGZdq{}}\PYG{p}{)}\PYG{p}{;}
            \PYG{n}{banca}\PYG{p}{.}\PYG{n+na}{restarSaldo}\PYG{p}{(}\PYG{l+m+mi}{360}\PYG{p}{)}\PYG{p}{;}
            \PYG{n}{sumarSaldo}\PYG{p}{(}\PYG{l+m+mi}{360}\PYG{p}{)}\PYG{p}{;}
        \PYG{p}{\PYGZcb{}}
        \PYG{n}{System}\PYG{p}{.}\PYG{n+na}{out}\PYG{p}{.}\PYG{n+na}{println}\PYG{p}{(}\PYG{n}{nombreHilo} \PYG{o}{+}
                           \PYG{l+s}{\PYGZdq{}}\PYG{l+s}{ queda con un saldo de }\PYG{l+s}{\PYGZdq{}} \PYG{o}{+} \PYG{n}{saldo}\PYG{p}{)}\PYG{p}{;}
        \PYG{n}{apuestaRealizada} \PYG{o}{=} \PYG{k+kc}{false}\PYG{p}{;}
    \PYG{p}{\PYGZcb{}}

    \PYG{n+nd}{@Override}
    \PYG{k+kd}{public} \PYG{k+kt}{void} \PYG{n+nf}{hacerApuesta}\PYG{p}{(}\PYG{p}{)} \PYG{p}{\PYGZob{}}
        \PYG{k}{if} \PYG{p}{(}\PYG{o}{!}\PYG{n}{banca}\PYG{p}{.}\PYG{n+na}{aceptaApuestas}\PYG{p}{(}\PYG{p}{)}\PYG{p}{)}
            \PYG{k}{return}\PYG{p}{;}
        \PYG{k}{if} \PYG{p}{(}\PYG{n}{apuestaRealizada}\PYG{p}{)}
            \PYG{k}{return}\PYG{p}{;}
        \PYG{c+cm}{/* Elegimos del 1 al 36 (no se puede elegir el 0*/}
        \PYG{n}{numeroElegido} \PYG{o}{=} \PYG{l+m+mi}{1} \PYG{o}{+} \PYG{n}{generador}\PYG{p}{.}\PYG{n+na}{nextInt}\PYG{p}{(}\PYG{l+m+mi}{36}\PYG{p}{)}\PYG{p}{;}
        \PYG{n}{banca}\PYG{p}{.}\PYG{n+na}{sumarSaldo}\PYG{p}{(}\PYG{l+m+mi}{10}\PYG{p}{)}\PYG{p}{;}
        \PYG{n}{restarSaldo}\PYG{p}{(}\PYG{l+m+mi}{10}\PYG{p}{)}\PYG{p}{;}
        \PYG{n}{apuestaRealizada} \PYG{o}{=} \PYG{k+kc}{true}\PYG{p}{;}
        \PYG{n}{banca}\PYG{p}{.}\PYG{n+na}{aceptarApuesta}\PYG{p}{(}\PYG{k}{this}\PYG{p}{)}\PYG{p}{;}
    \PYG{p}{\PYGZcb{}}
\PYG{p}{\PYGZcb{}}
\end{sphinxVerbatim}


\subsection{La clase JugadorParImpar}
\label{\detokenize{textos/anexos:la-clase-jugadorparimpar}}
\begin{sphinxVerbatim}[commandchars=\\\{\}]
\PYG{k+kd}{public} \PYG{k+kd}{class} \PYG{n+nc}{JugadorParImpar} \PYG{k+kd}{extends} \PYG{n}{Jugador} \PYG{p}{\PYGZob{}}

    \PYG{k+kd}{public} \PYG{n+nf}{JugadorParImpar}\PYG{p}{(}\PYG{k+kt}{long} \PYG{n}{saldoInicial}\PYG{p}{,}
                           \PYG{n}{Banca} \PYG{n}{b}\PYG{p}{)} \PYG{p}{\PYGZob{}}
        \PYG{k+kd}{super}\PYG{p}{(}\PYG{n}{saldoInicial}\PYG{p}{,} \PYG{n}{b}\PYG{p}{)}\PYG{p}{;}
    \PYG{p}{\PYGZcb{}}

    \PYG{k+kd}{protected} \PYG{k+kt}{boolean} \PYG{n}{jugamosAPares}\PYG{p}{;}

    \PYG{n+nd}{@Override}
    \PYG{k+kd}{public} \PYG{k+kt}{void} \PYG{n+nf}{hacerApuesta}\PYG{p}{(}\PYG{p}{)} \PYG{p}{\PYGZob{}}
        \PYG{k}{if} \PYG{p}{(}\PYG{o}{!}\PYG{n}{banca}\PYG{p}{.}\PYG{n+na}{aceptaApuestas}\PYG{p}{(}\PYG{p}{)}\PYG{p}{)}
            \PYG{k}{return}\PYG{p}{;}
        \PYG{k}{if} \PYG{p}{(}\PYG{n}{apuestaRealizada}\PYG{p}{)}
            \PYG{k}{return}\PYG{p}{;}
        \PYG{c+cm}{/* Elegimos una apuesta...*/}
        \PYG{k}{if} \PYG{p}{(}\PYG{n}{generador}\PYG{p}{.}\PYG{n+na}{nextBoolean}\PYG{p}{(}\PYG{p}{)} \PYG{o}{=}\PYG{o}{=} \PYG{k+kc}{true}\PYG{p}{)} \PYG{p}{\PYGZob{}}
            \PYG{c+c1}{//System.out.println(nombreHilo+\PYGZdq{} elige apostar a par\PYGZdq{});}
            \PYG{n}{jugamosAPares} \PYG{o}{=} \PYG{k+kc}{true}\PYG{p}{;}
        \PYG{p}{\PYGZcb{}} \PYG{k}{else} \PYG{p}{\PYGZob{}}
            \PYG{c+c1}{//System.out.println(nombreHilo+\PYGZdq{} elige apostar a impar\PYGZdq{});}
            \PYG{n}{jugamosAPares} \PYG{o}{=} \PYG{k+kc}{false}\PYG{p}{;}
        \PYG{p}{\PYGZcb{}}
        \PYG{n}{banca}\PYG{p}{.}\PYG{n+na}{sumarSaldo}\PYG{p}{(}\PYG{l+m+mi}{10}\PYG{p}{)}\PYG{p}{;}
        \PYG{n}{restarSaldo}\PYG{p}{(}\PYG{l+m+mi}{10}\PYG{p}{)}\PYG{p}{;}
        \PYG{n}{apuestaRealizada} \PYG{o}{=} \PYG{k+kc}{true}\PYG{p}{;}
        \PYG{c+cm}{/* Y pedimos a la banca que nos la acepte*/}
        \PYG{n}{banca}\PYG{p}{.}\PYG{n+na}{aceptarApuesta}\PYG{p}{(}\PYG{k}{this}\PYG{p}{)}\PYG{p}{;}
    \PYG{p}{\PYGZcb{}}

    \PYG{k+kd}{public} \PYG{k+kt}{boolean} \PYG{n+nf}{esGanador}\PYG{p}{(}\PYG{k+kt}{int} \PYG{n}{num}\PYG{p}{)} \PYG{p}{\PYGZob{}}
        \PYG{c+cm}{/* Este jugador necesita comprobar si ha salido el 0,}
\PYG{c+cm}{         * aunque no lo elige nunca, ya que si hemos apostado}
\PYG{c+cm}{         * Par podríamos pensar que hemos ganado cuando no es así */}
        \PYG{k}{if} \PYG{p}{(}\PYG{n}{num} \PYG{o}{=}\PYG{o}{=} \PYG{l+m+mi}{0}\PYG{p}{)} \PYG{p}{\PYGZob{}}
            \PYG{k}{return} \PYG{k+kc}{false}\PYG{p}{;}
        \PYG{p}{\PYGZcb{}} \PYG{k}{else} \PYG{p}{\PYGZob{}}
            \PYG{k}{if} \PYG{p}{(}\PYG{p}{(}\PYG{n}{num} \PYG{o}{\PYGZpc{}} \PYG{l+m+mi}{2} \PYG{o}{=}\PYG{o}{=} \PYG{l+m+mi}{0}\PYG{p}{)} \PYG{o}{\PYGZam{}}\PYG{o}{\PYGZam{}} \PYG{p}{(}\PYG{n}{jugamosAPares}\PYG{p}{)}\PYG{p}{)} \PYG{p}{\PYGZob{}}
                \PYG{k}{return} \PYG{k+kc}{true}\PYG{p}{;}
            \PYG{p}{\PYGZcb{}}
            \PYG{k}{if} \PYG{p}{(}\PYG{p}{(}\PYG{n}{num} \PYG{o}{\PYGZpc{}} \PYG{l+m+mi}{2} \PYG{o}{!}\PYG{o}{=} \PYG{l+m+mi}{0}\PYG{p}{)} \PYG{o}{\PYGZam{}}\PYG{o}{\PYGZam{}} \PYG{p}{(}\PYG{o}{!}\PYG{n}{jugamosAPares}\PYG{p}{)}\PYG{p}{)} \PYG{p}{\PYGZob{}}
                \PYG{k}{return} \PYG{k+kc}{true}\PYG{p}{;}
            \PYG{p}{\PYGZcb{}}
        \PYG{p}{\PYGZcb{}}
        \PYG{c+c1}{//Fin del else externo}
        \PYG{k}{return} \PYG{k+kc}{false}\PYG{p}{;}
        \PYG{c+c1}{//Fin de esGanador}
    \PYG{p}{\PYGZcb{}}

    \PYG{n+nd}{@Override}
    \PYG{k+kd}{public} \PYG{k+kt}{void} \PYG{n+nf}{comunicarNumero}\PYG{p}{(}\PYG{k+kt}{int} \PYG{n}{numero}\PYG{p}{)} \PYG{p}{\PYGZob{}}
        \PYG{k}{if} \PYG{p}{(}\PYG{n}{esGanador}\PYG{p}{(}\PYG{n}{numero}\PYG{p}{)}\PYG{p}{)} \PYG{p}{\PYGZob{}}
            \PYG{c+cm}{/*Ganamos y cogemos a la banca 20 euros*/}
            \PYG{n}{System}\PYG{p}{.}\PYG{n+na}{out}\PYG{p}{.}\PYG{n+na}{println}\PYG{p}{(}\PYG{n}{nombreHilo} \PYG{o}{+}
                               \PYG{l+s}{\PYGZdq{}}\PYG{l+s}{ gana 20 euros por acertar impar}\PYG{l+s}{\PYGZdq{}}\PYG{p}{)}\PYG{p}{;}
            \PYG{n}{banca}\PYG{p}{.}\PYG{n+na}{restarSaldo}\PYG{p}{(}\PYG{l+m+mi}{20}\PYG{p}{)}\PYG{p}{;}
            \PYG{k}{this}\PYG{p}{.}\PYG{n+na}{sumarSaldo}\PYG{p}{(}\PYG{l+m+mi}{20}\PYG{p}{)}\PYG{p}{;}
        \PYG{p}{\PYGZcb{}}
        \PYG{n}{System}\PYG{p}{.}\PYG{n+na}{out}\PYG{p}{.}\PYG{n+na}{println}\PYG{p}{(}\PYG{n}{nombreHilo} \PYG{o}{+}
                           \PYG{l+s}{\PYGZdq{}}\PYG{l+s}{ se queda con un saldo de }\PYG{l+s}{\PYGZdq{}} \PYG{o}{+} \PYG{n}{saldo}\PYG{p}{)}\PYG{p}{;}
        \PYG{c+cm}{/* Sea como sea, al terminar indicamos que ya no tenemos}
\PYG{c+cm}{         * una apuesta realizada. Es decir, permitirmos al}
\PYG{c+cm}{         * jugador volver a apostar	 */}
        \PYG{n}{apuestaRealizada} \PYG{o}{=} \PYG{k+kc}{false}\PYG{p}{;}
    \PYG{p}{\PYGZcb{}}
\PYG{p}{\PYGZcb{}}
\end{sphinxVerbatim}


\subsection{La clase JugadorMartingala}
\label{\detokenize{textos/anexos:la-clase-jugadormartingala}}
\begin{sphinxVerbatim}[commandchars=\\\{\}]
\PYG{k+kd}{public} \PYG{k+kd}{class} \PYG{n+nc}{JugadorMartingala} \PYG{k+kd}{extends} \PYG{n}{Jugador} \PYG{p}{\PYGZob{}}

    \PYG{k+kd}{private} \PYG{k+kt}{int} \PYG{n}{cantidadAApostar}\PYG{p}{;}

    \PYG{k+kd}{private} \PYG{k+kt}{int} \PYG{n}{numeroElegido}\PYG{p}{;}

    \PYG{k+kd}{public} \PYG{n+nf}{JugadorMartingala}\PYG{p}{(}\PYG{k+kt}{long} \PYG{n}{saldoInicial}\PYG{p}{,}
                             \PYG{n}{Banca} \PYG{n}{b}\PYG{p}{)} \PYG{p}{\PYGZob{}}
        \PYG{k+kd}{super}\PYG{p}{(}\PYG{n}{saldoInicial}\PYG{p}{,} \PYG{n}{b}\PYG{p}{)}\PYG{p}{;}
        \PYG{n}{cantidadAApostar} \PYG{o}{=} \PYG{l+m+mi}{1}\PYG{p}{;}
    \PYG{p}{\PYGZcb{}}

    \PYG{n+nd}{@Override}
    \PYG{k+kd}{public} \PYG{k+kt}{void} \PYG{n+nf}{comunicarNumero}\PYG{p}{(}\PYG{k+kt}{int} \PYG{n}{numero}\PYG{p}{)} \PYG{p}{\PYGZob{}}
        \PYG{k}{if} \PYG{p}{(}\PYG{n}{numero} \PYG{o}{=}\PYG{o}{=} \PYG{l+m+mi}{0}\PYG{p}{)} \PYG{p}{\PYGZob{}}
            \PYG{n}{System}\PYG{p}{.}\PYG{n+na}{out}\PYG{p}{.}\PYG{n+na}{println}\PYG{p}{(}\PYG{n}{nombreHilo} \PYG{o}{+} \PYG{l+s}{\PYGZdq{}}\PYG{l+s}{ pierde }\PYG{l+s}{\PYGZdq{}} \PYG{o}{+}
                               \PYG{n}{cantidadAApostar}\PYG{p}{)}\PYG{p}{;}
            \PYG{n}{cantidadAApostar} \PYG{o}{=} \PYG{n}{cantidadAApostar} \PYG{o}{*} \PYG{l+m+mi}{2}\PYG{p}{;}
            \PYG{k}{return}\PYG{p}{;}
        \PYG{p}{\PYGZcb{}}
        \PYG{k}{if} \PYG{p}{(}\PYG{n}{numeroElegido} \PYG{o}{=}\PYG{o}{=} \PYG{n}{numero}\PYG{p}{)} \PYG{p}{\PYGZob{}}
            \PYG{k+kt}{int} \PYG{n}{beneficios} \PYG{o}{=} \PYG{n}{cantidadAApostar} \PYG{o}{*} \PYG{l+m+mi}{36}\PYG{p}{;}
            \PYG{n}{banca}\PYG{p}{.}\PYG{n+na}{restarSaldo}\PYG{p}{(}\PYG{n}{beneficios}\PYG{p}{)}\PYG{p}{;}
            \PYG{n}{sumarSaldo}\PYG{p}{(}\PYG{n}{beneficios}\PYG{p}{)}\PYG{p}{;}
            \PYG{n}{cantidadAApostar} \PYG{o}{=} \PYG{l+m+mi}{1}\PYG{p}{;}
        \PYG{p}{\PYGZcb{}}
        \PYG{k}{if} \PYG{p}{(}\PYG{n}{numeroElegido} \PYG{o}{!}\PYG{o}{=} \PYG{n}{numero}\PYG{p}{)} \PYG{p}{\PYGZob{}}
            \PYG{c+c1}{//System.out.println(nombreHilo + \PYGZdq{} pierde \PYGZdq{}+cantidadAApostar);}
            \PYG{n}{cantidadAApostar} \PYG{o}{=} \PYG{n}{cantidadAApostar} \PYG{o}{*} \PYG{l+m+mi}{2}\PYG{p}{;}
        \PYG{p}{\PYGZcb{}}
        \PYG{n}{System}\PYG{p}{.}\PYG{n+na}{out}\PYG{p}{.}\PYG{n+na}{println}\PYG{p}{(}\PYG{n}{nombreHilo} \PYG{o}{+}
                           \PYG{l+s}{\PYGZdq{}}\PYG{l+s}{ queda con un saldo de }\PYG{l+s}{\PYGZdq{}} \PYG{o}{+} \PYG{n}{saldo}\PYG{p}{)}\PYG{p}{;}
        \PYG{n}{apuestaRealizada} \PYG{o}{=} \PYG{k+kc}{false}\PYG{p}{;}
    \PYG{p}{\PYGZcb{}}

    \PYG{n+nd}{@Override}
    \PYG{k+kd}{public} \PYG{k+kt}{void} \PYG{n+nf}{hacerApuesta}\PYG{p}{(}\PYG{p}{)} \PYG{p}{\PYGZob{}}
        \PYG{k}{if} \PYG{p}{(}\PYG{o}{!}\PYG{n}{banca}\PYG{p}{.}\PYG{n+na}{aceptaApuestas}\PYG{p}{(}\PYG{p}{)}\PYG{p}{)}
            \PYG{k}{return}\PYG{p}{;}
        \PYG{k}{if} \PYG{p}{(}\PYG{n}{apuestaRealizada}\PYG{p}{)}
            \PYG{k}{return}\PYG{p}{;}
        \PYG{c+cm}{/* Elegimos del 1 al 36 (no se puede elegir el 0*/}
        \PYG{n}{numeroElegido} \PYG{o}{=} \PYG{l+m+mi}{1} \PYG{o}{+} \PYG{n}{generador}\PYG{p}{.}\PYG{n+na}{nextInt}\PYG{p}{(}\PYG{l+m+mi}{36}\PYG{p}{)}\PYG{p}{;}
        \PYG{n}{banca}\PYG{p}{.}\PYG{n+na}{sumarSaldo}\PYG{p}{(}\PYG{n}{cantidadAApostar}\PYG{p}{)}\PYG{p}{;}
        \PYG{n}{restarSaldo}\PYG{p}{(}\PYG{n}{cantidadAApostar}\PYG{p}{)}\PYG{p}{;}
        \PYG{n}{apuestaRealizada} \PYG{o}{=} \PYG{k+kc}{true}\PYG{p}{;}
        \PYG{n}{banca}\PYG{p}{.}\PYG{n+na}{aceptarApuesta}\PYG{p}{(}\PYG{k}{this}\PYG{p}{)}\PYG{p}{;}
    \PYG{p}{\PYGZcb{}}
\PYG{p}{\PYGZcb{}}
\end{sphinxVerbatim}



\renewcommand{\indexname}{Índice}
\printindex
\end{document}